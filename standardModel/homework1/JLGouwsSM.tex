\documentclass[12pt,a4]{article}
\usepackage{physics, amsmath,amsfonts,amsthm,amssymb, mathtools,steinmetz, gensymb, siunitx}	% LOADS USEFUL MATH STUFF
\usepackage{xcolor,graphicx}
\usepackage{caption}
\usepackage{subcaption}
\usepackage[left=45pt, top=20pt, right=45pt, bottom=45pt ,a4paper]{geometry} 				% ADJUSTS PAGE
\usepackage{setspace}
\usepackage{tikz}
\usepackage{pgf,tikz,pgfplots,wrapfig}
\usepackage{mathrsfs}
\usepackage{fancyhdr}
\usepackage{float}
\usepackage{array}
\usepackage{booktabs,multirow}
\usepackage{bm}
\usepackage{tensor}
\usepackage{listings}
\usepackage{slashed}
\usepackage{tikz-feynman}
 \lstset{
    basicstyle=\ttfamily\small,
    numberstyle=\footnotesize,
    numbers=left,
    backgroundcolor=\color{gray!10},
    frame=single,
    tabsize=2,
    rulecolor=\color{black!30},
    title=\lstname,
    escapeinside={\%*}{*)},
    breaklines=true,
    breakatwhitespace=true,
    framextopmargin=2pt,
    framexbottommargin=2pt,
    inputencoding=utf8,
    extendedchars=true,
    literate={á}{{$\rho$}}1 {ã}{{\~a}}1 {é}{{\'e}}1,
}
\DeclareMathOperator{\sign}{sgn}

\usetikzlibrary{decorations.text, calc}
\pgfplotsset{compat=1.7}

\usetikzlibrary{decorations.pathreplacing,decorations.markings}
\usepgfplotslibrary{fillbetween}

\newcommand{\vect}[1]{\boldsymbol{#1}}

\usepackage{hyperref}

%\usepackage[style= ACM-Reference-Format, maxbibnames=6, minnames=1,maxnames = 1]{biblatex}
%\addbibresource{references.bib}


\hypersetup{pdfborder={0 0 0},colorlinks=true,linkcolor=black,urlcolor=cyan,}
\allowdisplaybreaks
%\hypersetup{
%
%    colorlinks=true,
%
%    linkcolor=blue,
%
%    filecolor=magenta,      
%
%    urlcolor=cyan,
%
%    pdftitle={An Example},
%
%    pdfpagemode=FullScreen,
%
%    }
%}

\title{
\textsc{Standard Model Homework 1}
}
\author{\textsc{J L Gouws}
}
\date{\today
\\[1cm]}



\usepackage{graphicx}
\usepackage{array}




\begin{document}
\thispagestyle{empty}

\maketitle

\begin{enumerate}
  \item
    \begin{enumerate}
      \item
        This is more orless a straight forward derivation of expanding and matching terms, first take the derivative and look at the parts separately:
        \begin{align*}
          D_\mu &= \partial_\mu - ig_2 \frac{1}{2} W^i_\mu \sigma^i - i g_1 Y B_\mu
        \end{align*}
        First for the $W$ boson part:
        \begin{align*}
          - ig_2 \frac{1}{2} W^i_\mu \sigma^i &= - ig_2 \frac{1}{2} W^1_\mu \sigma^1 - ig_2 \frac{1}{2} W^2_\mu \sigma^2  - ig_2 \frac{1}{2} W^3_\mu \sigma^3\\
                                              &= - ig_2 W^1_\mu T^1 - ig_2 W^2_\mu T^2  - ig_2  W^3_\mu T^3\\
                                              &= - \frac{ig_2}{2} (2W^1_\mu T^1 + 2W^2_\mu T^2)  - ig_2  W^3_\mu T^3\\
                                              &= - \frac{ig_2}{2} (W^1_\mu T^1 + W^1_\mu T^1 + i W^1_\mu T^2 - i W^1_\mu T^2 - i W^2_\mu T^1  + i W^2_\mu T^1 + W^2_\mu T^2 + W^2_\mu T^2)\\
                                              & \qquad + ig_2  W^3_\mu T^3\\
                                              &= - \frac{ig_2}{2} (W^1_\mu(T^1 + i T^2) - i W^2_\mu(T^1 + i T^2) + W^1_\mu(T^1 - i T^2) - i W^2_\mu(T^1 - i T^2))\\
                                              & \qquad + ig_2  W^3_\mu T^3\\
                                              &= - \frac{ig_2}{2} ((W^1_\mu - i W^2_\mu)(T^1 + i T^2) + (W^1_\mu + i W^2_\mu)(T^1 - i T^2))  - ig_2  W^3_\mu T^3\\
                                              &= - \frac{ig_2}{2} (\sqrt{2}W^+_\mu T^+ + \sqrt{2}W^-_\mu T^-)  - ig_2  W^3_\mu T^3\\
                                              &= - \frac{ig_2}{\sqrt{2}} (W^+_\mu T^+ + W^-_\mu T^-)  - ig_2  W^3_\mu T^3
        \end{align*}
        Now for the remaining terms in the covariant derivative:
        \begin{align*}
          - ig_2  W^3_\mu T^3 - i g_1 Y B_\mu &= -ig_2 \frac{1}{\cos \theta_W }\cos \theta_W W^3T^3 - ig_1 Y B_\mu\\
                                              &= -ig_2 \frac{1}{\cos \theta_W }Z_\mu T^3 - i g_1 B_\mu T^3- ig_1 Y B_\mu\\
                                              &= -ig_2 \frac{1}{\cos \theta_W }Z_\mu T^3 - i g_1 ( T^3 + Y ) B_\mu\\
                                              &= -ig_2 \frac{1}{\cos \theta_W }Z_\mu T^3 - i g_1 Q B_\mu\\
                                              &= -ig_2 \frac{1}{\cos \theta_W }Z_\mu T^3 - i g_1 Q (\cos \theta_W A_\mu + \sin^2 \theta_W B_\mu - \cos \theta_W B_\mu \sin \theta_W W^3_\mu)\\
                                              &= -ig_2 \frac{1}{\cos \theta_W }Z_\mu T^3 - i e Q A_\mu - i g_1 Q \sin \theta_W (\sin \theta_W B_\mu - \cos \theta_W B_\mu  W^3_\mu)\\
                                              &= -ig_2 \frac{1}{\cos \theta_W }Z_\mu T^3 - i e Q A_\mu + i g_1 Q \sin \theta_W Z_\mu\\
                                              &= -ig_2 \frac{1}{\cos \theta_W }Z_\mu T^3 - i e Q A_\mu + i g_2 \frac{\sin \theta_W}{\cos \theta_W} Q \sin \theta_W Z_\mu\\
                                              &= -ig_2 \frac{1}{\cos \theta_W }Z_\mu (T^3 - Q \sin^2 \theta_W) - i e Q A_\mu
%⇧        \begin{align*}
%⇧          \frac{1}{\sqrt{2}}W^+ T^+ + W^-T^- &= (W^1 - i W^2)(T^1 + i T^2) + (W^1 + i W^2)(T^1 - i T^2)\\
%⇧                           &= (W^1 - i W^2)(T^1 + i T^2) + (W^1 + i W^2)(T^1 - i T^2)\\
%⇧                           &= W^1(T^1 + i T^2) - i W^2(T^1 + i T^2) + W^1(T^1 - i T^2) + i W^2(T^1 - i T^2)\\
%⇧                           &= W^1T^1 + i W^1T^2 - i W^2T^1 + W^2T^2 + W^1T^1 - i W^1T^2 + i W^2T^1 + W^2T^2\\
%⇧                           &= 2 W^1T^1 + 2 W^2T^2\\
%⇧                           & W^1T^1 + 2 W^2T^2\\
%⇧        \end{align*}
        \end{align*}
        And finally throwing all of the parts together gives:
        \begin{align*}
          D_\mu &= \partial_\mu - \frac{ig_2}{\sqrt{2}} (W^+_\mu T^+ + W^-_\mu T^-) -ig_2 \frac{1}{\cos \theta_W }Z_\mu (T^3 - Q \sin^2 \theta_W) - i e A_\mu Q
        \end{align*}
      \item
        Keeping only the $Z$ boson part:
        \begin{align*}
          D_\mu &= - ig_2 \frac{1}{\cos \theta_W }Z_\mu (T^3 - Q \sin^2 \theta_W)
        \end{align*}
        And calculating:
        \begin{align*}
          D_\mu(Q_L) &= - ig_Z Z_\mu (T^3 - Q \sin^2 \theta_W) (Q_L)\\
                     &= - ig_Z Z_\mu \left(\left[\begin{matrix} 1/2 & 0 \\ 0 & -1/2\end{matrix}\right] - \left(\left[\begin{matrix} 1/2 & 0 \\ 0 & -1/2\end{matrix}\right] + \left[\begin{matrix} Y_B & 0 \\ 0 & Y_B\end{matrix}\right]\right) \sin^2 \theta_W\right) \left[\begin{matrix} u_L \\ d_L\end{matrix}\right]\\
                     &= - ig_Z Z_\mu \left(\left[\begin{matrix} 1/2 & 0 \\ 0 & -1/2\end{matrix}\right] - \left[\begin{matrix} 1/2 + 1/6 & 0 \\ 0 & 1/6 - 1/2\end{matrix}\right]  \sin^2 \theta_W\right) \left[\begin{matrix} u_L \\ d_L\end{matrix}\right]\\
                     &= - ig_Z Z_\mu \left(\left[\begin{matrix} 1/2 & 0 \\ 0 & -1/2\end{matrix}\right] - \left[\begin{matrix} 2/3 & 0 \\ 0 & -1/3 \end{matrix}\right]  \sin^2 \theta_W\right) \left[\begin{matrix} u_L \\ d_L\end{matrix}\right]\\
                     &= - ig_Z Z_\mu \left[\begin{matrix} 1/2 - (2 / 3)\sin^2 \theta_W& 0 \\ 0 & -1/2 - (-1/3)\sin^2 \theta_W\end{matrix}\right]  \left[\begin{matrix} u_L \\ d_L\end{matrix}\right]\\
                     &= - ig_Z Z_\mu \left[\begin{matrix} (1/2 - (2 / 3)\sin^2 \theta_W) u_L \\ (-1/2 - (-1/3) \sin^2 \theta_W) d_L \end{matrix}  \right] 
        \end{align*}
        From which it can be seen $Q_u = 2 / 3$ and $Q_d = - 1 / 3$.
      \item
        \begin{align*}
          \bar{Q_L} i \slashed{D} Q_L &= \left[\begin{matrix} u_L \\ d_L\end{matrix}\right]^\dagger \gamma^0 g_Z \slashed{Z} \left[\begin{matrix} (1/2 - Q_u\sin^2 \theta_W) u_L \\ (-1/2 - Q_d \sin^2 \theta_W) d_L \end{matrix}  \right] \\
                                      &= \left[\begin{matrix} u_L^\dagger\gamma^5 & d_L^\dagger \gamma^0 \end{matrix}\right]  g_Z \slashed{Z} \left[\begin{matrix} (1/2 - Q_u\sin^2 \theta_W) u_L \\ (-1/2 - Q_d \sin^2 \theta_W) d_L \end{matrix}  \right] \\
                                      &= \left[\begin{matrix} \bar{u}_L & \bar{d}_L \end{matrix}\right]  g_Z \slashed{Z} \left[\begin{matrix} (1/2 - Q_u\sin^2 \theta_W) u_L \\ (-1/2 - Q_d \sin^2 \theta_W) d_L \end{matrix}  \right] \\
                                      &= \bar{u}_L g_Z \slashed{Z}  (1/2 - Q_u\sin^2 \theta_W) u_L + \bar{d}_L  (- 1/2 - Q_d \sin^2 \theta_W) d_L \\
        \end{align*}
      \item
        The right handed Fermions transform trivially under $SU(2)$, but not under $U(1)$, according to the holy table of Standard Model.
        Therefore, the covariant derivative is:
        \begin{equation*}
          D_\mu = \partial_\mu - i g_1 Y B_\mu
        \end{equation*}
%        And acting on a fermion is:
%        \begin{equation*}
%          D_\mu\psi_R = \partial_\mu\psi_R - i g_1 Y B_\mu \psi_R
%        \end{equation*}
      \item
        For this question, 
        \begin{align*}
                          & \frac{Z_\mu}{\cos \theta_W} + \frac{\sin \theta_W}{\cos \theta_W} B_\mu=  \frac{A_\mu}{\sin \theta_W} - \frac{\cos \theta_W}{\sin \theta_W} B_\mu\\
          \Leftrightarrow & \frac{\cos \theta_W}{\sin \theta_W} B_\mu + \frac{\sin \theta_W}{\cos \theta_W} B_\mu =  \frac{A_\mu}{\sin \theta_W} - \frac{Z_\mu}{\cos \theta_W}\\
          \Leftrightarrow & \frac{g_2^2 + g_1^2}{g_1 g_2} B_\mu =  \frac{A_\mu}{\sin \theta_W} - \frac{Z_\mu}{\cos \theta_W}\\
          \Leftrightarrow & B_\mu  =  \frac{g_2}{g_1}\sin^2 \theta_W A_\mu - \frac{g_Z}{g_1} \sin^2 Z_\mu
        \end{align*}
        \begin{equation*}
          D_\mu = \partial_\mu - i g_1 Y ()
        \end{equation*}
%        In this case, $W$ is in the Trivial representation, so:
%        \begin{equation*}
%          Z_\mu = - \sin \theta_W B_\mu
%        \end{equation*}
%        And the covariant derivative is:
%        \begin{align*}
%          D_\mu &= \partial_\mu + i g_1 Y \frac{Z_\mu}{\sin \theta_W } \\
%                &= \partial_\mu + i g_Z \frac{\cos \theta_W}{g^2} g_1 Y \frac{Z_\mu}{\sin \theta_W } 
%        \end{align*}
        The covariant derivative here becomes:
        \begin{align*}
          D_\mu %&= \partial_\mu - i g_1 Y (g_2\sin^2 \theta_W A_\mu - g_Z \sin^2 Z_\mu)\\
                &= \partial_\mu - i g_1 Y \left(\frac{g_2}{g_1}\sin^2 \theta_W A_\mu - \frac{g_Z}{g_1} \sin^2\theta_W Z_\mu\right)\\
                &= \partial_\mu - i Y g_2\sin^2 \theta_W A_\mu - i g_Z(-Y \sin^2\theta_W) Z_\mu)
        \end{align*}
        Ignoring other terms gives the desired result.
      \item
        Determining the interaction between the $u_R$ quarks and the $Z$ boson gives, remembering that these quarks are scalars as far as the derivative goes:
        \begin{align*}
          \bar{u}_R i\slashed{D}  \bar{u}_R
                &= \bar{u}_R  g_Z \slashed{Z} (-\frac{2}{3} \sin^2\theta_W) ) \bar{u}_R
        \end{align*}
        For the down quark $Y = -\frac{1}{3} = Q$ from the holy table.
        Generalizing this the interaction term for right handed fermions:
        \begin{align*}
          \bar{\psi}_R i\slashed{D}  \bar{\psi}_R
                &= \bar{\psi}_R  g_Z \slashed{Z} (-Q \sin^2\theta_W) ) \bar{\psi}_R
        \end{align*}
        Here the physical charge is the actual charge because the right handed fermions do not interact with the Weak field which changes the physical charge in the case of left handed fermions.
      \item
        We have the following interactions:
        \begin{align*}
          g_Z \overline{\left(P_L \psi\right)} \slashed{Z}  (T_{3, \psi} - Q \sin^2 \theta_W)P_L\psi \qquad \text{ and } \qquad \overline{\left(P_R \psi\right)}  g_Z \slashed{Z} (-Q \sin^2\theta_W) ) P_R\bar{\psi} 
        \end{align*}
        If, these are added together:
        \begin{align*}
                          & g_Z \overline{\left(P_L \psi\right)} \slashed{Z}  T_{3, \psi} P_L\psi + g_Z \overline{\left(P_L \psi\right)} \slashed{Z}( - Q \sin^2 \theta_W )P_L\psi + \overline{\left(P_R \psi\right)}  g_Z \slashed{Z} (-Q \sin^2\theta_W) ) P_R\bar{\psi}\\
          =               & g_Z \overline{\left(P_L \psi\right)} \slashed{Z}  T_{3, \psi} P_L\psi + g_Z \overline{\left((P_R + P_L) \psi\right)} \slashed{Z}( - Q \sin^2 \theta_W )(P_R + P_L)\psi \\
          =               & g_Z \overline{\left(P_L \psi\right)} \slashed{Z}  T_{3, \psi} P_L\psi + g_Z \overline{\psi} \slashed{Z}( - Q \sin^2 \theta_W ) \psi 
        \end{align*}
        Now focus on the first term:
        \begin{align*}
                          & g_Z \overline{\left(\frac{1 - \gamma^5}{2} \psi\right)} \slashed{Z}  T_{3, \psi} \left(\frac{1 - \gamma^5}{2}\right)\psi \\
          =               & \frac{1}{4} g_Z \overline{\left((1 - \gamma^5 )\psi\right)} \slashed{Z}  T_{3, \psi} \left(1 - \gamma^5\right)\psi \\
          =               & \frac{1}{4} g_Z \psi^\dagger\left(1 - \gamma^5 \right)\gamma^0 \slashed{Z}  T_{3, \psi} \left(1 - \gamma^5\right)\psi \\
          =               & \frac{1}{4} g_Z \bar{\psi}\left(1 + \gamma^5 \right) \slashed{Z}  T_{3, \psi} \left(1 - \gamma^5\right)\psi \\
          =               & \frac{1}{4} g_Z \bar{\psi} \slashed{Z} \left(1 - \gamma^5 \right) T_{3, \psi} \left(1 - \gamma^5\right)\psi \\
          =               & \frac{1}{4} g_Z \bar{\psi} \slashed{Z}  T_{3, \psi} \left(1 - \gamma^5 \right) \left(1 - \gamma^5\right)\psi \\
          =               & \frac{1}{4} g_Z \bar{\psi} \slashed{Z}  T_{3, \psi} \left(1 - 2 \gamma^5  + 1\right) \psi \\
          =               & g_Z \bar{\psi} \slashed{Z}  \left(\frac{T_{3, \psi}}{2}  -  \frac{T_{3, \psi}}{2} \gamma^5 \right) \psi \\
        \end{align*}
        Finally combining this with the previous term:
        \begin{align*}
          =               & g_Z \bar{\psi} \slashed{Z}  \left(\frac{T_{3, \psi}}{2}  -  \frac{T_{3, \psi}}{2} \gamma^5 \right) \psi  + g_Z \overline{\psi} \slashed{Z}( - Q \sin^2 \theta_W ) \psi \\
          =               & g_Z \bar{\psi} \slashed{Z}  \left(\frac{T_{3, \psi}}{2} - Q \sin^2 \theta_W -  \frac{T_{3, \psi}}{2} \gamma^5 \right) \psi \\
          =               & g_Z \bar{\psi} \slashed{Z}  \left(g_V -  g_A \gamma^5 \right) \psi \\
        \end{align*}
    \end{enumerate}
  \item
    \begin{enumerate}
      \item
        The interaction term is:
        \begin{equation*}
          g_Z \bar{\psi} \slashed{Z} g_{V A} \bar{\psi}
        \end{equation*}
        And the vertex is thus:
        \begin{figure}[!ht]
          \centering
          \feynmandiagram [horizontal=a to b] {
            i1 -- [fermion] a -- [fermion] i2,
            a -- [boson] b
          };
        \end{figure}
        And the rule is to associate a factor of:
        \begin{equation*}
          - i g_Z \gamma^\mu g_{V A}
        \end{equation*}
        The fermion propagator gets the term:
        \begin{equation*}
          -\frac{i (\slashed{p} + m)}{p^2 - m^2+ i \epsilon}
        \end{equation*}
        \begin{figure}[!ht]
          \centering
          \feynmandiagram [horizontal=a to b] {
            a -- [fermion] b 
          };
        \end{figure}
        The boson gets the term:
        \begin{equation*}
          -\frac{i \eta_{\mu\nu}}{p^2 + i \epsilon}
        \end{equation*}
      \item
        The leading order Feynman diagram is:
        \begin{figure}[!ht]
          \centering
          \feynmandiagram [vertical=a to b] {
            i1 -- [fermion] a -- [fermion] i2,
            a -- [boson] b
          };
        \end{figure}
      \item
        The amplitude is given by:
        \begin{equation*}
          \mathcal{M}(Z_0 \to f \bar{f}) = - i g_Z \epsilon_\mu^\lambda \bar{u}^r(p) \gamma^\mu g_{V A} v^s(p)
        \end{equation*}
      \item
        \begin{align*}
          |\mathcal{M}|^2 &= - i g_Z \epsilon_\mu^\lambda \bar{u}^r(p) \gamma^\mu g_{V A} v^s(p) (- i g_Z \epsilon_\nu^\lambda \bar{u}^r(p) \gamma^\nu g_{V A} v^s(p))^*\\
                                   &= g_Z^2 \epsilon_\mu^\lambda \epsilon_\nu^{\lambda *} \bar{u}^r(p) \gamma^\mu g_{V A} v^s(p) ( \bar{u}^r(p) \gamma^\nu g_{V A} v^s(p))^*
        \end{align*}
        Averaging over the polarizations, thens spins gives:
        \begin{align*}
          \overline{|\mathcal{M}|}^2 
                                   &= \sum_{s}\sum_{r}\sum_{\lambda = 1}^{3} g_Z^2 \epsilon_\mu^\lambda \epsilon_\nu^{\lambda *} \bar{u}^r(p) \gamma^\mu g_{V A} v^s(p) ( \bar{u}^r(p) \gamma^\nu g_{V A} v^s(p))^*\\
                                   &= \sum_{s}\sum_{r}g_Z^2 \left(- \eta_{\mu \nu} + \frac{k_\mu k_\nu}{M_Z^2}\right) \bar{u}^r(p) \gamma^\mu g_{V A} v^s(p) ( \bar{u}^r(p) \gamma^\nu g_{V A} v^s(p))^*
        \end{align*}
        The $Z$ boson has a mass of $\SI{91.18}{\giga \eV}$ and the electron has mass $\SI{0.510}{\mega \eV}$. 
        The electron mass is clearly much smaller than the $Z$ boson mass.
        \begin{align*}
          T^{\mu\nu}  =& \Tr\left[\gamma^\mu g_{VA} (\gamma^\alpha q_\alpha - m) g_{VA} \gamma^\nu (\gamma^\beta p_\beta + m) \right]\\
                   \to & \Tr\left[\gamma^\mu (g_V - g_A \gamma^5) \gamma^\alpha q_\alpha \gamma^\nu (g_V - g_A \gamma^5) \gamma^\beta p_\beta \right]\\
                      =& g_V^2 \Tr\left[\gamma^\mu  \gamma^\alpha q_\alpha \gamma^\nu \gamma^\beta p_\beta \right] + g_A^2 \Tr\left[\gamma^\mu \gamma^5 \gamma^\alpha q_\alpha \gamma^\nu \gamma^5 \gamma^\beta p_\beta \right] -  g_A g_V\Tr\left[\gamma^\mu \gamma^\alpha q_\alpha \gamma^\nu  \gamma^5 \gamma^\beta p_\beta \right]\\
                       & \qquad -  g_A g_V\Tr\left[\gamma^\mu \gamma^5 \gamma^\alpha q_\alpha \gamma^\nu  \gamma^\beta p_\beta \right]\\
                      =& g_V^2 \Tr\left[\gamma^\mu  \gamma^\alpha \gamma^\nu \gamma^\beta \right]q_\alpha p_\beta - g_A^2 \Tr\left[\gamma^\mu \gamma^5 \gamma^\alpha \gamma^5\gamma^\nu \gamma^\beta  \right]q_\alpha  p_\beta-  g_A g_V\Tr\left[\gamma^5 \gamma^\beta \gamma^\mu \gamma^\alpha \gamma^\nu   \right]q_\alpha p_\beta\\
                       & \qquad -  g_A g_V\Tr\left[\gamma^5 \gamma^\alpha \gamma^\nu  \gamma^\beta \gamma^\mu \right]q_\alpha p_\beta \\
                      =& g_V^2 \Tr\left[\gamma^\mu  \gamma^\alpha \gamma^\nu \gamma^\beta \right]q_\alpha p_\beta + g_A^2 \Tr\left[\gamma^\mu \gamma^5 \gamma^5 \gamma^\alpha \gamma^\nu  \gamma^\beta  \right]q_\alpha  p_\beta-  g_A g_V(-4 i)\epsilon^{\beta \mu \alpha \nu} q_\alpha p_\beta\\
                       & \qquad -  g_A g_V (-4 i)\epsilon^{\alpha \nu\beta \mu } q_\alpha p_\beta \\
                      =& g_V^2 \Tr\left[\gamma^\mu  \gamma^\alpha \gamma^\nu \gamma^\beta \right]q_\alpha p_\beta + g_A^2 \Tr\left[\gamma^\mu \gamma^\alpha \gamma^\nu \gamma^\beta  \right]q_\alpha  p_\beta - 8 i g_A g_V q_\alpha p_\beta \epsilon^{\beta \mu \alpha \nu} \\
                      =& (g_V^2 + g_A^2)\Tr\left[\gamma^\mu  \gamma^\alpha \gamma^\nu \gamma^\beta \right]q_\alpha p_\beta  - 8 i g_A g_V q_\alpha p_\beta \epsilon^{\beta \mu \alpha \nu} \\
                      =& (g_V^2 + g_A^2)\left(\eta^{\mu \alpha} \eta^{\nu \beta} + \eta^{\mu\beta}  \eta^{\alpha \nu} - \eta^{\mu\nu}  \eta^{\alpha\beta}\right) q_\alpha p_\beta  - 8 i g_A g_V q_\alpha p_\beta \epsilon^{\beta \mu \alpha \nu} \\
                      =& (g_V^2 + g_A^2)\left( q_\mu p_\nu +  q_\nu p_\mu - \eta^{\mu\nu}  q_\alpha p^\alpha\right)   - 8 i g_A g_V q_\alpha p_\beta \epsilon^{\beta \mu \alpha \nu} \\
        \end{align*}
        The second term in $T^{\mu \nu}$ is antisymmetric in $\mu$ and $\nu$, but the bracketed terms in $|\bar{\mathcal{M}}|$ are symmetric in $\mu$ and $\nu$, therefore the $\epsilon$ term in $T^{\mu\nu}$ can be ignored.
        \begin{align*}
          |\bar{\mathcal{M}}|^2 = \frac{1}{3} g_Z^2 \left(\eta_{\mu} + \frac{k_\mu k_\nu }{M_Z^2}\right) (g_V^2 + g_A^2)\left( q_\mu p_\nu +  q_\nu p_\mu - \eta^{\mu\nu}  q_\alpha p^\alpha\right)
        \end{align*}
      \item
        \begin{tabular}{l c c c c}
          \toprule
          particles         & $T_3$ & Q & $g_V$ & $g_A$ \\
          \midrule
          $e, \mu, \tau$    & $1$ & 1 & $g_V$ & $g_A$ \\
          $u, c        $    & $+1/2$ & 2/3 & $g_V$ & $g_A$ \\
          $d, s , b    $    & $-1/2$ & -1/3 & $g_V$ & $g_A$ \\
          $\nu_e, \nu_\mu, \nu_\tau   $ & $1$ & 0 & $g_V$ & $g_A$ \\
          \bottomrule
        \end{tabular}
    \end{enumerate}
\end{enumerate}
\end{document}
