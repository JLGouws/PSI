\documentclass[12pt,a4]{article}
\usepackage{physics, amsmath,amsfonts,amsthm,amssymb, mathtools,steinmetz, gensymb, siunitx}	% LOADS USEFUL MATH STUFF
\usepackage{xcolor,graphicx}
\usepackage{caption}
\usepackage{subcaption}
\usepackage[left=45pt, top=20pt, right=45pt, bottom=45pt ,a4paper]{geometry} 				% ADJUSTS PAGE
\usepackage{setspace}
\usepackage{tikz}
\usepackage{pgf,tikz,pgfplots,wrapfig}
\usepackage{mathrsfs}
\usepackage{fancyhdr}
\usepackage{float}
\usepackage{array}
\usepackage{booktabs,multirow}
\usepackage{bm}
\usepackage{tensor}
\usepackage{listings}
\usepackage{slashed}
\usepackage{tikz-feynman}
 \lstset{
    basicstyle=\ttfamily\small,
    numberstyle=\footnotesize,
    numbers=left,
    backgroundcolor=\color{gray!10},
    frame=single,
    tabsize=2,
    rulecolor=\color{black!30},
    title=\lstname,
    escapeinside={\%*}{*)},
    breaklines=true,
    breakatwhitespace=true,
    framextopmargin=2pt,
    framexbottommargin=2pt,
    inputencoding=utf8,
    extendedchars=true,
    literate={á}{{$\rho$}}1 {ã}{{\~a}}1 {é}{{\'e}}1,
}
\DeclareMathOperator{\sign}{sgn}

\usetikzlibrary{decorations.text, calc}
\pgfplotsset{compat=1.7}

\usetikzlibrary{decorations.pathreplacing,decorations.markings}
\usepgfplotslibrary{fillbetween}

\newcommand{\vect}[1]{\boldsymbol{#1}}

\usepackage{hyperref}

%\usepackage[style= ACM-Reference-Format, maxbibnames=6, minnames=1,maxnames = 1]{biblatex}
%\addbibresource{references.bib}


\hypersetup{pdfborder={0 0 0},colorlinks=true,linkcolor=black,urlcolor=cyan,}
\allowdisplaybreaks
%\hypersetup{
%
%    colorlinks=true,
%
%    linkcolor=blue,
%
%    filecolor=magenta,      
%
%    urlcolor=cyan,
%
%    pdftitle={An Example},
%
%    pdfpagemode=FullScreen,
%
%    }
%}

\title{
\textsc{Standard Model Homework 1}
}
\author{\textsc{J L Gouws}
}
\date{\today
\\[1cm]}



\usepackage{graphicx}
\usepackage{array}




\begin{document}
\thispagestyle{empty}

\maketitle

\begin{enumerate}
  \item
    \begin{enumerate}
      \item
        This is more or less a straight forward derivation of expanding and matching terms, first take the derivative and look at the parts separately:
        \begin{align*}
          D_\mu &= \partial_\mu - ig_2 \frac{1}{2} W^i_\mu \sigma^i - i g_1 Y B_\mu
        \end{align*}
        First for the $W$ boson part:
        \begin{align*}
          - ig_2 \frac{1}{2} W^i_\mu \sigma^i &= - ig_2 \frac{1}{2} W^1_\mu \sigma^1 - ig_2 \frac{1}{2} W^2_\mu \sigma^2  - ig_2 \frac{1}{2} W^3_\mu \sigma^3\\
                                              &= - ig_2 W^1_\mu T^1 - ig_2 W^2_\mu T^2  - ig_2  W^3_\mu T^3\\
                                              &= - \frac{ig_2}{2} (2W^1_\mu T^1 + 2W^2_\mu T^2)  - ig_2  W^3_\mu T^3\\
                                              &= - \frac{ig_2}{2} (W^1_\mu T^1 + W^1_\mu T^1 + i W^1_\mu T^2 - i W^1_\mu T^2 - i W^2_\mu T^1  + i W^2_\mu T^1 + W^2_\mu T^2 + W^2_\mu T^2)\\
                                              & \qquad + ig_2  W^3_\mu T^3\\
                                              &= - \frac{ig_2}{2} (W^1_\mu(T^1 + i T^2) - i W^2_\mu(T^1 + i T^2) + W^1_\mu(T^1 - i T^2) - i W^2_\mu(T^1 - i T^2))\\
                                              & \qquad + ig_2  W^3_\mu T^3\\
                                              &= - \frac{ig_2}{2} ((W^1_\mu - i W^2_\mu)(T^1 + i T^2) + (W^1_\mu + i W^2_\mu)(T^1 - i T^2))  - ig_2  W^3_\mu T^3\\
                                              &= - \frac{ig_2}{2} (\sqrt{2}W^+_\mu T^+ + \sqrt{2}W^-_\mu T^-)  - ig_2  W^3_\mu T^3\\
                                              &= - \frac{ig_2}{\sqrt{2}} (W^+_\mu T^+ + W^-_\mu T^-)  - ig_2  W^3_\mu T^3
        \end{align*}
        Now for the remaining terms in the covariant derivative, keeping the $W^3$ term from the last calculation:
        \begin{align*}
          - ig_2  W^3_\mu T^3 - i g_1 Y B_\mu &= -ig_2 \frac{1}{\cos \theta_W }\cos \theta_W W^3T^3 - ig_1 Y B_\mu\\
                                              &= -ig_2 \frac{1}{\cos \theta_W }Z_\mu T^3 - i g_1 B_\mu T^3- ig_1 Y B_\mu\\
                                              &= -ig_2 \frac{1}{\cos \theta_W }Z_\mu T^3 - i g_1 ( T^3 + Y ) B_\mu\\
                                              &= -ig_2 \frac{1}{\cos \theta_W }Z_\mu T^3 - i g_1 Q B_\mu\\
                                              &= -ig_2 \frac{1}{\cos \theta_W }Z_\mu T^3 - i g_1 Q (\cos \theta_W A_\mu + \sin^2 \theta_W B_\mu - \cos \theta_W B_\mu \sin \theta_W W^3_\mu)\\
                                              &= -ig_2 \frac{1}{\cos \theta_W }Z_\mu T^3 - i e Q A_\mu - i g_1 Q \sin \theta_W (\sin \theta_W B_\mu - \cos \theta_W B_\mu  W^3_\mu)\\
                                              &= -ig_2 \frac{1}{\cos \theta_W }Z_\mu T^3 - i e Q A_\mu + i g_1 Q \sin \theta_W Z_\mu\\
                                              &= -ig_2 \frac{1}{\cos \theta_W }Z_\mu T^3 - i e Q A_\mu + i g_2 \frac{\sin \theta_W}{\cos \theta_W} Q \sin \theta_W Z_\mu\\
                                              &= -ig_2 \frac{1}{\cos \theta_W }Z_\mu (T^3 - Q \sin^2 \theta_W) - i e Q A_\mu
%⇧        \begin{align*}
%⇧          \frac{1}{\sqrt{2}}W^+ T^+ + W^-T^- &= (W^1 - i W^2)(T^1 + i T^2) + (W^1 + i W^2)(T^1 - i T^2)\\
%⇧                           &= (W^1 - i W^2)(T^1 + i T^2) + (W^1 + i W^2)(T^1 - i T^2)\\
%⇧                           &= W^1(T^1 + i T^2) - i W^2(T^1 + i T^2) + W^1(T^1 - i T^2) + i W^2(T^1 - i T^2)\\
%⇧                           &= W^1T^1 + i W^1T^2 - i W^2T^1 + W^2T^2 + W^1T^1 - i W^1T^2 + i W^2T^1 + W^2T^2\\
%⇧                           &= 2 W^1T^1 + 2 W^2T^2\\
%⇧                           & W^1T^1 + 2 W^2T^2\\
%⇧        \end{align*}
        \end{align*}
        And finally throwing all of the parts together gives:
        \begin{align*}
          D_\mu &= \partial_\mu - \frac{ig_2}{\sqrt{2}} (W^+_\mu T^+ + W^-_\mu T^-) -ig_2 \frac{1}{\cos \theta_W }Z_\mu (T^3 - Q \sin^2 \theta_W) - i e A_\mu Q
        \end{align*}
        I don't think the calculation was too instructive, but correct me if I am wrong.
      \item
        Keeping only the $Z$ boson part of the covariant derivative yields:
        \begin{align*}
          D_\mu &= - ig_2 \frac{1}{\cos \theta_W }Z_\mu (T^3 - Q \sin^2 \theta_W)
        \end{align*}
        And applying this to the quark doublet gives:
        \begin{align*}
          D_\mu(Q_L) &= - ig_Z Z_\mu (T^3 - Q \sin^2 \theta_W) (Q_L)\\
                     &= - ig_Z Z_\mu \left(\left[\begin{matrix} 1/2 & 0 \\ 0 & -1/2\end{matrix}\right] - \left(\left[\begin{matrix} 1/2 & 0 \\ 0 & -1/2\end{matrix}\right] + \left[\begin{matrix} Y_B & 0 \\ 0 & Y_B\end{matrix}\right]\right) \sin^2 \theta_W\right) \left[\begin{matrix} u_L \\ d_L\end{matrix}\right]\\
                     &= - ig_Z Z_\mu \left(\left[\begin{matrix} 1/2 & 0 \\ 0 & -1/2\end{matrix}\right] - \left[\begin{matrix} 1/2 + 1/6 & 0 \\ 0 & 1/6 - 1/2\end{matrix}\right]  \sin^2 \theta_W\right) \left[\begin{matrix} u_L \\ d_L\end{matrix}\right]\\
                     &= - ig_Z Z_\mu \left(\left[\begin{matrix} 1/2 & 0 \\ 0 & -1/2\end{matrix}\right] - \left[\begin{matrix} 2/3 & 0 \\ 0 & -1/3 \end{matrix}\right]  \sin^2 \theta_W\right) \left[\begin{matrix} u_L \\ d_L\end{matrix}\right]\\
                     &= - ig_Z Z_\mu \left[\begin{matrix} 1/2 - (2 / 3)\sin^2 \theta_W& 0 \\ 0 & -1/2 - (-1/3)\sin^2 \theta_W\end{matrix}\right]  \left[\begin{matrix} u_L \\ d_L\end{matrix}\right]\\
                     &= - ig_Z Z_\mu \left[\begin{matrix} (1/2 - (2 / 3)\sin^2 \theta_W) u_L \\ (-1/2 - (-1/3) \sin^2 \theta_W) d_L \end{matrix}  \right] 
        \end{align*}
        From which it can be seen $Q_u = 2 / 3$ and $Q_d = - 1 / 3$.
        Which is indeed the electric charge of the quarks.
      \item
        Applying the Dirac conjugate of the left quark doublet to the covariant derivative gives:
        \begin{align*}
          \bar{Q_L} i \slashed{D} Q_L &= \left[\begin{matrix} u_L \\ d_L\end{matrix}\right]^\dagger \gamma^0 g_Z \slashed{Z} \left[\begin{matrix} (1/2 - Q_u\sin^2 \theta_W) u_L \\ (-1/2 - Q_d \sin^2 \theta_W) d_L \end{matrix}  \right] \\
                                      &= \left[\begin{matrix} u_L^\dagger\gamma^5 & d_L^\dagger \gamma^0 \end{matrix}\right]  g_Z \slashed{Z} \left[\begin{matrix} (1/2 - Q_u\sin^2 \theta_W) u_L \\ (-1/2 - Q_d \sin^2 \theta_W) d_L \end{matrix}  \right] \\
                                      &= \left[\begin{matrix} \bar{u}_L & \bar{d}_L \end{matrix}\right]  g_Z \slashed{Z} \left[\begin{matrix} (1/2 - Q_u\sin^2 \theta_W) u_L \\ (-1/2 - Q_d \sin^2 \theta_W) d_L \end{matrix}  \right] \\
                                      &= \bar{u}_L g_Z \slashed{Z}  (1/2 - Q_u\sin^2 \theta_W) u_L + \bar{d}_L  (- 1/2 - Q_d \sin^2 \theta_W) d_L \\
        \end{align*}
      \item
        The right handed Fermions transform trivially under $SU(2)$, but not under $U(1)$, according to the holy table of Standard Model.
        Therefore, the covariant derivative is:
        \begin{equation*}
          D_\mu = \partial_\mu - i g_1 Y B_\mu
        \end{equation*}
%        And acting on a fermion is:
%        \begin{equation*}
%          D_\mu\psi_R = \partial_\mu\psi_R - i g_1 Y B_\mu \psi_R
%        \end{equation*}
      \item
        Here is an instructive mistake that I made, maybe you, the marker, should ignore this part for your sanity.
        In this case, $W$ is in the Trivial representation, so:
        \begin{equation*}
          Z_\mu = - \sin \theta_W B_\mu
        \end{equation*}
        And the covariant derivative is:
        \begin{align*}
          D_\mu &= \partial_\mu + i g_1 Y \frac{Z_\mu}{\sin \theta_W } \\
                &= \partial_\mu + i g_Z \frac{\cos \theta_W}{g^2} g_1 Y \frac{Z_\mu}{\sin \theta_W } 
        \end{align*}
        But, just because the representation is trivial does not mean the $W^3$ term can be ignored.
        Hence, an expression for the other terms eliminating the $W^3$ is required:
        \begin{align*}
                          & \frac{Z_\mu}{\cos \theta_W} + \frac{\sin \theta_W}{\cos \theta_W} B_\mu=  \frac{A_\mu}{\sin \theta_W} - \frac{\cos \theta_W}{\sin \theta_W} B_\mu\\
          \Leftrightarrow & \frac{\cos \theta_W}{\sin \theta_W} B_\mu + \frac{\sin \theta_W}{\cos \theta_W} B_\mu =  \frac{A_\mu}{\sin \theta_W} - \frac{Z_\mu}{\cos \theta_W}\\
          \Leftrightarrow & \frac{g_2^2 + g_1^2}{g_1 g_2} B_\mu =  \frac{A_\mu}{\sin \theta_W} - \frac{Z_\mu}{\cos \theta_W}\\
          \Leftrightarrow & B_\mu  =  \frac{g_2}{g_1}\sin^2 \theta_W A_\mu - \frac{g_Z}{g_1} \sin^2 Z_\mu
        \end{align*}
        And substituting this in gives the covariant derivative:
        \begin{align*}
          D_\mu %&= \partial_\mu - i g_1 Y (g_2\sin^2 \theta_W A_\mu - g_Z \sin^2 Z_\mu)\\
                &= \partial_\mu - i g_1 Y \left(\frac{g_2}{g_1}\sin^2 \theta_W A_\mu - \frac{g_Z}{g_1} \sin^2\theta_W Z_\mu\right)\\
                &= \partial_\mu - i Y g_2\sin^2 \theta_W A_\mu - i g_Z(-Y \sin^2\theta_W) Z_\mu)
        \end{align*}
        Ignoring other terms gives the desired result.
      \item
        Determining the interaction between the $u_R$ quarks and the $Z$ boson gives, remembering that these quarks are scalars as far as the derivative is concerned:
        \begin{align*}
          \bar{u}_R i\slashed{D}  \bar{u}_R
                &= \bar{u}_R  g_Z \slashed{Z} \left(-\frac{2}{3} \sin^2\theta_W\right)  \bar{u}_R
        \end{align*}
        For the down quark $Y = -\frac{1}{3} = Q$ from the holy table.
        Generalizing this interaction term for right handed fermions gives:
        \begin{align*}
          \bar{\psi}_R i\slashed{D}  \bar{\psi}_R
                &= \bar{\psi}_R  g_Z \slashed{Z} (-Q \sin^2\theta_W) ) \bar{\psi}_R
        \end{align*}
        Here the physical charge is the actual charge because the right handed fermions do not interact with the Weak field which changes the physical charge (gives an effective charge) in the case of left handed fermions.
      \item
        We have the following interactions:
        \begin{align*}
          g_Z \overline{\left(P_L \psi\right)} \slashed{Z}  (T_{3, \psi} - Q \sin^2 \theta_W)P_L\psi \qquad \text{ and } \qquad \overline{\left(P_R \psi\right)}  g_Z \slashed{Z} (-Q \sin^2\theta_W) ) P_R\bar{\psi} 
        \end{align*}
        If, these are added together we get:
        \begin{align*}
                          & g_Z \overline{\left(P_L \psi\right)} \slashed{Z}  T_{3, \psi} P_L\psi + g_Z \overline{\left(P_L \psi\right)} \slashed{Z}( - Q \sin^2 \theta_W )P_L\psi + \overline{\left(P_R \psi\right)}  g_Z \slashed{Z} (-Q \sin^2\theta_W) ) P_R\bar{\psi}\\
          =               & g_Z \overline{\left(P_L \psi\right)} \slashed{Z}  T_{3, \psi} P_L\psi + g_Z \overline{\left((P_R + P_L) \psi\right)} \slashed{Z}( - Q \sin^2 \theta_W )(P_R + P_L)\psi \\
          =               & g_Z \overline{\left(P_L \psi\right)} \slashed{Z}  T_{3, \psi} P_L\psi + g_Z \overline{\psi} \slashed{Z}( - Q \sin^2 \theta_W ) \psi 
        \end{align*}
        Now focussing on the first term:
        \begin{align*}
                          & g_Z \overline{\left(\frac{1 - \gamma^5}{2} \psi\right)} \slashed{Z}  T_{3, \psi} \left(\frac{1 - \gamma^5}{2}\right)\psi \\
          =               & \frac{1}{4} g_Z \overline{\left((1 - \gamma^5 )\psi\right)} \slashed{Z}  T_{3, \psi} \left(1 - \gamma^5\right)\psi \\
          =               & \frac{1}{4} g_Z \psi^\dagger\left(1 - \gamma^5 \right)\gamma^0 \slashed{Z}  T_{3, \psi} \left(1 - \gamma^5\right)\psi \\
          =               & \frac{1}{4} g_Z \bar{\psi}\left(1 + \gamma^5 \right) \slashed{Z}  T_{3, \psi} \left(1 - \gamma^5\right)\psi \\
          =               & \frac{1}{4} g_Z \bar{\psi} \slashed{Z} \left(1 - \gamma^5 \right) T_{3, \psi} \left(1 - \gamma^5\right)\psi \\
          =               & \frac{1}{4} g_Z \bar{\psi} \slashed{Z}  T_{3, \psi} \left(1 - \gamma^5 \right) \left(1 - \gamma^5\right)\psi \\
          =               & \frac{1}{4} g_Z \bar{\psi} \slashed{Z}  T_{3, \psi} \left(1 - 2 \gamma^5  + 1\right) \psi \\
          =               & g_Z \bar{\psi} \slashed{Z}  \left(\frac{T_{3, \psi}}{2}  -  \frac{T_{3, \psi}}{2} \gamma^5 \right) \psi 
        \end{align*}
        Finally combining this with the previous term:
        \begin{align*}
          =               & g_Z \bar{\psi} \slashed{Z}  \left(\frac{T_{3, \psi}}{2}  -  \frac{T_{3, \psi}}{2} \gamma^5 \right) \psi  + g_Z \overline{\psi} \slashed{Z}( - Q \sin^2 \theta_W ) \psi \\
          =               & g_Z \bar{\psi} \slashed{Z}  \left(\frac{T_{3, \psi}}{2} - Q \sin^2 \theta_W -  \frac{T_{3, \psi}}{2} \gamma^5 \right) \psi \\
          =               & g_Z \bar{\psi} \slashed{Z}  \left(g_V -  g_A \gamma^5 \right) \psi
        \end{align*}
    \end{enumerate}
  \item
    \begin{enumerate}
      \item
        The interaction term is:
        \begin{equation*}
          g_Z \bar{\psi} \slashed{Z} g_{V A} \bar{\psi}
        \end{equation*}
        And the vertex is thus:
        \begin{figure}[!ht]
          \centering
          \feynmandiagram [horizontal=a to b] {
            i1 [particle = $\bar{\psi}$] -- [fermion] a -- [fermion] i2 [particle = $\psi$],
            a -- [boson] b [particle = $Z$]
          };
        \end{figure}

        And the rule is to associate a factor of:
        \begin{equation*}
          - i g_Z \gamma^\mu g_{V A}
        \end{equation*}
        with the vertex.
%        The fermion propagator gets the term:
%        \begin{equation*}
%          -\frac{i (\slashed{p} + m)}{p^2 - m^2+ i \epsilon}
%        \end{equation*}
%        \begin{figure}[!ht]
%          \centering
%          \feynmandiagram [horizontal=a to b] {
%            a -- [fermion] b 
%          };
%        \end{figure}
%        The boson gets the term:
%        \begin{equation*}
%          -\frac{i \eta_{\mu\nu}}{p^2 + i \epsilon}
%        \end{equation*}
      \item
        The leading order Feynman diagram is the simple decay:
        \begin{figure}[!ht]
          \centering
          \feynmandiagram [vertical=a to b] {
            i1 [particle = $\bar{f}$] -- [fermion] a -- [fermion] i2 [particle = $f$],
            a -- [boson] b [particle = $Z$]
          };
        \end{figure}
      \item
        The amplitude is given by:
        \begin{equation*}
          \mathcal{M}(Z_0 \to f \bar{f}) = - i g_Z \epsilon_\mu^\lambda \bar{u}^r(p) \gamma^\mu g_{V A} v^s(p)
        \end{equation*}
        Just using the usual rules of sandwiching the interaction term between the two fermion.
      \item
        Here, we just times the amplitute by its complex conjugate or equivalently its hermitian conjugate.
        \begin{align*}
          |\mathcal{M}|^2 &= - i g_Z \epsilon_\mu^\lambda \bar{u}^r(p) \gamma^\mu g_{V A} v^s(q) (- i g_Z \epsilon_\nu^\lambda \bar{u}^r(p) \gamma^\nu g_{V A} v^s(q))^*\\
                                   &= g_Z^2 \epsilon_\mu^\lambda \epsilon_\nu^{\lambda *} \bar{u}^r(p) \gamma^\mu g_{V A} v^s(q) ( \bar{u}^r(p) \gamma^\nu g_{V A} v^s(q))^\dagger
        \end{align*}
        Averaging over the polarizations, and addin the spins gives:
        \begin{align*}
          \overline{|\mathcal{M}|}^2 
                                   &= \frac{1}{3}\sum_{s}\sum_{r}\sum_{\lambda = 1}^{3} g_Z^2 \epsilon_\mu^\lambda \epsilon_\nu^{\lambda *} \bar{u}^r(p) \gamma^\mu g_{V A} v^s(q) ( \bar{u}^r(p) \gamma^\nu g_{V A} v^s(q))^\dagger\\
                                   &= \frac{1}{3}\sum_{s}\sum_{r}g_Z^2 \left(- \eta_{\mu \nu} + \frac{k_\mu k_\nu}{M_Z^2}\right) \bar{u}^r(p) \gamma^\mu g_{V A} v^s(q) v^s(q)^\dagger g_{V A} (\gamma^\nu)^\dagger \gamma_0 u^r(p)   \\
                                   &= \frac{1}{3}\sum_{s}\sum_{r}g_Z^2 \left(- \eta_{\mu \nu} + \frac{k_\mu k_\nu}{M_Z^2}\right) \bar{u}^r(p) \gamma^\mu g_{V A} v^s(q) v^s(q)^\dagger g_{V A} \gamma_0 \gamma^\nu  u^r(p)   \\
                                   &= \frac{1}{3}\sum_{s}\sum_{r}g_Z^2 \left(- \eta_{\mu \nu} + \frac{k_\mu k_\nu}{M_Z^2}\right) \bar{u}^r(p) \gamma^\mu g_{V A} v^s(q) v^s(q)^\dagger (g_{V } - g_A \gamma^5) \gamma_0 \gamma^\nu  u^r(p)   \\
                                   &= \frac{1}{3}\sum_{s}\sum_{r}g_Z^2 \left(- \eta_{\mu \nu} + \frac{k_\mu k_\nu}{M_Z^2}\right) \bar{u}^r(p) \gamma^\mu g_{V A} v^s(q) v^s(q)^\dagger\gamma_0 (g_{V } + g_A \gamma^5)  \gamma^\nu  u^r(p)   \\
                                   &= \frac{1}{3}\sum_{s}\sum_{r}g_Z^2 \left(- \eta_{\mu \nu} + \frac{k_\mu k_\nu}{M_Z^2}\right) \bar{u}^r(p) \gamma^\mu g_{V A} v^s(q) \bar{v}^s(q) (g_{V } + g_A \gamma^5)  \gamma^\nu  u^r(p)   \\
                                   &= \frac{1}{3}\sum_{s}\sum_{r}g_Z^2 \left(- \eta_{\mu \nu} + \frac{k_\mu k_\nu}{M_Z^2}\right) \bar{u}^r(p) \gamma^\mu g_{V A} v^s(q) \bar{v}^s(q) \gamma^\nu (g_{V } - g_A \gamma^5)    u^r(p)   \\
                                   &= \frac{1}{3}\sum_{s}\sum_{r}g_Z^2 \left(- \eta_{\mu \nu} + \frac{k_\mu k_\nu}{M_Z^2}\right) \bar{u}^r(p) \gamma^\mu g_{V A} v^s(q) \bar{v}^s(q) \gamma^\nu g_{V A}   u^r(p)   \\
                                   &= \frac{1}{3}\sum_{r}g_Z^2 \left(- \eta_{\mu \nu} + \frac{k_\mu k_\nu}{M_Z^2}\right) \bar{u}^r(p) \gamma^\mu g_{V A} (\gamma^\alpha q_\alpha - m) \gamma^\nu g_{V A}   u^r(p)   \\
                                   &= \frac{1}{3}g_Z^2 \left(- \eta_{\mu \nu} + \frac{k_\mu k_\nu}{M_Z^2}\right) \Tr\left[ \gamma^\mu g_{V A} (\gamma^\alpha q_\alpha - m) \gamma^\nu g_{V A}   (\gamma^\beta p_\beta +m)\right] 
%                                   &= \frac{1}{3}\sum_{s}\sum_{r}g_Z^2 \left(- \eta_{\mu \nu} + \frac{k_\mu k_\nu}{M_Z^2}\right) \bar{u}^r(p) \gamma^\mu g_{V A} v^s(p) \bar{v}^s(p)(-1)\gamma_0 g_{V A} (\gamma^\nu)^\dagger \gamma_0 u^r(p)   \\
%                                   &= \frac{1}{3}\sum_{s}\sum_{r}g_Z^2 \left(- \eta_{\mu \nu} + \frac{k_\mu k_\nu}{M_Z^2}\right) \bar{u}^r(p) \gamma^\mu g_{V A} v^s(p) \bar{v}^s(p)(-1)\gamma_0 g_{V A} (\gamma^\nu)^\dagger \gamma_0 u^r(p)   \\
%                                   &= \frac{1}{3}\sum_{r}g_Z^2 \left(- \eta_{\mu \nu} + \frac{k_\mu k_\nu}{M_Z^2}\right) \bar{u}^r(p) \gamma^\mu g_{V A} (\gamma^\alpha q_\alpha - m)(-1)\gamma_0 g_{V A} \gamma_0\gamma^\nu  u^r(p)   \\
%                                   &= \frac{1}{3}g_Z^2 \left(- \eta_{\mu \nu} + \frac{k_\mu k_\nu}{M_Z^2}\right) \sum_{r}\bar{u}^r(p) \gamma^\mu g_{V A} (\gamma^\alpha q_\alpha - m)(-1)\gamma_0 g_{V A} \gamma_0 \gamma^\nu u^r(p)   \\
%                                   &= \frac{1}{3}g_Z^2 \left(- \eta_{\mu \nu} + \frac{k_\mu k_\nu}{M_Z^2}\right) \Tr\left[\gamma^\mu g_{V A} (\gamma^\alpha q_\alpha - m)(-1)\gamma_0 g_{V A} \gamma_0 \gamma^\nu(\gamma^\beta p_\beta - m) \right]  \\
        \end{align*}
        The $Z$ boson has a mass of $\SI{91.18}{\giga \eV}$ and the electron has mass $\SI{0.510}{\mega \eV}$. 
        The heaviest fermion, other than the top quark, is the bottom quark with mass $~ \SI{5}{\giga \eV}$.
        The $Z$ boson has a much higher mass, meaning, that upon decay, the fermions will have high momentum which will dominate the mass terms in the above expression, hence the fermion mass can be ignored.
        Now for calculating the trace part gives:
        \begin{align*}
          T^{\mu\nu}  =& \Tr\left[\gamma^\mu g_{VA} (\gamma^\alpha q_\alpha - m) g_{VA} \gamma^\nu (\gamma^\beta p_\beta + m) \right]\\
                   \to & \Tr\left[\gamma^\mu (g_V - g_A \gamma^5) \gamma^\alpha q_\alpha \gamma^\nu (g_V - g_A \gamma^5) \gamma^\beta p_\beta \right]\\
                      =& g_V^2 \Tr\left[\gamma^\mu  \gamma^\alpha q_\alpha \gamma^\nu \gamma^\beta p_\beta \right] + g_A^2 \Tr\left[\gamma^\mu \gamma^5 \gamma^\alpha q_\alpha \gamma^\nu \gamma^5 \gamma^\beta p_\beta \right] -  g_A g_V\Tr\left[\gamma^\mu \gamma^\alpha q_\alpha \gamma^\nu  \gamma^5 \gamma^\beta p_\beta \right]\\
                       & \qquad -  g_A g_V\Tr\left[\gamma^\mu \gamma^5 \gamma^\alpha q_\alpha \gamma^\nu  \gamma^\beta p_\beta \right]\\
                      =& g_V^2 \Tr\left[\gamma^\mu  \gamma^\alpha \gamma^\nu \gamma^\beta \right]q_\alpha p_\beta - g_A^2 \Tr\left[\gamma^\mu \gamma^5 \gamma^\alpha \gamma^5\gamma^\nu \gamma^\beta  \right]q_\alpha  p_\beta-  g_A g_V\Tr\left[\gamma^5 \gamma^\beta \gamma^\mu \gamma^\alpha \gamma^\nu   \right]q_\alpha p_\beta\\
                       & \qquad -  g_A g_V\Tr\left[\gamma^5 \gamma^\alpha \gamma^\nu  \gamma^\beta \gamma^\mu \right]q_\alpha p_\beta \\
                      =& g_V^2 \Tr\left[\gamma^\mu  \gamma^\alpha \gamma^\nu \gamma^\beta \right]q_\alpha p_\beta + g_A^2 \Tr\left[\gamma^\mu \gamma^5 \gamma^5 \gamma^\alpha \gamma^\nu  \gamma^\beta  \right]q_\alpha  p_\beta-  g_A g_V(-4 i)\epsilon^{\beta \mu \alpha \nu} q_\alpha p_\beta\\
                       & \qquad -  g_A g_V (-4 i)\epsilon^{\alpha \nu\beta \mu } q_\alpha p_\beta \\
                      =& g_V^2 \Tr\left[\gamma^\mu  \gamma^\alpha \gamma^\nu \gamma^\beta \right]q_\alpha p_\beta + g_A^2 \Tr\left[\gamma^\mu \gamma^\alpha \gamma^\nu \gamma^\beta  \right]q_\alpha  p_\beta - 8 i g_A g_V q_\alpha p_\beta \epsilon^{\beta \mu \alpha \nu} \\
                      =& (g_V^2 + g_A^2)\Tr\left[\gamma^\mu  \gamma^\alpha \gamma^\nu \gamma^\beta \right]q_\alpha p_\beta  - 8 i g_A g_V q_\alpha p_\beta \epsilon^{\beta \mu \alpha \nu} \\
                      =& 4 (g_V^2 + g_A^2)\left(\eta^{\mu \alpha} \eta^{\nu \beta} + \eta^{\mu\beta}  \eta^{\alpha \nu} - \eta^{\mu\nu}  \eta^{\alpha\beta}\right) q_\alpha p_\beta  - 8 i g_A g_V q_\alpha p_\beta \epsilon^{\beta \mu \alpha \nu} \\
                      =& 4 (g_V^2 + g_A^2)\left( q_\mu p_\nu +  q_\nu p_\mu - \eta^{\mu\nu}  q_\alpha p^\alpha\right)   - 8 i g_A g_V q_\alpha p_\beta \epsilon^{\beta \mu \alpha \nu}
        \end{align*}
        Note the identity given in the question seems to be missing a factor of $4$.
        The second term in $T^{\mu \nu}$ is antisymmetric in $\mu$ and $\nu$, but the bracketed terms in $|\bar{\mathcal{M}}|^2$ are symmetric in $\mu$ and $\nu$, therefore the $\epsilon$ term in $T^{\mu\nu}$ can be ignored.
        \begin{align*}
          |\bar{\mathcal{M}}|^2 = \frac{1}{3} g_Z^2 \left(\eta_{\mu} + \frac{k_\mu k_\nu }{M_Z^2}\right) (g_V^2 + g_A^2)\left( q_\mu p_\nu +  q_\nu p_\mu - \eta^{\mu\nu}  q_\alpha p^\alpha\right)
        \end{align*}
      \item
        The folowing table contains the data.
        \begin{table}[!h]
          \centering
          \begin{tabular}{l c c c c}
            \toprule
            particles         & $T_3$ & Q & $g_V$ & $g_A$ \\
            \midrule
            $e, \mu, \tau$    & $-1/2$   & -1 & $-1/4 + \sin^2\theta_W$ & $-1/4$ \\
            $u, c        $    & $+1/2$  & 2/3 & $+1/4 - (2 / 3) \sin^2\theta_W$ & $+1/4$ \\
            $d, s , b    $    & $-1/2$  & -1/3 & $-1/4 - (-1/3) \sin^2 \theta_W$ & $-1/4$ \\
            $\nu_e, \nu_\mu, \nu_\tau $ & $1/2$ & 0 & $1 / 4$ & $1 / 4$ \\
            \bottomrule
          \end{tabular}
        \end{table}
        The top quark has a mass ~$\SI{170}{\giga \eV}$ while the $Z$ boson's mass is only around $\SI{90}{\giga \eV}$ making the decay impossible--mass/energy will not be conserved.
      \item
        First the constant $g_Z$ is required:
        \begin{equation*}
          g_Z^2 = \frac{g_2^2}{\cos^2 \theta_W} = \frac{e^2}{\sin^2 \theta_W \cos^2 \theta_W} = \frac{4 \pi \alpha_Z^{EM}}{\cos^2 \theta_W \sin^2 \theta_W} = \frac{4 \pi \alpha_Z^{EM}}{(1 - \sin^2 \theta_W) \sin^2 \theta_W} = 0.5548
        \end{equation*}
        \begin{align*}
          \Gamma(Z \to e^+ e^-) 
                                &= \frac{1}{2 M_Z}\int \frac{1}{32 \pi^2} d\Omega \frac{4 g_Z^2 M_Z^2}{3} (g_V^2 + g_A^2) \\
                                &= \frac{M_Z}{12 \pi } g_Z^2 (g_V^2 + g_A^2) \\
                                &= \SI{0.08441180}{\giga \eV}
        \end{align*}
        For the neutrinos:
        \begin{align*}
          \Gamma(Z \to \nu+ \bar{\nu}) 
                                &= \frac{1}{2 M_Z}\int \frac{1}{32 \pi^2} d\Omega \frac{4 g_Z^2 M_Z^2}{3} (g_V^2 + g_A^2) \\
                                &= \frac{M_Z}{12 \pi } g_Z^2 (g_V^2 + g_A^2) \\
                                &= \SI{0.16775001}{\giga \eV}
        \end{align*}
        For the up quarks, remebering that there is an extra factor of three for colour:
        \begin{align*}
          \Gamma(Z \to u \bar{u}) 
                                &= 3 \times \frac{1}{2 M_Z}\int \frac{1}{32 \pi^2} d\Omega \frac{4 g_Z^2 M_Z^2}{3} (g_V^2 + g_A^2) \\
                                &= \frac{M_Z}{4 \pi } g_Z^2 (g_V^2 + g_A^2) \\
                                &= \SI{0.2892457556}{\giga \eV}
        \end{align*}
        And similarly for the down quarks.
        \begin{align*}
          \Gamma(Z \to d \bar{d}) 
                                &= 3 \times \frac{1}{2 M_Z}\int \frac{1}{32 \pi^2} d\Omega \frac{4 g_Z^2 M_Z^2}{3} (g_V^2 + g_A^2) \\
                                &= \frac{M_Z}{4 \pi } g_Z^2 (g_V^2 + g_A^2) \\
                                &= \SI{0.372583}{\giga \eV}
        \end{align*}
      \item
        The total decay width is, taking into account the generations, but ignoring the top quark:
        \begin{equation*}
          \Gamma_\text{total} = 3 \Gamma(Z \to e^+ e^-) + 3 \Gamma(Z \to \nu \bar{\nu}) + 2 \Gamma(Z \to u \bar{u}) +  3 \Gamma(Z \to d \bar{d}) = \SI{2.45}{\giga \eV}
        \end{equation*}
      \item
        The lifetime of the $Z_0$ boson will inversely proportional to the decay width or:
        \begin{equation*}
          \frac{\hbar}{\Gamma_{\text{total}}} = \SI{2.6833e-25}{\second}
        \end{equation*}
        Which comes from the Heisenberg uncertainty principle.
      \item
        The branching ratios are, blindly plugging in numbers:
        \begin{equation*}
          B(Z \to e^+ e^-) = 3.44 \%
        \end{equation*}
        \begin{equation*}
          B(Z \to \nu \bar{\nu}) = 6.84 \%
        \end{equation*}
        \begin{equation*}
          B(Z \to u \bar{u}) = 11.79 \%
        \end{equation*}
        \begin{equation*}
          B(Z \to d \bar{d}) =  15.19\%
        \end{equation*}
        Which are all pretty close to the experimental values.
        For example $B(Z \to e^+ e^-) = 3.3632$.
    \end{enumerate}
\end{enumerate}
\end{document}
