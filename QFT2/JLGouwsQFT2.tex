\documentclass[12pt,a4]{article}
\usepackage{physics, amsmath,amsfonts,amsthm,amssymb, mathtools,steinmetz, gensymb, siunitx}	% LOADS USEFUL MATH STUFF
\usepackage{xcolor,graphicx}
\usepackage{caption}
\usepackage{subcaption}
\usepackage[left=45pt, top=20pt, right=45pt, bottom=45pt ,a4paper]{geometry} 				% ADJUSTS PAGE
\usepackage{setspace}
\usepackage{tikz}
\usepackage{pgf,tikz,pgfplots,wrapfig}
\usepackage{mathrsfs}
\usepackage{fancyhdr}
\usepackage{tikz-feynman}
\usepackage{float}
\usepackage{array}
\usepackage{booktabs,multirow}
\usepackage{bm}
\usepackage{tensor}
\usepackage{slashed}
\usepackage{listings}
 \lstset{
    basicstyle=\ttfamily\small,
    numberstyle=\footnotesize,
    numbers=left,
    backgroundcolor=\color{gray!10},
    frame=single,
    tabsize=2,
    rulecolor=\color{black!30},
    title=\lstname,
    escapeinside={\%*}{*)},
    breaklines=true,
    breakatwhitespace=true,
    framextopmargin=2pt,
    framexbottommargin=2pt,
    inputencoding=utf8,
    extendedchars=true,
    literate={á}{{$\rho$}}1 {ã}{{\~a}}1 {é}{{\'e}}1,
}
\DeclareMathOperator{\sign}{sgn}

\usetikzlibrary{decorations.text, calc}
\pgfplotsset{compat=1.7}

\usetikzlibrary{decorations.pathreplacing,decorations.markings}
\usepgfplotslibrary{fillbetween}

\newcommand{\vect}[1]{\boldsymbol{#1}}

\usepackage{hyperref}

%\usepackage[style= ACM-Reference-Format, maxbibnames=6, minnames=1,maxnames = 1]{biblatex}
%\addbibresource{references.bib}


\hypersetup{pdfborder={0 0 0},colorlinks=true,linkcolor=black,urlcolor=cyan,}
\allowdisplaybreaks
%\hypersetup{
%
%    colorlinks=true,
%
%    linkcolor=blue,
%
%    filecolor=magenta,      
%
%    urlcolor=cyan,
%
%    pdftitle={An Example},
%
%    pdfpagemode=FullScreen,
%
%    }
%}

\title{
\textsc{QFT II Homework 1}
}
\author{\textsc{J L Gouws}
}
\date{\today
\\[1cm]}



\usepackage{graphicx}
\usepackage{array}




\begin{document}
\thispagestyle{empty}

\maketitle

\begin{enumerate}
  \item
    \begin{enumerate}
      \item 
        $S$ had dimension 1 and has the form:
        \begin{equation*}
          S = \int d^2x \bar{\psi}_a i \slashed{\partial} \psi^a + \frac{g^2}{2N} \left(\bar{\psi}_a \psi^a\right)^2
        \end{equation*}
%        Thus $\bar{\psi}_a \psi^a$ has dimension 1.% and $\psi$ has dimenstion $1/2$. and 
%        This gives $-$
        which implies $g^2$ must have the same dimension as $[\slashed{\partial}] = 1$ or $[g] = 1/ 2$ and note $[\psi] = 1$.
      \item
        Under the transformation $\psi_a \to \gamma^5\psi_a$ the Dirac conjugate of $\psi$ changes to $\bar{\psi}_a \to \left(\gamma^5 \psi_a \right)^\dagger \gamma^0 = \psi_a^\dagger {\gamma^5}^\dagger \gamma^0 = -\psi_a^\dagger \gamma^1 \gamma^0 \gamma^0 = -\psi_a^\dagger \gamma^0 \gamma^0 \gamma^1 = - \bar{\psi_a}\gamma^5$ so the the Lagrangian changes like:
        \begin{align*}
          \mathcal{L}_{NG} \to \mathcal{L}_{NG} &= \int d^2x -\bar{\psi}_a \gamma^5 \gamma^i\partial_i \gamma^5 \psi^a + \frac{g^2}{2N} \left(-\bar{\psi}_a \gamma^5 \gamma^5\psi^a\right)^2\\
                                                &= \int d^2x -\bar{\psi}_a \gamma^5 \gamma^i \gamma^5 \partial_i \psi^a + \frac{g^2}{2N} \left(-\bar{\psi}_a \psi^a\right)^2\\
                                                &= \int d^2x \bar{\psi}_a \gamma^5 \gamma^5 \gamma^i \partial_i \psi^a + \frac{g^2}{2N} \left(\bar{\psi}_a \psi^a\right)^2\\
                                                &= \int d^2x \bar{\psi}_a \gamma^5 \gamma^5 \gamma^i \partial_i \psi^a + \frac{g^2}{2N} \left(\bar{\psi}_a \psi^a\right)^2\\
                                                &= \int d^2x \bar{\psi}_a \slashed{\partial_i} \psi^a + \frac{g^2}{2N} \left(\bar{\psi}_a \psi^a\right)^2
        \end{align*}
        A mass term will have the form:
        \begin{align*}
           m \bar{\psi}_a \psi^a
        \end{align*}
        Which becomes negative under a chiral transformation and so the chiral symmetry forbids a mass term.
      \item
        The diagram for the two point function is:
        \begin{center}
          \begin{tikzpicture}[baseline={(current bounding box.center)},
              arrowlabel/.style={
                /tikzfeynman/momentum/.cd, % means that the following keys are read from the /tikzfeynman/momentum family
                arrow shorten=#1,arrow distance=1.5mm
              },
              arrowlabel/.default=0.4
            ]
            \begin{feynman}
              \vertex [dot] (a) {};
              \vertex  at (a) (c);
              \vertex  at (a) (d);
              \vertex [left=5em of a] (i3) {};
              \vertex [right=5em of a] (i4) {};
              \diagram * {
%                (i3) -- [fermion] (a) -- [out=40, in=0, min distance=0.5cm] (b) -- [out=180, in=140, min distance=0.5cm] (a) -- [fermion] (i4),
                (i3) -- [fermion, edge label=$p_1$] (d) -- [out=40, in=140, min distance=10em, fermion, edge label = {$k$}] (c)  -- [fermion, edge label = {$p_2$}] (i4),
  %                (a) -- [out=45, in=-45, loop, min distance=2cm] (a),%-- [plain, half left] (b) -- [plain, half left] (a),
              };
  %              \draw (a) arc [start angle=0, end angle=-180, radius=0.7cm];
            \end{feynman}
          \end{tikzpicture}
        \end{center}
        The matrix element of this is:
        \begin{equation*}
          \langle \Omega | \psi_{\alpha i} \psi_{\beta j}  | \Omega \rangle = \left(i {g \over \sqrt{n}}\right)^2 \frac{i}{\slashed{k}} = - \frac{g^2 / N}{(\slashed{p} - \slashed{q})_{\alpha \beta}} \delta_{ij}
        \end{equation*}
        Another diagram for the two point function is:
        \begin{center}
          \begin{tikzpicture}[baseline={(current bounding box.center)},
              arrowlabel/.style={
                /tikzfeynman/momentum/.cd, % means that the following keys are read from the /tikzfeynman/momentum family
                arrow shorten=#1,arrow distance=1.5mm
              },
              arrowlabel/.default=0.4
            ]
            \begin{feynman}
              \vertex [dot] (a) {};
              \vertex  at (a) (c);
              \vertex  at (a) (d);
              \vertex [left=5em of a] (i3) {};
              \vertex [right=5em of a] (i4) {};
              \diagram * {
%                (i3) -- [fermion] (a) -- [out=40, in=0, min distance=0.5cm] (b) -- [out=180, in=140, min distance=0.5cm] (a) -- [fermion] (i4),
                (i3) -- [fermion, edge label=$p_1$] (d) -- [out=140, in=40, min distance=10em, fermion, edge label = {$k$}] (c)  -- [fermion, edge label = {$p_2$}] (i4),
  %                (a) -- [out=45, in=-45, loop, min distance=2cm] (a),%-- [plain, half left] (b) -- [plain, half left] (a),
              };
  %              \draw (a) arc [start angle=0, end angle=-180, radius=0.7cm];
            \end{feynman}
          \end{tikzpicture}
        \end{center}
        With the opposite sign propagator, but the two diagrams subtract. Therefore, both the diagrams contribute constructively to the two point function so must be zero for other reasons.
      \item
        The functional integral for the Gross-Neveu theory is:
        \begin{align*}
          Z = \int D[\bar{\psi}, \psi]  \exp\left[\int d^2x \bar{\psi}_a i \slashed{\partial} \psi^a + \frac{g^2}{2N} \left(\bar{\psi}_a \psi^a\right)^2\right]
        \end{align*}
        And the functional integral for the scalar theory scalar field theory:
        \begin{align*}
          Z &= \int D[\sigma, \bar{\psi}, \psi] \exp\left[\int d^2x \bar{\psi}_a i (\slashed{\partial} - \sigma ) \psi^a - \frac{N}{2g^2} \sigma^2\right]\\
            &= \int D[\bar{\psi}, \psi] \int D \sigma \exp\left[\int d^2x \bar{\psi}_a i \slashed{\partial} \psi^a - \sigma \bar{\psi}_a \psi^a  - \frac{N}{2g^2}\sigma^2 \right]\\
            &= \int D[\bar{\psi}, \psi] \exp\left[\int d^2x \bar{\psi}_a i \slashed{\partial} \psi^a \right] \int D \sigma \exp\left[\int d^2x - \sigma \bar{\psi}_a \psi^a  - \frac{N}{2g^2}\sigma^2 \right]
        \end{align*}
        The integral in the action can be worked out with a discretization on a lattice, and taking a limit:
        \begin{align*}
          \int D \sigma \exp\left[\int d^2x - \sigma \bar{\psi}_a \psi^a  - \frac{N}{2g^2}\sigma^2 \right] &= \lim_{\epsilon \to 0 } \int D \sigma \exp\left[\sum_x \left(- \sigma(x) \bar{\psi}_a \psi^a  - \frac{N}{2g^2}\sigma^2(x) \right) \epsilon\right] \\ 
                                                                                                           &\propto \lim_{\epsilon \to 0 } \left(\int \prod_x  d \sigma(x)\right) e^\epsilon\exp\left[\sum_x - \sigma \bar{\psi}_a \psi^a  - \frac{N}{2g^2}\sigma^2 \right] \\ 
                                                                                                           &\propto \lim_{\epsilon \to 0 } \left(\prod_x e^\epsilon \int   d \sigma(x) \exp\left[- \sigma \bar{\psi}_a \psi^a  - \frac{N}{2g^2}\sigma^2 \right] \right)
        \end{align*}
        The $\sigma$ integral here is a Gaussian integral of the form:
        \begin{equation*}
          \int_{-\infty}^{\infty} dx \exp\left( -\frac{1}{ 2} a x^2 + Jx\right)  dx = \left ( {2\pi \over a } \right) ^{1\over 2} \exp\left( { J^2 \over 2a }\right)
        \end{equation*}
        Thus:
        \begin{align*}
          \int D \sigma \exp\left[\int d^2x - \sigma \bar{\psi}_a \psi^a  - \frac{N}{2g^2}\sigma^2 \right] &\propto \lim_{\epsilon \to 0 } \left(\prod_x e^\epsilon \left ( {2\pi g^2 \over N } \right) ^{1\over 2} \exp\left( \frac{g^2}{2N}(\bar{\psi}_a \psi^a)^2\right) \right)\\
                                                                                                           &\propto \lim_{\epsilon \to 0 } \left(\prod_x  \exp\left(\epsilon \frac{g^2}{2N}(\bar{\psi}_a \psi^a)^2\right) \right)\\
                                                                                                           &\propto \lim_{\epsilon \to 0 } \exp\left(\sum_x \epsilon \frac{g^2}{2N}(\bar{\psi}_a \psi^a)^2\right) \\
                                                                                                           &\propto  \exp\left(\int d^2 x \frac{g^2}{2N}(\bar{\psi}_a \psi^a)^2\right) 
        \end{align*}
        Therefore the path integral is:
        \begin{align*}
          Z &\propto \int D[\bar{\psi}, \psi] \exp\left[\int d^2x \bar{\psi}_a i \slashed{\partial} \psi^a \right] \exp\left[\int d^2x \frac{g^2}{2N}(\bar{\psi}_a \psi^a)^2 \right]\\
            &= \int D[\bar{\psi}, \psi] \exp\left[\int d^2x \bar{\psi}_a i \slashed{\partial} \psi^a + \frac{g^2}{2N}(\bar{\psi}_a \psi^a)^2 \right]
        \end{align*}
        Which is the same partition function up to some pontential overall normalization factor, which will not affect any physics.
      \item
        The functional integral for the scalar Gross-Neveu theory is:
        \begin{align*}
          Z &= \int D[\sigma, \bar{\psi}, \psi] \exp\left[\int d^2x \bar{\psi}_a i (\slashed{\partial} - \sigma ) \psi^a - \frac{N}{2g^2} \sigma^2\right]\\
            &= \int D[\sigma] \int D[\bar{\psi}, \psi] \exp\left[\int d^2x \bar{\psi}_a i (\slashed{\partial} - \sigma ) \psi^a - \frac{N}{2g^2} \sigma^2\right]\\
            &= \int D[\sigma] \exp\left[- \int d^2x \frac{N}{2g^2} \sigma^2\right]\int D[\bar{\psi}, \psi] \exp\left[\int d^2x \bar{\psi}_a i (\slashed{\partial} - \sigma ) \psi^a \right]\\
        \end{align*}
        And for Fermionic/Grassmann variables, using a short hand $\int \mathcal{D} \bar{\psi}_a \mathcal{D} \psi^a$ for the measure over all fermionic fields:
        \begin{align*}
          &\int \mathcal{D} \bar{\psi}_a \mathcal{D} \psi^a \exp\left[\sum_{a = 1}^n\int dx^2 \bar{\psi}_a (i \slashed{\partial} - \sigma) \psi^a \right] \\
          =& \int \mathcal{D} \bar{\psi}_a \mathcal{D} \psi^a \prod_{b = 1}^N \exp\left[\int dx^2 \bar{\psi}_b (i \slashed{\partial} - \sigma) \psi^b \right]\\
                                                                                                                                             =& \prod_{a = 1}^N \left[\int \mathcal{D} \bar{\psi}_a \mathcal{D} \psi_a  \lim_{\epsilon \to 0}\exp\left[\sum_x \epsilon \bar{\psi}_a (i \slashed{\partial} - \sigma) \psi^a \right]\right]\\
                                                                                                                                       \propto      &  \prod_{a = 1}^N \left[\lim_{\epsilon \to 0}  \prod_x e^\epsilon \int d \bar{\psi}_a(x) d \psi_a(x)  \exp\left[\bar{\psi}_a (i \slashed{\partial} - \sigma) \psi^a \right]\right]\\
                                                                                                                                       \propto      & \prod_{a = 1}^N \left[\lim_{\epsilon \to 0} \prod_xe^\epsilon \det(i \slashed{\partial} - \sigma)\right]\\
                                                                                                                                       \propto      & \prod_{a = 1}^N \left[\lim_{\epsilon \to 0} \prod_x e^\epsilon \exp\left[\tr(\log(i \slashed{\partial} - \sigma))\right]\right]\\
                                                                                                                                       \propto      & \prod_{a = 1}^N \left[\lim_{\epsilon \to 0} \exp\left[\sum_x \epsilon\tr(\log(i \slashed{\partial} - \sigma))\right]\right]\\
                                                                                                                                       \propto      & \prod_{a = 1}^N  \exp\left[\lim_{\epsilon \to 0} \sum_x \epsilon\tr(\log(i \slashed{\partial} - \sigma))\right]\\
                                                                                                                                       \propto      & \prod_{a = 1}^N\exp\left[\int d^2 x \tr(\log(i \slashed{\partial} - \sigma))\right]\\
                                                                                                                                       \propto      & \exp\left[N \int d^2 x \tr(\log(i \slashed{\partial} - \sigma))\right]\\
        \end{align*}
        Where $\tr$ indicates the full operator trace, of both the spinor indices and of the kernel.
        Putting this back in the partion function expression yields:
        \begin{align*}
          Z &\propto \int D[\sigma] \exp\left[- \int d^2x \frac{N}{2g^2} \sigma^2\right]\exp\left[ \int d^2 x N \tr(\log(i \slashed{\partial} - \sigma))\right]\\
            &= \int D[\sigma] \exp\left[- \int d^2x \frac{N}{2g^2} \sigma^2 +  \int d^2 x N \tr(\log(i \slashed{\partial} - \sigma))\right]\\
            &= \int D[\sigma] \exp\left[- \int d^2x \frac{N}{2g^2} \sigma^2 +  N \tr(\log(i \slashed{\partial} - \sigma))\right]
        \end{align*}
        Which is the functional integral presented in the question up to some overall constant normalization factor.
      \item
        The saddle point approximation requires the extremum of the action:
        \begin{align*}
          \frac{\delta S}{\delta \sigma} &= \frac{\partial \mathcal{L}}{\partial \sigma}\\
                                         &= -\frac{N}{g^2}\sigma + N \frac{\sigma}{4 \pi}\log\left(\frac{\Lambda^2}{\sigma^2}\right)
        \end{align*}
        Since the action is extremized for $\sigma_c$, the action's functional derivative should be zero at this extremum:
        \begin{align*}
                      & 0 = -\frac{N}{g^2}\sigma_c + N \frac{\sigma_c}{4 \pi}\log\left(\frac{\Lambda^2}{\sigma^2_c}\right)\\
        \end{align*}
        Clearly $\sigma_c = 0$ is a solution, but also:
        \begin{align*}
          \Rightarrow & \frac{4 \pi}{g^2} = \log\left(\frac{\Lambda^2}{\sigma^2_c}\right)\\
          \Rightarrow & \frac{\Lambda^2}{\sigma^2_c} = e^{\frac{4 \pi}{g^2}}\\
          \Rightarrow & \sigma^2_c = \Lambda^2 e^{- \frac{4 \pi}{g^2}}\\
          \Rightarrow & \sigma_c = \pm \Lambda e^{- \frac{2 \pi}{g^2}}
        \end{align*}
        These solutions are unphysical. 
        Physical solution no not depend on the regulator.
      \item
        The result that the main contribution to the functional integral is the path $\sigma_c = 0$ agrees with the previous results, because $\sigma$ is an effective mass field for the Gross-Neveu model as seen in question d).
        The results from $b)$ and $c)$ indicate that the Gross-Neveu fermions have no mass and this is also true to a good degree of approximation after any renoralization effects.
        Therefore, the effective mass for the Gross-Neveu fermions is zero and classically where no renormalization is needed they will have zero mass.
    \end{enumerate}
\end{enumerate}
\end{document}
