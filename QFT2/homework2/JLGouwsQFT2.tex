\documentclass[12pt,a4]{article}
\usepackage{physics, amsmath,amsfonts,amsthm,amssymb, mathtools,steinmetz, gensymb, siunitx}	% LOADS USEFUL MATH STUFF
\usepackage{xcolor,graphicx}
\usepackage{caption}
\usepackage{subcaption}
\usepackage[left=45pt, top=20pt, right=45pt, bottom=45pt ,a4paper]{geometry} 				% ADJUSTS PAGE
\usepackage{setspace}
\usepackage{tikz}
\usepackage{pgf,tikz,pgfplots,wrapfig}
\usepackage{mathrsfs}
\usepackage{fancyhdr}
\usepackage{tikz-feynman}
\usepackage{float}
\usepackage{array}
\usepackage{booktabs,multirow}
\usepackage{bm}
\usepackage{tensor}
\usepackage{slashed}
\usepackage{listings}
 \lstset{
    basicstyle=\ttfamily\small,
    numberstyle=\footnotesize,
    numbers=left,
    backgroundcolor=\color{gray!10},
    frame=single,
    tabsize=2,
    rulecolor=\color{black!30},
    title=\lstname,
    escapeinside={\%*}{*)},
    breaklines=true,
    breakatwhitespace=true,
    framextopmargin=2pt,
    framexbottommargin=2pt,
    inputencoding=utf8,
    extendedchars=true,
    literate={á}{{$\rho$}}1 {ã}{{\~a}}1 {é}{{\'e}}1,
}
\DeclareMathOperator{\sign}{sgn}

\usetikzlibrary{decorations.text, calc}
\pgfplotsset{compat=1.7}

\usetikzlibrary{decorations.pathreplacing,decorations.markings}
\usepgfplotslibrary{fillbetween}

\newcommand{\vect}[1]{\boldsymbol{#1}}

\usepackage{hyperref}

%\usepackage[style= ACM-Reference-Format, maxbibnames=6, minnames=1,maxnames = 1]{biblatex}
%\addbibresource{references.bib}


\hypersetup{pdfborder={0 0 0},colorlinks=true,linkcolor=black,urlcolor=cyan,}
\allowdisplaybreaks
%\hypersetup{
%
%    colorlinks=true,
%
%    linkcolor=blue,
%
%    filecolor=magenta,      
%
%    urlcolor=cyan,
%
%    pdftitle={An Example},
%
%    pdfpagemode=FullScreen,
%
%    }
%}

\title{
\textsc{QFT II Homework 2}
}
\author{\textsc{J L Gouws}
}
\date{\today
\\[1cm]}

\usepackage{graphicx}
\usepackage{array}

\begin{document}
\thispagestyle{empty}

\maketitle

\begin{enumerate}
  \item
    \begin{enumerate}
      \item
        The integral $I(p_1 + p_2, m_R;\Lambda)$ corresponds to the diagram:

        \begin{center}
          \feynmandiagram [horizontal=a to b][line width=1pt] {
            i1 -- [dotted, momentum=$p_1$] a -- [plain, reversed momentum=$k$, out = 50, in = 130, min distance = 20pt] b -- [dotted, momentum=$p_3$] i3,
            i2 -- [dotted, momentum=$p_2$] a -- [plain, momentum'=$k + p_1 + p_2$, out = -50, in = -130, min distance = 20pt] b -- [dotted, momentum=$p_4$] i4
          };
        \end{center}

        The interaction vertex comes from the interaction term $g_R \phi^4 / 4!$.
        This has the associated integral in Euclidean space:
        \begin{equation*}
          I(p_1 + p_2, m_R;\Lambda) = \int_{|k| < \Lambda} \frac{d^4 k}{(2 \pi)^4} \frac{1}{k^2 + m_R^2} \frac{1}{(k + p_1 + p_2)^2 + m_R^2}
        \end{equation*}
      \item
        For $m_R = 0$:
        \begin{align*}
          I(p, 0;\Lambda) &= \int_{|k| < \Lambda} \frac{d^4 k}{(2 \pi)^4} \frac{1}{k^2} \frac{1}{(k + p)^2}\\
                          &= \frac{2 \pi^2}{(2 \pi)^4} \int_{0 }^{\Lambda}  d k\frac{k^3}{k^2} \frac{1}{(k + p)^2}\\
                          &= \frac{2 \pi^2}{(2 \pi)^4} \int_{0 }^{\Lambda}  d k \frac{k}{(k + p)^2}\\
                          &= \frac{2 \pi^2}{(2 \pi)^4} \left[ \log\left(\left|k+p\right|\right)+\dfrac{p}{k+p}\right]_0^\Lambda\\
                          &= \frac{2 \pi^2}{(2 \pi)^4} \left[ \log\left(\left|\Lambda + p\right|\right)+\dfrac{p}{\Lambda +p } - \log p - 1\right]\\
        \end{align*}
        For large Lambda/momentum modes, the first $\log$ can be expanded about $p = 0$:
        \begin{align*}
          I(p, 0;\Lambda) &\approx \frac{2 \pi^2}{(2 \pi)^4} \left[ \log\left(\Lambda\right) + \frac{p}{\Lambda} + \dfrac{p}{\Lambda} - \log p - 1\right]\\
                          &\approx \frac{2 \pi^2}{(2 \pi)^4} \left[ \log\left(\Lambda\right)  - \log p \right]\\
                          &\approx \frac{2 \pi^2}{(2 \pi)^4} \left[ \frac{1}{2}\log\left(\frac{\Lambda^2}{p^2}\right)\right]\\
                          &\approx \frac{1}{(4 \pi)^2} \log\left(\frac{\Lambda^2}{p^2}\right)
        \end{align*}
        Where the constant terms have been ignored since they will be insignifican for large $\Lambda$.
      \item
        Differentiating the integral and swapping the order of integration and differentialtion yields:
        \begin{align*}
          \frac{\partial}{\partial (m^2_R)}I(p, m_R;\Lambda) &= \frac{\partial}{\partial (m^2_R)}\int_{|k| < \Lambda} \frac{d^4 k}{(2 \pi)^4} \frac{1}{k^2 + m_R^2} \frac{1}{(k + p)^2 + m_R^2}\\
                                                             &= \int_{|k| < \Lambda} \frac{d^4 k}{(2 \pi)^4} \frac{\partial}{\partial (m^2_R)}\left[\frac{1}{k^2 + m_R^2} \frac{1}{(k + p)^2 + m_R^2}\right]\\
                                                             &= - \int_{|k| < \Lambda} \frac{d^4 k}{(2 \pi)^4} \frac{1}{(k^2 + m_R^2)^2} \frac{1}{(k + p)^2 + m_R^2}\\
                                                             &  \qquad-\int_{|k| < \Lambda} \frac{d^4 k}{(2 \pi)^4} \left[\frac{1}{k^2 + m_R^2} \frac{1}{[(k - p)^2 + m_R^2]^2}\right]
        \end{align*}
        For large $\Lambda$ this integral goes like
        \begin{align*}
          \frac{\partial}{\partial (m^2_R)}I(p, m_R;\Lambda)  &\sim \left[\int_{|k| < \Lambda} k^3 d k \frac{1}{k^4} \frac{1}{k^2} + \int_{|k| < \Lambda} k^3d k \frac{1}{k^2} \frac{1}{k^4}\right]\\
                                                              &\sim \left[\int_{|k| < \Lambda} \frac{d k}{k^3}\right]\\
                                                              &\sim \frac{1}{\Lambda^2}
        \end{align*}
        Which goes to zero as $\Lambda$ tends to infinity indicating that the integral does not diverge for large momenta.
      \item
        For the four point generating functional:
        \begin{equation*}
          \hat \Gamma_R(p_i) = g_R \Rightarrow 0 = -\frac{\hbar g_R^2}{2}[I(p_1 + p_2, 0) + I(p_1 + p_3, 0) + I(p_1 + p_4, 0)] + \hbar C_1
        \end{equation*}
        The integrals here are:
        \begin{align*}
          I(p_a + p_b, 0;\Lambda) &= \int_{|k| < \Lambda} \frac{d^4 k}{(2 \pi)^4} \frac{1}{k^2 } \frac{1}{(k + p_a + p_b)^2 }\\
                                  &\sim \frac{1}{(4 \pi)^2} \log\left(\frac{\Lambda^2}{(p_a + p_b)^2}\right)\\
                                  &\sim \frac{1}{(4 \pi)^2} \log\left(\frac{\Lambda^2}{\mu^2}\right)
        \end{align*}
        So that:
        \begin{equation*}
          C_1 = \frac{ g_R^2}{2}\left[3 \times \frac{1}{(4 \pi)^2} \log\left(\frac{\Lambda^2}{\mu^2}\right)\right] = \frac{3g_R^2}{2}\frac{1}{(4 \pi)^2} \log\left(\frac{\Lambda^2}{\mu^2}\right)
        \end{equation*}
      \item
        \begin{align*}
          \hat \Gamma_R(p_i)  &= g_R -\frac{\hbar g_R^2}{2} \frac{1}{(4 \pi)^2}\left[\log\left(\frac{\Lambda^2}{(p_1 + p_2)^2}\right) + \log\left(\frac{\Lambda^2}{(p_1 + p_3)^2}\right) + \log\left(\frac{\Lambda^2}{(p_1 + p_4)^2}\right)\right]\\ 
                              &\quad +  \frac{3\hbar g_R^2}{2}\frac{1}{(4 \pi)^2} \log\left(\frac{\Lambda^2}{\mu^2}\right)\\
                              &=g_R - \frac{\hbar g_R^2}{2} \frac{1}{(4 \pi)^2}\left[\log\left(\frac{\mu^2}{(p_1 + p_2)^2}\right) + \log\left(\frac{\mu^2}{(p_1 + p_3)^2}\right) + \log\left(\frac{\mu^2}{(p_1 + p_4)^2}\right)\right]
        \end{align*}
      \item
        Consider the four point funcitons being evaluated with all the external momenta $\mu '$
        \begin{align}
          &\Gamma^{(4)}(\mu', g_R', \mu') = \Gamma^{(4)}(\mu', g_R, \mu)\nonumber \\
          &g_R'(\mu ') = g_R(\mu) - \frac{3\hbar g_R^2(\mu)}{2}\frac{1}{(4 \pi)^2}\log\frac{\mu^2}{\mu'^2} \label{eq:gAtDiffScales}
        \end{align}
      \item
        The beta function is given by:
        \begin{equation*}
          \beta(g_R) = \mu \frac{\partial g_R}{\partial\mu}
        \end{equation*}
        Taking Eq~\ref{eq:gAtDiffScales} and setting $\mu '  = \mu + d\mu$, then taylor expanding gives:
        \begin{align*}
                          & g_R'(\mu + \partial\mu) = g_R(\mu) - \frac{3\hbar g_R^2(\mu)}{2}\frac{1}{(4 \pi)^2}\left(\log \mu^2  - \log (\mu + d\mu)^2\right)\\
          \Leftrightarrow & g_R(\mu) + \frac{\partial g_R(\mu)}{\partial\mu} d\mu + \mathcal{O}(d\mu^2) = g_R(\mu) - \frac{3\hbar g_R^2(\mu)}{2}\frac{1}{(4 \pi)^2}\left(\log \mu^2 - \log (\mu + d\mu)^2\right)\\
          \Leftrightarrow & \frac{\partial g_R(\mu)}{\partial\mu}d \mu + \mathcal{O}(d\mu^2) = - \frac{3\hbar g_R^2(\mu)}{2}\frac{1}{(4 \pi)^2}\left(\log \mu^2 - \log \mu^2  - \frac{2}{\mu} d\mu)\right) +\mathcal{O}(d\mu^2)\\
          \Leftrightarrow & \frac{\partial g_R(\mu)}{\partial\mu}d \mu + \mathcal{O}(d\mu^2) =  \frac{1}{(4 \pi)^2}\frac{3\hbar g_R^2(\mu)}{\mu} d\mu +\mathcal{O}(d\mu^2)
        \end{align*}
        From this the $\beta$ function is:
        \begin{equation*}
          \beta(g_R) = \mu \frac{\partial g_R(\mu)}{\partial\mu} = \frac{3\hbar}{(4 \pi)^2} g_R^2(\mu)
        \end{equation*}
        And here is a plot of the $\beta$-function.
        \begin{figure}[!ht]
        \begin{center}
        \begin{tikzpicture}
          \begin{axis}[
                axis x line=center,
                axis y line=center,
                ticks=none,
                xlabel={$g_R$},
                ylabel={$\beta(g_R)$},
                xlabel style={below right},
                ylabel style={above left},
                xmin=-0.5,
                xmax=10,
                ymin=0,
                ymax=30]
                domain=0:120,
            ]
            \addplot[no marks, domain=0:11] {x^2};
          \end{axis}
        \end{tikzpicture}
        \end{center}
        \end{figure}

      \item
        From the graph it is evident that the coupling is stronger at higher energies, and weaker at lower energies since the $\beta$ function is alway positive.
        The coupling gets stronger slowly at lower energies, and gets increasingly stronger at higher energies.
        The growth probably slows down at the highest energy possible of experiment, but then more powerful methods of high-energy physics are probably required to see this.
      \item
        The integral for the tadpole diagram is:
        \begin{equation*}
          T(m_R) = \int_{|k| < \Lambda} \frac{d^D k}{(2 \pi)^D} \frac{1}{k^2 + m_R^2}
        \end{equation*}
        With $D = 4$ in this homework's specific case.
      \item
        To calculate the total solid angle of a $D$-hypersphere, take the $D$-dimensional Gaussian integral and change it to spherical coordinates.
        In this transformation, the radial differential is $dr$, and the angle differentials become $r f_i(\pmb{\theta}) d \theta_i$, where $f_i$ is some funciton of the angles in the hyper-spherical coordinate system and $i = 1, ..., D-1$.
        \begin{equation*}
          \int_{-\infty}^{\infty} dx_1 ... dx_D\exp\left(-\sum_{i = 1}^Dx_i^2\right) = \int \Omega_D \int_0^\infty dr r^{D - 1} e^{-r^2}
        \end{equation*}
        Where:
        \begin{equation*}
          \Omega_D = \prod_{i = 1}^{D - 1} f_i(\pmb{\theta}) d \theta_i
        \end{equation*}
        The expression on the lefthand side:
        \begin{equation*}
          \int_{-\infty}^{\infty} dx_1 ... dx_D\exp\left(-\sum_{i = 1}^Dx_i^2\right) 
        \end{equation*}
        Is the integral of a product of Gaussian integrals:
        \begin{equation*}
          \prod_{i = 1}^D\left[\int_{-\infty}^{\infty} dx_i \exp\left(x_i^2\right) \right] = \prod_{i = 1}^D\sqrt{\pi} = (\pi)^{D / 2}
        \end{equation*}
        The other intergal requiring evaluation is:
        \begin{align*}
          \int_0^\infty dr r^{D- 1} e^{-r^2} = \int_0^\infty \frac{1}{2} u^{-1 / 2} du u^{(D - 1)/2} e^{-u} = \frac{1}{2} \int_0^\infty  du u^{D/2 - 1} e^{-u} = \frac{1}{2} \Gamma(D / 2)
        \end{align*}
        Therefore:
        \begin{equation*}
          \pi^{D/ 2} = \frac{1}{2} \Gamma(D / 2)\int \Omega_D  \Rightarrow \int \Omega_D = \frac{2\pi^{D/ 2}}{ \Gamma(D / 2)}
        \end{equation*}
      \item
        For $4$ dimensions, the integral of the tadpole diagram is:
        \begin{align*}
          T(m_R)  & = \int_{|k| < \Lambda} \frac{d^4 k}{(2 \pi)^4} \frac{1}{k^2 + m_R^2} \\
                  & = \int \Omega_4 \int_{k < \Lambda}  \frac{d k}{(2 \pi)^4} \frac{k^3 }{k^2 + m_R^2}
        \end{align*}
        Since only the large Lambda behavior is of interest, the integral can be expanded about $m_R = 0$:
        \begin{align*}
          T(m_R)  & = \frac{2 \pi^{2}}{(2 \pi)^4}\int_{k < \Lambda}  d k \frac{k^3 }{k^2 + m_R^2}\\
                  & = \frac{2 \pi^{2}}{(2 \pi)^4}\frac{1}{2} (\Lambda^2 + m^2 \log m^2 - m^2 \log (m^2 + \Lambda^2))\\
                  & \approx \frac{2 \pi^{2}}{(2 \pi)^4}\frac{1}{2} \left(\Lambda^2 + m_R^2 \log m_R^2 - 2m_R^2 \log \Lambda  - \frac{m_R^4}{\Lambda^2}\right)\\
                  & \approx \frac{1}{(4 \pi)^2} \left(\Lambda^2 - m_R^2 \log \frac{\Lambda^2}{m_R^2} \right)\\
                  & \approx \frac{1}{(4 \pi)^2} \left(\Lambda^2 - m_R^2 \log \Lambda^2 + m_R^2\log m_R^2 \right)
%                  & = \frac{2 \pi^{2}}{(2 \pi)^4}\int_{k < \Lambda}  d k \left[\frac{k^3 }{k^2}  - \frac{k^3 }{k^4} m_R^2 + \frac{k^3}{k^6}m_R^4 + \mathcal{O}(m_R^6) \right]\\
%                  & \approx \frac{2 \pi^{2}}{(2 \pi)^4}\int_{k < \Lambda}  d k \left[k  - \frac{m_R^2}{k}  + \frac{m_R^4}{k^3}\right]\\
%                  & \approx \frac{2 \pi^{2}}{(2 \pi)^4} \left[\frac{1}{2}\Lambda^2  - m_R^2 \log \Lambda  - \frac{m_R^4}{2\Lambda^2}\right]
        \end{align*}
        Which are the quadratic and logarithmically divergent terms as well as the finite term for large $\Lambda$.
      \item
        Writing the full expression for the two point irreducible funciton gives:
        \begin{equation*}
          \hat\Gamma^{(2)}_R(p) = p^2 + m_R^2 + \frac{\hbar g_R }{(4\pi)^2}\frac{1}{2} \left(\Lambda^2 - m_R^2 \log \frac{\Lambda^2}{m_R^2}\right)+ \hbar B
        \end{equation*}
        Requiring $\hat\Gamma^{(2)}_R(0, m_R, g_R , \mu) = m_R^2$ gives:
        \begin{align*}
          & m_R^2 = m_R^2 + \frac{1}{2}\frac{\hbar g_R }{(4\pi)^2} \left(\Lambda^2 - m_R^2 \log \frac{\Lambda^2}{m_R^2}\right)+ \hbar B\\
          &\Rightarrow  B = - \frac{1}{2}\frac{ g_R }{(4\pi)^2}\left(\Lambda^2 - m_R^2 \log \frac{\Lambda^2}{m_R^2}\right)\\
          &\Rightarrow  B = - \frac{1}{2}\frac{ g_R }{(4\pi)^2}\left(\Lambda^2 - m_R^2\times \log \frac{\Lambda^2}{\mu^2}\right)
        \end{align*}
        From this:
        \begin{gather*}
          B_{1, 0}(g_R, \mu, \Lambda) = - \frac{1}{2}\frac{ g_R }{(4\pi)^2}\frac{\Lambda^2}{2} \qquad \quad B_{1, 1}(g_R, \mu, \Lambda) = \frac{1}{2} \frac{ g_R }{(4\pi)^2} \log \frac{\Lambda^2}{\mu^2}
        \end{gather*}
      \item
        The full two point effective action is now:
        \begin{equation*}
          \hat\Gamma^{(2)}_R(p) = p^2 + m_R^2 + \frac{1}{2} \frac{\hbar g_R }{(4\pi)^2} \left(\Lambda^2 - m_R^2 \log \frac{\Lambda^2}{m_R^2}\right)  - \frac{1}{2} \frac{ \hbar g_R }{(4\pi)^2}\left(\Lambda^2 - m_R^2\times \log \frac{\Lambda^2}{\mu^2}\right)
        \end{equation*}
        Which gives that:
        \begin{equation*}
          \hat\Gamma^{(2)}_R(p_i, g_R, \mu) = p^2 + m_R^2 + \frac{1}{2}\frac{\hbar g_R }{(4\pi)^2}  m_R^2 \log \frac{m_R^2}{\mu^2} = p^2 + m_R^2 - \frac{1}{2} \frac{\hbar g_R }{(4\pi)^2}  m_R^2 \log \frac{\mu^2}{m_R^2}
        \end{equation*}
        This is independent of $\Lambda$ and so has no divergences for $\Lambda \to \infty$.
        Notice from the last expression that performing an experiment at a higher energy scale, higher $\mu$, logarithmically decreases the mass in the experiment keeping the measured mass finite.
        The expression is also only linear in $p^2$ so will be finite for any experimentally feasible value of $p$.
      \item
        Requiring the two point irreducible function to be independent of the scale is equivlent to:
        \begin{equation*}
          \hat\Gamma^{(2)}_R(p_i, m_R', g_R', \mu') =  \hat\Gamma^{(2)}_R(p_i, m_R, g_R, \mu)
        \end{equation*}
        This leads to:
        \begin{equation*}
          m_R'^2(\mu') + \frac{\hbar g_R'(\mu') }{(4\pi)^2}  m_R'^2(\mu') \log \frac{m_R'^2(\mu')}{\mu'^2} = m_R^2(\mu) + \frac{\hbar g_R (\mu)}{(4\pi)^2}  m_R^2(\mu) \log \frac{m_R^2(\mu)}{\mu^2}
        \end{equation*}
        Now using the fact:
        \begin{equation*}
          g_R'(\mu ') = g_R(\mu) - \frac{3\hbar g_R^2(\mu)}{2}\log\frac{\mu^2}{\mu'^2}
        \end{equation*}
        The relation is:
        \begin{equation}
          m_R'^2(\mu') + \frac{1}{2}\frac{\hbar g_R(\mu)}{(4\pi)^2}  m_R'^2(\mu') \log \frac{m_R'^2(\mu')}{\mu'^2}= m_R^2(\mu) + \frac{1}{2}\frac{\hbar g_R (\mu)}{(4\pi)^2}  m_R^2(\mu) \log \frac{m_R^2(\mu)}{\mu^2} \label{eq:mAtDiffScales}
        \end{equation}
        Where the terms in order $\hbar^2$ are ignored.
      \item
        By definition:
        \begin{equation*}
          \Gamma^{(2)}_R(p) = p^2 + m_{\text{phys}}^2
        \end{equation*}
        from its relation to the particles propagator.
        Now evaluating this expression at scale $\mu^2 = m_R^2$ gives:
        \begin{align*}
          p^2 + m_{\text{phys}}^2 = \hat\Gamma^{(2)}_R(p_i, g_R, m_R^2) &= p^2 + m_R^2 + \frac{\hbar g_R }{(4\pi)^2}  m_R^2 \log \frac{m_R^2}{m_R^2}\\
                                                                        &= p^2 + m_R^2 + \frac{\hbar g_R }{(4\pi)^2}  m_R^2 \log(1) \\
                                                                        &= p^2 + m_R^2                                              
        \end{align*}
        Eliminating $p^2$ on both sides gives:
        \begin{align*}
          m_{\text{phys}}^2 = m_R^2
        \end{align*}
        or more generally:
        \begin{align*}
          m_{\text{phys}}^2 = m_R^2 + \frac{1}{2}\frac{\hbar g_R }{(4\pi)^2}  m_R^2 \log \frac{m_R^2}{\mu^2}
        \end{align*}
      \item
        Taking Eq.~\ref{eq:mAtDiffScales} and setting $\mu' = \mu + d \mu$ gives:
        \begin{align*}
            & m_R'^2(\mu + d \mu) + \frac{1}{2}\frac{\hbar g_R(\mu)}{(4\pi)^2}  m_R'^2(\mu + d \mu) \log \frac{m_R'^2(\mu + d \mu)}{(\mu + d \mu)^2}\\
          = &\left[ m_R^2(\mu) + \frac{d m_R^2}{d \mu} d \mu\right]\left(1 + \frac{1}{2}\frac{\hbar g_R(\mu)}{(4\pi)^2} \log \frac{m_R'^2(\mu + d \mu)}{(\mu + d \mu)^2}\right) + \mathcal{O}(d \mu^2)
        \end{align*}
        Looking only at the logarithmic term:
        \begin{align*}
          \log \frac{m_R'^2(\mu + d \mu)}{(\mu + d \mu)^2} &= \log m_R'^2(\mu + d \mu) - \log (\mu + d \mu)^2\\
                                                           &= \log \left(m_R^2(\mu) + \frac{d m_R^2(\mu)}{d \mu} d \mu\right) - \log (\mu^2 + 2 \mu d \mu)\\
                                                           &= \log m_R^2 + \frac{1}{m_R^2}\frac{d m_R^2(\mu)}{d \mu} d \mu - \log \mu^2 - \frac{2 d \mu}{\mu}\\
                                                           &= \log \frac{m_R^2}{\mu^2} + \frac{1}{m_R^2}\frac{d m_R^2(\mu)}{d \mu} d \mu - \frac{2 d \mu}{\mu}
        \end{align*}
        Putting this result back into the previous equation gives:
        \begin{align*}
          &   m_R'^2(\mu + d \mu) + \frac{1}{2}\frac{\hbar g_R(\mu)}{(4\pi)^2}  m_R'^2(\mu + d \mu) \log \frac{m_R'^2(\mu + d \mu)}{(\mu + d \mu)^2}\\
          &\qquad =  m_R^2(\mu) + \frac{d m_R^2}{d \mu} d \mu + \frac{1}{2}\frac{\hbar g_R}{(4\pi)^2} m_R^2 \log \frac{m_R^2}{\mu^2} + \frac{1}{2}\frac{\hbar g_R}{(4\pi)^2} \frac{d m_R^2}{d \mu} d \mu  - \frac{\hbar g_R}{(4\pi)^2} \frac{m_R^2}{\mu}d\mu\\
          & \qquad \qquad + \frac{1}{2}\frac{\hbar g_R}{(4\pi)^2} \log \left[\frac{m_R^2}{\mu^2}\right] \frac{d m_R^2}{d \mu} d \mu + \mathcal{O}(d \mu^2)
        \end{align*}
        And using this in Eq.~\ref{eq:mAtDiffScales} yields:
        \begin{align*}
                      & \frac{d m_R^2}{d \mu} d \mu + \frac{1}{2}\frac{\hbar g_R}{(4\pi)^2} \frac{d m_R^2}{d \mu} d \mu + \frac{1}{2}\frac{\hbar g_R}{(4\pi)^2} \log \frac{m_R^2}{\mu^2} \frac{d m_R^2}{d \mu} d \mu = \frac{\hbar g_R}{(4\pi)^2} \frac{m_R^2}{\mu} d \mu+ \mathcal{O}(d \mu^2)\\
          \Rightarrow & \frac{d m_R^2}{d \mu} \left(1 + \frac{1}{2}\frac{\hbar g_R}{(4\pi)^2} + \frac{1}{2}\frac{\hbar g_R}{(4\pi)^2} \log \frac{m_R^2}{\mu^2} \right) = \frac{\hbar g_R}{(4\pi)^2} \frac{m_R^2}{\mu}\\ 
          \Rightarrow & \frac{d m_R^2}{d \mu}  =  \frac{m_R^2}{\mu \left(\frac{(4\pi)^2}{\hbar g_R} + \frac{1}{2} +  \frac{1}{2}  \log \frac{m_R^2}{\mu^2} \right)}
        \end{align*}
        Now for the anomalous dimension:
        \begin{equation*}
          \gamma_{m^2}(g_R) \equiv \mu \frac{\partial}{\partial \mu} \log m_R^2 = \mu \frac{1}{m_R^2} \frac{\partial m_R^2}{\partial \mu} = \frac{1}{\left(\frac{(4\pi)^2}{\hbar g_R} + \frac{1}{2} +  \frac{1}{2}  \log \frac{m_R^2}{\mu^2} \right)}
        \end{equation*}
        I was kind of stuck on this problem and I am not sure if my answer is good, or even on the right track, please give me some hints if it is the wrong approach.
      \item
        I worked alone for this homework.
    \end{enumerate}
\end{enumerate}
\end{document}
