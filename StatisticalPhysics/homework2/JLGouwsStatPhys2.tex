\documentclass[12pt,a4]{article}
\usepackage{physics, amsmath,amsfonts,amsthm,amssymb, mathtools,steinmetz, gensymb, siunitx}	% LOADS USEFUL MATH STUFF
\usepackage{xcolor,graphicx}
\usepackage{caption}
\usepackage{subcaption}
\usepackage[left=45pt, top=20pt, right=45pt, bottom=45pt ,a4paper]{geometry} 				% ADJUSTS PAGE
\usepackage{setspace}
\usepackage{tikz}
\usepackage{pgf,tikz,pgfplots,wrapfig}
\usepackage{mathrsfs}
\usepackage{fancyhdr}
\usepackage{float}
\usepackage{array}
\usepackage{booktabs,multirow}
\usepackage{bm}
\usepackage{tensor}
\usepackage{listings}
 \lstset{
    basicstyle=\ttfamily\small,
    numberstyle=\footnotesize,
    numbers=left,
    backgroundcolor=\color{gray!10},
    frame=single,
    tabsize=2,
    rulecolor=\color{black!30},
    title=\lstname,
    escapeinside={\%*}{*)},
    breaklines=true,
    breakatwhitespace=true,
    framextopmargin=2pt,
    framexbottommargin=2pt,
    inputencoding=utf8,
    extendedchars=true,
    literate={á}{{$\rho$}}1 {ã}{{\~a}}1 {é}{{\'e}}1,
}
\DeclareMathOperator{\sign}{sgn}

\usetikzlibrary{decorations.text, calc}
\pgfplotsset{compat=1.7}

\usetikzlibrary{decorations.pathreplacing,decorations.markings}
\usepgfplotslibrary{fillbetween}

\newcommand{\vect}[1]{\boldsymbol{#1}}

\usepackage{hyperref}

%\usepackage[style= ACM-Reference-Format, maxbibnames=6, minnames=1,maxnames = 1]{biblatex}
%\addbibresource{references.bib}


\hypersetup{pdfborder={0 0 0},colorlinks=true,linkcolor=black,urlcolor=cyan,}
\allowdisplaybreaks
%\hypersetup{
%
%    colorlinks=true,
%
%    linkcolor=blue,
%
%    filecolor=magenta,      
%
%    urlcolor=cyan,
%
%    pdftitle={An Example},
%
%    pdfpagemode=FullScreen,
%
%    }
%}

\title{
\textsc{Statistical Physics Homework 2}
}
\author{\textsc{J L Gouws}
}
\date{\today
\\[1cm]}



\usepackage{graphicx}
\usepackage{array}




\begin{document}
\thispagestyle{empty}

\maketitle

\begin{enumerate}
  \item
    \begin{enumerate}
      \item
        \begin{align*}
          \sum_j B_{ij} 
%                        &= \sum_{j}\frac{1}{N} \sum_k B_k e^{i k \cdot (x_i - x_j)}\\
%                        &= \frac{1}{N} \sum_k B_k  e^{i k \cdot x_i} \sum_{j} e^{-i k \cdot x_j)}
                        &= \sum_{j}  \frac{1}{N^2} \sum_{k q} \hat B_{kq} e^{i k \cdot x_i} e^{i q \cdot x_i}\\
        \end{align*}
        Assuming that the model is limited to nearest neighbour interactions:
        \begin{align*}
          \sum_j B_{ij} 
                        &= \sum_{j}  \frac{1}{N^2} \sum_{k q} N B_k \delta_{k + q, 0} e^{i k \cdot x_i} e^{i q \cdot x_j}\\
                        &=   \frac{1}{N^2} \sum_{k q} N B_k \delta_{k + q, 0} e^{i k \cdot x_i} \sum_{j} e^{i q \cdot x_j}\\
                        &=   \frac{1}{N^2} \sum_{k q} N B_k \delta_{k + q, 0} e^{i k \cdot x_i} N \delta_{q,0}\\
                        &=   \frac{1}{N^2} \sum_{k} N^2 B_k \delta_{k, 0} e^{i k \cdot x_i} \\
                        &=   B_0 
        \end{align*}
      \item
        For the minimum, derivatives are taken:
        \begin{align*}
          \frac{\partial S}{\partial \phi_i} &= \frac{\partial }{\partial \phi_i}\left(\frac{\beta A^2}{2} \phi^T B \phi - \sum_j \ln(\cosh(\beta A (B \phi)_j))\right)\\
                                             &= \frac{\beta A^2}{2} \frac{\partial }{\partial \phi_i}\sum_{ij}\phi_i B_{ij} \phi_j - \frac{\partial }{\partial \phi_i}\sum_j \ln(\cosh(\beta A (B \phi)_j))\\
                                             &= \frac{\beta A^2}{2} \sum_{j} \left( B_{ij} \phi_j + \phi_jB_{ji} \right) - \sum_j \beta A B_{ji}\frac{\sinh(\beta A (B \phi)_j)}{\cosh(\beta A (B \phi)_j)}\\
                                             &= \frac{\beta A^2}{2} \sum_{j} \left( B_{ij} \phi_j + \phi_jB_{ji} \right) - \beta A\sum_j B_{ji}\tanh(\beta A (B \phi)_j)\\
                                             &= \beta A^2 \sum_{j} B_{ij} \phi_j - \beta A \sum_j B_{ji}\tanh(\beta A (B \phi)_j)
        \end{align*}
        This is zero at the critical point $\psi_i = \bar{\psi}$ for every $i$, and evaluating for very $i$ gives the result:
        \begin{align*}
          0 = \left.\frac{\partial S}{\partial \phi_i}\right|_{\phi_i = \bar{psi}} 
                                             &= \beta A^2 \sum_{j} B_{ij} \phi_{j} - \beta A \sum_j B_{ji}\tanh(\beta A \sum_k B_{jk} \phi_k)\\
                                             &= \beta A^2 \sum_{j} B_{ij} \bar{\psi} - \beta A \sum_j B_{ji}\tanh(\beta A \sum_k B_{jk} \bar\psi)\\
                                             &= \beta A^2 \bar{\psi} \sum_{j} B_{ij}  - \beta A \sum_j B_{ji}\tanh(\beta A \bar\psi \sum_k B_{jk} )\\
                                             &= \beta A M B_0  - \beta A \sum_j B_{ji}\tanh(\beta M B_0 )\\
                                             &= \beta A M B_0  - \beta A \tanh(\beta M B_0 ) \sum_j B_{ji}\\
                                             &= \beta A M B_0  - \beta A \tanh(\beta M B_0 ) B_0\\
                                             &= \beta A B_0 \left(M   -  \tanh(\beta M B_0 )\right)
        \end{align*}
        Or equivalently:
        \begin{align*}
                      & M =  \tanh(\beta M B_0 )\\
          \Rightarrow & M =  \tanh(\frac{1}{k_B T} M k_B T_c )\\
          \Rightarrow & M =  \tanh(\frac{M T_c }{T} )
        \end{align*}
      \item
    \end{enumerate}
\end{enumerate}
\end{document}
