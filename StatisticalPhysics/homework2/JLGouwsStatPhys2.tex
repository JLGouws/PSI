\documentclass[12pt,a4]{article}
\usepackage{physics, amsmath,amsfonts,amsthm,amssymb, mathtools,steinmetz, gensymb, siunitx}	% LOADS USEFUL MATH STUFF
\usepackage{xcolor,graphicx}
\usepackage{caption}
\usepackage{subcaption}
\usepackage[left=45pt, top=20pt, right=45pt, bottom=45pt ,a4paper]{geometry} 				% ADJUSTS PAGE
\usepackage{setspace}
\usepackage{tikz}
\usepackage{pgf,tikz,pgfplots,wrapfig}
\usepackage{mathrsfs}
\usepackage{fancyhdr}
\usepackage{float}
\usepackage{array}
\usepackage{booktabs,multirow}
\usepackage{bm}
\usepackage{tensor}
\usepackage{listings}
 \lstset{
    basicstyle=\ttfamily\small,
    numberstyle=\footnotesize,
    numbers=left,
    backgroundcolor=\color{gray!10},
    frame=single,
    tabsize=2,
    rulecolor=\color{black!30},
    title=\lstname,
    escapeinside={\%*}{*)},
    breaklines=true,
    breakatwhitespace=true,
    framextopmargin=2pt,
    framexbottommargin=2pt,
    inputencoding=utf8,
    extendedchars=true,
    literate={á}{{$\rho$}}1 {ã}{{\~a}}1 {é}{{\'e}}1,
}
\DeclareMathOperator{\sign}{sgn}
\DeclareMathOperator{\arcosh}{arcosh}
\DeclareMathOperator{\arsinh}{arsinh}
\DeclareMathOperator{\artanh}{artanh}
\DeclareMathOperator{\arsech}{arsech}
\DeclareMathOperator{\arcsch}{arcsch}
\DeclareMathOperator{\arcoth}{arcoth} 

\usetikzlibrary{decorations.text, calc}
\pgfplotsset{compat=1.7}

\usetikzlibrary{decorations.pathreplacing,decorations.markings}
\usepgfplotslibrary{fillbetween}

\newcommand{\vect}[1]{\boldsymbol{#1}}

\usepackage{hyperref}

%\usepackage[style= ACM-Reference-Format, maxbibnames=6, minnames=1,maxnames = 1]{biblatex}
%\addbibresource{references.bib}


\hypersetup{pdfborder={0 0 0},colorlinks=true,linkcolor=black,urlcolor=cyan,}
\allowdisplaybreaks
%\hypersetup{
%
%    colorlinks=true,
%
%    linkcolor=blue,
%
%    filecolor=magenta,      
%
%    urlcolor=cyan,
%
%    pdftitle={An Example},
%
%    pdfpagemode=FullScreen,
%
%    }
%}

\title{
\textsc{Statistical Physics Homework 2}
}
\author{\textsc{J L Gouws}
}
\date{\today
\\[1cm]}



\usepackage{graphicx}
\usepackage{array}




\begin{document}
\thispagestyle{empty}

\maketitle

\begin{enumerate}
  \item
    \begin{enumerate}
      \item
        \begin{align*}
          \sum_j B_{ij} 
%                        &= \sum_{j}\frac{1}{N} \sum_k B_k e^{i k \cdot (x_i - x_j)}\\
%                        &= \frac{1}{N} \sum_k B_k  e^{i k \cdot x_i} \sum_{j} e^{-i k \cdot x_j)}
                        &= \sum_{j}  \frac{1}{N^2} \sum_{k q} \hat B_{kq} e^{i k \cdot x_i} e^{i q \cdot x_i}\\
        \end{align*}
        Assuming that the model is limited to nearest neighbour interactions:
        \begin{align*}
          \sum_j B_{ij} 
                        &= \sum_{j}  \frac{1}{N^2} \sum_{k q} N B_k \delta_{k + q, 0} e^{i k \cdot x_i} e^{i q \cdot x_j}\\
                        &=   \frac{1}{N^2} \sum_{k q} N B_k \delta_{k + q, 0} e^{i k \cdot x_i} \sum_{j} e^{i q \cdot x_j}\\
                        &=   \frac{1}{N^2} \sum_{k q} N B_k \delta_{k + q, 0} e^{i k \cdot x_i} N \delta_{q,0}\\
                        &=   \frac{1}{N^2} \sum_{k} N^2 B_k \delta_{k, 0} e^{i k \cdot x_i} \\
                        &=   B_0 
        \end{align*}
      \item
        For the minimum, derivatives are taken:
        \begin{align*}
          \frac{\partial S}{\partial \phi_i} &= \frac{\partial }{\partial \phi_i}\left(\frac{\beta A^2}{2} \phi^T B \phi - \sum_j \ln(\cosh(\beta A (B \phi)_j))\right)\\
                                             &= \frac{\beta A^2}{2} \frac{\partial }{\partial \phi_i}\sum_{ij}\phi_i B_{ij} \phi_j - \frac{\partial }{\partial \phi_i}\sum_j \ln(\cosh(\beta A (B \phi)_j))\\
                                             &= \frac{\beta A^2}{2} \sum_{j} \left( B_{ij} \phi_j + \phi_jB_{ji} \right) - \sum_j \beta A B_{ji}\frac{\sinh(\beta A (B \phi)_j)}{\cosh(\beta A (B \phi)_j)}\\
                                             &= \frac{\beta A^2}{2} \sum_{j} \left( B_{ij} \phi_j + \phi_jB_{ji} \right) - \beta A\sum_j B_{ji}\tanh(\beta A (B \phi)_j)\\
                                             &= \beta A^2 \sum_{j} B_{ij} \phi_j - \beta A \sum_j B_{ji}\tanh(\beta A (B \phi)_j)
        \end{align*}
        This is zero at the critical point $\psi_i = \bar{\psi}$ for every $i$, and evaluating for very $i$ gives the result:
        \begin{align*}
          0 = \left.\frac{\partial S}{\partial \phi_i}\right|_{\phi_i = \bar{psi}} 
                                             &= \beta A^2 \sum_{j} B_{ij} \phi_{j} - \beta A \sum_j B_{ji}\tanh(\beta A \sum_k B_{jk} \phi_k)\\
                                             &= \beta A^2 \sum_{j} B_{ij} \bar{\psi} - \beta A \sum_j B_{ji}\tanh(\beta A \sum_k B_{jk} \bar\psi)\\
                                             &= \beta A^2 \bar{\psi} \sum_{j} B_{ij}  - \beta A \sum_j B_{ji}\tanh(\beta A \bar\psi \sum_k B_{jk} )\\
                                             &= \beta A M B_0  - \beta A \sum_j B_{ji}\tanh(\beta M B_0 )\\
                                             &= \beta A M B_0  - \beta A \tanh(\beta M B_0 ) \sum_j B_{ji}\\
                                             &= \beta A M B_0  - \beta A \tanh(\beta M B_0 ) B_0\\
                                             &= \beta A B_0 \left(M   -  \tanh(\beta M B_0 )\right)
        \end{align*}
        Or equivalently:
        \begin{align}
                      & M =  \tanh(\beta M B_0 ) \nonumber \\
          \Rightarrow & M =  \tanh(\frac{1}{k_B T} M k_B T_c ) \nonumber \\
          \Rightarrow & M =  \tanh(\frac{M T_c }{T} ) \label{eq:magnetization}
        \end{align}
      \item
        Taking the second derivative:
        \begin{align*}
          \frac{\partial^2 S}{\partial \phi_j \partial \phi_i} 
                                            &= \frac{\partial}{\partial \phi_j } \left(\beta A^2 \sum_{k} B_{ik} \phi_k - \beta A \sum_k B_{ki}\tanh(\beta A (B \phi)_k)\right)\\
                                            &= \beta A^2  B_{ij}  - \beta^2 A^2 \sum_k B_{ki} B_{kj}\sech^2(\beta A (B \phi)_k)\\
        \end{align*}
        Evaluating this at the critical point gives:
        \begin{align*}
          \left.\frac{\partial^2 S}{\partial \phi_j \partial \phi_i}\right|_{\bar{\psi}} 
                                            &= \beta A^2  B_{ij}  - \beta^2 A^2 \sum_k B_{ki} B_{kj}\sech^2\left(\frac{MT_c}{T}\right)\\
                                            &= \beta A^2  B_{ij}  - \beta^2 A^2 \sum_k B_{ki} B_{kj}\sech^2\left(\artanh(M)\right)\\
                                            &= \beta A^2  \left(B_{ij}  - \beta \sech\left(\sqrt{1 - M^2}\right) \sum_k B_{ik} B_{kj} \right)\\
                                            &= \beta A^2  \left(B_{ij}  - \beta \sech\left(\sqrt{1 - M^2}\right) (B^2)_{ij} \right)
        \end{align*}
        Using Eq.~\ref{eq:magnetization} in the second step. 
        Since $\beta A^2 > 0$ the term that determines the positive definiteness is:
        \begin{align*}
          B_{ij}  - \beta \sech\left(\sqrt{1 - M^2}\right) (B^2)_{ij}
        \end{align*}
        Near the critical temperature $\beta = 1 / k_B(T_c - t) = 1 / (B_0 - k_B t) \sim 1 / B_0 + k_B t / B_0^2 $, the eigen values for the above matrix are:
        \begin{align*}
          B_{k}  - \beta \sech\left(\sqrt{1 - M^2}\right) B^2_{k} & \approx B_{k}  - \left(\frac{1}{B_0} +\frac{k_B t}{B_0^2}\right) \sech\left(\sqrt{1 - M^2}\right) B^2_{k}\\
                                                                  & \geq B_{k}  - \left(\frac{1}{B_0} + \frac{k_B t}{B_0^2}\right) B^2_{k}\\
                                                                  & = B_{k}  - \frac{B_{k}}{B_0}B_{k} - k_B t\frac{B^2_{k}}{B_0^2}\\
        \end{align*}
        Since $\sech^2(x) \leq 1$
      \item
        Using the approximation:
        \begin{align*}
          S(\phi) \approx S(\psi) + \frac{1}{2}\sum_{i,j}(\phi - \psi)_i (\phi - \psi)_j \frac{\partial^2 S}{\partial \phi_i \partial \phi_j}(\psi)
        \end{align*}
        The first term is:
        \begin{align*}
          S(\psi) &= \frac{\beta}{2} (A\bar{\psi})^2 \sum_{i j} B_{ij} - \sum_i \ln(\cosh(\beta A\bar{\psi}\sum_jB_{ij}))\\
                  &= \frac{\beta}{2} M^2 \sum_{i } B_{0} - \sum_i \ln(\cosh(\beta M B_0))\\
                  &= \frac{\beta}{2} M^2 N B_{0} - N \ln(\cosh(\beta M B_0))
        \end{align*}
        And the hessian term is:
        \begin{align*}
          \frac{1}{2}\sum_{i,j}(\phi - \psi)_i (\phi - \psi)_j \frac{\partial^2 S}{\partial \phi_i \partial \phi_j}(\psi) &= \beta A^2  \frac{1}{2}\sum_{i,j}(\phi - \psi)_i B_{ij} (\phi - \psi)_j  \\
               & \qquad   - \beta \sech\left(\sqrt{1 - M^2}\right) \frac{1}{2}\sum_{i,j} (\phi - \psi)_i (B^2)_{ij} (\phi - \psi)_j
        \end{align*}
      \item
        The integral in the denomenator is:
        \begin{align*}
          \int_{\mathbb{R}^N} d^N\phi \exp\left[-\frac{\beta A^2}{2}(\phi - \psi)^T\left(B  - \beta \sech (\sqrt{1 - M})B^2)\right) (\phi - \psi)\right]
        \end{align*}
        The integral is invariant under the change of variables $\tilde\phi = \phi - \psi$:
        \begin{align*}
          &\int_{\mathbb{R}^N} d^N\tilde\phi \exp\left[-\frac{\beta A^2}{2}\tilde\phi^T\left(B  - \beta \sech (\sqrt{1 - M})B^2\right) \tilde\phi\right]\\
          & \qquad = \sqrt{\frac{(2 \pi)^n}{\det\left( B  - \beta \sech (\sqrt{1 - M})B^2\right)}}
        \end{align*}
    \end{enumerate}
\end{enumerate}
\end{document}
