\documentclass[12pt,a4]{article}
\usepackage{physics, amsmath,amsfonts,amsthm,amssymb, mathtools,steinmetz, gensymb, siunitx}	% LOADS USEFUL MATH STUFF
\usepackage{xcolor,graphicx}
\usepackage[left=45pt, top=20pt, right=45pt, bottom=45pt ,a4paper]{geometry} 				% ADJUSTS PAGE
\usepackage{setspace}
\usepackage{caption}
\usepackage{tikz}
\usepackage{pgf,tikz,pgfplots,wrapfig}
\usetikzlibrary{intersections}
\usepackage{tkz-euclide}
\usepackage{mathrsfs}
\usepackage{fancyhdr}
\usepackage{float}
\usepackage{array}
\usepackage{tensor}
\usepackage{booktabs,multirow}
\usepackage{bm}

\usetikzlibrary{decorations.text, calc}
\pgfplotsset{compat=1.7}

\usetikzlibrary{decorations.pathreplacing,decorations.markings}
\usepgfplotslibrary{fillbetween}
\allowdisplaybreaks

\newcommand{\vect}[1]{\boldsymbol{#1}}

\usepackage{hyperref}
%\usepackage[style= ACM-Reference-Format, maxbibnames=6, minnames=1,maxnames = 1]{biblatex}
%\addbibresource{references.bib}

\hypersetup{
    colorlinks=true,
    linkcolor=blue,
    filecolor=magenta,      
    urlcolor=cyan,
    pdfpagemode=FullScreen,
    }

\title{
\textsc{QT Homework 2}
}
\author{\textsc{J L Gouws}
}
\date{\today
\\[1cm]}



\usepackage{graphicx}
\usepackage{array}
\usepackage[compat=1.1.0]{tikz-feynman}




\begin{document}
\thispagestyle{empty}

\maketitle

%\feynmandiagram [horizontal=a to b] {
%  i1 -- [plain] a -- [plain] i2,
%  a -- [plain] b,
%  f1 -- [plain] b -- [plain] f2,
%};

\begin{enumerate}
  \item
\begin{enumerate}
  \item
    If
    \begin{equation*}
      \tensor{\Lambda}{^{\mu}_\nu} = \delta^\mu_\nu + \tensor{\omega}{^\mu_\nu}
    \end{equation*}
    Then a vector tansforms as:
    \begin{equation*}
      {x'}^\mu = \tensor{\Lambda}{^{\mu}_\nu} x^\nu = x^\mu + \tensor{\omega}{^\mu_\nu} x^\nu
    \end{equation*}
    Then we want the length of the vector to be preserved under the.
    \begin{align*}
      {x}^\mu{x}_\mu = {x'}^\mu{x'}_\mu  &= (x^\mu + \tensor{\omega}{^\mu_\nu} x^\nu) (x_\mu + \tensor{\omega}{_\mu^\nu} x_\nu)\\
                                         &= x^\mu x_\mu + x_\mu \tensor{\omega}{^\mu_\nu} x^\nu + x^\mu \tensor{\omega}{_\mu^\nu} x_\nu\\
                                         &\Rightarrow 0 = x_\mu \tensor{\omega}{^\mu_\nu} x^\nu + x^\mu \tensor{\omega}{_\mu^\nu} x_\nu\\
                                         &\Rightarrow 0 = x_\mu \tensor{\omega}{^\mu_\nu} x^\nu + x^\nu \tensor{\omega}{_\nu^\mu} x_\mu\\
                                         &\Rightarrow 0 = x_\mu (\tensor{\omega}{^\mu_\nu} + \tensor{\omega}{_\nu^\mu}) x^\nu\\
                                         &\Rightarrow 0 = \omega_{\mu\nu} +  \omega_{\mu\nu}\\
                                         &\Rightarrow \omega_{\mu\nu} = -\omega_{\nu\mu}
    \end{align*}
  \item
    The expectation values of observables should be the same:
    \begin{align*}
      \langle \psi | O | \phi \rangle  &= \langle \psi | U^\dagger U O U^\dagger U | \phi \rangle = \langle \psi' |  U O U^\dagger | \phi' \rangle\\
                                      &= \langle \psi' | O' | \phi' \rangle\\
                                      &\Rightarrow O' = U O U^\dagger
    \end{align*}
  \item
    We know that, writing things a bit more explicitly:
    \begin{equation*}
      U(\Lambda, a)U(\delta + \omega, \epsilon) U^\dagger(\Lambda, a)= U(\Lambda, a)U(\delta + \omega, \epsilon) U(T^{-1}(\Lambda, a)) = U(T(\Lambda, a)T(\delta + \omega, \epsilon)T^{-1}(\Lambda, a))
    \end{equation*}
    Now we can see how $T(\Lambda, a)T(\delta + \omega, \epsilon)T^{-1}(\Lambda, a)$ operates on $x$:
    \begin{align*}
      T(\Lambda, a)T(\delta + \omega, \epsilon)T^{-1}(\Lambda, a)x &\leadsto \tensor{T(\Lambda, a)}{^\alpha_\beta}\tensor{T(\delta + \omega, \epsilon)}{^\beta_\gamma}\tensor{T^{-1}(\Lambda, a)}{^\gamma_\delta}x^\delta\\
                                                                   &= \tensor{T(\Lambda, a)}{^\alpha_\beta}\tensor{T(\delta + \omega, \epsilon)}{^\beta_\gamma}\left[\tensor{\left(\Lambda^{-1}\right)}{^\gamma_\delta}x^\delta - \tensor{\left(\Lambda^{-1}\right)}{^\gamma_\delta}a^\delta\right]\\
                                                                   &= \tensor{T(\Lambda, a)}{^\alpha_\beta}\left[\tensor{\left(\Lambda^{-1}\right)}{^\beta_\delta}x^\delta - \tensor{\left(\Lambda^{-1}\right)}{^\beta_\delta}a^\delta \right.\\
                                                                   &\qquad + \left.\tensor{\omega}{^\beta_\gamma}\tensor{\left(\Lambda^{-1}\right)}{^\gamma_\delta}x^\delta - \tensor{\omega}{^\beta_\gamma}\tensor{\left(\Lambda^{-1}\right)}{^\gamma_\delta}a^\delta + \epsilon^\beta\right]\\
                                                                   &= \tensor{T(\Lambda, a)}{^\alpha_\beta}\left[\tensor{\left(\Lambda^{-1}\right)}{^\beta_\delta}x^\delta - \tensor{\left(\Lambda^{-1}\right)}{^\beta_\delta}a^\delta \right.\\
                                                                   &\qquad + \left. \tensor{\omega}{^\beta_\gamma}\tensor{\left(\Lambda^{-1}\right)}{^\gamma_\delta}x^\delta - \tensor{\omega}{^\beta_\gamma}\tensor{\left(\Lambda^{-1}\right)}{^\gamma_\delta}a^\delta + \epsilon^\beta\right]\\
                                                                   &= \tensor{\Lambda}{^\alpha_\beta}\tensor{\left(\Lambda^{-1}\right)}{^\beta_\delta}x^\delta - \tensor{\Lambda}{^\alpha_\beta}\tensor{\left(\Lambda^{-1}\right)}{^\beta_\delta}a^\delta + \tensor{\Lambda}{^\alpha_\beta}\tensor{\omega}{^\beta_\gamma}\tensor{\left(\Lambda^{-1}\right)}{^\gamma_\delta}x^\delta \\
                                                                   &\qquad   - \tensor{\Lambda}{^\alpha_\beta}\tensor{\omega}{^\beta_\gamma}\tensor{\left(\Lambda^{-1}\right)}{^\gamma_\delta}a^\delta + \tensor{\Lambda}{^\alpha_\beta}\epsilon^\beta + a^\alpha\\
                                                                   &= \delta{^\alpha_\delta}x^\delta - \delta{^\alpha_\delta}a^\delta + \tensor{\Lambda}{^\alpha_\beta}\tensor{\omega}{^\beta_\gamma}\tensor{\left(\Lambda^{-1}\right)}{^\gamma_\delta}x^\delta \\
                                                                   &\qquad - \tensor{\Lambda}{^\alpha_\beta}\tensor{\omega}{^\beta_\gamma}\tensor{\left(\Lambda^{-1}\right)}{^\gamma_\delta}a^\delta + \tensor{\Lambda}{^\alpha_\beta}\epsilon^\beta + a^\alpha\\
                                                                   &= \delta{^\alpha_\delta}x^\delta + \tensor{\Lambda}{^\alpha_\beta}\tensor{\omega}{^\beta_\gamma}\tensor{\left(\Lambda^{-1}\right)}{^\gamma_\delta}x^\delta \\
                                                                   &\qquad - \tensor{\Lambda}{^\alpha_\beta}\tensor{\omega}{^\beta_\gamma}\tensor{\left(\Lambda^{-1}\right)}{^\gamma_\delta}a^\delta + \tensor{\Lambda}{^\alpha_\beta}\epsilon^\beta\\
                                                                   &= T(\delta +\Lambda \omega \Lambda^{-1},\Lambda \epsilon - \Lambda \omega \Lambda^{-1})
    \end{align*}
    Hence:
    \begin{equation*}
      U(\Lambda, a)U(\delta + \omega, \epsilon) U^\dagger(\Lambda, a) = U(\delta +\Lambda \omega \Lambda^{-1},\Lambda \epsilon - \Lambda \omega \Lambda^{-1}a)
    \end{equation*}
  \item
    \begin{align*}
      U(\Lambda, a) U(\delta + \omega, \epsilon) U^\dagger(\Lambda, a) = \mathbf{1} + \frac{i}{2} \omega_{\mu\nu}U(\Lambda, a) J^{\mu\nu} U^{\dagger}(\Lambda, a) + i\epsilon_\mu U(\Lambda, a)P^\mu U^\dagger(\Lambda, a)
    \end{align*}
    But this is equal to:
    \begin{align*}
        & U(\delta +\Lambda \omega \Lambda^{-1},\Lambda \epsilon - \Lambda \omega \Lambda^{-1}a)\\
      = & \delta^\alpha_\beta + \frac{i}{2} \tensor{\Lambda}{_\alpha^\gamma} \tensor{\omega}{_\gamma_\delta}\tensor{(\Lambda^{-1})}{^\delta_\epsilon} J^{\alpha\epsilon} + i\tensor{\Lambda}{_\nu^\mu} \epsilon_\mu P^\nu - i\tensor{\Lambda}{_\alpha^\gamma}\tensor{\omega}{_\gamma_\delta}\tensor{(\Lambda^{-1})}{^\delta_\epsilon} a^\epsilon P^\alpha\\
      = & \delta^\alpha_\beta + \frac{i}{2} \tensor{\Lambda}{_\alpha^\gamma} \tensor{\omega}{_\gamma_\delta}\tensor{\Lambda}{_\epsilon^\delta} J^{\alpha\epsilon} + i\tensor{\Lambda}{_\nu^\mu} \epsilon_\mu P^\nu - i\tensor{\Lambda}{_\alpha^\gamma}\tensor{\omega}{_\gamma_\delta}\tensor{\Lambda}{_\epsilon^\delta} a^\epsilon P^\alpha
    \end{align*}
    Since $\omega$ and $\epsilon$ are arbitrary:
    \begin{align*}
      \omega_{\mu\nu}U(\Lambda, a) J^{\mu\nu} U^{\dagger}(\Lambda, a) &= \tensor{\Lambda}{_\rho^\mu} \tensor{\Lambda}{_\sigma^\nu} \omega_{\mu\nu}J^{\rho\sigma} - 2\tensor{\Lambda}{_\rho^\mu}\tensor{\Lambda}{_\sigma^\nu} \omega_{\mu\nu} a^\sigma P^\rho\\
                                                                      &= \tensor{\Lambda}{_\rho^\mu} \tensor{\Lambda}{_\sigma^\nu} \omega_{\mu\nu}J^{\rho\sigma} + \tensor{\Lambda}{_\rho^\mu}\tensor{\Lambda}{_\sigma^\nu} \omega_{\nu\mu} a^\sigma P^\rho - \tensor{\Lambda}{_\rho^\mu}\tensor{\Lambda}{_\sigma^\nu} \omega_{\mu\nu} a^\sigma P^\rho\\
                                                                      &= \tensor{\Lambda}{_\rho^\mu} \tensor{\Lambda}{_\sigma^\nu} \omega_{\mu\nu}J^{\rho\sigma} + \tensor{\Lambda}{_\sigma^\nu}\tensor{\Lambda}{_\rho^\mu} \omega_{\mu\nu} a^\rho P^\sigma - \tensor{\Lambda}{_\rho^\mu}\tensor{\Lambda}{_\sigma^\nu} \omega_{\mu\nu} a^\sigma P^\rho\\
                                                                      &= \tensor{\Lambda}{_\rho^\mu} \tensor{\Lambda}{_\sigma^\nu} \omega_{\mu\nu}(J^{\rho\sigma} + a^\rho P^\sigma - a^\sigma P^\rho)
    \end{align*}
    This must hold for any $\omega$, thus:
    \begin{align*}
      U(\Lambda, a) J^{\mu\nu} U^{\dagger}(\Lambda, a) &= \tensor{\Lambda}{_\rho^\mu} \tensor{\Lambda}{_\sigma^\nu} (J^{\rho\sigma} + a^\rho P^\sigma - a^\sigma P^\rho)\\
    \end{align*}
    And similarly for $P$:
    \begin{align*}
      U(\Lambda, a) P^{\mu} U^{\dagger}(\Lambda, a) = \tensor{\Lambda}{_\nu^\mu} P^\nu
    \end{align*}
  \item
% \mathbf{1} + \frac{i}{2} \omega_{\mu\nu} J^{\mu\nu} + i\epsilon_\mu P^\mu 
    For the $P$ one:
    \begin{align*}
      & U(\Lambda, a) P^{\mu} = \tensor{\Lambda}{_\nu^\mu}U P^\nu(\Lambda, a)\\
      \Rightarrow & (\mathbf{1} + \frac{i}{2} \omega_{\nu\xi} J^{\nu\xi} + i\epsilon_\nu P^\nu) P^\mu = (\delta^\mu_\nu + \tensor{\omega}{_\nu^\mu} )P^\nu (\mathbf{1} + \frac{i}{2} \omega_{\mu\xi} J^{\mu\xi} + i\epsilon_\mu P^\mu)\\
      \Rightarrow & P^\mu+ \frac{i}{2} \omega_{\nu\xi} J^{\nu\xi}P^\mu + i\epsilon_\nu P^\nu P^\mu  = (P^\mu + \tensor{\omega}{_\nu^\mu}P^\nu ) (\mathbf{1} + \frac{i}{2} \omega_{\mu\xi} J^{\mu\xi} + i\epsilon_\mu P^\mu)\\
      \Rightarrow & \frac{i}{2} \omega_{\nu\xi} J^{\nu\xi}P^\mu + i\epsilon_\nu P^\nu P^\mu  =  \frac{i}{2} \omega_{\nu\xi} P^\mu J^{\nu\xi} + i\epsilon_\nu P^\mu P^\nu  + \tensor{\omega}{_\nu^\mu}P^\nu\\
    \end{align*}
    Since $\omega$ and $\epsilon $ independent and arbitrary:
    \begin{equation*}
      i\epsilon_\nu (P^\nu P^\mu - P^\mu P^\nu) = 0 \Rightarrow [P^\mu, P^\nu] = 0
    \end{equation*}
    And:
    \begin{align*}
                  & \frac{i}{2} \omega_{\nu\xi} J^{\nu\xi}P^\mu =  \frac{i}{2} \omega_{\nu\xi} P^\mu J^{\nu\xi} + \tensor{\omega}{_\nu^\mu}P^\nu\\
      \Rightarrow & i \omega_{\nu\xi} (J^{\nu\xi}P^\mu - P^\mu J^{\nu\xi}) = 2\eta^{\mu\xi}\tensor{\omega}{_\nu_\xi}P^\nu\\
      \Rightarrow & i \omega_{\nu\xi} (J^{\nu\xi}P^\mu - P^\mu J^{\nu\xi}) = \eta^{\mu\xi}\tensor{\omega}{_\nu_\xi}P^\nu - \eta^{\mu\xi}\tensor{\omega}{_\xi_\nu}P^\nu\\
      \Rightarrow & i \omega_{\nu\xi} [J^{\nu\xi},P^\mu] = \eta^{\mu\xi}\tensor{\omega}{_\nu_\xi}P^\nu - \eta^{\mu\nu}\tensor{\omega}{_\nu_\xi}P^\xi\\
      \Rightarrow & i [J^{\nu\xi},P^\mu] = \eta^{\mu\xi}P^\nu - \eta^{\mu\nu}P^\xi\\
    \end{align*}
    Now expanding the other relation:
    \begin{align*}
      U(\Lambda, a) J^{\mu\nu}  &= (\mathbf{1} + \frac{i}{2} \omega_{\zeta\xi} J^{\zeta\xi} + i\epsilon_\xi P^\xi)J^{\mu\nu}\\
                                &= J^{\mu\nu} + \frac{i}{2} \omega_{\zeta\xi} J^{\zeta\xi}J^{\mu\nu} + i\epsilon_\xi P^\xi J^{\mu\nu}
    \end{align*}
    And:
    \begin{align*}
        & \tensor{\Lambda}{_\rho^\mu} \tensor{\Lambda}{_\sigma^\nu} (J^{\rho\sigma} + \epsilon^\rho P^\sigma - \epsilon^\sigma P^\rho) U(\Lambda, a)\\
      = & (\delta{_\rho^\mu} +\tensor{\omega}{_\rho^\mu}) (\delta {_\sigma^\nu} + \tensor{\omega}{_\sigma^\nu}) (J^{\rho\sigma} + \epsilon^\rho P^\sigma - \epsilon^\sigma P^\rho) (\mathbf{1} + \frac{i}{2} \omega_{\zeta\xi} J^{\zeta\xi} + i\epsilon_\xi P^\xi)\\
      = & J^{\mu\nu} + \epsilon^\mu P^\nu - \epsilon^\nu P^\mu + \frac{i}{2} \omega_{\zeta\xi} J^{\mu\nu} J^{\zeta\xi} + i\epsilon_\xi J^{\mu\nu}  P^\xi + \tensor{\omega}{_\rho^\mu}J^{\rho\nu} + \tensor{\omega}{_\sigma^\nu}J^{\mu\sigma} + \mathcal{O}(\epsilon \omega)%(\delta{_\rho^\mu} + \tensor{\omega}{_\rho^\mu}) (\delta {_\sigma^\nu} + \tensor{\omega}{_\sigma^\nu}) (J^{\rho\sigma} + \epsilon^\rho P^\sigma - \epsilon^\sigma P^\rho) (\mathbf{1} + \frac{i}{2} \omega_{\zeta\xi} J^{\zeta\xi} + i\epsilon_\xi P^\xi)\\
    \end{align*}
    And looking at the terms with $\omega$ coefficients:
    \begin{align*}
                  & \frac{i}{2} \omega_{\zeta\xi} J^{\zeta\xi}J^{\mu\nu} = \frac{i}{2} \omega_{\zeta\xi} J^{\mu\nu} J^{\zeta\xi} + \tensor{\omega}{_\rho^\mu}J^{\rho\nu} + \tensor{\omega}{_\sigma^\nu}J^{\mu\sigma}\\
      \Rightarrow & i \omega_{\zeta\xi} (J^{\zeta\xi}J^{\mu\nu} - J^{\mu\nu} J^{\zeta\xi}) = 2\tensor{\omega}{_\rho^\mu}J^{\rho\nu} + 2\tensor{\omega}{_\sigma^\nu}J^{\mu\sigma}\\
      \Rightarrow & i \omega_{\zeta\xi} (J^{\zeta\xi}J^{\mu\nu} - J^{\mu\nu} J^{\zeta\xi}) = 2\eta^{\mu\xi}\tensor{\omega}{_\zeta_\xi}J^{\zeta\nu} + 2\eta^{\nu\xi}\tensor{\omega}{_\zeta_\xi}J^{\mu\zeta}\\
      \Rightarrow & i \omega_{\zeta\xi} (J^{\zeta\xi}J^{\mu\nu} - J^{\mu\nu} J^{\zeta\xi}) = \eta^{\mu\xi}\tensor{\omega}{_\zeta_\xi}J^{\zeta\nu} - \eta^{\mu\xi}\tensor{\omega}{_\xi_\zeta}J^{\zeta\nu} + \eta^{\nu\xi}\tensor{\omega}{_\zeta_\xi}J^{\mu\zeta} - \eta^{\nu\xi}\tensor{\omega}{_\xi_\zeta}J^{\mu\zeta}\\
      \Rightarrow & i \omega_{\zeta\xi} [J^{\zeta\xi},J^{\mu\nu}]  = \tensor{\omega}{_\zeta_\xi}(\eta^{\mu\xi}J^{\zeta\nu} - \eta^{\mu\zeta}J^{\xi\nu} + \eta^{\nu\xi}J^{\mu\zeta} - \eta^{\nu\zeta}J^{\mu\xi})\\
      \Rightarrow & i[J^{\zeta\xi},J^{\mu\nu}]  = \eta^{\mu\xi}J^{\zeta\nu} - \eta^{\mu\zeta}J^{\xi\nu} + \eta^{\nu\xi}J^{\mu\zeta} - \eta^{\nu\zeta}J^{\mu\xi}
    \end{align*}
%    Note that:
%    \begin{align*}
%      i [J^{\zeta\xi},J^{\mu\nu}] = 2\eta^{\mu\xi}J^{\zeta\nu} + 2\eta^{\nu\xi}J^{\mu\zeta} = 2\eta^{\alpha\xi}(\delta^\mu_\alpha J^{\zeta\nu} + \delta^\nu_\alpha J^{\mu\zeta})
%    \end{align*}
%    And for spacial indices:
  \item
    Since $J^{\mu\nu}$ is multiplied by an antisymmetric matrix, it can be taken as antisymmetric itself because only the antisymmetric part will contribute to the Lorentz transform.
    Calling $J_1 = J^{23}$, $J_2 = J^{31}$ and $J_3 = J^{12}$:
    Then:
    \begin{equation*}
      \tensor{\epsilon}{^i^j^s} J_s = J^{ij} \Rightarrow \tensor{\epsilon}{_i_j_k}\tensor{\epsilon}{^i^j^s} J_s = \tensor{\epsilon}{_i_j_k}J^{ij} \Rightarrow 2 \delta_{sk} J_s = \tensor{\epsilon}{_i_j_k}J^{ij} \Rightarrow J_k = \frac{1}{2}\tensor{\epsilon}{_i_j_k}J^{ij}
    \end{equation*}
    Now taking the spatial part of the commutator:
    \begin{align*}
      i [J^{ij},J^{kl}] 
                        &= 2\eta^{k j} J^{il} + 2\eta^{l j}J^{ki}\\
%                        &= -2\delta^{m j}(\delta^k_m J^{il} + \delta^l_m J^{ki})\\
%                        &= -\tensor{\varepsilon}{_n_p^m} \tensor{\varepsilon}{_n_p^j}(\delta^k_m J^{il} + \delta^l_m J^{ki})\\
%                        &= -\tensor{\varepsilon}{_n_p^k} \tensor{\varepsilon}{_n_p^j} J^{il} -\tensor{\varepsilon}{_n_p^l} \tensor{\varepsilon}{_n_p^j}  J^{ki}\\
%                        &= -\tensor{\varepsilon}{_n_p^k} \tensor{\varepsilon}{_n_p^j} J^{il} -\tensor{\varepsilon}{_n_p^l} \tensor{\varepsilon}{_n_p^j}  J^{ki}\\
%                        &= \tensor{\varepsilon}{_n_p^k} \tensor{\varepsilon}{_n_p^j} J^{li} -\tensor{\varepsilon}{_n_p^l} \tensor{\varepsilon}{_n_p^j}  J^{ki}\\
                        &= -2(\delta^{kj} J^{il} + \delta^{lj} J^{ki})\\
                        &= -2(\delta^{kj} J^{il} - \delta^{lj} J^{ik})\\
                        &= -2(\delta^{kj}\delta^{nl}  - \delta^{lj}\delta^{nk} )J^{in}\\
                        &= -2(\delta^{jk}\delta^{nl}  - \delta^{jl}\delta^{nk} )J^{in}\\
                        &= -2\epsilon^{jnq}\epsilon^{klq}J^{in}\\
                        &\Rightarrow i [\epsilon_{rij} J^{ij},J^{kl}] = -2\epsilon_{rij}\epsilon^{jnq}\epsilon^{klq}J^{in}\\
                        &\Rightarrow i [\epsilon_{rij} J^{ij},J^{kl}] = -2\epsilon_{rij}\epsilon^{nqj}\epsilon^{klq}J^{in}\\
                        &\Rightarrow i [\epsilon_{rij} J^{ij},J^{kl}] = -2(\delta^{rn}\delta^{iq}  - \delta^{rq}\delta^{in} )\epsilon^{klq}J^{in}\\
                        &\Rightarrow i [\epsilon_{rij} J^{ij},J^{kl}] = -2\epsilon^{klq}J^{qr}\\
                        &\Rightarrow i [\epsilon_{rij} J^{ij},J^{kl}] = 2\epsilon^{klq}J^{rq}
    \end{align*}
    The last three steps follow from the symmetry of $\delta$(contraction of a symmetric and antisymmetric tensor is zero) and the anitsymmetry of $J$.
    \begin{align*}
                      i [\epsilon_{rij} J^{ij},\epsilon_{pkl} J^{kl}] &= 2\epsilon^{pkl}\epsilon^{klq}J^{rq}\\
                                                                      &= 4\delta^{pq} J^{rq}\\
                                                                      &= 4 J^{rp}\\
                                                                      &= 2 (J^{rp} - J^{pr})\\
                                                                      &= 2 (\delta^{rs}\delta^{pt} - \delta^{rt}\delta^{ps})J^{st}\\
                                                                      &= 2 \epsilon^{rpq}\epsilon^{stq}J^{st}\\
                                                                      &= 2 \epsilon^{rpq}\epsilon^{qst}J^{st}
%      \Rightarrow i [J^{ij},J^{kl}]
%      i[J^{ij},J^{kl}]  &= \eta^{kj}J^{il} - \eta^{ki}J^{jl} + \eta^{lj}J^{ki} - \eta^{li}J^{kj}\\
%      i[J^{ij},J^{kl}]  &= \delta^{kj}J^{il} - \delta^{ki}J^{jl} + \delta^{lj}J^{ki} - \delta^{li}J^{kj}\\
%      i[J^{ij},J^{kl}]  &= (\delta^{kj}\delta^{ni} - \delta^{ki}\delta^{nj})J^{nl} + (\delta^{lj}\delta ^{ni} - \delta^{li}\delta^{nj})J^{kn}\\
%      i[J^{ij},J^{kl}]  &= (\epsilon^{knp}\epsilon^{jip})J^{nl} + (\epsilon^{lnp}\epsilon^{jip})J^{kn}\\
%      \Rightarrow & i[\epsilon^{rij}J^{ij},J^{kl}]  = (\epsilon^{knp}\epsilon^{rij}\epsilon^{jip})J^{nl} + (\epsilon^{lnp}\epsilon^{rij}\epsilon^{jip})J^{kn}\\
%      \Rightarrow & i[\epsilon^{rij}J^{ij},J^{kl}]  = 2\epsilon^{knr}J^{nl} + 2\epsilon^{lnr}J^{kn}\\
%      \Rightarrow & i[\epsilon^{rij}J^{ij},\epsilon^{qkl}J^{kl}]  = 2\epsilon^{knr}\epsilon^{qkl}J^{nl} + 2\epsilon^{qkl}\epsilon^{lnr}J^{kn}\\
    \end{align*}
    And after multiplying each side by $1/4$:
    \begin{align*}
          i \left[\frac{1}{2}\epsilon^{rij} J^{ij},\frac{1}{2}\epsilon^{pkl} J^{kl}\right] = \frac{1}{2} \epsilon^{rpq}\epsilon^{qst}J^{st}
          \Rightarrow [J^r,J^p] = -i\epsilon^{rpq}J_q
    \end{align*}
    I did this problem in this way first, but it didn't seem as convincing.
    Now for the case where one of $J^{23}, J^{31}, J^{12}$ is commuted with itself:
    \begin{align*}
      [J^{ij},J^{ij}]  &= -i(-\delta^{ij}J^{ij} + \delta^{ii}J^{jj} - \delta^{jj}J^{ii} + \delta^{ji}J^{ij})\\
                       &= -i(-\delta^{ij}J^{ij} + 0 - 0 + \delta^{ij}J^{ij})\\
                       &= 0 
    \end{align*}
    And with each other, for $i\neq j$ and $j \neq k$ (otherwise equation gives $0=0$):
    \begin{align*}
%      \Rightarrow & i[J^{ij},J^{jk}]  = \eta^{jj}J^{ik} - \eta^{ji}J^{jk} + \eta^{kj}J^{ji} - \eta^{ki}J^{jj}
      [J^{ij},J^{jk}]  &= -i(-\delta^{jj}J^{ik} + \delta^{ji}J^{jk} - \delta^{kj}J^{ji} + \delta^{ki}J^{jj})\\
                       &= i J^{ik}\\
                       &= - i J^{ki}
    \end{align*}
%%    Or:
%%    \begin{align*}
%%%      \Rightarrow & i[J^{ij},J^{jk}]  = \eta^{jj}J^{ik} - \eta^{ji}J^{jk} + \eta^{kj}J^{ji} - \eta^{ki}J^{jj}
%%      [J^{lm},J^{mn}] = - i J^{ln}
%%      \Rightarrow [\epsilon_{ilm}J^{lm},\epsilon_{jmn}J^{mn}] = - i \epsilon_{ilm}\epsilon_{jmn}J^{ln} = - i \epsilon_{ijk}\epsilon_{kln}J^{ln}
%%    \end{align*}
    Which gives the identity, because even permutations of $\{ij\}, \{jk\}, \{ki\}$ give the factor $-i$, for example:
    \begin{align*}
      [J_{1},J_{2}] = [J_{23},J_{31}] = - i J_{12} = - i J_{3}
    \end{align*}
    and odd permutations, give the other sign, which is evident in switching the order of terms in the commutator:
    \begin{align*}
      [J^{ij},J^{jk}] = -[J^{jk},J^{ij}] = - i J^{ki} \Rightarrow [J^{jk},J^{ij}] = i J^{ki}
    \end{align*}
    For which an explicit example is:
    \begin{align*}
      [J_{1},J_{3}] = [J_{23},J_{12}] = i J_{31} = i J_{2}
    \end{align*}
  \item
    Using the Leibniz rule to compute the commutator:
    \begin{align*}
      [P^2, P^\nu] = [\eta_{\mu\xi}P^\mu P^\xi, P^\nu] = \eta_{\mu\xi}[P^\mu P^\xi, P^\nu] = \eta_{\mu\xi}[P^\mu , P^\nu] P^\xi + \eta_{\mu\xi}P^\mu[ P^\xi, P^\nu] = 0 + 0 = 0
    \end{align*}
    And:
    \begin{align*}
      i[\eta_{\lambda\xi}P^\lambda P^\xi, J^{\mu\nu}] &= i\eta_{\lambda\xi}[P^\lambda P^\xi, J^{\mu\nu}]\\
                                                      &= i\eta_{\lambda\xi}[P^\lambda P^\xi, J^{\mu\nu}]\\
                                                      &= i\eta_{\lambda\xi}([P^\lambda, J^{\mu\nu}] P^\xi + P^\lambda [P^\xi, J^{\mu\nu}] )\\
                                                      &= i\eta_{\lambda\xi}(\eta^{\lambda\mu}P^\nu P^\xi - \eta^{\lambda\nu}P^\mu P^\xi + \eta^{\xi\mu} P^\lambda P^{\nu} - \eta^{\xi\nu} P^\lambda P^{\mu})\\
                                                      &= i(\delta^{\mu}_\xi P^\nu P^\xi - \delta^{\nu}_\xi P^\mu P^\xi + \delta^{\mu}_\lambda P^\lambda P^{\nu} - \delta^{\nu}_\lambda P^\lambda P^{\mu})\\
                                                      &= i( P^\nu P^\mu -  P^\mu P^\nu + P^\mu P^{\nu} -  P^\nu P^{\mu})\\
                                                      &= 0 
    \end{align*}
\end{enumerate}
\item
  I worked alone for this homework.
\end{enumerate}

\end{document}
