\documentclass[12pt,a4]{article}
\usepackage{physics, amsmath,amsfonts,amsthm,amssymb, mathtools,steinmetz, gensymb, siunitx}	% LOADS USEFUL MATH STUFF
\usepackage{xcolor,graphicx}
\usepackage[left=45pt, top=20pt, right=45pt, bottom=45pt ,a4paper]{geometry} 				% ADJUSTS PAGE
\usepackage{setspace}
\usepackage{caption}
\usepackage{tikz}
\usepackage{pgf,tikz,pgfplots,wrapfig}
\usepackage{mathrsfs}
\usepackage{fancyhdr}
\usepackage{float}
\usepackage{array}
\usepackage{booktabs,multirow}
\usepackage{bm}

\usetikzlibrary{decorations.text, calc}
\pgfplotsset{compat=1.7}

\usetikzlibrary{decorations.pathreplacing,decorations.markings}
\usepgfplotslibrary{fillbetween}

\newcommand{\vect}[1]{\boldsymbol{#1}}

\usepackage{hyperref}
%\usepackage[style= ACM-Reference-Format, maxbibnames=6, minnames=1,maxnames = 1]{biblatex}
%\addbibresource{references.bib}

\hypersetup{
    colorlinks=true,
    linkcolor=blue,
    filecolor=magenta,      
    urlcolor=cyan,
    pdftitle={Overleaf Example},
    pdfpagemode=FullScreen,
    }

\title{
\textsc{QM Homework 1}
}
\author{\textsc{J L Gouws}
}
\date{\today
\\[1cm]}



\usepackage{graphicx}
\usepackage{array}




\begin{document}
\thispagestyle{empty}

\maketitle

\begin{enumerate}
  \item
    \begin{enumerate}
      \item 
        The time evolution of the position operator in the Heisenberg picture is:
        \begin{align*}
          \frac{d}{d t}      x &= \frac{i}{\hbar}\left[H, x\right]\\
                               &= \frac{i}{\hbar}\left[\frac{ p^2}{2m}, x\right]\\
                               &= \frac{i}{2m\hbar}\left( p\left[ p, x\right] + \left[ p, x\right]  p\right)\\
                               &= \frac{i}{2m\hbar}\left(-2 i \hbar p\right)\\
                               &= \frac{ p}{m}
        \end{align*}
        Note also that:
        \begin{align*}
          \frac{d}{d t}      p &= \frac{i}{\hbar}\left[H, p\right]\\
                               &= \frac{i}{\hbar}\left[\frac{1}{2} m\omega^2 x^2, p\right]\\
                               &= \frac{m\omega^2 i}{2\hbar}\left[ x^2, p\right]\\
                               &= \frac{m\omega^2 i}{2\hbar}\left( 2 i \hbar x \right)\\
                               &= -\omega^2 x
        \end{align*}
        So that:
        \begin{align*}
          \frac{d^2}{d t^2} x &= \frac{1}{m}\frac{d}{dt}\hat p\\
                                   &= \omega^2 x
        \end{align*}
        This equation is exactly the classical simple harmonic oscillator with solution:
        \begin{equation*}
          x(t) = A \cos (\omega t) + B \sin (\omega t)
        \end{equation*}
        The coefficients can be seen by taking the derivative of $x$ and setting $t$ to zero:
        \begin{equation*}
          x(t) =  \cos (\omega t) x(0) + \frac{\sin(\omega t)}{m \omega} p(0)
        \end{equation*}
      \item
        The correlation can now be written explicitly:
        \begin{equation*}
          C(t) = \langle x(t) x(0) \rangle = \langle \cos (\omega t) x^2(0) + \frac{\sin(\omega t)}{m \omega} x(0) p(0)\rangle - i \hbar \frac{\sin(\omega t)}{m \omega}
        \end{equation*}
        At $t = 0$ the oscillator must coincide with t a Schr\"odinger picture that has the wave function:\footnote{\url{https://en.wikipedia.org/wiki/Quantum_harmonic_oscillator}}
        \begin{equation*}
          \psi_0(x) = \left(\frac{m \omega}{\pi \hbar}\right) ^{1/4} e^{ - \frac{m \omega x^2}{2 \hbar}}
        \end{equation*}
        Now:
        \begin{align*}
          C(t) &= \cos (\omega t) \left(\frac{m \omega}{\pi \hbar}\right) ^{1/2} \int x^2 dx e^{ - \frac{m \omega x^2}{\hbar}}\\
               & \qquad + \left(\frac{m \omega}{\pi \hbar} \right)^{1/2}\frac{\sin(\omega t)}{m \omega} \int dx e^{ - \frac{m \omega x^2}{ 2 \hbar}} x \left(-i \hbar\frac{\partial}{\partial x}\right) e^{ - \frac{m \omega x^2}{ 2 \hbar}} - i \hbar \frac{\sin(\omega t)}{m \omega}\\
               &= \cos (\omega t) \left(\frac{m \omega}{\pi \hbar}\right) ^{1/2} \int x^2 dx e^{ - \frac{m \omega x^2}{\hbar}}\\
               &\qquad + i \left(\frac{m \omega}{\pi \hbar} \right)^{1/2} \sin(\omega t) \int dx e^{ - \frac{m \omega x^2}{ \hbar}} x^2    - i \hbar \frac{\sin(\omega t)}{m \omega}\\
               &= e^{i \omega t} \left(\frac{m \omega}{\pi \hbar}\right) ^{1/2} \int x^2 dx e^{ - \frac{m \omega x^2}{\hbar}} - i \hbar \frac{\sin(\omega t)}{m \omega}\\
               &= e^{i \omega t} \left(\frac{m \omega}{\pi \hbar}\right) ^{1/2}\frac{1}{2}\sqrt{\left(\frac{\hbar ^3\pi}{m^3 \omega^3}\right)}  - i \hbar \frac{\sin(\omega t)}{m \omega}\\
               &= e^{i \omega t} \frac{\hbar}{ 2 m \omega}  - i \hbar \frac{\sin(\omega t)}{m \omega}\\
               &=  \frac{\hbar}{ 2 m \omega}e^{-i \omega t} 
        \end{align*}
        For finding the expectation value for $C$ in the ground state:
        \begin{align*}
          \langle 0 |C(t) |0 \rangle =  \cos (\omega t) \langle 0 | x(0)^2 | 0 \rangle + \frac{\sin(\omega t)}{m \omega} \langle 0 | p(0) x(0)| 0 \rangle
        \end{align*}
        For the simple harmonic oscillator the position and momentum operators can be written in terms of the ladder operators:
        \begin{equation*}
          x = \sqrt{\frac{\hbar}{2 m \omega}} (a^\dagger + a) \qquad p = i \sqrt{\frac{\hbar m \omega}{2}} (a^\dagger - a)
        \end{equation*}
        For a product of these two operators to have a nonzero ground state expection value, we need the ladder operators to appear in the combination $a a^\dagger$.
        Now:
        \begin{align*}
          \langle 0 |C(t) |0 \rangle &=  \cos (\omega t) \langle 0 | x(0)^2 | 0 \rangle + \frac{\sin(\omega t)}{m \omega} \langle 0 | p(0) x(0)| 0 \rangle\\
                                     &=  \cos (\omega t) \langle 0 | \frac{\hbar}{2 m \omega} a a^\dagger | 0 \rangle + \frac{\sin(\omega t)}{m \omega} \langle 0 |\left( -i\frac{\hbar}{2} a a^\dagger \right) |0 \rangle\\
                                     &=  \frac{\hbar}{2 m \omega}\cos (\omega t) \langle 1 |   1 \rangle + -i\frac{\hbar}{2m \omega}\sin(\omega t) \langle 1    |1 \rangle\\
                                     &=  \frac{\hbar}{ 2 m \omega}e^{-i \omega t} 
        \end{align*}
        
  \end{enumerate}
  \item
    I did the previous question with ladder operators thanks to your explanation! So, thank you for that. I hope I got the products correct for this question.
    The Pauli matrices are (in a representation where $|z = 1\rangle = (1, 0)^T$ and $|z = -1\rangle = (0, 1)^T$):
    \begin{align*}
      \sigma_{x} = 
                       \left(\begin{matrix}
                        0 &  1 \\
                        1 &  0
                      \end{matrix} \right)
                      \qquad
      \sigma_{y} = 
                       \left(\begin{matrix}
                        0 & -i \\
                        i &  0
                      \end{matrix} \right)
                      \qquad
      \sigma_{z} = 
                       \left(\begin{matrix}
                        1 &  0 \\
                        0 & -1
                      \end{matrix} \right)
    \end{align*}
    The single particle operators are, for $x$:
    \begin{align*}
      \sigma_{x}^{1} = \sigma_x \otimes I_2 =
                       \left(\begin{matrix}
                        0 &  0 & 1 &  0\\
                        0 &  0 & 0 &  1\\
                        1 &  0 & 0 &  0\\
                        0 &  1 & 0 &  0
                      \end{matrix} \right)
                      \qquad
      \sigma_{x}^{2} = I_2 \otimes \sigma_x = 
                      \left(\begin{matrix}
                        0 &  1 & 0 &  0\\
                        1 &  0 & 0 &  0\\
                        0 &  0 & 0 &  1\\
                        0 &  0 & 1 &  0
                      \end{matrix} \right)
    \end{align*}
    For $y$:
    \begin{align*}
      \sigma_{y}^{1} = 
                       \left(\begin{matrix}
                        0 &  0 &-i &  0\\
                        0 &  0 & 0 & -i\\
                         i&  0 & 0 &  0\\
                        0 &  i & 0 &  0
                      \end{matrix} \right)
                      \qquad
      \sigma_{y}^{2} =  
                      \left(\begin{matrix}
                        0 &  -i & 0 &  0\\
                        i&  0 & 0 &  0\\
                        0 &  0 & 0 &-i\\
                        0 &  0 &i &  0
                      \end{matrix} \right)
    \end{align*}
    And for $z$:
    \begin{align*}
      \sigma_{z}^{1} =                       
                       \left(\begin{matrix}
                        1 &  0 & 0 &  0\\
                        0 &  1 & 0 &  0\\
                        0 &  0 &-1 &  0\\
                        0 &  0 & 0 & -1
                      \end{matrix} \right)
                      \qquad
      \sigma_{z}^{2} =  
                       \left(\begin{matrix}
                        1 &  0 & 0 &  0\\
                        0 & -1 & 0 &  0\\
                        0 &  0 & 1 &  0\\
                        0 &  0 & 0 & -1
                      \end{matrix} \right)
    \end{align*}
    For the first matrix measuring $\sigma_x$:
    \begin{align*}
      \sigma_{xx} &= \sigma_x \otimes I_2 +  I_2 \otimes \sigma_x\\
                  &= \sigma_x^1 + \sigma_x^2\\
                  &=
                      \left( \begin{matrix}
                        0 & 0 & 1 & 0\\
                        0 & 0 & 0 & 1\\
                        1 & 0 & 0 & 0\\
                        0 & 1 & 0 & 0
                      \end{matrix} \right)
                      +
                      \left( \begin{matrix}
                        0 & 1 & 0 & 0\\
                        1 & 0 & 0 & 0\\
                        0 & 0 & 0 & 1\\
                        0 & 0 & 1 & 0
                      \end{matrix} \right)\\
                  &=  
                      \left(\begin{matrix}
                        0 & 1 & 1 & 0\\
                        1 & 0 & 0 & 1\\
                        1 & 0 & 0 & 1\\
                        0 & 1 & 1 & 0
                      \end{matrix} \right)
    \end{align*}
    and also:
    \begin{align*}
      \sigma_{xy} =   \left(\begin{matrix}
                        0 & -i & 1 &  0\\
                        i &  0 & 0 &  1\\
                        1 &  0 & 0 & -i\\
                        0 &  1 & i &  0
                      \end{matrix} \right)
                      \qquad
      \sigma_{xz} =   \left(\begin{matrix}
                        1 &  0 & 1 &  0\\
                        0 & -1 & 0 &  1\\
                        1 &  0 & 1 &  0\\
                        0 &  1 & 0 & -1
                      \end{matrix} \right)
    \end{align*}
    For $y$ in the first index:
    \begin{align*}
      \sigma_{yx} =   \left(\begin{matrix}
                        0 &  1 &-i &  0\\
                        1 &  0 & 0 & -i\\
                        i &  0 & 0 &  1\\
                        0 &  i & 1 &  0
                      \end{matrix} \right)
                      \qquad
      \sigma_{yy} =   \left(\begin{matrix}
                        0 & -i &-i &  0\\
                        i &  0 & 0 & -i\\
                        i &  0 & 0 & -i\\
                        0 &  i & i &  0
                      \end{matrix} \right)
                      \qquad
      \sigma_{yz} =   \left(\begin{matrix}
                        1 &  0 &-i &  0\\
                        0 & -1 & 0 & -i\\
                        i &  0 & 1 &  0\\
                        0 &  i & 0 & -1
                      \end{matrix} \right)
    \end{align*}
    For $z$ in the first index:
    \begin{align*}
      \sigma_{zx} =   \left(\begin{matrix}
                        1 &  1 & 0 &  0\\
                        1 &  1 & 0 &  0\\
                        0 &  0 &-1 &  1\\
                        0 &  0 & 1 & -1
                      \end{matrix} \right)
                      \qquad
      \sigma_{zy} =   \left(\begin{matrix}
                        1 & -i & 0 &  0\\
                        i &  1 & 0 &  0\\
                        0 &  0 &-1 & -i\\
                        0 &  0 & i & -1
                      \end{matrix} \right)
                      \qquad
      \sigma_{zz} =   \left(\begin{matrix}
                        2 &  0 & 0 &  0\\
                        0 &  0 & 0 &  0\\
                        0 &  0 & 0 &  0\\
                        0 &  0 & 0 & -2
                      \end{matrix} \right)
    \end{align*}
  \item
    The path integral is more explicitly:
    \begin{align*}
      \int D[q(t)] e^{\frac{i}{\hbar} S[q, \dot q, t]} &=\lim_{N \to \infty, \Delta t\to 0} \left(\frac{m}{2 \pi i \hbar \Delta t}\right)^{N / 2}\prod_{i = 1}^{N - 1} \int  dq_i \exp\left\{\frac{i}{\hbar}\int dt L(q, \dot q, t)\right\}\\
                                                       &=\lim_{N \to \infty, \Delta t \to 0}\left(\frac{m}{2 \pi i \hbar \Delta t}\right)^{N / 2} \prod_{i = 1}^{N - 1}\int  dq_i \exp\left\{\frac{i}{\hbar}\sum_{i = 1}^{N}\int_{t_{i-1}}^{t_i} dt \frac{1}{2} m \dot q^2\right\}
    \end{align*}
    Now, the integral in the exponent is over a very short time interval so the argument will be approximately constant over this time period.
    More specifically:
    \begin{equation*}
      \dot q \leadsto \frac{q_i - q_{i - 1}}{\Delta t}
    \end{equation*}
    Now:
    \begin{align*}
      \int D[q(t)] e^{\frac{i}{\hbar} S[q, \dot q, t]} &=\lim_{N \to \infty, \Delta t \to 0}\left(\frac{m}{2 \pi i \hbar \Delta t}\right)^{N / 2} \prod_{i = 1}^{N}\int dq_i \exp\left\{\frac{i}{\hbar}\sum_{i = 1}^{N}\frac{1}{2} m \left(\frac{q_i - q_{i - 1}}{\Delta t}\right)^2 \int_{t_{i-1}}^{t_i} dt \right\}\\
                                                       &=\lim_{N \to \infty, \Delta t \to 0}\left(\frac{m}{2 \pi i \hbar \Delta t}\right)^{N / 2} \prod_{i = 1}^{N}\int dq_i \exp\left\{\frac{im}{2\hbar\Delta t}\sum_{i = 1}^{N}\left(q_i - q_{i - 1}\right)^2 \right\}
    \end{align*}
    Now note that\footnote{I used the integral identity from the tutorial, but it is wrong in the tutorial. This is the actual relation that I got from this \href{https://en.wikipedia.org/wiki/Common_integrals_in_quantum_field_theory}{Wikipedia page}}:
    \begin{align*}
      I(k) \coloneqq&\int_{-\infty}^\infty dq_k \exp\left\{\frac{im}{2\hbar \Delta t}\left(q_{k+1} - q_{k}\right)^2 + \frac{im}{2\hbar k\Delta t}\left(q_{k} - q_{0}\right)^2\right\} \\
                                                                                                                        =& \int_{-\infty}^\infty dq_i \exp\left\{\frac{im(k + 1)}{2\hbar k \Delta t} q_{k}^2 -\frac{im}{\hbar k\Delta t} (kq_{k + 1} + q_{0}) q_k + \frac{im}{2\hbar k\Delta t} \left(q_{0}^2 + k q_{k + 1}^2\right) \right\}\\
                                                                                                                        =& \left(\frac{2 \pi i \hbar  k\Delta t}{ m (k + 1)}\right)^{1/2}\exp\left\{\left[-i \frac{m(kq_{k + 1} + q_{0})^2}{2\hbar k(k + 1)\Delta t} \right]\right\}  \exp\left\{\frac{im}{2\hbar k  \Delta t}\left(k q_{k + 1}^2 + q_{0}^2\right)\right\} \\
                                                                                                                        =& \left(\frac{2 \pi i \hbar  k\Delta t}{ m (k + 1)}\right)^{1/2}\exp\left\{\frac{im}{2 \hbar k (k + 1) \Delta t}\left[\left(k q_{k + 1}^2 + q_{0}^2\right)(k + 1) - (kq_{k + 1} + q_{0})^2 \right]\right\}\\
                                                                                                                        =& \left(\frac{2 \pi i \hbar  \Delta t}{ m }\right)^{1/2}\left(\frac{k}{k+1}\right)^{1/2}\exp\left\{\frac{im}{2 \hbar  (k + 1) \Delta t}\left(q_{k + 1} - q_{0}\right)^2\right\}
%                                                                                                                        =& \left(\frac{2 \pi i \hbar (k + 1)\Delta t}{2 m}\right)^{1/2}e^{ \frac{im(q_{i + 1} - q_{i-1})^2}{2\hbar\cdot 2\Delta t} }  
    \end{align*}
    This integral can be expanded recursively:
    \begin{align*}
      I(k) &= \int_{-\infty}^\infty dq_k \exp\left\{\frac{im}{2\hbar \Delta t}\left(q_{k+1} - q_{k}\right)^2\right\} \left(\frac{2 \hbar i \Delta t}{m}\right)^{-1/2}\left( \frac{k - 1}{k}\right)^{-1/2}I(k-1) \\
           &= \int_{-\infty}^\infty dq_k \exp\left\{\frac{im}{2\hbar \Delta t}\left(q_{k+1} - q_{k}\right)^2\right\}\left(\frac{2 \hbar i \Delta t}{m}\right)^{-1/2}\left( \frac{k - 1}{k}\right)^{-1/2}\\
           &\qquad\times \int_{-\infty}^\infty dq_{k-1}\exp\left\{\frac{im}{2\hbar \Delta t}\left(q_{k} - q_{k-1}\right)^2 + \frac{im}{2\hbar (k-1)\Delta t}\left(q_{k-1} - q_{0}\right)^2\right\}\\
           &= \int_{-\infty}^\infty dq_k \int_{-\infty}^\infty dq_{k-1}\exp\left\{\frac{im}{2\hbar \Delta t}\left(q_{k+1} - q_{k}\right)^2 + \frac{im}{2\hbar \Delta t}\left(q_{k} - q_{k-1}\right)^2\right\}\\
           &\qquad \times \left(\frac{2 \hbar i \Delta t}{m}\right)^{-2/2}\left( \frac{k - 2}{k}\right)^{-1/2} I(k - 2)
    \end{align*}
    And this recursion process can be continued down to $I(1)$ with the result:
    \begin{align}
      I(k) &= \prod_{j = 2}^{k}\left(\int_{-\infty}^\infty dq_j\right) \exp\left\{\frac{im}{2\hbar \Delta t}\sum_{i = 2}^{k}\left(q_{i+1} - q_{i}\right)^2 \right\} \nonumber\\
           &\qquad \times \left(\frac{2 \hbar i \Delta t}{m}\right)^{-(k - 1)/2}\left( \frac{2}{k}\right)^{-1/2} I(1) \nonumber \\
           &= \left(\frac{2 \hbar i \Delta t}{m}\right)^{-(k - 1)/2}\left(\frac{1}{k}\right)^{-1/2} \prod_{j = 2}^{k}\left(\int_{-\infty}^\infty dq_j\right) \exp\left\{\frac{im}{2\hbar \Delta t}\sum_{i = 2}^{k}\left(q_{j+1} - q_{j}\right)^2 \right\} \nonumber\\
           &\qquad \times \int_{-\infty}^\infty dq_1 \exp\left\{\frac{im}{2\hbar \Delta t}\left(q_{2} - q_{1}\right)^2 + \frac{im}{2\hbar \Delta t}\left(q_{1} - q_{0}\right)^2\right\} \nonumber\\
           &= \left(\frac{2 \hbar i \Delta t}{m}\right)^{-(k - 1)/2}\left(\frac{1}{k}\right)^{-1/2} \prod_{j = 1}^{k}\left(\int_{-\infty}^\infty dq_j\right) \exp\left\{\frac{im}{2\hbar \Delta t}\sum_{j = 1}^{k + 1}\left(q_{j} - q_{j - 1}\right)^2 \right\}\label{eq:pathint}\\
           &=\left(\frac{2 \pi i \hbar  \Delta t}{ m }\right)^{1/2}\left(\frac{k}{k+1}\right)^{1/2}\exp\left\{\frac{im}{2 \hbar  (k + 1) \Delta t}\left(q_{k + 1} - q_{0}\right)^2\right\} \nonumber
    \end{align}
    The integral in Eq.~\ref{eq:pathint} is the integral that appears in the path integral for the propagator with $k = N-1$:
    \begin{align*}
      &\prod_{j = 1}^{N - 1}\left(\int_{-\infty}^\infty dq_j\right) \exp\left\{\frac{im}{2\hbar \Delta t}\sum_{j = 1}^{N}\left(q_{j} - q_{j - 1}\right)^2 \right\} \\
      &\qquad =  \left(\frac{2 \pi i \hbar  \Delta t}{ m }\right)^{(N - 1)/2} \left(\frac{1}{N}\right)^{1/2}\exp\left\{\frac{im}{2 \hbar  t}\left(q_{A} - q_{B}\right)^2\right\} \nonumber
    \end{align*}
    Where $t \equiv N \Delta t$. This expression determines the path integral for the amplitude to find a particle in state $q_A$ at $q_B$.
    Now:
    \begin{align*}
      \int D[q(t)] e^{\frac{i}{\hbar} S[q, \dot q, t]} &=\lim_{N \to \infty, \Delta t \to 0}\left(\frac{m}{2 \pi i \hbar N \Delta t}\right)^{1 / 2}  \exp\left\{\frac{im}{2 \hbar  t}\left(q_{A} - q_{B}\right)^2\right\}\\
                                                       &=\left(\frac{m}{2 \pi i \hbar t}\right)^{1 / 2}  \exp\left\{\frac{im}{2 \hbar  t}\left(q_{A} - q_{B}\right)^2\right\}
    \end{align*}
  \item
    My circuit can be found at this \href{https://quantum-computing.ibm.com/composer/files/new?initial=N4IgdghgtgpiBcIBSAZA4gewK4HcDOAtAAoDKAkgEwAMFAzCADQgCOEeUCIA8kQKIByARQCCJALIACCgDoqAbgA6YAJZgAxgBssAExgSFLGBuUAjAIzTVag4rBLmAJxgBzCcwDaFALq21T12ruACw\%2BSgAWbu5UoWAADgAUscoSAPQStACUkdG\%2BWInJabQMEgRJxWUSVFke0cUeZjERNTGwbFhO2V4lAHwSgTlKjCC6eH7KsQAuyhhgnCAAvkA}{link}.
    The circuit consists of a Hadamard gate that works as a beam splitter.
    A phase shifter follows and then a controlled U gate.
    The U gate is currently being used as a Y rotation gate--only the first argument should be changed, but a U gate was chosen to allow flexibility.
    The "interference" pattern goes away when the theta parameter of the U gate is set to $\pi$ and the $\left|P_0 \right\rangle$ and $\left\|P_1 \right\rangle$ states become orthogonal.
\end{enumerate}

\end{document}
