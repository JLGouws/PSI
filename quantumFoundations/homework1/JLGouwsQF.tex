\documentclass[12pt,a4]{article}
\usepackage{physics, amsmath,amsfonts,amsthm,amssymb, mathtools,steinmetz, gensymb, siunitx}	% LOADS USEFUL MATH STUFF
\usepackage{xcolor,graphicx}
\usepackage{caption}
\usepackage{subcaption}
\usepackage[left=45pt, top=20pt, right=45pt, bottom=45pt ,a4paper]{geometry} 				% ADJUSTS PAGE
\usepackage{setspace}
\usepackage{tikz}
\usepackage{pgf,tikz,pgfplots,wrapfig}
\usepackage{mathrsfs}
\usepackage{fancyhdr}
\usepackage{float}
\usepackage{array}
\usepackage{booktabs,multirow}
\usepackage{bm}
\usepackage{tensor}
\usepackage{listings}
 \lstset{
    basicstyle=\ttfamily\small,
    numberstyle=\footnotesize,
    numbers=left,
    backgroundcolor=\color{gray!10},
    frame=single,
    tabsize=2,
    rulecolor=\color{black!30},
    title=\lstname,
    escapeinside={\%*}{*)},
    breaklines=true,
    breakatwhitespace=true,
    framextopmargin=2pt,
    framexbottommargin=2pt,
    inputencoding=utf8,
    extendedchars=true,
    literate={á}{{$\rho$}}1 {ã}{{\~a}}1 {é}{{\'e}}1,
}
\DeclareMathOperator{\sign}{sgn}

\usetikzlibrary{decorations.text, calc}
\pgfplotsset{compat=1.7}

\usetikzlibrary{decorations.pathreplacing,decorations.markings}
\usepgfplotslibrary{fillbetween}

\newcommand{\vect}[1]{\boldsymbol{#1}}

\usepackage{hyperref}

%\usepackage[style= ACM-Reference-Format, maxbibnames=6, minnames=1,maxnames = 1]{biblatex}
%\addbibresource{references.bib}


\hypersetup{pdfborder={0 0 0},colorlinks=true,linkcolor=black,urlcolor=cyan,}
\allowdisplaybreaks
%\hypersetup{
%
%    colorlinks=true,
%
%    linkcolor=blue,
%
%    filecolor=magenta,      
%
%    urlcolor=cyan,
%
%    pdftitle={An Example},
%
%    pdfpagemode=FullScreen,
%
%    }
%}

\title{
\textsc{Quantum Foundations Homework 1}
}
\author{\textsc{J L Gouws}
}
\date{\today
\\[1cm]}



\usepackage{graphicx}
\usepackage{array}




\begin{document}
\thispagestyle{empty}

\maketitle

\begin{enumerate}
  \item
    \begin{enumerate}
      \item
        Note, that the notation in these answers is not absolutely explicit, and it uses some liberty.
        The notation does not show the entaglement with the ready states of the detectors from the beginning, and this continues in all the answers appearing later.
        \begin{align*}
              & \left| \text{in} \right\rangle \\
          \to & \frac{1}{\sqrt{2}} \left(\left| v \right \rangle + i \left| u \right \rangle \right) \\
          \to & \frac{1}{\sqrt{2}} \left(i\left| v \right \rangle - \left| u \right \rangle \right) \\
          \to & \frac{1}{\sqrt{2}} \left(i\left| v \right \rangle - e^{i \phi}\left| u \right \rangle \right) \\
          \to & \frac{1}{\sqrt{2}} \left(\frac{i}{\sqrt{2}}\left(\left| C_0 \right \rangle \left| D \right \rangle + i \left| C \right \rangle \left| D_0 \right \rangle\right) - \frac{1}{\sqrt{2}} e^{i \phi}\left( \left| C \right \rangle\left| D_0 \right \rangle + i\left| C_0 \right \rangle \left| D \right \rangle \right) \right) \\
            = & \frac{1}{2} \left( i \left| C_0 \right \rangle \left| D \right \rangle - \left| C \right \rangle \left| D_0 \right \rangle - e^{i \phi} \left( \left| C \right \rangle \left| D_0 \right \rangle + i \left| C_0 \right \rangle \left| D \right \rangle \right) \right) \\
            = & i \frac{(1 - e^{i \phi})}{2} \left| C_0 \right \rangle \left| D \right \rangle - \frac{(1 + e^{i \phi})}{2} \left| C \right \rangle \left| D_0 \right \rangle  
        \end{align*}
        From this it follows that:
        \begin{equation*}
          P (D) = \left| i \frac{(1 - e^{i \phi})}{2} \right|^2 = \sin^2 (\phi / 2) \qquad \text{ and } \qquad  P (C) = \left| - \frac{(1 + e^{i \phi})}{2} \right|^2 = \cos^2 (\phi / 2)
        \end{equation*}
      \item
        Here again, the notation is not absolutely explicit, but hopefully it is clear.
        The following solution uses one ket to denote the location of the photon, it might be better to indicate the state of the boxes as $\left| U_0 \right \rangle\left| V_0 \right \rangle$ for empty boxes and $\left| U \right \rangle\left| V_0 \right \rangle$ to denote a particle in box $U$ and so forth.
        The events are exclusive and this notation is a sense enforces this.
        \begin{align*}
              & \left| \text{in} \right\rangle \\
          \to & \frac{1}{\sqrt{2}} \left(\left| v \right \rangle + i \left| u \right \rangle \right) \\
          \to & \frac{1}{\sqrt{2}} \left(\left| v \right \rangle - \left| u \right \rangle \right) \\
          \to & \frac{1}{\sqrt{2}} \left(\left| v \right \rangle \left| V \right \rangle - \left| u \right \rangle \left| U \right \rangle\right) \\
          \to & \frac{1}{\sqrt{2}} \left(i\left| v \right \rangle \left| V \right \rangle - e^{i \phi}\left| u \right \rangle \left| U \right \rangle\right) \\
          \to & \frac{1}{\sqrt{2}} \left(\frac{i}{\sqrt{2}}\left(\left| C_0 \right \rangle \left| D \right \rangle + i \left| C \right \rangle \left| D_0 \right \rangle\right) \left| V \right \rangle - \frac{1}{\sqrt{2}} e^{i \phi}\left( \left| C \right \rangle\left| D_0 \right \rangle + i\left| C_0 \right \rangle \left| D \right \rangle \right) \left| U \right \rangle \right) \\
            = & \frac{1}{2} \left( i \left| C_0 \right \rangle \left| D \right \rangle \left| V \right \rangle - \left| C \right \rangle \left| D_0 \right \rangle \left| V \right \rangle - e^{i \phi} \left| C \right \rangle \left| D_0 \right \rangle \left| U \right \rangle - i e^{i \phi} \left| C_0 \right \rangle \left| D \right \rangle \left| U \right \rangle \right)
        \end{align*}
        Now for the conditional probablilities:
        \begin{align*}
          P(C | U) &= \frac{P(C \cap U)}{P(U)} = \frac{\left|\frac{-e^{i \phi}}{2}\right|^2}{\left|\frac{-e^{i \phi}}{2}\right|^2 + \left|\frac{-ie^{i \phi}}{2}\right|^2} = \frac{1}{2}\\
          P(D | U) &= \frac{1}{2}\\
          P(C | V) &= \frac{1}{2}\\
          P(D | V) &= \frac{1}{2}
        \end{align*}
      \item
        The presence of the boxes completely removes the $\phi$ dependence.
        The presence of the photon in the box indicates the path that the photon took, because the photon being placed in the box is essentially a measurement.
        The "measurement" therefore destroys interference at the detectors $C$ and $D$ and hence the loss of the $\phi$ dependence.

      \item
        Here, we want to calculate $P(C)$, which is straight forward because $U$ and $V$ are disjoint.
        \begin{equation*}
          P(C) = P(C \cap U) + P(C \cap V) = P(C | U) P(U) + P(C | V) P(V) = \frac{1}{2} \cdot \frac{1}{2} + \frac{1}{2}\cdot \frac{1}{2} = \frac{1}{2}
        \end{equation*}
      \item
        Ignoring the measurement, clearly gives quite different answers to doing the measurement.
        This essentially comes down to the measurement done by box $U$ or box $V$ destroying interference, regardless of the result being checked.
      \item
        Making a measurement in the $\left| \pm \right \rangle$ basis gives no information about which path the particle took.
        Therefore it is as if the boxes are U and V are not there and the result should be the same as the first question where there are no boxes.
      \item
        The final state after the photon has traversed the experimental equipment is:
        \begin{align*}
          \frac{1}{2} \left(i\left(\left| C_0 \right \rangle \left| D \right \rangle + i \left| C \right \rangle \left| D_0 \right \rangle\right) \left| V \right \rangle - e^{i \phi}\left( \left| C \right \rangle\left| D_0 \right \rangle + i\left| C_0 \right \rangle \left| D \right \rangle \right) \left| U \right \rangle \right)
        \end{align*}
        Which is required in the $\left| \pm \right\rangle$ basis.
        Getting this into a $\left| \pm \right\rangle$ basis requires applying the identity operator $I_{sys} = I_{C} \otimes I_{D} \otimes \left(\left| + \right\rangle \left\langle + \right| + \left| - \right\rangle \left\langle - \right|\right)$ operator and calculating:
        \begin{align*}
              & \frac{1}{2} \left(i\left(\left| C_0 \right \rangle \left| D \right \rangle + i \left| C \right \rangle \left| D_0 \right \rangle\right) \left| V \right \rangle - e^{i \phi}\left( \left| C \right \rangle\left| D_0 \right \rangle + i\left| C_0 \right \rangle \left| D \right \rangle \right) \left| U \right \rangle \right) \\
          \to & \frac{1}{2\sqrt{2}} \left(i\left(\left| C_0 \right \rangle \left| D \right \rangle + i \left| C \right \rangle \left| D_0 \right \rangle\right) \left| + \right \rangle - e^{i \phi}\left( \left| C \right \rangle\left| D_0 \right \rangle + i\left| C_0 \right \rangle \left| D \right \rangle \right) \left| + \right \rangle \right) \\
              & \quad + \frac{1}{2\sqrt{2}} \left(-i\left(\left| C_0 \right \rangle \left| D \right \rangle + i \left| C \right \rangle \left| D_0 \right \rangle\right) \left| - \right \rangle - e^{i \phi}\left( \left| C \right \rangle\left| D_0 \right \rangle + i\left| C_0 \right \rangle \left| D \right \rangle \right) \left| - \right \rangle \right) \\
          \to & \frac{1}{2\sqrt{2}} \left(i(1 - e^{i\phi}) \left| C_0 \right \rangle \left| D \right \rangle \left| + \right \rangle - (1 + e^{i \phi}) \left| C \right \rangle \left| D_0 \right \rangle \left| + \right \rangle \right) \\
              & \quad + \frac{1}{2\sqrt{2}} \left(-i(1 + e^{i\phi})\left| C_0 \right \rangle \left| D \right \rangle \left| - \right \rangle + (1 - e^{i\phi}) \left| C \right \rangle \left| D_0 \right \rangle \left| - \right \rangle\right) 
        \end{align*}
        Now the conditional probabilites follow straight forwardly:
        \begin{align*}
          P(C | +) &= \frac{P(C \cap +)}{P(+)} = \frac{\left|\frac{1 + e^{i \phi}}{2\sqrt{2}}\right|^2}{\left|\frac{1 + e^{i \phi}}{2\sqrt{2}}\right|^2 + \left|i\frac{1 - e^{i \phi}}{2\sqrt{2}}\right|^2} = \cos^2(\phi / 2)\\
          P(D | +) &= \sin^2(\phi / 2)\\
          P(C | -) &= \sin^2(\phi / 2)\\
          P(D | -) &= \cos^2(\phi / 2)
        \end{align*}
      \item
        The significance (I believe) of the $\left| \pm \right\rangle$ basis is that it is a basis of maximal uncertainty; for example, if $U$ and $V$ were states of two qubits, the basis would be a basis of Bell states.
        Having knowledge of this measurement means that we know the particle is in a superposition of being on the paths $u$ and $v$ like in part a).
%        I am not sure about what the next part of the question is asking.
%        Measuring the $\left| \pm \right\rangle$ state and measuring $\left| C\right\rangle$ or $\left| D\right\rangle$ are quite different, and $P()$.
        However, in this case $P(C) = P(D) = 1/2$, still the same as part d) above, so making the measurement in $\left|\pm\right\rangle$ does still destroy the interference pattern.
        Therefore, ignoring the outcome of the $\left| \pm \right\rangle$ measurement is very different from not performing it at all.
        This is quite difficult to interpret (for me at least), but similar situations exist in variations of the double-slit experiment (with polarizers).
    \end{enumerate}
  \item
    Assume that there is a machine that can clone states, in symbols:
    \begin{align*}
      & \left|\psi\right\rangle \left|M\right\rangle \to \left|\psi\right\rangle \left|\psi\right\rangle \left|M\right\rangle\\
      & \left|\phi\right\rangle \left|M\right\rangle \to \left|\phi\right\rangle \left|\phi\right\rangle \left|M\right\rangle
    \end{align*}
    Now if we have a linear combination $\left|s\right\rangle = \alpha\left|\psi\right\rangle + \beta \left|\phi\right\rangle$, and try to clone it, there are two ways:
    \begin{align*}
      \left|s\right\rangle\left|M\right\rangle =& \left(\alpha\left|\psi\right\rangle + \beta \left|\phi\right\rangle\right) \left|M\right\rangle\\
                                               =& \alpha\left|\psi\right\rangle \left|M\right\rangle + \beta \left|\phi\right\rangle \left|M\right\rangle\\
                                            \to & \alpha\left|\psi\right\rangle \left|\psi\right\rangle \left|M\right\rangle + \beta \left|\phi\right\rangle\left|\phi\right\rangle \left|M\right\rangle\\
                                               =& (\alpha\left|\psi\right\rangle \left|\psi\right\rangle  + \beta \left|\phi\right\rangle\left|\phi\right\rangle) \left|M\right\rangle
    \end{align*}
    It is also possible to clone the system in this way:
    \begin{align*}
      \left|s\right\rangle\left|M\right\rangle &\to \left|s\right\rangle\left|s\right\rangle\left|M\right\rangle \\
                                               &= \left(\alpha\left|\psi\right\rangle + \beta \left|\phi\right\rangle\right) \left(\alpha\left|\psi\right\rangle + \beta \left|\phi\right\rangle\right)\left|M\right\rangle \\
                                               &= \left(\alpha^2\left|\psi\right\rangle \left|\psi\right\rangle  + 2 \alpha \beta \left|\psi\right\rangle \left|\phi\right\rangle + \beta^2 \left|\phi\right\rangle \left|\phi\right\rangle\right)\left|M\right\rangle
    \end{align*}
    Which are two clearly different states, and thus the operation of cloning is not well defined.
  \item
    \begin{enumerate}
      \item
        When Alice gets the second qubit $S$, the total system is in state:
        \begin{align*}
          \left|\phi\right\rangle_S \left|\phi^+\right\rangle &= \frac{1}{\sqrt{2}}\left(\alpha\left| 0\right\rangle_S + \beta\left| 1\right \rangle_S\right)\frac{1}{\sqrt{2}}\left(\left| 0\right\rangle_A\left| 0 \right \rangle_B + \left| 1\right\rangle_A\left| 1 \right \rangle_B\right)\\
                                                              &= \frac{1}{2}\left(\left(\alpha\left| 0\right\rangle_S + \beta\left| 1\right \rangle_S\right)\left| 0\right\rangle_A\left| 0 \right \rangle_B + \left(\alpha\left| 0\right\rangle_S + \beta\left| 1\right \rangle_S\right)\left| 1\right\rangle_A\left| 1 \right \rangle_B\right)\\
                                                              &= \frac{1}{2}\left(\alpha\left| 0\right\rangle_S \left| 0\right\rangle_A\left| 0 \right \rangle_B + \beta\left| 1\right \rangle_S\left| 0\right\rangle_A\left| 0 \right \rangle_B + \alpha\left| 0\right\rangle_S\left| 1\right\rangle_A\left| 1 \right \rangle_B + \beta\left| 1\right \rangle_S\left| 1\right\rangle_A\left| 1 \right \rangle_B\right)
        \end{align*}
        Now if Alice measures the product state in one of the states $\left|\phi^{\pm}\right\rangle_{SA}$ or $\left|\psi^{\pm}\right\rangle_{SA}$ (applying the corresponding bra to $\left|\phi\right\rangle_S \left|\phi^+\right\rangle$), the state will collapse to:
        \begin{align}
          \left|\phi^{+}\right\rangle_{SA} &\to \frac{1}{2 \sqrt{2}}\left(\alpha \left| 0 \right\rangle_B + \beta \left|1 \right \rangle_B\right) \label{eq:aliceCase1}\\
          \left|\phi^{-}\right\rangle_{SA} &\to \frac{1}{2 \sqrt{2}}\left(\alpha \left| 0 \right\rangle_B - \beta \left|1 \right \rangle_B\right) \label{eq:aliceCase2}\\
          \left|\psi^{+}\right\rangle_{SA} &\to \frac{1}{2 \sqrt{2}}\left(\alpha \left| 1 \right\rangle_B + \beta \left|0 \right \rangle_B\right) \label{eq:aliceCase3}\\
          \left|\psi^{-}\right\rangle_{SA} &\to \frac{1}{2 \sqrt{2}}\left(\alpha \left| 1 \right\rangle_B - \beta \left|0 \right \rangle_B\right) \label{eq:aliceCase4}
        \end{align}
        And thus Alice precisely knows the superposition state of Bob's qubit depending on which case was measured.
      \item
        In Case~\ref{eq:aliceCase1}, Bob's qubit is in the same state as $\left|\phi\right\rangle_S$.
        Therefore, Bob only has to apply the identity operator $\left|0\right\rangle_B\left\langle 0 \right|_B + \left|1\right\rangle_B\left\langle 1 \right|_B$ to his state.
        In Case~\ref{eq:aliceCase2}, Bob must change the sign of the qubit $\left|1\right\rangle_B$.
        Therefore, Bob has to apply the operator $\left|0\right\rangle_B\left\langle 0 \right|_B - \left|1\right\rangle_B\left\langle 1 \right|_B$ to his state, which corresponds to the $\sigma_z$ matrix.
        In Case~\ref{eq:aliceCase3}, Bob must switch the two qubits.
        Therefore, Bob has to apply the operator $\left|1\right\rangle_B\left\langle 0 \right|_B + \left|0\right\rangle_B\left\langle 1 \right|_B$ to his state, which is just the $\sigma_x$ matrix.
        In the final Case~\ref{eq:aliceCase4}, Bob must change the sign of the qubit $\left|0\right\rangle_B$ and then switch the two qubits.
        Therefore, Bob has to apply the operator $-\left|1\right\rangle_B\left\langle 0 \right|_B + \left|0\right\rangle_B\left\langle 1 \right|_B$ to his state, which is just the $i\sigma_y$ matrix.
        In all of these cases Bob just has to apply the corresponding unitary.
    \end{enumerate}
\end{enumerate}
\end{document}
