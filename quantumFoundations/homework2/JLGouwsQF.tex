\documentclass[12pt,a4]{article}
\usepackage{physics, amsmath,amsfonts,amsthm,amssymb, mathtools,steinmetz, gensymb, siunitx}	% LOADS USEFUL MATH STUFF
\usepackage{xcolor,graphicx}
\usepackage{caption}
\usepackage{subcaption}
\usepackage[left=45pt, top=20pt, right=45pt, bottom=45pt ,a4paper]{geometry} 				% ADJUSTS PAGE
\usepackage{setspace}
\usepackage{tikz}
\usepackage{pgf,tikz,pgfplots,wrapfig}
\usepackage{mathrsfs}
\usepackage{fancyhdr}
\usepackage{float}
\usepackage{array}
\usepackage{booktabs,multirow}
\usepackage{bm}
\usepackage{tensor}
\usepackage{listings}
 \lstset{
    basicstyle=\ttfamily\small,
    numberstyle=\footnotesize,
    numbers=left,
    backgroundcolor=\color{gray!10},
    frame=single,
    tabsize=2,
    rulecolor=\color{black!30},
    title=\lstname,
    escapeinside={\%*}{*)},
    breaklines=true,
    breakatwhitespace=true,
    framextopmargin=2pt,
    framexbottommargin=2pt,
    inputencoding=utf8,
    extendedchars=true,
    literate={á}{{$\rho$}}1 {ã}{{\~a}}1 {é}{{\'e}}1,
}
\DeclareMathOperator{\sign}{sgn}

\usetikzlibrary{decorations.text, calc}
\pgfplotsset{compat=1.7}

\usetikzlibrary{decorations.pathreplacing,decorations.markings}
\usepgfplotslibrary{fillbetween}

\newcommand{\vect}[1]{\boldsymbol{#1}}

\usepackage{hyperref}

%\usepackage[style= ACM-Reference-Format, maxbibnames=6, minnames=1,maxnames = 1]{biblatex}
%\addbibresource{references.bib}


\hypersetup{pdfborder={0 0 0},colorlinks=true,linkcolor=black,urlcolor=cyan,}
\allowdisplaybreaks
%\hypersetup{
%
%    colorlinks=true,
%
%    linkcolor=blue,
%
%    filecolor=magenta,      
%
%    urlcolor=cyan,
%
%    pdftitle={An Example},
%
%    pdfpagemode=FullScreen,
%
%    }
%}

\title{
\textsc{Quantum Foundations Homework 2}
}
\author{\textsc{J L Gouws}
}
\date{\today
\\[1cm]}



\usepackage{graphicx}
\usepackage{array}




\begin{document}
\thispagestyle{empty}

\maketitle

\begin{enumerate}
  \item
    \begin{enumerate}
      \item
        \begin{align*}
          & C(a, b) + C(a', b) + C(a, b') - C(a', b')\\
        = & p(\text{agree}|a, b) - p(\text{disagree}|a, b) + p(\text{agree}|a', b) - p(\text{disagree}|a', b)\\
          & \qquad + p(\text{agree}|a, b') - p(\text{disagree}|a, b') - p(\text{agree}|a', b') + p(\text{disagree}|a', b')\\
        = & p(\text{agree}|a, b) + p(\text{agree}|a', b) + p(\text{agree}|a, b')  + p(\text{disagree}|a', b')\\
          & \qquad  - (p(\text{disagree}|a, b) + p(\text{disagree}|a', b)+ p(\text{disagree}|a, b') + p(\text{agree}|a', b'))\\
        \end{align*}
        Now, notice:
        \begin{align*}
          1 &= \frac{1}{4} (1) + \frac{1}{4} (1) + \frac{1}{4} (1)  + \frac{1}{4}(1)  \\
            &= \frac{1}{4}(p(\text{agree}|a, b) + p(\text{disagree}|a, b)) + \frac{1}{4} (p(\text{agree}|a', b) + p(\text{disagree}|a', b))\\
            &\qquad + \frac{1}{4} (p(\text{agree}|a, b') + p(\text{agree}|a, b'))  + \frac{1}{4}(p(\text{agree}|a', b') + p(\text{disagree}|a', b')) \\
            &= \frac{1}{4}p(\text{agree}|a, b) + \frac{1}{4} p(\text{agree}|a', b) + \frac{1}{4}p(\text{agree}|a, b') + \frac{1}{4}p(\text{disagree}|a', b') \\
            &\qquad + \frac{1}{4} p(\text{disagree}|a, b) + \frac{1}{4} p(\text{disagree}|a', b) + \frac{1}{4} p(\text{disagree}|a, b')  + \frac{1}{4}p(\text{agree}|a', b') \\
            &\le \frac{3}{4} + \frac{1}{4} p(\text{disagree}|a, b) + \frac{1}{4} p(\text{disagree}|a', b) + \frac{1}{4} p(\text{disagree}|a, b')  + \frac{1}{4}p(\text{agree}|a', b')
        \end{align*}
        Therefore:
        \begin{equation*}
          \frac{1}{4} \le \frac{1}{4} p(\text{disagree}|a, b) + \frac{1}{4} p(\text{disagree}|a', b) + \frac{1}{4} p(\text{disagree}|a, b')  + \frac{1}{4}p(\text{agree}|a', b')
        \end{equation*}
        Or:
        \begin{equation*}
          - 1 \ge - p(\text{disagree}|a, b) - p(\text{disagree}|a', b) - p(\text{disagree}|a, b') - p(\text{agree}|a', b')
        \end{equation*}
        And also:

        \begin{align*}
%         p(\text{agree}|a, b) + p(\text{agree}|a', b) + p(\text{agree}|a, b') + p(\text{disagree}|a', b') \le 3
          1 \le p(\text{disagree}|a, b) + p(\text{disagree}|a', b) + p(\text{disagree}|a, b') + p(\text{agree}|a', b') \le 4
        \end{align*}
        Clearly:
        \begin{align*}
          & C(a, b) + C(a', b) + C(a, b') - C(a', b')\\
        = & p(\text{agree}|a, b) + p(\text{agree}|a', b) + p(\text{agree}|a, b')  + p(\text{disagree}|a', b')\\
          & \qquad  - (p(\text{disagree}|a, b) + p(\text{disagree}|a', b)+ p(\text{disagree}|a, b') + p(\text{agree}|a', b')\\
      \le & 2
        \end{align*}
        For the lower bound:
        The 
      \item
        By definition of conditional probablility:
        \begin{align*}
                           & P(A | a, b, B, \lambda) =\frac{ P(A, B | a, b , \lambda)}{P(B | a, b, \lambda)} \\
           \Leftrightarrow & P(A | a, \lambda) = \frac{ P(A, B | a, b \lambda)}{P(B |b, \lambda)} \\
           \Leftrightarrow & P(A | a, \lambda) P(B |b, \lambda) = P(A, B | a, b \lambda)
        \end{align*}
      \item
        \begin{align*}
          C(a,b) &= \int d\lambda p(\lambda) (p(\text{agree} | a, b, \lambda) - p(\text{disagree} | a, b, \lambda))\\
                 &= \int d\lambda p(\lambda) (p(+, + | a, b, \lambda) + p(-, - | a, b, \lambda) - p(+, -| a, b, \lambda) - p(-, +| a, b, \lambda))\\
                 &= \int d\lambda p(\lambda) \left(p(+| a, \lambda)p(+|b, \lambda) + p(-| a,\lambda)p(-| b,\lambda) \right.\\
                 & \qquad \left. - p(+| a, \lambda)p(- | b, \lambda) - p(- | a \lambda)p(+ | b \lambda)\right)\\
                 &= \int d\lambda p(\lambda) \left(p(+| a, \lambda)(p(+|b, \lambda) - p(- | b, \lambda)) + p(-| a,\lambda)p(-| b,\lambda) \right.\\
                 & \qquad \left.  - p(- | a \lambda)p(+ | b \lambda)\right)\\
                 &= \int d\lambda p(\lambda) \left(p(+| a, \lambda)(p(+|b, \lambda) - p(- | b, \lambda)) + p(-| a,\lambda)(p(-| b,\lambda) - p(+ | b \lambda))\right)\\
                 &= \int d\lambda p(\lambda) (p(+| a, \lambda) - p(-| a,\lambda))(p(+|b, \lambda) - p(- | b, \lambda))  
        \end{align*}
      \item
        Here we get:
        \begin{align*}
          C(a,b) &= \int d\lambda p(\lambda) \bar{A}(a, \lambda)\bar{B}(b, \lambda) 
        \end{align*}
        So that:
        \begin{align*}
          C(a,b) + C(a,b') &= \int d\lambda p(\lambda) \bar{A}(a, \lambda)\bar{B}(b, \lambda) + \int d\lambda p(\lambda) \bar{A}(a, \lambda)\bar{B}(b', \lambda)\\
                           &= \int d\lambda p(\lambda) \bar{A}(a, \lambda)\left[\bar{B}(b, \lambda) + \bar{B}(b', \lambda)\right]
        \end{align*}
        And therefore:
        \begin{align*}
          |C(a,b) + C(a,b')| 
                             &= \left|\int d\lambda p(\lambda) \bar{A}(a, \lambda)\left[\bar{B}(b, \lambda) + \bar{B}(b', \lambda)\right]\right|\\
                             &= \int d\lambda p(\lambda) \left|\bar{A}(a, \lambda)\right| \left|\bar{B}(b, \lambda) + \bar{B}(b', \lambda)\right|\\
                             &\leq \int d\lambda p(\lambda) \left|\bar{B}(b, \lambda) + \bar{B}(b', \lambda)\right|
        \end{align*}
        Since:
        \begin{equation*}
          0 \le \left|\bar{A}(a, \lambda)\right| \le 1
        \end{equation*}
        The next inequality is the same as above, just substitute the symbols $+ \to -$ and $a \to a'$, but the alcalulation appears below for reference:
        \begin{align*}
          C(a',b) - C(a',b') &= \int d\lambda p(\lambda) \bar{A}(a', \lambda)\bar{B}(b, \lambda) - \int d\lambda p(\lambda) \bar{A}(a', \lambda)\bar{B}(b', \lambda)\\
                             &= \int d\lambda p(\lambda) \bar{A}(a', \lambda)\left[\bar{B}(b, \lambda) - \bar{B}(b', \lambda)\right]
        \end{align*}
        And therefore:
        \begin{align*}
          |C(a',b) - C(a',b')| 
                             &= \left|\int d\lambda p(\lambda) \bar{A}(a', \lambda)\left[\bar{B}(b, \lambda) - \bar{B}(b', \lambda)\right]\right|\\
                             &= \int d\lambda p(\lambda) \left|\bar{A}(a', \lambda)\right| \left|\bar{B}(b, \lambda) - \bar{B}(b', \lambda)\right|\\
                             &\leq \int d\lambda p(\lambda) \left|\bar{B}(b, \lambda) - \bar{B}(b', \lambda)\right|
        \end{align*}
        Since:
        \begin{equation*}
          0 \le \left|\bar{A}(a, \lambda)\right| \le 1
        \end{equation*}
      \item
        \begin{align*}
          & |C(a,b) + C(a,b')| + |C(a',b) - C(a',b')|\\
          \leq & \int d\lambda p(\lambda) \left|\bar{B}(b, \lambda) + \bar{B}(b', \lambda)\right| + \int d\lambda p(\lambda) \left|\bar{B}(b, \lambda) - \bar{B}(b', \lambda)\right|\\
             = & \int d\lambda p(\lambda) \left[\left|\bar{B}(b, \lambda) + \bar{B}(b', \lambda)\right| + \left|\bar{B}(b, \lambda) - \bar{B}(b', \lambda)\right|\right]\\
          \leq & \int d\lambda p(\lambda) \left[\left|\bar{B}(b, \lambda) + \bar{B}(b', \lambda)\right| + \left|\bar{B}(b, \lambda) - \bar{B}(b', \lambda)\right|\right]\\
        \end{align*}
        Now take a look at:
        \begin{equation*}
          \left|\bar{B}(b, \lambda) + \bar{B}(b', \lambda)\right| + \left|\bar{B}(b, \lambda) - \bar{B}(b', \lambda)\right|
        \end{equation*}
        It is true that:
        \begin{equation*}
          \left|\bar{B}(b, \lambda) + \bar{B}(b', \lambda)\right| + \left|\bar{B}(b, \lambda) - \bar{B}(b', \lambda)\right| \le 2
        \end{equation*}
        by some nonsense\footnote{If $a,b$ are real numbers, note $|a + b| + |a - b| = |-a + b| + |-a - b|= |-a - b| + |-a + b| = |a - b| + |a + b|$ and therefore: $|a + b| + |a - b| = \begin{cases}|a + b| + |a - b| &,  \text{ if } a \ge 0\\ |-a + b| + |-a - b| &, \text{ if } a < 0\end{cases} = ||a| + b| + ||a| - b|$ and similarly for $b$. Therefore $|a + b| + |a - b| = ||a| + |b|| + ||a| - |b|| = |a| + |b| + \max\{|a|, |b|\} - \min\{|a|, |b|\} = \max\{|a|, |b|\} + \min\{|a|, |b|\} + \max\{|a|, |b|\} - \min\{|a|, |b|\} = 2 \max\{|a|, |b|\}$. Hence, if $a,b$ are two real numbers that are from two respective absolutely bounded sets $A, B$: $|a + b| + |a - b| \leq \max\limits_{a \in A}\max\limits_{b \in B}| (|a + b| + |a - b|) =\max\limits_{a \in A}\max\limits_{b \in B}2 \max\{|a|, |b|\} = 2 \max\{\max\limits_{a \in A}|a|, \max\limits_{b \in B} |b|\}$}.
        %= |-a - b| + |-a + b| = |a - b| + |a + b|
%        The triangle inequality:
%        \begin{equation*}
%          |y| \le |x| + |x - y|
%        \end{equation*}
%        This is just applying a bunch of triangle inequalities.
%        \begin{align*}
%          \left|\bar{B}(b, \lambda) + \bar{B}(b', \lambda)\right| \leq \left|\bar{B}(b, \lambda)\right| + \left| \bar{B}(b', \lambda)\right| \leq 1 + 1 = 2 \\
%        \end{align*}
%        Therefore:
%        \begin{align*}
%          |C(a,b) + C(a,b')| 
%                             &\leq \int d\lambda p(\lambda) \left|\bar{B}(b, \lambda) + \bar{B}(b', \lambda)\right|\\
%                             &\leq \int d\lambda p(\lambda)2 \\
%                             &\leq 2 
%        \end{align*}
        And by the triangle inequality:
        \begin{align*}
          |C(a,b) + C(a,b') + C(a',b) - C(a',b')| \leq |C(a,b) + C(a,b')| + |C(a',b) - C(a',b')| \leq 2
        \end{align*}
    \end{enumerate}
  \item
    \begin{enumerate}
    \end{enumerate}
\end{enumerate}
\end{document}
