\documentclass[12pt,a4]{article}
\usepackage{physics, amsmath,amsfonts,amsthm,amssymb, mathtools,steinmetz, gensymb, siunitx}	% LOADS USEFUL MATH STUFF
\usepackage{xcolor,graphicx}
\usepackage{caption}
\usepackage{subcaption}
\usepackage[left=45pt, top=20pt, right=45pt, bottom=45pt ,a4paper]{geometry} 				% ADJUSTS PAGE
\usepackage{setspace}
\usepackage{tikz}
\usepackage{pgf,tikz,pgfplots,wrapfig}
\usepackage{mathrsfs}
\usepackage{fancyhdr}
\usepackage{float}
\usepackage{array}
\usepackage{booktabs,multirow}
\usepackage{bm}
\usepackage{tensor}
\usepackage{listings}
 \lstset{
    basicstyle=\ttfamily\small,
    numberstyle=\footnotesize,
    numbers=left,
    backgroundcolor=\color{gray!10},
    frame=single,
    tabsize=2,
    rulecolor=\color{black!30},
    title=\lstname,
    escapeinside={\%*}{*)},
    breaklines=true,
    breakatwhitespace=true,
    framextopmargin=2pt,
    framexbottommargin=2pt,
    inputencoding=utf8,
    extendedchars=true,
    literate={á}{{$\rho$}}1 {ã}{{\~a}}1 {é}{{\'e}}1,
}
\DeclareMathOperator{\sign}{sgn}

\usetikzlibrary{decorations.text, calc}
\pgfplotsset{compat=1.7}

\usetikzlibrary{decorations.pathreplacing,decorations.markings}
\usepgfplotslibrary{fillbetween}

\newcommand{\vect}[1]{\boldsymbol{#1}}

\usepackage{hyperref}

%\usepackage[style= ACM-Reference-Format, maxbibnames=6, minnames=1,maxnames = 1]{biblatex}
%\addbibresource{references.bib}


\hypersetup{pdfborder={0 0 0},colorlinks=true,linkcolor=black,urlcolor=cyan,}
\allowdisplaybreaks
%\hypersetup{
%
%    colorlinks=true,
%
%    linkcolor=blue,
%
%    filecolor=magenta,      
%
%    urlcolor=cyan,
%
%    pdftitle={An Example},
%
%    pdfpagemode=FullScreen,
%
%    }
%}

\title{
\textsc{GR Homework 1}
}
\author{\textsc{J L Gouws}
}
\date{\today
\\[1cm]}



\usepackage{graphicx}
\usepackage{array}




\begin{document}
\thispagestyle{empty}

\maketitle

\begin{enumerate}
  \item
    \begin{enumerate}
      \item 
        The first index of $F$ must be raised to evaluate the differential equation:
        \begin{align*}
          \tensor{F}{^\mu_\nu} = \eta^{\mu\rho} F_{\rho\nu}
        \end{align*}
        This only has two non-zero components:
        \begin{align*}
          \tensor{F}{^0_1} = \eta^{00} F_{01} = - F_{01} = E \qquad \text{and} \qquad \tensor{F}{^1_0} = \eta^{11} F_{10} = F_{10} = E
        \end{align*}

        Now the proper velocity and acceleration of the particle is:
        \begin{equation*}
          u^\mu = \frac{d x^\mu}{d \tau} = (\cosh(g \tau), \sinh(g \tau), 0, 0) \Rightarrow \frac{d u^\mu}{d \tau} = (g \sinh(g \tau), g\cosh(g \tau), 0, 0)
        \end{equation*}
        Now if $\tau$ is the proper time for this trajectory $u^\mu u_\mu  = -1$:
        \begin{align*}
          \eta^{\mu\nu}u_\mu u_\nu = (-\cosh(g \tau), \sinh(g \tau), 0, 0)^T (\cosh(g \tau), \sinh(g \tau), 0, 0) = \sinh^2(g\tau) - \cosh^2(g \tau) = -1
        \end{align*}
        As required.
        Now using the expression in the Lorentz force equation yields:
        \begin{align*}
          k\tensor{F}{^\mu_\nu}v^\nu = (Ek\sinh(g \tau), Ek\cosh(g \tau), 0, 0)
        \end{align*}
        Thus, this trajectory satisfies the Lorentz force law provided:
        \begin{equation*}
          g = Ek
        \end{equation*}
      \item
        Lowering the index of the 4-accleration gives:
        \begin{equation*}
          a^\mu = (g \sinh(g \tau), g\cosh(g \tau), 0, 0) \Rightarrow a_\mu = \eta_{\mu\nu}a^\nu = (-g \sinh(g \tau), g\cosh(g \tau), 0, 0)
        \end{equation*}
        And now taking the inner product:
        \begin{equation*}
          a^\mu a_\mu = -g^2 \sinh^2(g \tau) + g^2\cosh^2(g \tau) = g^2
        \end{equation*}
      \item
        Figure~\ref{fig:partTrajectories} gives the trajectories for the particle with different acclerations.
        \begin{figure}[H]
          \centering
          \includegraphics[width = 0.4\textwidth]{ParticleTrajectories.pdf}
          \caption{The trajectories of the charge for different values of $g$.}
          \label{fig:partTrajectories}
        \end{figure}
        The particle goes off to positive infinity for large times.
        The speed of the particle tends to $1$, as the particle cannot travel faster than the speed of light.
      \item
        Figure~\ref{fig:constCoords} shows the lines of constant $T$ and $X$.
        Note when $X = -\frac{1}{g}$ the coordinate system becomes degenerate--any $T$ will map to $(x = 0, t = 0)$ so the Rhindler coordinate system breaks down at this point and another chart is necessary to give the rest of space coordinates.
        The Rhindler coordinate system, thus, is only valid for $X \in (-1/g, \infty)$ and $T \in (-\infty, \infty)$.
        \begin{figure}[H]
          \centering
          \includegraphics[width = 0.4\textwidth]{ConstCoords.pdf}
          \caption{Lines of constant $X$ and $T$}
          \label{fig:constCoords}
        \end{figure}
      \item 
        The metric and its transformation is given by:
        \begin{align*}
          ds^2 &= -dt^2 + dx^2 + d\rho^2 + \rho^2d\phi^2\\
               &= -\left(\frac{dt}{dT}\right)^2dT^2 + \left(\frac{dx}{dX}\right)^2 dX^2 + d\rho^2 + \rho^2d\phi^2
        \end{align*}
        Now:
        \begin{equation*}
          \frac{d t}{dT} = \frac{d}{dT} \left(\frac{1}{g} + X\right) \sinh(gT) = \left(1 + gX\right) \cosh(gT) + \frac{d X}{dT} \sinh(gT)
        \end{equation*}
        And:
        \begin{equation*}
          \frac{d x}{dX} = \frac{d}{dX} \left(\frac{1}{g} + X\right) \cosh(gT) =  \cosh(gT) + \left(1 + gX\right) \sinh(gT) \frac{dT}{dX}
        \end{equation*}
        And using these in the expression for the metric:
        \begin{align*}
          ds^2  &= -\left(\left(1 + gX\right) \cosh(gT) + \frac{d X}{dT} \sinh(gT)\right)^2dT^2\\
                &\qquad + \left(\cosh(gT) + \left(1 + gX\right) \sinh(gT) \frac{dT}{dX}\right)^2 dX^2 + d\rho^2 + \rho^2d\phi^2\\
                &= -\left(1 + gX\right)^2 \cosh^2(gT)dT^2 - \left(1 + gX\right) \cosh(gT) \sinh(gT)dXdT - \sinh^2(gT) dX^2\\
                &\qquad + \cosh^2(gT)dX^2 + \cosh(gT)\left(1 + gX\right) \sinh(gT) dT dX + \left(1 + gX\right)^2 \sinh^2(gT) dT^2\\
                &\qquad \quad + d\rho^2 + \rho^2d\phi^2\\
                &= -\left(1 + gX\right)^2 (\cosh^2(gT) - \sinh^2(gT))dT^2  + (\cosh^2(gT) - \sinh^2(gT)) dX^2 + d\rho^2 + \rho^2d\phi^2\\
                &= -\left(1 + gX\right)^2 dT^2  +  dX^2 + d\rho^2 + \rho^2d\phi^2
        \end{align*}
    \end{enumerate}

  \item
    \begin{enumerate}
      \item
        The 4-vector potential is:
        \begin{gather*}
          A^\mu = \frac{1}{4 \pi}\int \frac{J^\mu (\mathbf{r}', t_r')}{|\mathbf{r} - \mathbf{r}'|} d^3r'
        \end{gather*}
        Since nothing can travel faster than the speed of light, and it is expected that electromagnetic interactions will travel at the speed of light, it is expecetd that anything observed will have come the past at some $t_r = t - |\mathbf{r} - \mathbf{r}|'$.
        For a moving point charge:
        \begin{gather*}
          J^\mu (\mathbf{r}', t') = \int d\tau' Q u^\mu \delta^4(x'^\mu - x_Q^\mu)
        \end{gather*}
        And a sneaky dirac delta insertion in the vector potential can clean things up:
        \begin{align*}
%          \phi(\mathbf{r}, t) = \frac{1}{4 \pi}\int\int\frac{Q \delta^3(\mathbf{r'} - \mathbf{r'}_Q(t'))}{|\mathbf{r} - \mathbf{r}'|}\delta(t_r' - t') dt' d^3\mathbf{r}' \\
%          \mathbf{A}(\mathbf{r}, t) = \frac{1}{4 \pi}\int\int\frac{Q \mathbf{v}'_Q(t') \delta^3(\mathbf{r'} - \mathbf{r'}_Q(t'))}{|\mathbf{r} - \mathbf{r}'|} \delta(t_r' - t')dt' d^3r'\\
          A^\mu &= \frac{1}{4 \pi}\int \frac{J^\mu (\mathbf{r}', t_r')}{|\mathbf{r} - \mathbf{r}'|} \delta(t_r' - t')dt' d^3 r'\\
                &= \frac{1}{4 \pi}\int J^\mu (\mathbf{r}', t_r') \frac{\delta((t - t') - |\mathbf{r} - \mathbf{r}'|)}{|\mathbf{r} - \mathbf{r}'|}dt' d^3 r'
        \end{align*}
        Using the identity, where $x_0$ is a zero of $f$:
        \begin{equation}
          \delta(f(x)) = \frac{\delta(x - x_0)}{|f'(x_0)|}
          \label{eq:diracdeltafunction}
        \end{equation}
        And let $\Delta t = (t - t')$ then:
        \begin{equation*}
          \delta(\Delta t^2 - |\mathbf{r} - \mathbf{r}'|^2) = \frac{\delta(t - t')}{2 \Delta t |_{|\mathbf{r} - \mathbf{r}'|}} = \frac{\delta(t - t')}{2 |\mathbf{r} - \mathbf{r}'|}
        \end{equation*}
        Thus
        \begin{align*}
          A^\mu &=\frac{1}{4 \pi}\int J^\mu (\mathbf{r}', t_r') \frac{\delta((t - t') - |\mathbf{r} - \mathbf{r}'|)}{|\mathbf{r} - \mathbf{r}'|}dt' d^3 r'\\
                &=\frac{1}{2 \pi}\int J^\mu (\mathbf{r}', t_r') \delta((t - t')^2 - |\mathbf{r} - \mathbf{r}'|^2)dt' d^3 r'\\
                &=\frac{1}{2 \pi}\int J^\mu (\mathbf{r}', t_r') \delta((x^\nu - x'^\nu)(x_\nu - x'_\nu))d^4x'\\
                &=\frac{1}{2 \pi}\int Q u^\mu \delta^4(x'^\nu - x_Q'^\nu)|_{t'_r} \delta((x^\nu - x'^\nu)(x_\nu - x'_\nu))d \tau' d^4x' \\
                &=\frac{1}{2 \pi}Q \left.\int  u^\mu \delta((x_Q'^\nu - x'^\nu)(x'\tensor{}{_Q_\nu} - x'_\nu)) d \tau' \right|_{t'_r}
        \end{align*}
        Now $(x_Q'^\mu - x'^\mu)(x'\tensor{}{_Q_\mu} - x'_\mu)$ is implicitly a function of $\tau$, therefore using Equation~\ref{eq:diracdeltafunction}:
        \begin{align*}
          \delta((x_Q'^\mu - x'^\mu)(x'\tensor{}{_Q_\mu} - x'_\mu)) = \frac{\delta(\tau - \tau')}{|(x_Q'^\mu - x'^\mu)\partial_\tau x'\tensor{}{_Q_\mu} + (x_Q'^\mu - x'_\mu)\partial_\tau x_Q'^\mu|} = \frac{\delta(\tau - \tau')}{2|(x'^\mu - x_Q'^\mu)u_\mu|}
        \end{align*}
        And using this in the expression for $A^\mu$ and carrying out the $\tau '$ integral over the delta function:
        \begin{align*}
          A^\mu &=\frac{1}{4 \pi}Q \frac{u^\mu}{|(x'^\nu - x_Q'^\nu)u_\nu|}
        \end{align*}
        Choosing units that set $4 \pi = 1$ gives the desired relation.
      \item
        Now $u^\mu = (\cosh(g \tau), \sinh(g \tau), 0, 0) = (g x_Q, g t_Q, 0, 0) = g(x_Q,t_Q, 0, 0)$. 
        Now the index must be lowered with the (Minkowski) metric:
        \begin{equation*}
          \eta_{\mu\nu} u^\mu = -1 u_0 dt + 1 u_1 dx + 1 u_2 d\rho + \rho^2 u_3 d\phi = g(-x_Q,t_Q, 0, 0)^{T}
        \end{equation*}
        Where the transpose is there to indicate that this is a form.
        Now:
        \begin{align*}
          R^\nu u_\nu &= (x^\nu - x_Q^\nu)u_\nu\\
                      &= (t - t_Q, x - x_Q, y, z) g (-x_Q, t_Q, 0, 0)^T\\
                      &= g(-t x_Q + t_Qx_Q +x t_Q - x_Qt_Q)\\
                      &= g(x t_Q -t x_Q)
        \end{align*}
        Thus combining:
        \begin{align*}
          A^\mu = - \frac{Q g (-x_Q, t_Q, 0, 0)}{g(x t_Q -t x_Q)} = \frac{Q  (-x_Q, t_Q, 0, 0)}{t x_Q - x t_Q} = \frac{Q}{\xi} (-x_Q, t_Q, 0, 0)
        \end{align*}
      \item
        I did this by hand first and then I decided to calculate $t_Q$ with Mathematica.
        \begin{equation*}
          t_Q = t - \sqrt{\rho^2 + (x - x_Q)^2}
        \end{equation*}
        Now using this in the expression for $x_Q$:
        \begin{align*}
          x_Q^2 &= L^2 + \left(t - \sqrt{\rho^2 + (x - x_Q)^2}\right)^2\\
                &= L^2 + t^2 + \rho^2 + (x - x_Q)^2 - 2 t \sqrt{\rho^2 + (x - x_Q)^2}\\
                &= L^2 + t^2 + \rho^2 + x^2 - 2x_Qx + x_Q^2 - 2 t \sqrt{\rho^2 + (x - x_Q)^2}
        \end{align*}
        Rearranging and squaring this:
        \begin{align*}
                      & 2x_Qx - L^2 - t^2 - \rho^2 - x^2 = -2 t \sqrt{\rho^2 + (x - x_Q)^2}\\
          \Rightarrow & x_Qx - (L^2 + t^2 + \rho^2 + x^2)/2 = - t \sqrt{\rho^2 + (x - x_Q)^2}\\
          \Rightarrow & x_Q^2x^2 - x_Qx (L^2 + t^2 + \rho^2 + x^2) + (L^2 + t^2 + \rho^2 + x^2)^2/4 = t^2 (\rho^2 + (x - x_Q)^2)\\
          \Rightarrow & x_Q^2x^2 - x_Qx (L^2 + t^2 + \rho^2 + x^2) + (L^2 + t^2 + \rho^2 + x^2)^2/4 = t^2 \rho^2 + t^2 x^2 - 2 t^2x_Qx + t^2x_Q^2\\
          \Rightarrow & x_Q^2(x^2 - t^2) + x_Qx (2t^2 - (L^2 + t^2 + \rho^2 + x^2)) + (L^2 + t^2 + \rho^2 + x^2)^2/4 = t^2 \rho^2 + t^2 x^2 \\
          \Rightarrow & x_Q^2(x^2 - t^2) - x_Qx (L^2 - t^2 + \rho^2 + x^2) + (L^2 + t^2 + \rho^2 + x^2)^2/4 = t^2 \rho^2 + t^2 x^2 \\
          \Rightarrow & x_Q^2(x^2 - t^2) - x_Qx \delta + (L^2 + t^2 + \rho^2 + x^2)^2/4 = t^2 \rho^2 + t^2 x^2 \\
        \end{align*}
        And now solving for $x_Q$:
        \begin{align*}
          x_Q   &= \frac{x\delta \pm \left[x^2\delta^2 - ((L^2 + t^2 + \rho^2 + x^2)^2 - 4 t^2 ( \rho^2 + x^2)) (x^2 - t^2)\right]^{1/2}}{2(x^2 - t^2)}\\
                &= \frac{x\delta \pm \left[x^2\delta^2 - \left[(L^2 + t^2 + \rho^2 + x^2)^2 - 4 t^2 ( \rho^2 + x^2)\right] (x^2 - t^2)\right]^{1/2}}{2(x^2 - t^2)}\\
                &= \frac{x\delta \pm \left[x^2\delta^2 - \left[(L^2 + t^2 + \rho^2 + x^2)^2 - 4 t^2 (\rho^2 + x^2 + L^2) + 4t^2L^2\right] (x^2 - t^2)\right]^{1/2}}{2(x^2 - t^2)}\\
%                &= \frac{x\delta \pm \left[x^2\delta^2 - \left[L^4 + t^4 + \rho^4 + x^4 + 2 L^2 t^2 + 2 L^2 x^2 + 2 L^2 \rho^2 - 2 t^2\rho^2 - 2t^2 x^2 + 2 \rho^2 x^2\right] (x^2 - t^2)\right]^{1/2}}{2(x^2 - t^2)}\\
                &= \frac{x\delta \pm \left[x^2\delta^2 - \left(\delta^2 + 4L^2t^2\right) (x^2 - t^2)\right]^{1/2}}{2(x^2 - t^2)}\\
                &= \frac{x\delta \pm \left[\left(\delta^2 + 4L^2t^2 - 4L^2 x^2 \right)t^2 \right]^{1/2}}{2(x^2 - t^2)}\\
%                &= \frac{x\delta \pm t\left[\left(\delta^2 - 4L^2t^2 - 4L^2 x^2 \right) \right]^{1/2}}{2(x^2 - t^2)}\\
                &= \frac{x\delta \pm t\left[\left((L^2 - t^2 + x^2 + \rho^2)^2 - 4L^2(-t^2 + x^2 + \rho^2)\right) + 4L^2 \rho^2 \right]^{1/2}}{2(x^2 - t^2)}\\
                &= \frac{x\delta \pm t\left[(x^2 + \rho^2 - L^2 - t^2) ^2  + 4L^2 \rho^2 \right]^{1/2}}{2(x^2 - t^2)}\\
                &= \frac{x\delta \pm 2t\left[(x^2 + \rho^2 - L^2 - t^2) ^2/4  + L^2 \rho^2 \right]^{1/2}}{2(x^2 - t^2)}\\
                &= \frac{x\delta \pm 2t\xi}{2(x^2 - t^2)}
%                &= \frac{x\delta \pm \left[(\delta^2 - \delta (2x^2 + 2\rho^2) + (2x^2 + 2\rho^2))t^2 - x^2 (\delta (2x^2 + 2\rho^2) - (2x^2 + 2\rho^2)) + 4 t^2(\rho^2 + x^2) (x^2 - t^2)\right]^{1/2}}{2(x^2 - t^2)}\\
        \end{align*}
        Now, the correct sign should be chosen.
        Both the plus and minus sign seem to solve the problem provided the expression for $t_Q$ is given the same sign:
        \begin{equation*}
          t_Q = \frac{t\delta \pm 2x\xi}{2(x^2 - t^2)}
        \end{equation*}
        The for this expression calculation is analogous to the above.
        The only difference appears to be the following.
        Look at the large $t$ limit, $\xi \sim t^2$ so that:
        \begin{equation*}
          x_Q \leadsto \mp 2 t
        \end{equation*}
        So that for the top sign as $t \to \infty$ the particle goes in the negative direction, and the bottom sign indicates that the particle goes to positive infinity.
        In the $\tau$ parameterization of $x_Q$, $x_Q$ goes to positive infinity indicating that the lower sign corresponds to the parametrization in question 1.
      \item
        Since the expression for the potential has $t$, $x$ and $\rho$ dependence it is possible for the $F_{10}$, $F_{01}$, $F_{20}$, $F_{02}$, $F_{12}$ and $F_{21}$ components to be non-zero.
        All other components are automatically zero.
        Doing the calculation explicitly shows that all the above mentioned are non-zero--see the Mathematica notebook.
        $F_{[10]}$ and $F_{[20]}$ correspond to the Electric field in the $x$ and $\rho$ direction, but maybe surprisingly the $F_{[21]}$ components are non-zero indicating a magnetic field in the $\phi$ direction.
    \end{enumerate}
  \item
    \begin{enumerate}
      \item
        Now the transformation is:
        \begin{equation*}
          A_\mu dx^\mu = A_\mu \frac{\partial x^\mu}{\partial X^\mu} dX^\mu = A'_\mu dX^\mu
        \end{equation*}
        Where $X$ stands for the Rhindler coordinates, and $A'_\mu = A_\mu \frac{\partial x^\mu}{\partial X^\mu}$.
        The transformation matrix is, see the Mathematica notebook:
        \begin{equation*}
          \frac{\partial x^\mu}{\partial X^\mu}
          =
          \left(
          \begin{matrix*}
            (1+g X) \cosh(g T)&\sinh g T)&0&0\\
            (1+g X) \sinh g T)&\cosh g T)&0&0\\
            0&0&1&0\\
            0&0&0&1
          \end{matrix*}
          \right)
        \end{equation*}
        And using this and writing the potential in the Rhindler coordintes gives the transformation.
      \item
        The gauge function should satisfy the following properties:
        \begin{equation*}
          \partial_T f = 0 \qquad \partial_X f = -\frac{Q}{L + X} \qquad \partial_\rho f = 0 \qquad \partial_\phi f = 0
        \end{equation*}
        Thus, $f$ should only have explicit $X$ dependence, and a function that will give the correct $\partial_X f$ criterion is:
        \begin{equation*}
          f = - Q \ln(L + X)
        \end{equation*}
      \item
        Taking the Taylor series in $g$ of the potential, which I left to Mathematica.
        \begin{align*}
          \phi &= \frac{Q}{r} -\frac{g Q X}{2 r}+ \frac{3 g ^2Q (X^2+r^2) }{8 r^2}\\
               &= \frac{Q}{r}\left(1 -\frac{1}{2 r}gX+ \frac{3 g ^2 (X^2+r^2) }{8 r}\right)
        \end{align*}
      \item
        Figure~\ref{fig:equipgraphs} gives plots of equipotentials for various values of $g$.
        \begin{figure}[H]
          \centering
          \begin{subfigure}[a]{.4\textwidth}
            \includegraphics[width=\linewidth]{g01equip.pdf}
          \end{subfigure}
          \begin{subfigure}[a]{.4\textwidth}
            \includegraphics[width=\linewidth]{g025equip.pdf}
          \end{subfigure}
          \begin{subfigure}[a]{.4\textwidth}
            \includegraphics[width=\linewidth]{g05equip.pdf}
          \end{subfigure}
          \begin{subfigure}[a]{.4\textwidth}
            \includegraphics[width=\linewidth]{g1equip.pdf}
          \end{subfigure}
          \begin{subfigure}[a]{.4\textwidth}
            \includegraphics[width=\linewidth]{g0equip.pdf}
          \end{subfigure}
%          \subfigure[]{\includegraphics[width=0.24\textwidth]{g025equip.pdf}}
%          \subfigure[]{\includegraphics[width=0.24\textwidth]{g05equip.pdf}}
%          \subfigure[]{\includegraphics[width=0.24\textwidth]{g1equip.pdf}}
          \caption{Equipotentials for the accelerating charge in an orthonormal coordinate system}
          \label{fig:equipgraphs}
        \end{figure}
        Figure~\ref{fig:equipgraphs} shows that the graphs seem streched out to the left and compressed to the right.
        The electric field will be weaker at points in the direction opposite to the charge's motion and stronger at points that are to the right of the charge.
        In the exreme cases of high $g$ the potential is constant to the left of the charge indicating that there will be no electric field to the left of the charge.
        This occurence is (probably) equivalent to the fact that no light escape regions of high enough gravitational acceleration.
        With $g$ low and zero, the equipotentials become circles reducting to the static limit.
      \item
        The $F'_{M N}$ components are given by $F'_{MN} = \partial_{M}A'_N - \partial_{N}A'_M$.
        Which I gave to Mathematica to compute:
        \begin{equation*}
          F_{10} = \frac{4Q(1 + g X)(X(2 + gX) - g \rho^2)}{r^2((2 + gX)^2 + g^2 \rho^2)^{3/2}} \qquad
          F_{20} = \frac{8Q(1 + g X)\rho}{r^2((2 + gX)^2 + g^2 \rho^2)^{3/2}}
        \end{equation*}
        The only thing to keep track of here is that the indices here should be raised by the inverse metric:
        \begin{equation*}
          F'^{M N} = g^{M P}g^{N\Sigma}F'_{P\Sigma}
        \end{equation*}
        And $\det (g_\alpha \beta) = -(1 + gX)^2\rho^2$, the rest is mess of algebra, for which Mathematica gives:
        
        \begin{lstlisting}[language=Mathematica, mathescape]
In[41]:= FmunuPrime =
  (D[AdmuPrime /. {r^2 ->X^2 + $\rho$^2}, {{T,X,$\rho$, $\phi$}}] 
  -Transpose[D[AdmuPrime /. {r^2 ->X^2 + $\rho$^2}, 
            {{T,X,$\rho$, $\phi$}}]])//FullSimplify
Out[41]= 
  {{0,(4 Q (1+g X) (X (2+g X)-g $\rho$^2))/((X^2+ρ^2) ((2+g X)^2
  +g^2 ρ^2))^(3/2),(8 Q (1+g X)^2 ρ)/((X^2+ρ^2) ((2+g X)^2+g^2 ρ^2))^(3/2),0},{-((4 Q (1+g X) (X (2+g X)-g ρ^2))/((X^2+ρ^2) ((2+g X)^2+g^2 ρ^2))^(3/2)),0,0,0},{-((8 Q (1+g X)^2 ρ)/((X^2+ρ^2) ((2+g X)^2+g^2 ρ^2))^(3/2)),0,0,0},{0,0,0,0}}
In[42]:= ds2 = DiagonalMatrix[{-(1+g X)^2,1,1,$\rho$^2}];
(*Thank you to Ruhi for this idea*);
In[43]:= del={D[#,T] &, D[#,X] &,D[#,$\rho$] &,D[#,$\phi$] &}; 
In[44]:= gu = Inverse[ds2]//FullSimplify;
Total[del[[#]][Sqrt[-Det[ds2]]gu.FmunuPrime.gu[[#]]]&/@Range[4]]
        //FullSimplify
Out[45]= {0,0,0,0}
        \end{lstlisting}
    \end{enumerate}
\end{enumerate}
\end{document}
