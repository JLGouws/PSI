\documentclass[12pt,a4]{article}
\usepackage{physics, amsmath,amsfonts,amsthm,amssymb, mathtools,steinmetz, gensymb, siunitx}	% LOADS USEFUL MATH STUFF
\usepackage{xcolor,graphicx}
\usepackage{caption}
\usepackage{subcaption}
\usepackage[left=45pt, top=20pt, right=45pt, bottom=45pt ,a4paper]{geometry} 				% ADJUSTS PAGE
\usepackage{setspace}
\usepackage{tikz}
\usepackage{pgf,tikz,pgfplots,wrapfig}
\usepackage{mathrsfs}
\usepackage{fancyhdr}
\usepackage{float}
\usepackage{array}
\usepackage{booktabs,multirow}
\usepackage{bm}
\usepackage{tensor}
\usepackage{listings}
 \lstset{
    basicstyle=\ttfamily\small,
    numberstyle=\footnotesize,
    numbers=left,
    backgroundcolor=\color{gray!10},
    frame=single,
    tabsize=2,
    rulecolor=\color{black!30},
    title=\lstname,
    escapeinside={\%*}{*)},
    breaklines=true,
    breakatwhitespace=true,
    framextopmargin=2pt,
    framexbottommargin=2pt,
    inputencoding=utf8,
    extendedchars=true,
    literate={á}{{$\rho$}}1 {ã}{{\~a}}1 {é}{{\'e}}1,
}
\DeclareMathOperator{\sign}{sgn}
\DeclareMathOperator{\Div}{div}
\newcommand{\e}{\mathrm{d}}

\usetikzlibrary{decorations.text, calc}
\pgfplotsset{compat=1.7}

\usetikzlibrary{decorations.pathreplacing,decorations.markings}
\usepgfplotslibrary{fillbetween}

\newcommand{\vect}[1]{\boldsymbol{#1}}

\usepackage{hyperref}

%\usepackage[style= ACM-Reference-Format, maxbibnames=6, minnames=1,maxnames = 1]{biblatex}
%\addbibresource{references.bib}


\hypersetup{pdfborder={0 0 0},colorlinks=true,linkcolor=black,urlcolor=cyan,}
\allowdisplaybreaks
%\hypersetup{
%
%    colorlinks=true,
%
%    linkcolor=blue,
%
%    filecolor=magenta,      
%
%    urlcolor=cyan,
%
%    pdftitle={An Example},
%
%    pdfpagemode=FullScreen,
%
%    }
%}

\title{
\textsc{Mathematical Physics Homework 1}
}
\author{\textsc{J L Gouws}
}
\date{\today
\\[1cm]}



\usepackage{graphicx}
\usepackage{array}




\begin{document}
\thispagestyle{empty}

\maketitle

\begin{enumerate}
  \item
    \begin{enumerate}
      \item
        Using the coordinate fomula for the hodge star:
        \begin{align*}
          * \e x &= \frac{1}{(3 - 1)!} \sqrt{|\det{\delta_{ij}}|} \Delta^{x j_1} \varepsilon_{j_1 j_2 j_3} \e x^{j_2} \wedge \e x^{j_3}\\
                 &= \frac{1}{2} \varepsilon_{x j_2 j_3} \e x^{j_2} \wedge \e x^{j_3}\\
                 &= \frac{1}{2} \left(\e y \wedge \e z - \e z \wedge \e y\right)\\
                 &= \e y \wedge \e z
        \end{align*}
        Since the inverse metric is just $\Delta^{ab} = \delta^{ab}$ and $\det \delta = 1$.
        Similarly:
        \begin{align*}
          * \e y &= \frac{1}{(3 - 1)!} \sqrt{|\det{\delta_{ij}}|} \Delta^{y j_1} \varepsilon_{j_1 j_2 j_3} \e x^{j_2} \wedge \e x^{j_3}\\
                 &= \frac{1}{2} \varepsilon_{y j_2 j_3} \e x^{j_2} \wedge \e x^{j_3}\\
                 &= \e z \wedge \e x
        \end{align*}
        And finally:
        \begin{align*}
          * \e z &= \varepsilon_{z j_2 j_3} \e x^{j_2} \wedge \e x^{j_3}\\
                 &= \e x \wedge \e y
        \end{align*}
        The Hodge star operator's action on the two and three forms is:
        \begin{align*}
          * (\e y \wedge \e z)\\
                  &= *(* \e x)\\
                  &= (-1)^{1(3 - 2)}\sign(\det \delta) \e x\\
                  &= \e x
        \end{align*}
        The factor $(-1)^{1(3 - 2)}\sign(\det \delta)$ is the same for all of the 2-forms, therefore:
        \begin{equation*}
          *(\e z \wedge \e x) = \e y \qquad \text{ and } \qquad *(\e x \wedge \e y) = \e z
        \end{equation*}
        For the top form:
        \begin{equation*}
          * 1 = \sqrt{\det \delta} dx \wedge dy \wedge dz = dx \wedge dy \wedge dz \Rightarrow *(dx \wedge dy \wedge dz) = ** 1 = 1
        \end{equation*}
        For a general smooth top form where $f \in \Omega^0(M)$:
        \begin{equation*}
          *(f dx \wedge dy \wedge dz) = f
        \end{equation*}
      \item
        This is a straight-forward calculation of $\e f$.
        \begin{equation*}
          \e f^\sharp = \left(\frac{\partial f}{\partial x} \e x + \frac{\partial f}{\partial y} \e y + \frac{\partial f}{\partial z} \e z\right)^\sharp = \frac{\partial f}{\partial x}\frac{\partial}{\partial y} + \frac{\partial f}{\partial y}\frac{\partial }{\partial y}  + \frac{\partial f}{\partial z}\frac{\partial}{\partial y} \equiv \nabla f
        \end{equation*}
        For the one form $\alpha = \alpha_x \e x + \alpha_y \e y + \alpha_z \e z$:
        \begin{align*}
          &* \e \alpha \\
                      =& *(\partial_y\alpha_x \e y \wedge \e x + \partial_z \alpha_x \e z \wedge \e x + \partial_x\alpha_y \e x \wedge \e y + \partial_z \alpha_y \e z \wedge \e y + \partial_x\alpha_z \e x \wedge \e z + \partial_y \alpha_z \e y \wedge \e z)\\
                      =& *\left[(\partial_x\alpha_y - \partial_y\alpha_x) \e x \wedge \e y + (\partial_y \alpha_z - \partial_z \alpha_y)\e y \wedge \e z + (\partial_z \alpha_x - \partial_x \alpha_z)\e z \wedge \e x \right]\\
                      =& (\partial_x\alpha_y - \partial_y\alpha_x) \e z + (\partial_y \alpha_z - \partial_z \alpha_y) \e x + (\partial_z \alpha_x - \partial_x \alpha_z)\e y 
        \end{align*}
        Which is the formula for the curl of a vector with components $\alpha$.
        Similarly:
        \begin{align*}
          * \e * \alpha 
                      &= *\left[\e (\alpha_x \e y \wedge \e z + \alpha_y \e z \wedge \e x + \alpha_z \e x \wedge \e y\right]\\
                      &= *\left[\partial_x \alpha_x \e x \wedge \e y \wedge \e z + \partial_y \alpha_y \e y \wedge \e z \wedge \e x + \partial_z \alpha_z \e z \wedge \e x \wedge \e y\right]\\
                      &= *\left[(\partial_x \alpha_x + \partial_y \alpha_y + \partial_z \alpha_z ) \e x \wedge \e y \wedge \e z\right]\\
                      &= \partial_x \alpha_x + \partial_y \alpha_y + \partial_z \alpha_z
        \end{align*}
        Which is just the gradient of $\alpha$.
        \begin{align*}
          * (\alpha \wedge \beta)
                      &= *\left[(\beta_x \e x + \beta_y \e y + \beta_z \e z) \wedge (\alpha_x \e x + \alpha_y \e y + \alpha_z \e z)\right]\\
                      &= *\left[(\beta_x\alpha_y - \beta_y\alpha_x) \e x \wedge \e y + (\beta_y \alpha_z - \beta_z \alpha_y)\e y \wedge \e z + (\beta_z \alpha_x - \beta_x \alpha_z)\e z \wedge \e x \right]\\
                      &= (\beta_x\alpha_y - \beta_y\alpha_x) \e z + (\beta_y \alpha_z - \beta_z \alpha_y)\e x + (\beta_z \alpha_x - \beta_x \alpha_z)\e y
        \end{align*}
        Which is just the cross product formula.
    \end{enumerate}
  \item
    \begin{enumerate}
      \item
        Under the coordinate reflection $(\e x, \e y, \e z) \to (- \e x, -\e y, -\e z)$, the magnetic field transforms as:
        \begin{align*}
          B &= B_x \e y \wedge \e z + B_y \e z \wedge \e x + B_z \e x \wedge \e y \\
            &\to  B_x (- \e y) \wedge ( - \e z) + B_y (-\e z) \wedge (- \e x) + B_z (- \e x) \wedge (-\e y)\\
            &=  B_x \e y \wedge \e z + B_y \e z \wedge \e x + B_z \e x \wedge \e y\\
            &=  B
        \end{align*}
        Making $B$ a pseudovector.
      \item
        The equation:
        \begin{equation*}
          \e B = 0 \Leftrightarrow * \e * (*B) = 0 \Leftrightarrow \nabla \cdot (* B)^\sharp = 0
        \end{equation*}
        The next equation is:
        \begin{equation*}
          \e E + \frac{\partial B}{\partial t} = 0 \Leftrightarrow * \e E +  \frac{\partial * B}{\partial t} = 0 \Leftrightarrow  \nabla \times E^\sharp + \frac{\partial (* B)^\sharp}{\partial t} = 0
        \end{equation*}
        Since the ``metric", $\delta_{ij}$ is time independent.
        Next:
        \begin{equation*}
          * \e * E = \mu_0 \rho \Leftrightarrow  \nabla \cdot E^\sharp = \mu_0 \rho
        \end{equation*}
        Finally:
        \begin{equation*}
          * \e * B - \frac{\partial E}{\partial t} = \mu_0 J \Leftrightarrow * \e (* B) - \frac{\partial E}{\partial t} = \mu_0 J \Leftrightarrow * \nabla \times (*B)^\sharp - \frac{\partial E^\sharp}{\partial t} = \mu_0 J^\sharp
        \end{equation*}
      \item
        Here $\e B = 0$ says that $B$ is closed so Poincare's lemma says $B$ is exact, or in other words there is a 1-form, $A$, such that $B = \e A$.
        Also:
        \begin{align*}
                      & \e E + \frac{\partial B}{\partial t} = 0\\
          \Rightarrow & \e E + \frac{\partial \e A}{\partial t} = 0\\
          \Rightarrow & \e E + \e \frac{\partial A}{\partial t} = 0\\
          \Rightarrow & \e \left(E + \frac{\partial A}{\partial t}\right) = 0
        \end{align*}
        Therefore, $E + \partial_t A$ is a closed 1-form so by Poincare's lemma, there exists a 0-form $\phi$ or $-\varphi$ such that:
        \begin{align*}
                      & E + \frac{\partial A}{\partial t} = - \e \varphi\\
          \Rightarrow & E  = - \e \varphi - \frac{\partial A}{\partial t}
        \end{align*}
      \item
        For the 0-form:
        \begin{equation*}
          *1 = \e t \wedge \e x \wedge \e y \wedge \e z
        \end{equation*}
        And hence the top form:
        \begin{equation*}
          *(\e t \wedge \e x \wedge \e y \wedge \e z) = *(*1) = (-1)^{0(4-1)}(-1) 1 = -1
        \end{equation*}
        For the one forms, the computations are straight forward:
        \begin{align*}
          *\e t
                 &= \frac{1}{(4 - 1)!} \sqrt{|\det{\eta_{ij}}|} H^{t j_1} \varepsilon_{j_1 j_2 j_3 j_4} \e x^{j_2} \wedge \e x^{j_3} \wedge \e x^{j_4}\\
                 &= - \frac{1}{3!}  \varepsilon_{t j_2 j_3 j_4} \e x^{j_2} \wedge \e x^{j_3} \wedge \e x^{j_4}\\
                 &= - \frac{1}{6}  \left(\varepsilon_{t x y z} \e x \wedge \e y \wedge \e z + \varepsilon_{t x z y} \e x \wedge \e z \wedge \e y + \varepsilon_{t y x z} \e y \wedge \e x \wedge \e z\right.\\
                 &\qquad \qquad +\left. \varepsilon_{t y z x} \e z \wedge \e z \wedge \e x + \varepsilon_{t z x y} \e z \wedge \e x \wedge \e y + \varepsilon_{t z y x} \e z \wedge \e y \wedge \e x\right)\\
                 &= -  \e x \wedge \e y \wedge \e z
        \end{align*}
        \begin{align*}
          *\e x
                 &= \frac{1}{(4 - 1)!} \sqrt{|\det{\eta_{ij}}|} H^{x j_1} \varepsilon_{j_1 j_2 j_3 j_4} \e x^{j_2} \wedge \e x^{j_3} \wedge \e x^{j_4}\\
                 &= - \frac{1}{3!}  \varepsilon_{x j_2 j_3 j_4} \e x^{j_2} \wedge \e x^{j_3} \wedge \e x^{j_4}\\
                 &= \frac{1}{6}  \left(\varepsilon_{x t y z} \e t \wedge \e y \wedge \e z + \varepsilon_{x t z y} \e t \wedge \e z \wedge \e y + \varepsilon_{x y t z} \e y \wedge \e t \wedge \e z\right.\\
                 &\qquad \qquad +\left. \varepsilon_{x y z t} \e y \wedge \e z \wedge \e t + \varepsilon_{x z t y} \e z \wedge \e t \wedge \e y + \varepsilon_{x z y t} \e z \wedge \e y \wedge \e t\right)\\
                 &= -  \e t \wedge \e y \wedge \e z
        \end{align*}
        \begin{align*}
          *\e y
                 &= \frac{1}{(4 - 1)!} \sqrt{|\det{\eta_{ij}}|} H^{y j_1} \varepsilon_{j_1 j_2 j_3 j_4} \e x^{j_2} \wedge \e x^{j_3} \wedge \e x^{j_4}\\
                 &= \frac{1}{3!}  \varepsilon_{y j_2 j_3 j_4} \e x^{j_2} \wedge \e x^{j_3} \wedge \e x^{j_4}\\
                 &= \frac{1}{6}  \left(\varepsilon_{y t x z} \e t \wedge \e x \wedge \e z + \varepsilon_{y t z x} \e t \wedge \e z \wedge \e x + \varepsilon_{y x t z} \e x \wedge \e t \wedge \e z\right.\\
                 &\qquad \qquad +\left. \varepsilon_{y x z t} \e x \wedge \e z \wedge \e t + \varepsilon_{y z t x} \e z \wedge \e t \wedge \e x + \varepsilon_{y z x t} \e z \wedge \e x \wedge \e t\right)\\
                 &=  \e t \wedge \e x \wedge \e z
        \end{align*}
        Similarly for $z$.
        \begin{align*}
          *\e z
                 &=  - \e t \wedge \e x \wedge \e y
        \end{align*}
        Hence, for the three forms:
        \begin{align*}
          *(\e x \wedge \e y \wedge \e z) 
            &= *(- *\e t)\\
            &= - (-1)^{1(4 - 1)} \sign(-1) \e t\\
            &= - \e t
        \end{align*}
        \begin{align*}
          *(\e t \wedge \e y \wedge \e z) 
            &= - \e x
        \end{align*}
        \begin{align*}
          *(\e t \wedge \e x \wedge \e z) 
            &= \e y
        \end{align*}
        \begin{align*}
          *(\e t \wedge \e x \wedge \e y) 
            &= \e z
        \end{align*}
        For the 2-forms:
        \begin{align*}
          *(\e t \wedge \e x)
                 &= \frac{1}{(4 - 2)!} \sqrt{|\det{\eta_{ij}}|} H^{t j_1} H^{x j_2} \varepsilon_{j_1 j_2 j_3 j_4} \e x^{j_3} \wedge \e x^{j_4}\\
                 &= - \frac{1}{2} \varepsilon_{t x j_3 j_4} \e x^{j_3} \wedge \e x^{j_4}\\
                 &= - \frac{1}{2} \left(\varepsilon_{t x y z} \e y \wedge \e  z + \varepsilon_{t x z y} \e z \wedge \e y\right)\\
                 &= - \frac{1}{2} \left( \e y \wedge \e  z - \e z \wedge \e y\right)\\
                 &= - \e y \wedge \e  z 
        \end{align*}
        \begin{align*}
          *(\e t \wedge \e y)
                 &= \frac{1}{(4 - 2)!} \sqrt{|\det{\eta_{ij}}|} H^{t j_1} H^{y j_2} \varepsilon_{j_1 j_2 j_3 j_4} \e x^{j_3} \wedge \e x^{j_4}\\
                 &= - \frac{1}{2} \varepsilon_{t y j_3 j_4} \e x^{j_3} \wedge \e x^{j_4}\\
                 &= - \frac{1}{2} \left(\varepsilon_{t y x z} \e x \wedge \e  z + \varepsilon_{t y z x} \e z \wedge \e x\right)\\
                 &= - \frac{1}{2} \left( - \e x \wedge \e  z + \e z \wedge \e x\right)\\
                 &= \e x \wedge \e  z 
        \end{align*}
        \begin{align*}
          *(\e t \wedge \e z)
                 &= \frac{1}{(4 - 2)!} \sqrt{|\det{\eta_{ij}}|} H^{t j_1} H^{z j_2} \varepsilon_{j_1 j_2 j_3 j_4} \e x^{j_3} \wedge \e x^{j_4}\\
                 &= - \frac{1}{2} \varepsilon_{t z j_3 j_4} \e x^{j_3} \wedge \e x^{j_4}\\
                 &= - \frac{1}{2} \left(\varepsilon_{t z x y} \e x \wedge \e  y + \varepsilon_{t z y x} \e y \wedge \e x\right)\\
                 &= - \frac{1}{2} \left( \e x \wedge \e  y - \e y \wedge \e x\right)\\
                 &= - \e x \wedge \e  y 
        \end{align*}
        And the other ones are given by the star of the star:
        \begin{align*}
         *(\e y \wedge \e  z) 
                 &= *(- *(\e t \wedge \e x))\\
                 &= -* *(\e t \wedge \e x)\\
                 &= -(-1)^{2 (4 -2)}\sign (-1)\e t \wedge \e x)\\
                 &= \e t \wedge \e x
        \end{align*}
        \begin{align*}
         *(\e x \wedge \e  z) 
                 &= * *(\e t \wedge \e y)\\
                 &= (-1)^{2 (4 -2)}\sign (-1)\e t \wedge \e y)\\
                 &= - \e t \wedge \e y
        \end{align*}
        \begin{align*}
         *(\e x \wedge \e  y) 
                 &= *(- *(\e t \wedge \e z))\\
                 &= -* *(\e t \wedge \e z)\\
                 &= -(-1)^{2 (4 -2)}\sign (-1)\e t \wedge \e z)\\
                 &= \e t \wedge \e z 
        \end{align*}
%        \begin{align*}
%          *\e x
%                 &= \frac{1}{(4 - 1)!} \sqrt{|\det{\eta_{ij}}|} H^{x j_1} \varepsilon_{j_1 j_2 j_3 j_4} \e x^{j_2} \wedge \e x^{j_3} \wedge \e x^{j_4}\\
%                 &= - \frac{1}{3!}  \varepsilon_{x j_2 j_3 j_4} \e x^{j_2} \wedge \e x^{j_3} \wedge \e x^{j_4}\\
%                 &= \frac{1}{6}  \left(\varepsilon_{x t y z} \e t \wedge \e y \wedge \e z + \varepsilon_{x t z y} \e t \wedge \e z \wedge \e y + \varepsilon_{x y t z} \e y \wedge \e t \wedge \e z\right.\\
%                 &\qquad \qquad +\left. \varepsilon_{x y z t} \e y \wedge \e z \wedge \e t + \varepsilon_{x z t y} \e z \wedge \e t \wedge \e y + \varepsilon_{x z y t} \e z \wedge \e y \wedge \e t\right)\\
%                 &= -  \e t \wedge \e y \wedge \e z
%        \end{align*}
%        \begin{align*}
%          *\e y
%                 &= \frac{1}{(4 - 1)!} \sqrt{|\det{\eta_{ij}}|} H^{y j_1} \varepsilon_{j_1 j_2 j_3 j_4} \e x^{j_2} \wedge \e x^{j_3} \wedge \e x^{j_4}\\
%                 &= \frac{1}{3!}  \varepsilon_{y j_2 j_3 j_4} \e x^{j_2} \wedge \e x^{j_3} \wedge \e x^{j_4}\\
%                 &= \frac{1}{6}  \left(\varepsilon_{y t x z} \e t \wedge \e x \wedge \e z + \varepsilon_{y t z x} \e t \wedge \e z \wedge \e x + \varepsilon_{y x t z} \e x \wedge \e t \wedge \e z\right.\\
%                 &\qquad \qquad +\left. \varepsilon_{y x z t} \e x \wedge \e z \wedge \e t + \varepsilon_{y z t x} \e z \wedge \e t \wedge \e x + \varepsilon_{y z x t} \e z \wedge \e x \wedge \e t\right)\\
%                 &=  \e t \wedge \e x \wedge \e z
%        \end{align*}
      \item
        Using the definitions of $F, B$ and $E$, the terms can be expanded and after using the defintion of the wedge product in terms of the tensor product the form is clear.
        \begin{align*}
          F &= B_x \e y \wedge \e z + B_y \e z \wedge \e x + B_z \e x \wedge \e y + E \wedge \e t\\
            &= B_x \e y \wedge \e z + B_y \e z \wedge \e x + B_z \e x \wedge \e y + E_x \e x \wedge \e t  + E_y \e y \wedge \e t  + E_z \e z \wedge \e t\\
            &= B_x \e y \otimes \e z - B_x \e z \otimes \e y + B_y \e z \otimes \e x - B_y \e x \otimes \e z + B_z \e x \otimes \e y - B_z \e y \otimes \e x\\
            &\qquad \quad + E_x \e x \otimes \e t - E_x \e t \otimes \e x + E_y \e y \otimes \e t - E_y \e t \otimes \e x + E_z \e z \otimes \e t - E_z \e t \otimes \e z
        \end{align*}
        Which is a $(0,2)$ (hopefully the $2$ is in the correct place, it has two lower indices in component form) Tensor with the same compnents as the Electromagnetic field strength tensor.
      \item
        I was not sure how to answer this question, and my answer is probably a bit long.
        For the first equation $\e F = 0$, I work out the copmonets explicitly.
        \begin{align*}
                          &\e F = 0\\
          \Leftrightarrow &\e (B + E \wedge \e t) = 0\\
          \Leftrightarrow &\e B + \e (E \wedge \e t) = 0\\
          \Leftrightarrow &\e B + (\e E) \wedge \e t - E \wedge \e(\e t) = 0\\
          \Leftrightarrow &\e B + (\e E) \wedge \e t  = 0
        \end{align*}
        Calculating $\e B$ gives:
        \begin{align*}
            & \e(B_x \e y \wedge \e z + B_y \e z \wedge \e x + B_z \e x \wedge \e y) \\
          = & \partial_t B_x  \e t \wedge \e y \wedge \e z + \partial_t B_y  \e t \wedge \e z \wedge \e x + \partial_t B_z  \e t \wedge \e x \wedge \e y\\
            & \qquad + \partial_x B_x \e x \wedge \e y \wedge \e z + \partial_y B_y \e y \wedge \e z \wedge \e x  + \partial_z B_z \e z \wedge \e x \wedge \e y\\
          = & \partial_t B_x  \e t \wedge \e y \wedge \e z + \partial_t B_y  \e t \wedge \e z \wedge \e x + \partial_t B_z  \e t \wedge \e x \wedge \e y\\
            & \qquad + \partial_x B_x \e x \wedge \e y \wedge \e z + \partial_y B_y \e x \wedge \e t \wedge \e z + \partial_z B_z \e x \wedge \e y \wedge \e z\\
          = & \partial_t B_z  \e x \wedge \e y \wedge \e t + \partial_t B_x  \e y \wedge \e z \wedge \e t + \partial_t B_y  \e z \wedge \e x \wedge \e t \\
            & \qquad + (\partial_x B_x + \partial_y B_y + \partial_z B_z )\e x \wedge \e y \wedge \e z
        \end{align*}
        Calculating $\e E$ gives, and ignoring the time parts, which will vanish after the wedge with $\e t$:
        \begin{align*}
          &\e (E_x \e x + E_y \e y + E_z \e z) \\
          = &(\partial_x E_y - \partial_y E_x) \e x \wedge \e y + (\partial_y  E_z - \partial_z  E_y)\e y \wedge \e z + (\partial_z  E_x - \partial_x  E_z)\e z \wedge \e x
        \end{align*}
        From matching the components:
        \begin{align*}
            (\partial_x B_x + \partial_y B_y + \partial_z B_z )\e x \wedge \e y \wedge \e z = 0 \Leftrightarrow \nabla \cdot B = 0
        \end{align*}
        And also:
        \begin{align*}
          &\partial_t B_z  \e x \wedge \e y \wedge \e t + \partial_t B_x  \e y \wedge \e z \wedge \e t + \partial_t B_y  \e z \wedge \e x \wedge \e t \\
          =& - (\partial_x E_y - \partial_y E_x) \e x \wedge \e y \wedge \e t + (\partial_y  E_z - \partial_z  E_y)\e y \wedge \e z \wedge \e t + (\partial_z  E_x - \partial_x  E_z)\e z \wedge \e x \wedge \e t \\
        \end{align*}
        Or equivalently:
        \begin{equation*}
          \frac{\partial B}{\partial t} + \nabla \times E = 0
        \end{equation*}
        Arguably (after looking at this problem a fifth time) doing it this way works and is clearer, but quite terse.
        The notation $\text{op}^{(3)}$ means the 3-D version of the operator.
        \begin{align*}
         &\begin{cases}
            \e ^{(3)} B = 0\\
            \e ^{(3)} E + \frac{\partial B}{\partial t} = 0
          \end{cases}\\
          \Leftrightarrow &
          \begin{cases}
            \e ^{(3)} B = 0\\
           (\e ^{(3)} E )\wedge \e t + \frac{\partial B}{\partial t} \wedge \e t = 0
          \end{cases}\\
          \Leftrightarrow &
            \e ^{(3)} B + (\e ^{(3)} E )\wedge \e t + \frac{\partial B}{\partial t} \wedge \e t = 0\\
          \Leftrightarrow &
            \e ^{(3)} B  + \frac{\partial B}{\partial t} (\wedge \e t) + \e (E \wedge \e t) = 0\\
          \Leftrightarrow &
            \e B   + \e (E \wedge \e t) = 0\\
          \Leftrightarrow &
            \e F = 0
        \end{align*}
        The other equation is:
        \begin{equation*}
          \e * F = \mu_0 J
        \end{equation*}
        I tried to compare this to the formulation given earlier, there is probably a cleaner way to do this, but I think this works.
        For which (I use a formula from \href{https://mathoverflow.net/questions/162366/distributing-the-hodge-map-over-the-wedge-product}{here}, and I hope the divergence notation is clear):
        \begin{align*}
          \e * F &= \e * B + \e * (E \wedge \e t)\\
                 &= \e * B + \e \left(\iota_{\e t^\sharp} * E \right)\\
                 &= \e * B + \e \iota_{-\frac{\partial}{\partial t}} (-E_x \e t \wedge \e y \wedge \e z + E_y \e t \wedge \e x \wedge \e z - E_z\e t \wedge \e x \wedge \e y)\\
                 &= \e * B - \e (-E_x \e y \wedge \e z + E_y \e x \wedge \e z - E_z \e x \wedge \e y)\\
                 &= \e * B + (\Div E ) \e x \wedge \e y \wedge \e z + \frac{\partial E_x}{\partial t} \e t \wedge \e y \wedge \e z - \frac{\partial E_y}{\partial t} \e t \wedge \e x \wedge \e z + \frac{\partial E_z}{\partial t} \e t \wedge \e x \wedge \e y\\
                 &= \e * B - (\Div E ) *\e t + \e t \wedge \left(\frac{\partial E_x}{\partial t}  \e y \wedge \e z + \frac{\partial E_y}{\partial t} \e z \wedge \e x + \frac{\partial E_z}{\partial t} \e x \wedge  \e y\right)\\
                 &= \e * B - (\Div E ) *\e t + \e t \wedge \left[*^{(3)}\frac{\partial E}{\partial t} \right]
%                &= \e * B - (\Div E ) *\e t - \frac{\partial E_x}{\partial t} * \e x - \frac{\partial E_y}{\partial t} * \e y - \frac{\partial E_z}{\partial t} *\e z
        \end{align*}
        Which looks promising and we already have:
        \begin{equation*}
          - (\Div E ) *\e t = - \mu_0 \rho *\e t
        \end{equation*}
        as this is the part of the equation, $\e * F = \mu_0 J$, that has no part in the $\e t$ ``direction"/hyper-volume element.
        For the $\e * B$ part:
        \begin{align*}
          \e * B &= \e (\e t \wedge (*^{(3)} B))\\
                 &= (-1)^1\e t \wedge \e^{(3)}(*^{(3)} B)
        \end{align*}
        Since the time components of the exterior derivative will be killed by the wedge with $\e t$.
        And for the part of $J$ dual to space 1-forms:
        \begin{align*}
          J &= - J_x \e t \wedge \e y \wedge \e z +  J_y \e t \wedge \e x \wedge \e z -  J_z \e t \wedge \e x \wedge \e y\\
            &= - \e t \wedge \left(J_x \e y \wedge \e z +  J_y \e z \wedge \e x +  J_z \e x \wedge \e y\right)\\
            &= - \e t \wedge *^{(3)}J^{(3)}
        \end{align*}
        Thus for the remaining part of $\e * F = \mu_0 J$:
        \begin{align*}
                          & -\e t \wedge \e^{(3)}(*^{(3)} B) + \e t \wedge \left[*^{(3)}\frac{\partial E}{\partial t} \right]= - \mu_0 \e t \wedge *^{(3)}J^{(3)}\\
          \Leftrightarrow & -\e t \wedge \left(\e^{(3)}(*^{(3)} B) + \left[*^{(3)}\frac{\partial E}{\partial t} \right]\right) = \e t \wedge \left(- \mu_0 *^{(3)}J^{(3)}\right)\\
          \Leftrightarrow & \e^{(3)}(*^{(3)} B) - \left[*^{(3)}\frac{\partial E}{\partial t} \right] = \mu_0 *^{(3)}J^{(3)}\\
          \Leftrightarrow & *^{(3)} \e^{(3)}(*^{(3)} B) + \frac{\partial E}{\partial t} = - \mu_0 J^{(3)}
        \end{align*}
        Which was the version of Maxwell's equations given in 2 b). 
        That calculation was pretty much \href{https://en.wikipedia.org/wiki/Abstract\_nonsense}{nonsense} (using the word a bit out of context) and quite terse, I hope it was semi-followable, though.
%        Now using $* (f \alpha) = f* \alpha$:
%        \begin{align*}
%          * F &= *(B_x \e y \wedge \e z + B_y \e z \wedge \e x + B_z \e x \wedge \e y + E_x \e x \wedge \e t  + E_y \e y \wedge \e t  + E_z \e z \wedge \e t)\\
%              &= B_x \e t \wedge \e x + B_y \e y \wedge \e t + B_z \e t \wedge \e z + E_x \e z \wedge \e y  + E_y \e z \wedge \e x  + E_z \e x \wedge \e y)\\
%          \Rightarrow  \\
%
%         \e * F  &= \partial_y B_x \e y \wedge  \e t \wedge \e x + B_y \e y \wedge \e t + B_z \e t \wedge \e z + E_x \e z \wedge \e y  + E_y \e z \wedge \e x  + E_z \e x \wedge \e y)
%        \end{align*}
      \item
        $F$ is closed by Maxwell's equations ($\e F = 0$), and therefore exact (in flat Minkowski space considered here, this is not true in curved space-times) $F = \e A^{(4)}$ for some 1-form $A^{(4)}$.\\

        Let $A^{(4)} = - \varphi \e t + A^{(3)}$, where $\varphi$ and $A^{(3)} = A^{(3)}_i \e x^i$ are the potentials defined earlier which is the usual $4$-potential.
        Then:
        \begin{align*}
          \e^{4} A^{(4)} &= - \e^{3} \varphi \wedge \e t + \e^{(4)} A^{(3)}\\
                         &= - \e^{3} \varphi \wedge \e t + \e^{(3)} A^{(3)} + \frac{\partial A^{(3)_i}}{\partial t} \e t \wedge \e x^i\\
                         &= - \e^{3} \varphi \wedge \e t + B^{(3)} + \frac{\partial A^{(3)_i}}{\partial t} \e t \wedge \e x^i\\
                         &= - \e^{3} \varphi \wedge \e t + B^{(3)} - \frac{\partial A^{(3)_i}}{\partial t} \e x^i \wedge \e t\\
                         &=  B^{(3)} + (- \e^{3} \varphi  - \frac{\partial A^{(3)}}{\partial t} ) \wedge \e t\\
                         &= B + E \wedge \e t\\
                         &= F
        \end{align*}
        That was the order in which I calculated, but as far as answering the question goes reading the answer backwards makes more sense.
    \end{enumerate}
\end{enumerate}
\end{document}
