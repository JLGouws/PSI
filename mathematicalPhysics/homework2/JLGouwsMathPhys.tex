\documentclass[12pt,a4]{article}
\usepackage{physics, amsmath,amsfonts,amsthm,amssymb, mathtools,steinmetz, gensymb, siunitx}	% LOADS USEFUL MATH STUFF
\usepackage{xcolor,graphicx}
\usepackage{caption}
\usepackage{subcaption}
\usepackage[left=45pt, top=20pt, right=45pt, bottom=45pt ,a4paper]{geometry} 				% ADJUSTS PAGE
\usepackage{setspace}
\usepackage{tikz}
\usepackage{pgf,tikz,pgfplots,wrapfig}
\usepackage{mathrsfs}
\usepackage{fancyhdr}
\usepackage{float}
\usepackage{array}
\usepackage{booktabs,multirow}
\usepackage{bm}
\usepackage{tensor}
\usepackage{listings}
 \lstset{
    basicstyle=\ttfamily\small,
    numberstyle=\footnotesize,
    numbers=left,
    backgroundcolor=\color{gray!10},
    frame=single,
    tabsize=2,
    rulecolor=\color{black!30},
    title=\lstname,
    escapeinside={\%*}{*)},
    breaklines=true,
    breakatwhitespace=true,
    framextopmargin=2pt,
    framexbottommargin=2pt,
    inputencoding=utf8,
    extendedchars=true,
    literate={á}{{$\rho$}}1 {ã}{{\~a}}1 {é}{{\'e}}1,
}
\DeclareMathOperator{\sign}{sgn}
\DeclareMathOperator{\Div}{div}
\newcommand{\e}{\mathrm{d}}

\usetikzlibrary{decorations.text, calc}
\pgfplotsset{compat=1.7}

\usetikzlibrary{decorations.pathreplacing,decorations.markings}
\usepgfplotslibrary{fillbetween}

\newcommand{\vect}[1]{\boldsymbol{#1}}

\usepackage{hyperref}

%\usepackage[style= ACM-Reference-Format, maxbibnames=6, minnames=1,maxnames = 1]{biblatex}
%\addbibresource{references.bib}


\hypersetup{pdfborder={0 0 0},colorlinks=true,linkcolor=black,urlcolor=cyan,}
\allowdisplaybreaks

\title{
\textsc{Mathematical Physics Homework 2}
}
\author{\textsc{J L Gouws}
}
\date{\today
\\[1cm]}



\usepackage{graphicx}
\usepackage{array}




\begin{document}
\thispagestyle{empty}

\maketitle

\begin{enumerate}
  \item
    Using the definition of the canonical 1-form and properties of the pullback:
    \begin{align*}
      \mathbf{F} L^* \theta 
        &= \mathbf{F} L^* (p_i \e q^i) \\
        &= \mathbf{F} L^* (p_i) \mathbf{F} L^* (\e q^i) \\
        &= \mathbf{F} L^* (p_i) \e \mathbf{F} L^* q^i \\
        &= (p_i\circ \mathbf{F} L)   \e  (q^i \circ \mathbf{F} L) 
    \end{align*}
    Now using the definition of the $p$ coodinate functions and looking at their pullback:
    \begin{equation*}
      p_i( \mathbf{F} L(\mathbf{q}, \mathbf{v})) = p_i\left(\mathbf{q}, \frac{\partial \hat L}{\partial v_i} \e q^i\right) = \frac{\partial \hat L}{\partial v_i}(\hat{\mathbf{q}}, \hat{\mathbf{v}})
    \end{equation*}
    And:
    \begin{equation*}
      q^i( \mathbf{F} L(\mathbf{q}, \mathbf{v})) = q^i\left(\mathbf{q}, \frac{\partial \hat L}{\partial v_i} \e q^i\right) = q^i
    \end{equation*}
    From this it can be concluded that:
    \begin{align*}
      \mathbf{F} L^* \theta 
        &= \frac{\partial \hat L}{\partial v_i} \e q^i 
    \end{align*}
  \item
    Since $\omega = - \e \theta$ the calculation is straightforward since the exterior derivative commutes with the pullback:
    \begin{align*}
      \omega_L = 
        \mathbf{F} L^* \omega
        &= \mathbf{F} L^*(- \e \theta) \\
        &= - \e \left(\frac{\partial \hat L}{\partial v^i}(\hat q, \hat v) \e q^i \right) \\
        &= - \frac{\partial^2 \hat L}{\partial q^j \partial v^i} \e q^j \wedge \e q^i - \frac{\partial^2 \hat L}{\partial v^j \partial v^i} \e q^j \wedge \e q^i  \\
        &= \frac{\partial^2 \hat L}{\partial q^j \partial v^i} \e q^i \wedge \e q^j + \frac{\partial^2 \hat L}{\partial v^j \partial v^i} \e q^i \wedge \e v^j 
    \end{align*}
  \item
    Now to find the matrix of $\omega_L$:
    \begin{align*}
      \omega_L
        &= \frac{\partial^2 \hat L}{\partial q^j \partial v^i} \e q^i \wedge \e q^j + \frac{\partial^2 \hat L}{\partial v^j \partial v^i} \e q^i \wedge \e v^j \\
        &= \frac{\partial^2 \hat L}{\partial q^j \partial v^i} (\e q^i \otimes \e q^j - \e q^j \otimes \e q^ji + \frac{\partial^2 \hat L}{\partial v^j \partial v^i} (\e q^i \otimes \e v^j - \e v^j \otimes \e q^i) \\
        &= \frac{\partial^2 \hat L}{\partial q^{[j} \partial v^{i]}} \e q^i \otimes \e q^j + \frac{\partial^2 \hat L}{\partial v^j \partial v^i} \e q^i \otimes \e v^j - \frac{\partial^2 \hat L}{\partial v^j \partial v^i}\e v^j \otimes \e q^i \\
    \end{align*}
    Therefore the matrix in block diagonal form is:
    \begin{equation*}
      \omega_{ij} = 
        \left(
          \begin{matrix}
            \frac{\partial^2 \hat L}{\partial q^j \partial v^i} & \frac{\partial^2 \hat L}{\partial v^j \partial v^i} \\
            - \frac{\partial^2 \hat L}{\partial v^j \partial v^i} & 0 
          \end{matrix}
        \right) 
    \end{equation*}
    Which has determinant:
    \begin{equation*}
      \det \omega_{ij} = 
            \left(\frac{\partial^2 \hat L}{\partial v^j \partial v^i} \right)^2
    \end{equation*}
    And since $\omega_L$ is non-degenerate iff the matrix is non-singular, in this case it is non degenerate iff:
    \begin{equation*}
      \left(\frac{\partial^2 \hat L}{\partial v^j \partial v^i} \right)^2 \neq 0 \Leftrightarrow \frac{\partial^2 \hat L}{\partial v^j \partial v^i}  \neq 0
    \end{equation*}
  \item
    This is just putting in the definitions:
    \begin{equation*}
      E(\mathbf{q}, \mathbf{v}) = \frac{\partial \hat L}{\partial v^i} \e q^i (\mathbf{v}) - L(\mathbf{q}, \mathbf{v}) = \frac{\partial \hat L}{\partial v^i} v^i - L(\mathbf{q}, \mathbf{v})
    \end{equation*}
  \item
    First working out the right hand side of the definiton of $X_E$ gives:
    \begin{align*}
      \e E(\mathbf{q}, \mathbf{v}) &= \e \left(\frac{\partial \hat L}{\partial v^i} v^i - L(\mathbf{q}, \mathbf{v})\right)\\
                                   &= \frac{\partial^2 \hat L}{\partial q^i\partial v^j} v^j \e q^i + \frac{\partial^2 \hat L}{\partial v^i\partial v^j} v^j \e v^i + \frac{\partial \hat L}{\partial v^i} \e v^i - \frac{\partial \hat L}{\partial q^i} \e q^i - \frac{\partial \hat L}{\partial v^i} \e v^i\\
                                   &= \left(\frac{\partial^2 \hat L}{\partial q^i\partial v^j} v^j - \frac{\partial \hat L}{\partial q^i} \right)\e q^i + \frac{\partial^2 \hat L}{\partial v^i\partial v^j} v^j \e v^i
    \end{align*}
    And for the right hand side let:
    \begin{equation*}
      X_E = A^i \frac{\partial}{\partial q^i} + B^j \frac{\partial}{\partial v^j}
    \end{equation*}
    So that:
    \begin{align*}
      i_{X_E}\omega  %= A^i \frac{\partial^2 \hat L}{\partial q^j \partial v^i} \e q^i \wedge \e q^j + \frac{\partial^2 \hat L}{\partial v^j \partial v^i} \e q^i \wedge \e v^j
                    &= A^i \frac{\partial^2 \hat L}{\partial q^{[j} \partial v^{i]}} \e q^j  + A^i \frac{\partial^2 \hat L}{\partial v^j \partial v^i} \e v^j - B^j \frac{\partial^2 \hat L}{\partial v^j \partial v^i} \e q^i
    \end{align*}
    It follows immediately that $A^i = v^i$, and relating the terms with the $\e q^i$ differentials gives:
    \begin{align*}
                      & v^j \frac{\partial^2 \hat L}{\partial q^{[i} \partial v^{j]}} - B^j \frac{\partial^2 \hat L}{\partial v^j \partial v^i} = \frac{\partial^2 \hat L}{\partial q^i\partial v^j} v^j - \frac{\partial \hat L}{\partial q^i} \\
      \Leftrightarrow & - v^j \frac{\partial^2 \hat L}{\partial q^j \partial v^i} - B^j \frac{\partial^2 \hat L}{\partial v^j \partial v^i} = - \frac{\partial \hat L}{\partial q^i} \\
      \Leftrightarrow &  B^j  = \left(\frac{\partial^2 \hat L}{\partial v^j \partial v^i}\right)^{-1}\left[\frac{\partial \hat L}{\partial q^i} - v^j \frac{\partial^2 \hat L}{\partial q^j \partial v^i}\right] 
    \end{align*}
    Since the matrix $\frac{\partial^2 \hat L}{\partial v^j \partial v^i}$ is non singular.
    And the integral cuves satisfy:
    \begin{equation*}
      \dot{\gamma} = X_E \Rightarrow 
      \begin{matrix}
        \dot{q}^i \frac{\partial}{\partial q^i} = A^i \frac{\partial}{\partial q^i}\\
        \dot{v}^i \frac{\partial}{\partial v^i} = B^i \frac{\partial}{\partial v^i}
      \end{matrix}
    \end{equation*}
    And hence:
    \begin{equation*}
      \dot{q}^i = v^i
    \end{equation*}
    and the velocity component:
    \begin{align*}
                      & \dot{v}^j  = \left(\frac{\partial^2 \hat L}{\partial v^j \partial v^i}\right)^{-1}\left[\frac{\partial \hat L}{\partial q^i} - v^j \frac{\partial^2 \hat L}{\partial q^j \partial v^i}\right]\\
      \Leftrightarrow & \frac{\partial^2 \hat L}{\partial v^j \partial v^i}\dot{v}^j  = \frac{\partial \hat L}{\partial q^i} - v^j \frac{\partial^2 \hat L}{\partial q^j \partial v^i}\\
      \Leftrightarrow & \dot{v}^j \frac{\partial}{ \partial v^j}\frac{\partial \hat L}{ \partial v^i} + \dot q^j \frac{\partial}{\partial q^j}\frac{\partial \hat L}{\partial v^i} =  \frac{\partial \hat L}{\partial q^i}\\
      \Leftrightarrow & \frac{\partial}{ \partial t}\left[\frac{\partial \hat L}{ \partial v^i}(q(t), v(t)) \right] =  \frac{\partial \hat L}{\partial q^i}\\
    \end{align*}

\end{enumerate}
\end{document}
