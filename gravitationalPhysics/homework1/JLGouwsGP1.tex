\documentclass[12pt,a4]{article}
\usepackage{physics, amsmath,amsfonts,amsthm,amssymb, mathtools,steinmetz, gensymb, siunitx}	% LOADS USEFUL MATH STUFF
\usepackage{xcolor,graphicx}
\usepackage{caption}
\usepackage{subcaption}
\usepackage[left=45pt, top=20pt, right=45pt, bottom=45pt ,a4paper]{geometry} 				% ADJUSTS PAGE
\usepackage{setspace}
\usepackage{tikz}
\usepackage{pgf,tikz,pgfplots,wrapfig}
\usepackage{mathrsfs}
\usepackage{fancyhdr}
\usepackage{float}
\usepackage{array}
\usepackage{booktabs,multirow}
\usepackage{bm}
\usepackage{tensor}
\usepackage{listings}
 \lstset{
    basicstyle=\ttfamily\small,
    numberstyle=\footnotesize,
    numbers=left,
    backgroundcolor=\color{gray!10},
    frame=single,
    tabsize=2,
    rulecolor=\color{black!30},
    title=\lstname,
    escapeinside={\%*}{*)},
    breaklines=true,
    breakatwhitespace=true,
    framextopmargin=2pt,
    framexbottommargin=2pt,
    inputencoding=utf8,
    extendedchars=true,
    literate={á}{{$\rho$}}1 {ã}{{\~a}}1 {é}{{\'e}}1,
}
\DeclareMathOperator{\sign}{sgn}

\usetikzlibrary{decorations.text, calc}
\pgfplotsset{compat=1.7}

\usetikzlibrary{decorations.pathreplacing,decorations.markings}
\usepgfplotslibrary{fillbetween}

\newcommand{\vect}[1]{\boldsymbol{#1}}

\usepackage{hyperref}

%\usepackage[style= ACM-Reference-Format, maxbibnames=6, minnames=1,maxnames = 1]{biblatex}
%\addbibresource{references.bib}


\hypersetup{pdfborder={0 0 0},colorlinks=true,linkcolor=black,urlcolor=cyan,}
\allowdisplaybreaks
%\hypersetup{
%
%    colorlinks=true,
%
%    linkcolor=blue,
%
%    filecolor=magenta,      
%
%    urlcolor=cyan,
%
%    pdftitle={An Example},
%
%    pdfpagemode=FullScreen,
%
%    }
%}

\title{
\textsc{Gravitational Physics Homework 1}
}
\author{\textsc{J L Gouws}
}
\date{\today
\\[1cm]}



\usepackage{graphicx}
\usepackage{array}




\begin{document}
\thispagestyle{empty}

\maketitle

\begin{enumerate}
  \item
    The orthonormal basis one forms that make the metric into the Minkoski metric are:
    \begin{align*}
      \underline{\omega}^0 &= dt\\
      \underline{\omega}^1 &= \frac{a(t)}{\sqrt{1- k r^2}} dr\\
      \underline{\omega}^2 &= a(t) r d\theta\\
      \underline{\omega}^3 &= a(t) r \sin \theta d \phi
    \end{align*}
  \item
%    \begin{align*}
%      d \underline{\omega}   & = \frac{a'(t)}{\sqrt{1 - k r^2}} dt \wedge dr + a'(t) r d t \wedge d \theta + a(t) dr \wedge d \theta + a'(t) r \sin \theta dt \wedge d \phi\\
%                              & \qquad  + a(t) r \cos \theta d \theta \wedge d \phi + a(t) \sin \theta d r \wedge d \phi\\
%                             & = \frac{a'(t)}{a(t)} \underline{\omega}^0 \wedge \underline{\omega}^1 + \frac{a'(t)}{a(t)} \underline{\omega}^0 \wedge \underline{\omega}^2 + \frac{\sqrt{1 - k r^2}}{a(t)r} \underline{\omega}^1 \wedge \underline{\omega}^2 + \frac{a'(t)}{a(t)} \underline{\omega}^0 \wedge \underline{\omega}^3\\
%                              & \qquad  + \frac{\cot{\theta}}{a(t) r} \underline{\omega}^2 \wedge \underline{\omega}^3 + \frac{\sqrt{1 - k r^2}}{a(t) r} \underline{\omega}^{1} \wedge \underline{\omega}^3
%    \end{align*}
%    We have the following components:
    First, using Cartan's equation requires the exterior derivative of the $\underline{\omega}^a$s:
    \begin{align*}
      &d \underline{\omega} ^ 0 = 0\\
      &d \underline{\omega} ^ 1 = \frac{a'(t)}{a(t)} \underline{\omega}^0 \wedge \underline{\omega}^1\\
      &d \underline{\omega} ^ 2 = \frac{a'(t)}{a(t)} \underline{\omega}^0 \wedge \underline{\omega}^2 + \frac{\sqrt{1 - k r^2}}{a(t)r} \underline{\omega}^1 \wedge \underline{\omega}^2\\
      &d \underline{\omega} ^ 3 = \frac{a'(t)}{a(t)} \underline{\omega}^0 \wedge \underline{\omega}^3 + \frac{\cot{\theta}}{a(t) r} \underline{\omega}^2 \wedge \underline{\omega}^3 + \frac{\sqrt{1 - k r^2}}{a(t) r} \underline{\omega}^{1} \wedge \underline{\omega}^3
    \end{align*}
    Now, using these in Cartan's torsion equation:
    \begin{equation*}
      \underline{T}^a = d \underline{\omega}^a + \tensor{\underline{\theta}}{^a_b} \wedge \underline{\omega}^b
    \end{equation*}
    And setting the torsion to zero yields:
%    \begin{equation*}
%      0 = d \underline{\omega}^0 = - \tensor{\underline{\theta}}{^0_b} \wedge \underline{\omega}^b \Rightarrow \tensor{\underline{\theta}}{^0_b} = \tensor{\underline{\theta}}{^b_0} = 0
%    \end{equation*}
    \begin{equation*}
      d \underline{\omega} ^ 1 = \frac{a'(t)}{a(t)} \underline{\omega}^0 \wedge \underline{\omega}^1 = - \tensor{\underline{\theta}}{^1_b} \wedge \underline{\omega}^b \Rightarrow  \tensor{\underline{\theta}}{^1_0} = \frac{a'(t)}{a(t)} \underline{\omega}^1 = \frac{a'(t)}{\sqrt{1 - k r^2}} dr
    \end{equation*}
    We can also determine from this that $\tensor{\underline{\theta}}{^1_1}$ has no component in $\underline{\omega}^1$, $\tensor{\underline{\theta}}{^1_2}$ has no component in $\underline{\omega}^2$ and $\tensor{\underline{\theta}}{^1_3}$ has no component in $\underline{\omega}^3$.
    And also requiring the connection one-forms be antisymmetric:
    \begin{equation*}
      \tensor{\underline{\theta}}{^0_1} = \tensor{\underline{\theta}}{_0_1} = -\tensor{\underline{\theta}}{_1_0} = \tensor{\underline{\theta}}{^1_0}
    \end{equation*}
    For the next orthonormal basis 1-form:
    \begin{align*}
      d \underline{\omega} ^ 2 &= \frac{a'(t)}{a(t)} \underline{\omega}^0 \wedge \underline{\omega}^2 + \frac{\sqrt{1 - k r^2}}{a(t)r} \underline{\omega}^1 \wedge \underline{\omega}^2\\
                               &= - \tensor{\underline{\theta}}{^2_b} \wedge \underline{\omega}^b\\
                               &= - \tensor{\underline{\theta}}{^2_0} \wedge \underline{\omega}^0 -\tensor{\underline{\theta}}{^2_1} \wedge \underline{\omega}^1 - \tensor{\underline{\theta}}{^2_2} \wedge \underline{\omega}^2 - \tensor{\underline{\theta}}{^2_3} \wedge \underline{\omega}^3 
    \end{align*}
    Therefore:
    \begin{align*}
      \tensor{\underline{\theta}}{^2_0} = \frac{a'(t)}{a(t)} \underline{\omega}^2 \qquad &\text{ and } \qquad \tensor{\underline{\theta}}{^2_1} = \frac{\sqrt{1 - k r^2}}{a(t)r} \underline{\omega}^2\\
      \tensor{\underline{\theta}}{^0_2} = \frac{a'(t)}{a(t)} \underline{\omega}^2 \qquad &\text{ and } \qquad \tensor{\underline{\theta}}{^1_2} = -\frac{\sqrt{1 - k r^2}}{a(t)r} \underline{\omega}^2\\
    \end{align*}
    And similarly:
    \begin{gather*}
      \tensor{\underline{\theta}}{^3_0} = \frac{a'(t)}{a(t)} \underline{\omega}^3 \quad \text{ and } \quad \tensor{\underline{\theta}}{^3_1} = \frac{\sqrt{1 - k r^2}}{a(t) r}  \underline{\omega}^3 \quad \text{ and } \quad \tensor{\underline{\theta}}{^3_2} = \frac{\cot \theta }{a(t)r} \underline{\omega}^3 \\
      \tensor{\underline{\theta}}{^0_3} = \frac{a'(t)}{a(t)} \underline{\omega}^3 \quad \text{ and } \quad \tensor{\underline{\theta}}{^1_3} = -\frac{\sqrt{1 - k r^2}}{a(t) r} \underline{\omega}^3 \quad \text{ and } \quad \tensor{\underline{\theta}}{^2_3} = -\frac{\cot \theta }{a(t)r} \underline{\omega}^3 
    \end{gather*}
    All other connection 1-form components are zero, because otherwise there would be more terms in the exterior derivatives of the orthonormal basis 1-froms.
  \item
    The above connection 1-forms can be used to determine the curvature 2-forms.
    First:
    \begin{align*}
      \tensor{\underline{R}}{^1_0} &= d \tensor{\underline{\theta}}{^1_0} + \tensor{\underline{\theta}}{^1_c} \wedge \tensor{\underline{\theta}}{^c_0}\\
                                   &= \frac{a''(t)}{\sqrt{1 - k r^2}} dt \wedge dr + \tensor{\underline{\theta}}{^1_c} \wedge \tensor{\underline{\theta}}{^c_0}\\
                                   &= \frac{a''(t)}{\sqrt{1 - k r^2}} dt \wedge dr + \tensor{\underline{\theta}}{^1_c} \wedge \tensor{\underline{\theta}}{^c_0}\\
                                   &= \frac{a''(t)}{\sqrt{1 - k r^2}} dt \wedge dr\\
                                   &= \frac{a''(t)}{a(t)} \underline{\omega}^0 \wedge \underline{\omega}^1
    \end{align*}
    Next:
    \begin{align*}
      \tensor{\underline{R}}{^2_0} &= d \tensor{\underline{\theta}}{^2_0} + \tensor{\underline{\theta}}{^2_c} \wedge \tensor{\underline{\theta}}{^c_0}\\
                                   &= d \left(a'(t) r d \theta \right) + \frac{\sqrt{1 - k r^2}}{a(t)r} \underline{\omega}^2 \wedge \frac{a'(t)}{a(t)} \underline{\omega}^1 \\
                                   &= a''(t) r d t \wedge d \theta + a'(t) d r \wedge d \theta + a'(t) d \theta \wedge d r\\
                                   &= a''(t) r d t \wedge d \theta \\
                                   &= \frac{a''(t)}{a(t)} \underline{\omega}^0 \wedge \underline{\omega}^2 
    \end{align*}
    And:
    \begin{align*}
      \tensor{\underline{R}}{^3_0} &= d \tensor{\underline{\theta}}{^3_0} + \tensor{\underline{\theta}}{^3_c} \wedge \tensor{\underline{\theta}}{^c_0}\\
                                   &= d \left(a'(t) r \sin \theta d \phi\right) + \frac{\sqrt{1 - k r^2}}{a(t) r}  \underline{\omega}^3 \wedge \frac{a'(t)}{a(t)} \underline{\omega}^1 + \frac{\cot \theta }{a(t)r} \underline{\omega}^3 \wedge \frac{a'(t)}{a(t)} \underline{\omega}^2\\
                                   &= a''(t) r \sin \theta d t \wedge d \phi + a'(t) \sin \theta dr \wedge d\phi + a'(t) r \cos \theta d \theta \wedge d \phi\\
                                   & \qquad + a'(t) r \sin \theta d\phi \wedge  dr + a'(t)r\cos \theta d \phi \wedge d\theta \\
                                   &= a''(t) r \sin \theta d t \wedge d \phi \\
                                   &= \frac{a''(t)}{a(t)}  \underline{\omega}^0 \wedge \underline{\omega}^3
    \end{align*}
    The other 2-forms follow from similar calculations
%    \begin{align*}
%      \tensor{\underline{R}}{^1_2} &= d \tensor{\underline{\theta}}{^1_2} + \tensor{\underline{\theta}}{^1_c} \wedge \tensor{\underline{\theta}}{^c_2}\\
%                                   &= d\left(-\sqrt{1 - k r^2} d\theta \right) + \frac{a'(t)}{\sqrt{1 - k r^2}} dr \wedge a'(t)r d\theta \\
%                                   &= \frac{r(a'^2(t) - k)}{\sqrt{1 - k r^2}} d r \wedge d\theta
%    \end{align*}

    \begin{align*}
      \tensor{\underline{R}}{^2_1} &= d \tensor{\underline{\theta}}{^2_1} + \tensor{\underline{\theta}}{^2_c} \wedge \tensor{\underline{\theta}}{^c_1}\\
                                   &= d\left(\sqrt{1 - k r^2} d\theta \right) + a'(t)r d\theta \wedge \frac{a'(t)}{\sqrt{1 - k r^2}} dr \\
                                   &= - \frac{r(k + a'^2(t))}{\sqrt{1 - k r^2}} d r \wedge d\theta\\
                                   &= -\frac{k + a'^2(t)}{a^2(t)} \underline{\omega}^1 \wedge \underline{\omega}^2
    \end{align*}
    And the next 2-form with $1$ in the lower index:
    \begin{align*}
      \tensor{\underline{R}}{^3_1} &= d \tensor{\underline{\theta}}{^3_1} + \tensor{\underline{\theta}}{^3_c} \wedge \tensor{\underline{\theta}}{^c_1}\\
                                   &= d\left(\sin \theta \sqrt{1 - k r^2}  d\phi\right) + \frac{a'(t)}{a(t)} \underline{\omega}^3 \wedge \frac{a'(t)}{a(t)} \underline{\omega}^1 + \frac{\cot \theta }{a(t)r} \underline{\omega}^3 \wedge \frac{\sqrt{1 - k r^2}}{a(t)r} \underline{\omega}^2\\
                                   &= - \frac{k r\sin \theta }{\sqrt{1 - k r^2}} d r \wedge d \phi + \cos \theta \sqrt{1 - k r^2} d\theta \wedge d \phi + \frac{a'(t)^2 r \sin \theta }{\sqrt{1 - k r^2}}  d \phi \wedge  d r + \sqrt{1 - k r^2}  \cos \theta d\phi  \wedge d\theta\\
%                                   &= \frac{r\sin \theta (k - a'^2(t))}{\sqrt{1 - k r^2}} d r \wedge d \phi \\
                                   &= - \frac{r\sin \theta (k + a'^2(t))}{\sqrt{1 - k r^2}} d r \wedge d \phi \\
                                   &= -\frac{(k + a'^2(t))}{a^2(t)} \underline{\omega}^1 \wedge \underline{\omega}^3
    \end{align*}
    And finally:
    \begin{align*}
      \tensor{\underline{R}}{^3_2} &= d \tensor{\underline{\theta}}{^3_2} + \tensor{\underline{\theta}}{^3_c} \wedge \tensor{\underline{\theta}}{^c_2}\\
                                   &= d\left(\cos \theta d\phi \right) + \frac{a'(t)}{a(t)} \underline{\omega}^3 \wedge \frac{a'(t)}{a(t)} \underline{\omega}^2 - \frac{\sqrt{1 - k r^2}}{a(t) r} \underline{\omega}^3 \wedge \frac{\sqrt{1 - k r^2}}{a(t)r} \underline{\omega}^2 \\
                                   &= -\sin \theta d \theta \wedge d\phi + a'^2(t) r^2 \sin \theta d \phi \wedge d\theta - \left(1 - k r^2\right)\sin \theta d\phi \wedge d\theta \\
                                   &=  r^2 \sin \theta\left( k + a'^2(t) \right) d\theta \wedge d \phi  \\
                                   &=  \frac{\left( k + a'^2(t) \right)}{a^2(t)} \underline{\omega}^2 \wedge \underline{\omega}^3
    \end{align*}
%    \begin{align*}
%      \tensor{\underline{R}}{^2_3} &= d \tensor{\underline{\theta}}{^2_3} + \tensor{\underline{\theta}}{^2_c} \wedge \tensor{\underline{\theta}}{^c_3}\\
%                                   &= d\left(-\cos \theta d\phi \right) + \frac{a'(t)}{a(t)} \underline{\omega}^2 \wedge \frac{a'(t)}{a(t)} \underline{\omega}^3  - \frac{\sqrt{1 - k r^2}}{a(t) r} \underline{\omega}^3 \wedge \frac{\sqrt{1 - k r^2}}{a(t)r} \underline{\omega}^2 \\
%                                   &= \sin \theta d \theta \wedge d\phi + a'^2(t) r^2 \sin \theta d \phi \wedge d\theta - \left(1 - k r^2\right)\sin \theta d\phi \wedge d\theta \\
%                                   &=  r^2 \sin \theta\left( k + a'^2(t) \right) d\theta \wedge d \phi  
%    \end{align*}
    These are all the components of the curvature 2-form, all other components are related by symmetries, or they are zero.
  \item
    From the above, the Riemann tensor components in the orthonomal basis are evident, note that the number indices here mean orthonormal basis and coordinate symbol indices mean coordinate basis components.
    \begin{align*}
      \tensor{R}{^0_1_0_1} &= \frac{a''(t)}{a(t)}\\
      \tensor{R}{^0_2_0_2} &= \frac{a''(t)}{a(t)}\\
      \tensor{R}{^0_3_0_3} &= \frac{a''(t)}{a(t)}\\
      \tensor{R}{^1_2_1_2} &= \frac{k + a'^2(t)}{a^2(t)} \\
      \tensor{R}{^1_3_1_3} &= \frac{k + a'^2(t)}{a^2(t)}\\
      \tensor{R}{^2_3_2_3} &= \frac{ k + a'^2(t) }{a^2(t)}
    \end{align*}
    The other components are related by symmetry if they have the same indices in a different permutation, otherwise the components are zero.
    And in the coordintate basis:
    \begin{align*}
      &\tensor{R}{^t_r_t_r} = \frac{a''(t)a(t)}{1 - k r^2}\\
      &\tensor{R}{^t_\theta_t_\theta} = a''(t)a(t) r^2\\
      &\tensor{R}{^t_\phi_t_\phi} = a''(t)a(t)r^2 \sin^2 \theta \\
      &\tensor{R}{^r_\theta_r_\theta} =  (k - a'^2(t))r^2 \\
      &\tensor{R}{^r_\phi_r_\phi} =      (k + a'^2(t)) r^2 \sin^2 \theta\\
      &\tensor{R}{^\theta_\phi_\theta_\phi} =  \left( k + a'^2(t) \right) r^2 \sin ^2 \theta 
    \end{align*}
    The other components of the orthonormal basis Riemann tensor are:
    \begin{align*}
      &\tensor{R}{^1_0_1_0} = -\frac{a''(t)}{a(t)} \Rightarrow \tensor{R}{^r_t_r_t} = -\frac{a''(t)}{a(t)} \\
      &\tensor{R}{^2_0_2_0} = -\frac{a''(t)}{a(t)} \Rightarrow \tensor{R}{^\theta_t_\theta_t} = -\frac{a''(t)}{a(t)} \\
      &\tensor{R}{^3_0_3_0} = -\frac{a''(t)}{a(t)} \Rightarrow \tensor{R}{^\phi_t_\phi_t} = -\frac{a''(t)}{a(t)}\\
      &\tensor{R}{^2_1_2_1} = - \frac{(k - a'^2(t))}{a^2(t)} \Rightarrow \tensor{R}{^\theta_r_\theta_r} = \frac{k - a'^2(t)}{1 - kr^2}\\
      &\tensor{R}{^3_1_3_1} = - \frac{(k - a'^2(t))}{a^2(t)} \Rightarrow \tensor{R}{^\phi_r_\phi_r} = \frac{k - a'^2(t)}{1 - kr^2}\\
      &\tensor{R}{^3_2_3_2} = - \frac{\left( k + a'^2(t) \right)}{a^2(t)} \Rightarrow \tensor{R}{^\phi_\theta_\phi_\theta} = \left( k + a'^2(t) \right) r^2
    \end{align*}
    Therefore the components of the Ricci tensor in the coordinate basis are:
    \begin{align*}
      R_{t t} &= \tensor{R}{^r_t _r _t} + \tensor{R}{^\theta_t_\theta_t} + \tensor{R}{^\phi_t_\phi_t}= - 3 \frac{a''(t)}{a(t)}\\
      R_{r r} &= \frac{a''(t)a(t) + 2a'^2(t) + 2k}{1 - k r^2}\\
      R_{\theta \theta} &= -r^2(a''(t) a(t) + 2a'^2(t) + 2 k)\\
      R_{\phi\phi} &= r^2 (a''(t) a(t) + 2a'^2(t) + 2 k) \sin\theta
    \end{align*}
    The Ricci tensor has no cross terms since the other components would require repetition of two lower indices in the Riemann tensor and all these components are zero.
\end{enumerate}
\end{document}
