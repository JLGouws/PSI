\documentclass[12pt,a4]{article}
\usepackage{physics, amsmath,amsfonts,amsthm,amssymb, mathtools,steinmetz, gensymb, siunitx}	% LOADS USEFUL MATH STUFF
\usepackage{xcolor,graphicx}
\usepackage{caption}
\usepackage{subcaption}
\usepackage[left=45pt, top=20pt, right=45pt, bottom=45pt ,a4paper]{geometry} 				% ADJUSTS PAGE
\usepackage{setspace}
\usepackage{tikz}
\usepackage{pgf,tikz,pgfplots,wrapfig}
\usepackage{mathrsfs}
\usepackage{fancyhdr}
\usepackage{float}
\usepackage{array}
\usepackage{booktabs,multirow}
\usepackage{bm}
\usepackage{tensor}
\usepackage{listings}
 \lstset{
    basicstyle=\ttfamily\small,
    numberstyle=\footnotesize,
    numbers=left,
    backgroundcolor=\color{gray!10},
    frame=single,
    tabsize=2,
    rulecolor=\color{black!30},
    title=\lstname,
    escapeinside={\%*}{*)},
    breaklines=true,
    breakatwhitespace=true,
    framextopmargin=2pt,
    framexbottommargin=2pt,
    inputencoding=utf8,
    extendedchars=true,
    literate={á}{{$\rho$}}1 {ã}{{\~a}}1 {é}{{\'e}}1,
}
\DeclareMathOperator{\sign}{sgn}

\usetikzlibrary{decorations.text, calc}
\pgfplotsset{compat=1.7}

\usetikzlibrary{decorations.pathreplacing,decorations.markings}
\usepgfplotslibrary{fillbetween}

\newcommand{\vect}[1]{\boldsymbol{#1}}
\newcommand{\e}{\mathrm{d}}

\usepackage{hyperref}

%\usepackage[style= ACM-Reference-Format, maxbibnames=6, minnames=1,maxnames = 1]{biblatex}
%\addbibresource{references.bib}


\hypersetup{pdfborder={0 0 0},colorlinks=true,linkcolor=black,urlcolor=cyan,}
\allowdisplaybreaks
%\hypersetup{
%
%    colorlinks=true,
%
%    linkcolor=blue,
%
%    filecolor=magenta,      
%
%    urlcolor=cyan,
%
%    pdftitle={An Example},
%
%    pdfpagemode=FullScreen,
%
%    }
%}

\title{
\textsc{Gravitational Physics Homework 2}
}
\author{\textsc{J L Gouws}
}
\date{\today
\\[1cm]}



\usepackage{graphicx}
\usepackage{array}




\begin{document}
\thispagestyle{empty}

\maketitle

\begin{enumerate}
  \item
    \begin{enumerate}
      \item
        Let $\omega = \e \lambda$ be exact, then:
        \begin{equation*}
          \e \omega = \e( \e \lambda )
        \end{equation*}
        For $n = 1$, the only case that has to be checked is for 1-forms because there are no "$-1$"-forms to make "0"-forms exact.
        All 1-forms are necesarilly exact: $\e (\alpha \e x) = \frac{\partial \alpha}{\partial x} \e x \wedge \e x \equiv 0$.
        The question now is for every one-form $\alpha(x) \e x$, is there a function $\beta(x)$ such that $\alpha = \frac{\partial \beta}{\partial x}$.
        Such a function is:
        \begin{equation*}
          \beta(x) = \int_0^x \alpha(x') \e x'
        \end{equation*}
        Since $\alpha$ is smooth and hence integrable, this is indeed a smooth $0$ form.

        This argument breaks down on the circle, because forms have to be smooth (and continuous) on the whole circle.
        A counter example of a closed, but non exact 1-form on the circle is $\e \theta$, where $\theta$ is the angle along the circle.
        Here if one tries to construct the 0-form $\beta(\theta) = \theta + C$ it satisfies $\frac{\partial \beta}{\theta} \e \theta = \e \theta$, but is not actually a 0-form on the circle $S^1$.
        The points $\theta = 0$ and $\theta = 2 \pi$ are the same points on the circle, but $\beta(0) = C$ and $\beta(2 \pi) = 2 \pi + C$ and so $\beta$ is not single valued and hence a well defined function on $S^1$.
      \item
        \begin{align*}
          g &= \frac{1}{4 \pi } \int_{S^2} F^{(2)} \\
            &= \frac{1}{4 \pi } \int_{S^2} \e A^{(1)} \\
            &= \frac{1}{4 \pi } \int_{\partial S^2} A^{(1)} \\
            &= 0 
        \end{align*}
        Since the sphere has not boundary.
      \item
        \begin{align*}
          \e F^{(4)} &= \e (Q \sin \theta \e \theta \wedge \e \phi)\\
                     &= Q \cos \theta \e \theta \wedge \e \theta \wedge \e \phi\\
                     &= 0 
        \end{align*}
        And for the other term, note the metric is $\e t^2 - \e r^2 - r^2 \e \theta - r^2 \sin^2 \theta \e \phi$
        \begin{align*}
          \e * F^{(4)} &= \e (* (Q \sin \theta \e \theta \wedge \e \phi))\\
                       &= \e \left(\frac{1}{(4 - 2)!}\frac{1}{2} \left(\tensor{\epsilon}{^\theta ^\phi_\mu_\nu} Q \sin \theta \e x^\mu \wedge \e x^\nu + \tensor{\epsilon}{^\phi^\theta _\mu_\nu} Q \sin \theta \e x^\mu \wedge \e x^\nu\right)\right)\\
                       &= \frac{1}{4}\e \left(Q \sin \theta \left(\tensor{\epsilon}{^\theta ^\phi_\mu_\nu}  \e x^\mu \wedge \e x^\nu + \tensor{\epsilon}{^\phi^\theta _\mu_\nu} \e x^\mu \wedge \e x^\nu\right)\right)\\
                       &= \frac{1}{4}\e \left(Q \sin \theta \left( \tensor{\epsilon}{^\theta ^\phi_t_r} \e t \wedge \e r + \tensor{\epsilon}{^\theta ^\phi_r_t} \e r \wedge \e t + \tensor{\epsilon}{^\phi^\theta _t_r} \e t \wedge \e r + \tensor{\epsilon}{^\phi^\theta _r_t} \e r \wedge \e t\right)\right)
    %                   &= \e \left(\tensor{\epsilon}{^\theta ^\phi_\mu_\nu} Q \sin \theta \e x^\mu \wedge \e x^\nu \right)\\
    %                   &= \e \left(\tensor{\epsilon}{^\theta ^\phi_t_r} Q \sin \theta \e t \wedge \e r - \tensor{\epsilon}{^\theta ^\phi_r_t} Q \sin \theta \e r \wedge \e t\right)\\
    %                   &= Q \cos \theta \e \theta \wedge \e \theta \wedge \e \phi\\
    %                   &= 0 
        \end{align*}
        Now $\sqrt{g} = r^2 \sin \theta$ and $g^{\theta \theta} = -\frac{1}{r^2}$ and $g^{\phi\phi} = -\frac{1}{r^2 \sin^2\theta }$
        Therefore $\tensor{\epsilon}{^\theta ^\phi_t_r} = \sqrt{g} g^{\theta\theta} g^{\phi\phi} \sigma_{\theta\phi t r}$ and:
        \begin{align*}
          \e * F^{(4)} &= \e \left(Q \frac{Q}{r^2} \e t \wedge \e r\right)\\
                       &= 0 
        \end{align*}
      \item
        Here:
        \begin{equation*}
          \e \theta = \frac{\partial \theta }{ \partial x} \e x
        \end{equation*}
        \begin{equation*}
          x =  r \sin \theta \cos \phi \Rightarrow \e x = \sin \theta \cos \phi \e r + r \cos \theta \cos \phi \e \theta -  r \sin \theta \sin \phi \e \phi
        \end{equation*}
      \item
        The \href{https://en.wikipedia.org/wiki/Spherical_coordinate_system}{inverse Jacobian matrix} is:
        \begin{equation*}
          J = \frac{\partial(r, \theta, \phi)}{\partial (x, y, z)} =
          \left(
          \begin{matrix}
            \dfrac{x}{r}&\dfrac{y}{r}&\dfrac{z}{r}\\
            \dfrac{xz}{r^2\sqrt{x^2+y^2}}&\dfrac{yz}{r^2\sqrt{x^2+y^2}}&\dfrac{-\left(x^2 + y^2\right)}{r^2\sqrt{x^2+y^2}}\\
            \dfrac{-y}{x^2+y^2}&\dfrac{x}{x^2+y^2}&0
          \end{matrix}
          \right)
        \end{equation*}
        Therefore:
        \begin{equation*}
          \e \theta = \dfrac{xz}{r^2\sqrt{x^2+y^2}} \e x + \dfrac{yz}{r^2\sqrt{x^2+y^2}} \e y + \dfrac{-\left(x^2 + y^2\right)}{r^2\sqrt{x^2+y^2}} \e z
        \end{equation*}
        and:
        \begin{equation*}
          \e \phi = \dfrac{-y}{x^2+y^2} \e x + \dfrac{x}{x^2+y^2} \e y
        \end{equation*}
        And obviously:
        \begin{equation*}
          \sin \theta = \frac{\sqrt{x^2 + y^2}}{r}
        \end{equation*}
        From which, calculating $F$ is straight forward:
        \begin{align*}
          F &= Q \frac{\sqrt{x^2 + y^2}}{r} \left(\dfrac{xz}{r^2\sqrt{x^2+y^2}} \e x + \dfrac{yz}{r^2\sqrt{x^2+y^2}} \e y + \dfrac{-\left(x^2 + y^2\right)}{r^2\sqrt{x^2+y^2}} \e z\right) \wedge \left(\dfrac{-y}{x^2+y^2} \e x + \dfrac{x}{x^2+y^2} \e y\right)\\
            &= Q \left(\dfrac{z}{r^3} \e x \wedge \e y + \dfrac{y}{r^3} \e z \wedge \e x + \dfrac{x}{r^3} \e y \wedge \e z\right)
        \end{align*}
      \item
        For the integral of the 2-form on the sphere:
        \begin{align*}
          g =
          \frac{1}{4 \pi}\int_{S^2} F^{(2)} =
          \frac{1}{4 \pi}\int_{S^2} Q \sin \theta \e \theta \wedge \e \phi =
          \frac{1}{4 \pi}\int_{S^2} Q \e \Omega =
          \frac{4 \pi}{4 \pi} Q  = Q
        \end{align*}
        \begin{enumerate}
          \item
            If $F$ were globally exact, the integral would vanish, which it does not, contradicting $F$ begin globally exact.
          \item
            ...
        \end{enumerate}
      \item
        An attempt at constructing such a potential is:
        \begin{equation*}
          A = Q \cos \theta \e \phi
        \end{equation*}
        But this potential is not well defined on the $\theta = 0$ or $z$ axis, because $\e \phi$ does not make sense there.
        Thus the ponetial must vanish at the poles $\theta = 0$ and $\theta = \pi$.
        \begin{equation*}
          A_+ = Q (1 - \cos \theta) \e \phi
        \end{equation*}
        A 1-form such that $F^{(4)} = \e A^{(2)}$ is:
        \begin{equation*}
          A_+ = Q (1 - \cos \theta) \e \phi
        \end{equation*}
        and:
        \begin{equation*}
          A_- = -Q (1 + \cos \theta) \e \phi
        \end{equation*}
      \item
        \begin{align*}
                      & -Q (1 + \cos \theta) \e \phi =  Q (1 - \cos \theta) \e \phi + \frac{1}{ie} \gamma^{-1} \e \gamma\\
          \Rightarrow &  \gamma^{-1} \e \gamma = - 2Q i e \e \phi
        \end{align*}
        Say $\gamma = e^{i g \phi}$ then:
        \begin{align*}
                      &  \gamma^{-1} \e \gamma = - 2Q i e \e \phi\\
          \Rightarrow &  i  g \e \phi= - 2Q i e \e \phi\\
          \Rightarrow &  i( g + 2 Q e)  \e \phi = 0\\
          \Rightarrow &  g + 2 Q e = 0\\
          \Rightarrow &  Q = - \frac{g}{2 e}
        \end{align*}

        
    \end{enumerate}
  \item
    \begin{enumerate}
      \item
        When $n = 0$, $A_\sigma = 0$ and $f = \frac{r^2 - 2 m r}{r^2} = 1 - \frac{2 m}{r}$ so that:
        \begin{align*}
          \e s^2 = \left(1 - \frac{2 m}{r}\right)\e t \otimes \e t - \left(1 - \frac{2 m}{r}\right)^{-1}\e r \otimes \e r - r^2 (\e \theta \otimes \e \theta + \sin^2 \theta \e \phi \otimes \e \phi)
        \end{align*}
        Which is the Schartzschild solution for a body with mass $m$.
      \item
        With $\sigma = \pm 1$, the potential is $A_\sigma = n(\cos \theta \pm 1) \e \phi$
        And the time component of the metric becomes (the radial and angular part has not $\sigma$ dependence, and so its regulartiy will not be affected by the value of 
        $\sigma$):
        \begin{align*}
          f(\e t + 2 A_\sigma)^2 &= f(\e t + 2 n (\cos\theta \pm 1) \e \phi)^2\\
                                 &= f(\e t \otimes \e t + 2 n (\cos\theta \pm 1) \e t \otimes \e \phi + 2 n (\cos\theta \pm \sigma) \e \phi \otimes \e t\\
                                 &\qquad + 4 n^2 (\cos\theta \pm 1)^2 \e \phi \otimes \e \phi)
        \end{align*}
        The problem here is that again, $\e \phi$ is not well defined at $\theta = 0$, the north pole, and $\theta = \pi$, the south pole.
    \end{enumerate}
\end{enumerate}
\end{document}
