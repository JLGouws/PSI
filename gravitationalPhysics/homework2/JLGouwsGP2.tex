\documentclass[12pt,a4]{article}
\usepackage{physics, amsmath,amsfonts,amsthm,amssymb, mathtools,steinmetz, gensymb, siunitx}	% LOADS USEFUL MATH STUFF
\usepackage{xcolor,graphicx}
\usepackage{caption}
\usepackage{subcaption}
\usepackage[left=45pt, top=20pt, right=45pt, bottom=45pt ,a4paper]{geometry} 				% ADJUSTS PAGE
\usepackage{setspace}
\usepackage{tikz}
\usepackage{pdfpages}
\usepackage{pgf,tikz,pgfplots,wrapfig}
\usepackage{mathrsfs}
\usepackage{fancyhdr}
\usepackage{float}
\usepackage{array}
\usepackage{booktabs,multirow}
\usepackage{bm}
\usepackage{tensor}
\usepackage{listings}
 \lstset{
    basicstyle=\ttfamily\small,
    numberstyle=\footnotesize,
    numbers=left,
    backgroundcolor=\color{gray!10},
    frame=single,
    tabsize=2,
    rulecolor=\color{black!30},
    title=\lstname,
    escapeinside={\%*}{*)},
    breaklines=true,
    breakatwhitespace=true,
    framextopmargin=2pt,
    framexbottommargin=2pt,
    inputencoding=utf8,
    extendedchars=true,
    literate={á}{{$\rho$}}1 {ã}{{\~a}}1 {é}{{\'e}}1,
}
\DeclareMathOperator{\sign}{sgn}

\usetikzlibrary{decorations.text, calc}
\pgfplotsset{compat=1.7}

\usetikzlibrary{decorations.pathreplacing,decorations.markings}
\usepgfplotslibrary{fillbetween}

\newcommand{\vect}[1]{\boldsymbol{#1}}
\newcommand{\e}{\mathrm{d}}

\usepackage{hyperref}

%\usepackage[style= ACM-Reference-Format, maxbibnames=6, minnames=1,maxnames = 1]{biblatex}
%\addbibresource{references.bib}


\hypersetup{pdfborder={0 0 0},colorlinks=true,linkcolor=black,urlcolor=cyan,}
\allowdisplaybreaks
%\hypersetup{
%
%    colorlinks=true,
%
%    linkcolor=blue,
%
%    filecolor=magenta,      
%
%    urlcolor=cyan,
%
%    pdftitle={An Example},
%
%    pdfpagemode=FullScreen,
%
%    }
%}

\title{
\textsc{Gravitational Physics Homework 2}
}
\author{\textsc{J L Gouws}
}
\date{\today
\\[1cm]}



\usepackage{graphicx}
\usepackage{array}




\begin{document}
\thispagestyle{empty}

\maketitle

\begin{enumerate}
  \item
    \begin{enumerate}
      \item
        Let $\omega = \e \lambda$ be exact, then:
        \begin{equation*}
          \e \omega = \e( \e \lambda )
        \end{equation*}
        For $n = 1$, the only case that has to be checked is 1-forms because there are no ``$-1$"-forms to make ``0"-forms exact.
        All 1-forms are necesarilly closed: $\e (\alpha \e x) = \frac{\partial \alpha}{\partial x} \e x \wedge \e x \equiv 0$.
        The question now is for every one-form $\alpha(x) \e x$, is there a function $\beta(x)$ such that $\alpha = \frac{\partial \beta}{\partial x}$.
        Such a function is:
        \begin{equation*}
          \beta(x) = \int_0^x \alpha(x') \e x'
        \end{equation*}
        Since $\alpha$ is smooth and hence integrable, this is indeed a smooth $0$ form.

        This argument breaks down on the circle, because forms have to be smooth (and continuous) on the whole circle.
        A counter example of a closed, but non exact 1-form on the circle is $\e \theta$, where $\theta$ is the angle along the circle.
        Here if one tries to construct the 0-form $\beta(\theta) = \theta + C$ it satisfies $\e \beta = \frac{\partial \beta}{\partial \theta} \e \theta = \e \theta$, but is not actually a 0-form on the circle $S^1$.
        The points $\theta = 0$ and $\theta = 2 \pi$ are the same points on the circle, but $\beta(0) = C$ and $\beta(2 \pi) = 2 \pi + C$ and so $\beta$ is not single valued and hence a well defined function on $S^1$.
      \item
        Using the generalized Stokes theorem:
        \begin{align*}
          g &= \frac{1}{4 \pi } \int_{S^2} F^{(2)} \\
            &= \frac{1}{4 \pi } \int_{S^2} \e A^{(1)} \\
            &= \frac{1}{4 \pi } \int_{\partial S^2} A^{(1)} \\
            &= 0 
        \end{align*}
        Since the sphere has not boundary.
      \item
        \begin{align*}
          \e F^{(4)} &= \e (Q \sin \theta \e \theta \wedge \e \phi)\\
                     &= Q \cos \theta \e \theta \wedge \e \theta \wedge \e \phi\\
                     &= 0 
        \end{align*}
        And for the other term, note the metric is $\e t^2 - \e r^2 - r^2 \e \theta - r^2 \sin^2 \theta \e \phi$
        \begin{align*}
          \e * F^{(4)} &= \e (* (Q \sin \theta \e \theta \wedge \e \phi))\\
                       &= \e \left(\frac{1}{(4 - 2)!}\frac{1}{2} \left(\tensor{\epsilon}{^\theta ^\phi_\mu_\nu} Q \sin \theta \e x^\mu \wedge \e x^\nu + \tensor{\epsilon}{^\phi^\theta _\mu_\nu} Q \sin \theta \e x^\mu \wedge \e x^\nu\right)\right)\\
                       &= \frac{1}{4}\e \left(Q \sin \theta \left(\tensor{\epsilon}{^\theta ^\phi_\mu_\nu}  \e x^\mu \wedge \e x^\nu + \tensor{\epsilon}{^\phi^\theta _\mu_\nu} \e x^\mu \wedge \e x^\nu\right)\right)\\
                       &= \frac{1}{4}\e \left(Q \sin \theta \left( \tensor{\epsilon}{^\theta ^\phi_t_r} \e t \wedge \e r + \tensor{\epsilon}{^\theta ^\phi_r_t} \e r \wedge \e t + \tensor{\epsilon}{^\phi^\theta _t_r} \e t \wedge \e r + \tensor{\epsilon}{^\phi^\theta _r_t} \e r \wedge \e t\right)\right)
    %                   &= \e \left(\tensor{\epsilon}{^\theta ^\phi_\mu_\nu} Q \sin \theta \e x^\mu \wedge \e x^\nu \right)\\
    %                   &= \e \left(\tensor{\epsilon}{^\theta ^\phi_t_r} Q \sin \theta \e t \wedge \e r - \tensor{\epsilon}{^\theta ^\phi_r_t} Q \sin \theta \e r \wedge \e t\right)\\
    %                   &= Q \cos \theta \e \theta \wedge \e \theta \wedge \e \phi\\
    %                   &= 0 
        \end{align*}
        Now $\sqrt{g} = r^2 \sin \theta$ and $g^{\theta \theta} = -\frac{1}{r^2}$ and $g^{\phi\phi} = -\frac{1}{r^2 \sin^2\theta }$.
        Therefore $\tensor{\epsilon}{^\theta ^\phi_t_r} = \sqrt{g} g^{\theta\theta} g^{\phi\phi} \sigma_{\theta\phi t r}$ and:
        \begin{align*}
          \e * F^{(4)} &= \e \left(\frac{Q}{r^2} \e t \wedge \e r\right)\\
                       &= 0 
        \end{align*}
        Since the coefficient only has $r$ dependence, and taking the derivative of this gives the differential $\e r$ which will make the whole term zero when wedged with $\e r$.
%      \item
%        Here:
%        \begin{equation*}
%          \e \theta = \frac{\partial \theta }{ \partial x} \e x
%        \end{equation*}
%        \begin{equation*}
%          x =  r \sin \theta \cos \phi \Rightarrow \e x = \sin \theta \cos \phi \e r + r \cos \theta \cos \phi \e \theta -  r \sin \theta \sin \phi \e \phi
%        \end{equation*}
      \item
        The \href{https://en.wikipedia.org/wiki/Spherical_coordinate_system}{inverse Jacobian matrix} is:
        \begin{equation*}
          J = \frac{\partial(r, \theta, \phi)}{\partial (x, y, z)} =
          \left(
          \begin{matrix}
            \dfrac{x}{r}&\dfrac{y}{r}&\dfrac{z}{r}\\
            \dfrac{xz}{r^2\sqrt{x^2+y^2}}&\dfrac{yz}{r^2\sqrt{x^2+y^2}}&\dfrac{-\left(x^2 + y^2\right)}{r^2\sqrt{x^2+y^2}}\\
            \dfrac{-y}{x^2+y^2}&\dfrac{x}{x^2+y^2}&0
          \end{matrix}
          \right)
        \end{equation*}
        Therefore:
        \begin{equation*}
          \e \theta = \dfrac{xz}{r^2\sqrt{x^2+y^2}} \e x + \dfrac{yz}{r^2\sqrt{x^2+y^2}} \e y + \dfrac{-\left(x^2 + y^2\right)}{r^2\sqrt{x^2+y^2}} \e z
        \end{equation*}
        and:
        \begin{equation*}
          \e \phi = \dfrac{-y}{x^2+y^2} \e x + \dfrac{x}{x^2+y^2} \e y
        \end{equation*}
        And obviously:
        \begin{equation*}
          \sin \theta = \frac{\sqrt{x^2 + y^2}}{r}
        \end{equation*}
        From which, calculating $F$ is straightforward:
        \begin{align*}
          F &= Q \frac{\sqrt{x^2 + y^2}}{r} \left(\dfrac{xz}{r^2\sqrt{x^2+y^2}} \e x + \dfrac{yz}{r^2\sqrt{x^2+y^2}} \e y\right.\\
            & \qquad + \left.\dfrac{-\left(x^2 + y^2\right)}{r^2\sqrt{x^2+y^2}} \e z\right) \wedge \left(\dfrac{-y}{x^2+y^2} \e x + \dfrac{x}{x^2+y^2} \e y\right)\\
            &= Q \left(\dfrac{z}{r^3} \e x \wedge \e y + \dfrac{y}{r^3} \e z \wedge \e x + \dfrac{x}{r^3} \e y \wedge \e z\right)
        \end{align*}
        For a static electric field $\phi(\mathbf{r}) = -\frac{Q_{em}}{r}$
        An electric monople has electric field asoociated with it:
        \begin{align*}
          E &= \e^{(3)} \phi\\
            &= -\frac{\partial}{\partial x}\left(\frac{Q_{em}}{\sqrt{x^2 + y^2 + z^2}}\right) \e x-\frac{\partial}{\partial y}\left(\frac{Q_{em}}{\sqrt{x^2 + y^2 + z^2}}\right) \e y-\frac{\partial}{\partial z}\left(\frac{Q_{em}}{\sqrt{x^2 + y^2 + z^2}}\right) \e z\\
            &= \frac{Q_{em}x}{\sqrt{x^2 + y^2 + z^2}^3} \e x+\frac{Q_{em} y}{\sqrt{x^2 + y^2 + z^2}^3} \e y+\frac{Q_{em}z}{\sqrt{x^2 + y^2 + z^2}^3} \e z\\
            &= \frac{Q_{\text{EM}} x}{r^3} \e x + \frac{Q_{\text{EM}} y}{r^3}\e y +\frac{Q_{\text{EM}} z}{r^3}\e z
        \end{align*}
        Which is the same form as the magnetic monopole.
      \item
        For the integral of the 2-form on the sphere:
        \begin{align*}
          g =
          \frac{1}{4 \pi}\int_{S^2} F^{(2)} =
          \frac{1}{4 \pi}\int_{S^2} Q \sin \theta \e \theta \wedge \e \phi =
          \frac{1}{4 \pi}\int_{S^2} Q \e \Omega =
          \frac{4 \pi}{4 \pi} Q  = Q
        \end{align*}
        \begin{enumerate}
          \item
            If $F$ were globally exact, the integral would vanish, which it does not, contradicting $F$ begin globally exact.
          \item
%            Therefore like the closed but non-exact form on the circle $\e \theta$, $F$ must not be smooth at one point, say $\theta = 0$.
            This is an argument by contradiction.
            Note that $\e F^{(3)} = 0 \Rightarrow F^{(3)} = \e A^{(3)}$ by Poincare's lemma. 
            Then if $F^{(3)}$ is pulled back by the inclusion map $i_S : \mathbb{S}^2 \hookrightarrow \mathbb{R}^{3}$ implies $F^{(2)} = i_S^*F^{(3)} = i_S^*\e A^{(3)} = \e i_S^*A^{(3)}$ so that $F^{(2)}$ is globally exact on the sphere which is a contradiction.
%            $F$ is not globally exact on the 2-sphere, therefore it is not globally exact in $\mathbb{R}^3$, it it were globally exact on $\mathbb{R}^3$, the projection of the potential on the 2-shpere would make $F$ exact on the 2-sphere.
%            Maxwell's equations give $\e F^{(3)} = 0$ meaning that $F$ is closed in flat space, therefore if $F$ is a smooth $2$-form (i.e. a field defined everywhere), $F$ should be globally exact in $\mathbb{R}^{(3)}$ by Poincare's lemma, a contradiction.
%            Since $F$ cannot be globally defined in $\mathbb{R}^{3}$ its extension to $\mathbb{R}^{4}$ cannot be globally defined too.\\


            Alternaitively, looking at the expression in Cartesian coordinates $r^3$ goes to zero faster that $x$, so $F$ cannot be defined at $r = 0$.
          \item
            Since $F^{(3)}$ is not defined at $r = 0$, it's domain of definition is: $\mathbb{R}^3 \setminus \{0\}$.
            And $F^{(4)}$ is not defined at $r = 0$, it's domain of definition is: $\mathbb{R}^4 \setminus \{0\}$.
%            Drawing by analogy on the case for $S^1$ the breakdown of the form happened at $\theta = 0$ because this is where the coordinates become degenerate.
%            In $S^2$ the coordinates also become degenerate at $\theta = 0$.
%            The domain of $F^{(3)}$ is $\mathbb{R}^3 \setminus \{(r, \theta, \phi) | \theta = 0\}$.
%            And the domain of $F^{(4)}$ is $\mathbb{R}^4 \setminus \{(t, r, \theta, \phi) | \theta = 0\}$.
        \end{enumerate}
      \item
        An attempt at constructing such a potential is:
        \begin{equation*}
          A = Q \cos \theta \e \phi
        \end{equation*}
        But this potential is not well defined on the $\theta = 0$ or $z$ axis, because $\e \phi$ is not well defined there.
        Thus the ponetial must vanish at the poles $\theta = 0$ and $\theta = \pi$.
        For the potential defined at the north pole $\theta = 0$:
        \begin{equation*}
          A_+ = Q (1 - \cos \theta) \e \phi
        \end{equation*}
%        A 1-form such that $F^{(4)} = \e A^{(2)}$ is:
%        \begin{equation*}
%          A_+ = Q (1 - \cos \theta) \e \phi
%        \end{equation*}
        and for $\theta = \pi$, i.e. defined at the south pole:
        \begin{equation*}
          A_- = -Q (1 + \cos \theta) \e \phi
        \end{equation*}
        With $A^{3}_{\pm}$, one is essentially folliating the spacetime with spheres which break down along either the north or south half axis.
        This gives a singularity along the half axis which is like a string, this is a singularity of a line of 1D points $z > 0$, where as $F^{(3)}$ only has a singularity at one point $r = 0$, or a zero dimensional singularity.
      \item
        Let us look at the difference between the potentials in the region in which they are both defined:
        \begin{align*}
                      & A_- - A_+ \\
                     =& -Q (1 + \cos \theta) \e \phi -  Q (1 - \cos \theta) \e \phi \\
                     =& - 2 Q \e \phi
        \end{align*}
        This can be written as:
        \begin{equation*}
          - 2 Q \e \phi = \frac{1}{ie} \e (-i e 2 Q \phi) = \frac{1}{ie} \e \log e^{(-i e 2 Q \phi)} = \frac{1}{ie} \e \log \gamma = \frac{1}{ie} \gamma^{-1} \e \gamma
        \end{equation*}
        Where $\gamma = e^{(-i e 2 Q \phi)}$ and is therefore a $U(1)$ gauge transformation.
        Therefore:
        \begin{align*}
          A_- = A_+ + \frac{1}{ie} \gamma^{-1} \e \gamma
        \end{align*}
        For $\gamma$ to be single valued, it must be invariant under roations by $\phi = 2 \pi$:
        This implies that $2 Q e$ must be an integer so that $e = \frac{n}{2Q}$, and electric charged is conserved by the monopole.
%        \begin{align*}
%                      & -Q (1 + \cos \theta) \e \phi =  Q (1 - \cos \theta) \e \phi + \frac{1}{ie} \gamma^{-1} \e \gamma\\
%          \Rightarrow &  \gamma^{-1} \e \gamma = - 2Q i e \e \phi
%        \end{align*}
%        Say $\gamma = e^{i k \phi}$ then:
%        \begin{align*}
%                      &  \gamma^{-1} \e \gamma = - 2Q i e \e \phi\\
%          \Rightarrow &  i  k \e \phi= - 2Q i e \e \phi\\
%          \Rightarrow &  i( k + 2 Q e)  \e \phi = 0\\
%          \Rightarrow &  k + 2 Q e = 0\\
%          \Rightarrow &  Q = - \frac{k}{2 e}
%        \end{align*}
%        But now for $\gamma$ to be single valued:
%        \begin{equation*}
%          e^{i k \phi} = e^{i (k + 2 \pi n) \phi} 
%        \end{equation*}
%        So that:
%        \begin{align*}
%          Q = - \frac{2 \pi n}{2 e}
%        \end{align*}
        
      \item
        It is required that for lambda to satisfy:
        \begin{align*}
                      & - 2 Q \e \phi = \frac{1}{e} \e \lambda\\
          \Rightarrow & \int_{\phi = 0}^{2 \pi} - 2 Q \e \phi = \frac{1}{e} \int_{\phi = 0}^{2 \pi} \e \lambda\\
          \Rightarrow &  - 4 \pi Q e  =  \lambda(\phi = 2 \pi) - \lambda(\phi = 0)\\
          \Rightarrow &  \lambda(\phi = 2 \pi) \neq \lambda(\phi = 0)
        \end{align*}
        Since $\phi = 0$ and $\phi = 2 \pi$ are identified, $\lambda$ is not single valued.
      \item
        The gauge transformation moves the string from the north pole to the south pole, if something is happending in the area near the north pole, the string can be gauged away to the south pole and vice-versa.
        This makes the string unobservable because it can always be removed from a local patch by a gauge transformation which will not affect any physics.
    \end{enumerate}
  \item
    \begin{enumerate}
      \item
        When $n = 0$, $A_\sigma = 0$ and $f = \frac{r^2 - 2 m r}{r^2} = 1 - \frac{2 m}{r}$ so that:
        \begin{align*}
          \e s^2 = \left(1 - \frac{2 m}{r}\right)\e t \otimes \e t - \left(1 - \frac{2 m}{r}\right)^{-1}\e r \otimes \e r - r^2 (\e \theta \otimes \e \theta + \sin^2 \theta \e \phi \otimes \e \phi)
        \end{align*}
        Which is the Schartzschild solution for a body with mass $m$.
      \item
        With $\sigma = \pm 1$, the potential is $A_\sigma = n(\cos \theta \pm 1) \e \phi$
        And the time component of the metric becomes (the radial and angular part has not $\sigma$ dependence, and so its regulartiy will not be affected by the value of 
        $\sigma$):
        \begin{align*}
          f(\e t + 2 A_\sigma)^2 &= f(\e t + 2 n (\cos\theta \pm 1) \e \phi)^2\\
                                 &= f(\e t \otimes \e t + 2 n (\cos\theta \pm 1) \e t \otimes \e \phi + 2 n (\cos\theta \pm \sigma) \e \phi \otimes \e t\\
                                 &\qquad + 4 n^2 (\cos\theta \pm 1)^2 \e \phi \otimes \e \phi)
        \end{align*}
        Now looking at the $t$  constant surface which sends $\e t$ to zero and:
        \begin{align*}
          f(2 A_\sigma)^2 &= 4 n^2 (\cos\theta \pm 1)^2 \e \phi \otimes \e \phi)
        \end{align*}
        The problem here is that again, $\e \phi$ is not well defined at $\theta = 0$, the north pole, and $\theta = \pi$, the south pole.
        For $\sigma = 0$:
        \begin{align*}
          f(2 A_\sigma)^2 = 4 n^2 \cos^2\theta  \e \phi \otimes \e \phi
        \end{align*}
        And since $\e \phi$ is not well defined at $\theta = 0$ and $\theta = \pi$ this metric is not regular at both the north and the south pole and so there is a string singularity across the whole space time.
      \item
        In the coordinate transformation:
        \begin{equation*}
          t \mapsto t_\sigma := t - 2 n \sigma \phi \Rightarrow \e t \mapsto \e t - 2 n \sigma \e \phi
        \end{equation*}
        Here the metric becomes:
        \begin{align*}
          ds^2 &= f(\e t - 2 n \sigma \e \phi + 2 A_\sigma)^2 - \frac{\e r^2}{f} - (r^2 + n^2)(\e \theta^2 + \sin^2 \theta \e \phi)\\
               &= f(\e t - 2 n \sigma \e \phi + 2n (\cos \theta + \sigma) \e \phi)^2 - \frac{\e r^2}{f} - (r^2 + n^2)(\e \theta^2 + \sin^2 \theta \e \phi)\\
               &= f(\e t  + 2n \cos \theta  \e \phi)^2 - \frac{\e r^2}{f} - (r^2 + n^2)(\e \theta^2 + \sin^2 \theta \e \phi)
        \end{align*}
        For the north pole axis, the coefficient of $\e \phi$ must be zero when $\theta = 0$ and for this $2n \cos \theta = 2n$, therefore:
        \begin{equation*}
          t \mapsto t_+ := t_\sigma - 2n \phi = t - 2 n \sigma \phi - 2n \phi \Rightarrow \e t \mapsto \e t - 2 n \sigma \e \phi - 2 n \e \phi
        \end{equation*}
        And hence the metric becomes:
        \begin{align*}
          ds^2 &= f(\e t - 2 n \sigma \e \phi - 2 n \e \phi + 2 A_\sigma)^2 - \frac{\e r^2}{f} - (r^2 + n^2)(\e \theta^2 + \sin^2 \theta \e \phi)\\
               &= f(\e t - 2 n \sigma \e \phi - 2 n \e \phi + 2n (\cos \theta + \sigma) \e \phi)^2 - \frac{\e r^2}{f} - (r^2 + n^2)(\e \theta^2 + \sin^2 \theta \e \phi)\\
               &= f(\e t + 2n (\cos \theta - 1) \e \phi)^2 - \frac{\e r^2}{f} - (r^2 + n^2)(\e \theta^2 + \sin^2 \theta \e \phi)
        \end{align*}
        For the south pole axis, the coefficient of $\e \phi$ must be zero when $\theta = \pi$ and $2n \cos \theta = - 2n$, therefore:
        \begin{equation*}
          t \mapsto t_- := t_\sigma + 2n \phi = t - 2 n \sigma \phi + 2n \phi \Rightarrow \e t \mapsto \e t - 2 n \sigma \e \phi + 2 n \e \phi
        \end{equation*}
        And hence, for the metric:
        \begin{align*}
          ds^2 &= f(\e t - 2 n \sigma \e \phi + 2 n \e \phi + 2 A_\sigma)^2 - \frac{\e r^2}{f} - (r^2 + n^2)(\e \theta^2 + \sin^2 \theta \e \phi)\\
               &= f(\e t - 2 n \sigma \e \phi + 2 n \e \phi + 2n (\cos \theta + \sigma) \e \phi)^2 - \frac{\e r^2}{f} - (r^2 + n^2)(\e \theta^2 + \sin^2 \theta \e \phi)\\
               &= f(\e t + 2n (\cos \theta + 1) \e \phi)^2 - \frac{\e r^2}{f} - (r^2 + n^2)(\e \theta^2 + \sin^2 \theta \e \phi)
        \end{align*}
        Therefore:
        \begin{equation*}
          t_+ = \e t - 2 n \sigma \e \phi + 2 n \phi - 2 n \phi = t_- - 4n \phi
        \end{equation*}
        For the Misner string to be unobservable, $t_+$ and $t_-$ must be related by a gauge transformation, that is single valued over azimuthal rotations by an angle of $2 \pi$.
        And since $\phi$ is identified with $\phi + 2 \pi n$ for all integers $n$ and in particular $n = 0$:
        \begin{align*}
          \begin{cases*}
            t_+ = t_- & $\phi = 0$\\
            t_+ = t_- - 8 \pi n & $\phi = 2 \pi n$
          \end{cases*}
        \end{align*}
        There is a connection to the Kaluza-Klein monopole here in the sense that there is not a dimension which has been compactified.
        I am not sure about this, but it is probably possible to set $g_{tt} = 1$ and get a naked singularity.

      \item
        For $k^\flat$:
        \begin{align*}
          k^\flat &= 2 n f (\cos \theta + \sigma) \e t + 4 n^2 f (\cos \theta + \sigma)^2 \e \phi- (r^2 + n ^2) \sin^2 \theta \e \phi\\
                  &= 2 n \frac{r^2 - 2m r - n^2}{r^2 + n^2} (\cos \theta + \sigma) \e t + 4 n^2 \frac{r^2 - 2m r - n^2}{r^2 + n^2} (\cos \theta + \sigma)^2 \e \phi- (r^2 + n ^2) \sin^2 \theta \e \phi
        \end{align*}
        Therefore:
        \begin{align*}
          k^\flat_{\text{ren}} :=
                  k^\flat - k^\flat |_{m = 0}
                  &= 2 n \frac{- 2m r }{r^2 + n^2} (\cos \theta + \sigma) \e t + 4 n^2 \frac{- 2m r }{r^2 + n^2} (\cos \theta + \sigma)^2 \e \phi 
        \end{align*}
        Then:
        \begin{align*}
          \e k^\flat_{\text{ren}} 
              &= 4 nm \dfrac{r^2-n^2}{\left(r^2+n^2\right)^2}(\cos \theta + \sigma) \e r \wedge \e t + 4 n m\frac{ r }{r^2 + n^2} \sin \theta \e \theta \wedge \e t \\
              &\qquad + 8 m n^2 \frac{ r^2 - n^2 }{(r^2 + n^2)^2} (\cos \theta + \sigma)^2 \e r \wedge \e \phi + 16 m n^2 \frac{r }{r^2 + n^2} (\cos \theta + \sigma)\sin\theta \e \theta \wedge \e \phi 
        \end{align*}
        Now, only one term of the hodge dual of this form, the first, is required since all the others will be zero on the sphere.
        Taking the large $r$ limit:
        \begin{equation*}
          \det(\e s^2) = r^4 \sin^2 \theta
        \end{equation*}
        Which I worked out with other metric components in Mathematica, see the end of the pdf.
        Using these in a straight forward calculation gives:
        \begin{align*}
            & *\left(4 nm \dfrac{r^2-n^2}{\left(r^2+n^2\right)^2}(\cos \theta + \sigma) \e r \wedge \e t\right)\\
          = &\frac{1}{2}4 nm \dfrac{r^2-n^2}{\left(r^2+n^2\right)^2}(\cos \theta + \sigma)\sqrt{|\det g|} g^{r a} g^{t b} \varepsilon_{abcd} \e x^c \wedge \e x^d \\
        \to & - 2 nm \dfrac{1}{r^2}(\cos \theta + \sigma)r^2 \sin \theta \varepsilon_{rtcd} \e x^c \wedge \e x^d\\
          = & - 2 nm (\cos \theta + \sigma) \sin \theta \varepsilon_{rtcd} \e x^c \wedge \e x^d \\
          = & 2 nm (\cos \theta + \sigma) \sin \theta \varepsilon_{trcd} \e x^c \wedge \e x^d\\
          = & 4 nm (\cos \theta + \sigma) \sin \theta  \e \theta \wedge \e \phi
        \end{align*}
        And integrating this over the sphere:
        \begin{align*}
            & -\frac{1}{16 \pi}\int_{S^2_\infty} 4 nm (\cos \theta + \sigma) \sin \theta  \e \theta \wedge \e \phi\\
          = & -\frac{1}{16 \pi}\left(\int_{S^2_\infty} 4 nm \cos \theta \sin \theta  \e \theta \wedge \e \phi + \int_{S^2_\infty} 4 nm \sigma \sin \theta  \e \theta \wedge \e \phi\right)\\
          = & -\frac{1}{16 \pi}\left(\int_{S^2_\infty} 4 nm \sigma \e \Omega\right)\\
          = & -\frac{1}{16 \pi}16 \pi nm \sigma \\
          = & - nm \sigma 
        \end{align*}
    \end{enumerate}
\end{enumerate}
\includepdf[pages=-]{gphw2Mathematica.pdf}
\end{document}
