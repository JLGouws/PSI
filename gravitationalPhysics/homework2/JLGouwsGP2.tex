\documentclass[12pt,a4]{article}
\usepackage{physics, amsmath,amsfonts,amsthm,amssymb, mathtools,steinmetz, gensymb, siunitx}	% LOADS USEFUL MATH STUFF
\usepackage{xcolor,graphicx}
\usepackage{caption}
\usepackage{subcaption}
\usepackage[left=45pt, top=20pt, right=45pt, bottom=45pt ,a4paper]{geometry} 				% ADJUSTS PAGE
\usepackage{setspace}
\usepackage{tikz}
\usepackage{pgf,tikz,pgfplots,wrapfig}
\usepackage{mathrsfs}
\usepackage{fancyhdr}
\usepackage{float}
\usepackage{array}
\usepackage{booktabs,multirow}
\usepackage{bm}
\usepackage{tensor}
\usepackage{listings}
 \lstset{
    basicstyle=\ttfamily\small,
    numberstyle=\footnotesize,
    numbers=left,
    backgroundcolor=\color{gray!10},
    frame=single,
    tabsize=2,
    rulecolor=\color{black!30},
    title=\lstname,
    escapeinside={\%*}{*)},
    breaklines=true,
    breakatwhitespace=true,
    framextopmargin=2pt,
    framexbottommargin=2pt,
    inputencoding=utf8,
    extendedchars=true,
    literate={á}{{$\rho$}}1 {ã}{{\~a}}1 {é}{{\'e}}1,
}
\DeclareMathOperator{\sign}{sgn}

\usetikzlibrary{decorations.text, calc}
\pgfplotsset{compat=1.7}

\usetikzlibrary{decorations.pathreplacing,decorations.markings}
\usepgfplotslibrary{fillbetween}

\newcommand{\vect}[1]{\boldsymbol{#1}}
\newcommand{\e}{\mathrm{d}}

\usepackage{hyperref}

%\usepackage[style= ACM-Reference-Format, maxbibnames=6, minnames=1,maxnames = 1]{biblatex}
%\addbibresource{references.bib}


\hypersetup{pdfborder={0 0 0},colorlinks=true,linkcolor=black,urlcolor=cyan,}
\allowdisplaybreaks
%\hypersetup{
%
%    colorlinks=true,
%
%    linkcolor=blue,
%
%    filecolor=magenta,      
%
%    urlcolor=cyan,
%
%    pdftitle={An Example},
%
%    pdfpagemode=FullScreen,
%
%    }
%}

\title{
\textsc{Gravitational Physics Homework 2}
}
\author{\textsc{J L Gouws}
}
\date{\today
\\[1cm]}



\usepackage{graphicx}
\usepackage{array}




\begin{document}
\thispagestyle{empty}

\maketitle

\begin{enumerate}
  \item
    Let $\omega = \e \lambda$ be exact, then:
    \begin{equation*}
      \e \omega = \e( \e \lambda )
    \end{equation*}
    For $n = 1$, the only case that has to be checked is for 1-forms because there are no "$-1$"-forms to make "0"-forms exact.
    All 1-forms are necesarilly exact: $\e (\alpha \e x) = \frac{\partial \alpha}{\partial x} \e x \wedge \e x \equiv 0$.
    The question now is for every one-form $\alpha(x) \e x$, is there a function $\beta(x)$ such that $\alpha = \frac{\partial \beta}{\partial x}$.
    Such a function is:
    \begin{equation*}
      \beta(x) = \int_0^x \alpha(x') \e x'
    \end{equation*}
    Since $\alpha$ is smooth and hence integrable, this is indeed a smooth $0$ form.

    This argument breaks down on the circle, because forms have to be smooth (and continuous) on the whole circle.
    A counter example of a closed, but non exact 1-form on the circle is $\e \theta$, where $\theta$ is the angle along the circle.
    Here if one tries to construct the 0-form $\beta(\theta) = \theta + C$ it satisfies $\frac{\partial \beta}{\theta} \e \theta = \e \theta$, but is not actually a 0-form on the circle $S^1$.
    The points $\theta = 0$ and $\theta = 2 \pi$ are the same points on the circle, but $\beta(0) = C$ and $\beta(2 \pi) = 2 \pi + C$ and so $\beta$ is not single valued and hence a well defined function on $S^1$.
  \item
    \begin{align*}
      g &= \frac{1}{4 \pi } \int_{S^2} F^{(2)} \\
        &= \frac{1}{4 \pi } \int_{S^2} \e A^{(1)} \\
        &= \frac{1}{4 \pi } \int_{\partial S^2} A^{(1)} \\
        &= 0 
    \end{align*}
    Since the sphere has not boundary.
  \item
    \begin{align*}
      \e F^{(4)} &= \e (Q \sin \theta \e \theta \wedge \e \phi)\\
                 &= Q \cos \theta \e \theta \wedge \e \theta \wedge \e \phi\\
                 &= 0 
    \end{align*}
    And for the other term, note the metric is $\e t^2 - \e r^2 - r^2 \e \theta - r^2 \sin^2 \theta \e \phi$
    \begin{align*}
      \e * F^{(4)} &= \e (* (Q \sin \theta \e \theta \wedge \e \phi))\\
                   &= \e \left(\frac{1}{(4 - 2)!}\frac{1}{2} \left(\tensor{\epsilon}{^\theta ^\phi_\mu_\nu} Q \sin \theta \e x^\mu \wedge \e x^\nu + \tensor{\epsilon}{^\phi^\theta _\mu_\nu} Q \sin \theta \e x^\mu \wedge \e x^\nu\right)\right)\\
                   &= \frac{1}{4}\e \left(Q \sin \theta \left(\tensor{\epsilon}{^\theta ^\phi_\mu_\nu}  \e x^\mu \wedge \e x^\nu + \tensor{\epsilon}{^\phi^\theta _\mu_\nu} \e x^\mu \wedge \e x^\nu\right)\right)\\
                   &= \frac{1}{4}\e \left(Q \sin \theta \left( \tensor{\epsilon}{^\theta ^\phi_t_r} \e t \wedge \e r + \tensor{\epsilon}{^\theta ^\phi_r_t} \e r \wedge \e t + \tensor{\epsilon}{^\phi^\theta _t_r} \e t \wedge \e r + \tensor{\epsilon}{^\phi^\theta _r_t} \e r \wedge \e t\right)\right)
%                   &= \e \left(\tensor{\epsilon}{^\theta ^\phi_\mu_\nu} Q \sin \theta \e x^\mu \wedge \e x^\nu \right)\\
%                   &= \e \left(\tensor{\epsilon}{^\theta ^\phi_t_r} Q \sin \theta \e t \wedge \e r - \tensor{\epsilon}{^\theta ^\phi_r_t} Q \sin \theta \e r \wedge \e t\right)\\
%                   &= Q \cos \theta \e \theta \wedge \e \theta \wedge \e \phi\\
%                   &= 0 
    \end{align*}
    Now $\sqrt{g} = r^2 \sin \theta$ and $g^{\theta \theta} = -\frac{1}{r^2}$ and $g^{\phi\phi} = -\frac{1}{r^2 \sin^2\theta }$
    Therefore $\tensor{\epsilon}{^\theta ^\phi_t_r} = \sqrt{g} g^{\theta\theta} g^{\phi\phi} \sigma_{\theta\phi t r}$ and:
    \begin{align*}
      \e * F^{(4)} &= \e \left(Q \frac{Q}{r^2} \e t \wedge \e r\right)\\
                   &= 0 
    \end{align*}
  \item
    Here:
    \begin{equation*}
      \e \theta = \frac{\partial \theta }{ \partial x} \e x
    \end{equation*}
    \begin{equation*}
      x =  r \sin \theta \cos \phi \Rightarrow \e x = \sin \theta \cos \phi \e r + r \cos \theta \cos \phi \e \theta -  r \sin \theta \sin \phi \e \phi
    \end{equation*}
    
\end{enumerate}
\end{document}
