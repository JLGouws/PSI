\documentclass[12pt,a4]{article}
\usepackage{physics, amsmath,amsfonts,amsthm,amssymb, mathtools,steinmetz, gensymb, siunitx}	% LOADS USEFUL MATH STUFF
\usepackage{xcolor,graphicx}
\usepackage[left=45pt, top=20pt, right=45pt, bottom=45pt ,a4paper]{geometry} 				% ADJUSTS PAGE
\usepackage{setspace}
\usepackage{caption}
\usepackage{tikz}
\usepackage{pgf,tikz,pgfplots,wrapfig}
\usepackage{mathrsfs}
\usepackage{fancyhdr}
\usepackage{float}
\usepackage{array}
\usepackage{booktabs,multirow}
\usepackage{bm}

\usetikzlibrary{decorations.text, calc}
\pgfplotsset{compat=1.7}

\usetikzlibrary{decorations.pathreplacing,decorations.markings}
\usepgfplotslibrary{fillbetween}

\newcommand{\vect}[1]{\boldsymbol{#1}}

\usepackage{hyperref}
%\usepackage[style= ACM-Reference-Format, maxbibnames=6, minnames=1,maxnames = 1]{biblatex}
%\addbibresource{references.bib}


\hypersetup{pdfborder={0 0 0},colorlinks=true,linkcolor=black,urlcolor=cyan,}
%\hypersetup{
%
%    colorlinks=true,
%
%    linkcolor=blue,
%
%    filecolor=magenta,      
%
%    urlcolor=cyan,
%
%    pdftitle={An Example},
%
%    pdfpagemode=FullScreen,
%
%    }
%}

\title{
\textsc{CM Homework 1}
}
\author{\textsc{J L Gouws}
}
\date{\today
\\[1cm]}



\usepackage{graphicx}
\usepackage{array}




\begin{document}
\thispagestyle{empty}

\maketitle

\begin{enumerate}
  \item
    \begin{enumerate}
      \item 
        The bead has velocity in the $\hat\phi$ direction of $\rho\omega$, leading to a kinetic energy term: $\frac{1}{2} m \omega^2 \rho^2$.
        In the $\hat\rho$ direction, the velocity is simply $\dot\rho$, leading to a kinetc eneregy term: $\frac{1}{2} m \dot\rho^2$.
        In the $z$ direction, the velocity is $\dot z$, but since the bead is constrained to the piece of wire, this velocity can be determined in terms of $\rho$.
        Using the constraint equation results in:
        \begin{align*}
          \frac{dz}{dt} &= D_t\left[\alpha \rho^2 \right]\\
                        &= 2 \alpha \rho D_t\left[\rho \right]\\
                        &= 2 \alpha \rho \dot\rho \\
                        &\Rightarrow T_z = \frac{1}{2} m 4 \alpha^2 \rho^2 \dot \rho^2
        \end{align*}

        If we choose $z=0$ as the point of zero potential, then:
        \begin{equation*}
          V = m g h = m g \alpha \rho^2
        \end{equation*}

        Combining this gives the Lagrangian of the system:
        \begin{align*}
          L &= T - V\\
            &= \frac{1}{2} m \dot\rho^2 + \frac{1}{2} m \omega^2 \rho^2 + \frac{1}{2}m4 \alpha^2 \rho^2 \dot \rho^2 - mg \alpha \rho^2\\
            &= \frac{1}{2} m \left(\left[1 + 4 \alpha^2 \rho^2\right]\dot\rho^2 + \rho^2\omega^2\right) - mg\alpha\rho^2
        \end{align*}
        This is a simplistic derivation that might run into problems with more complicated systems and coordinates.
        A better approach would be to do it interms of rectangular coordiates(the basis vectors are fixed throughout space) and have a constraint equation with a Lagrange multiplier, and doing a change of coordinates.
      \item
        This Lagrangian has no explicit time dependence, and so there will be a conserved quantity that is ``canonically conjugate'' to time, which is just the generalized energy:
        \begin{align*}
          Q_s &= \dot q^ip_i - L\\
              &= \dot\rho m \left(1 + 4 \alpha^2 \rho^2\right)\dot\rho - L \\
              &= \frac{1}{2} m \left(1 + 4 \alpha^2 \rho^2\right)\dot\rho^2 - \frac{1}{2} m\omega \rho^2 \omega + m g \alpha \rho^2 \\
              &= \frac{1}{2} m \left(1 + 4 \alpha^2 \rho^2\right)\dot\rho^2 + \frac{1}{2} m\omega \rho^2 \left(2 g \alpha - \omega^2\right) \\
              &= \frac{1}{2} m \left(1 + 4 \alpha^2 \rho^2\right)\dot\rho^2 + \frac{1}{2} m\omega \rho^2\left(\omega_0^2  - \omega^2\right) \\
              &= \frac{1}{2} m \left(1 + 4 \alpha^2 \rho^2\right)\dot\rho^2 + U_{\text{eff}}
        \end{align*}
        With $\omega_0 = \sqrt{2 g \alpha}$.
      \item 
%        $\quad$\\
%        \begin{center}
%          \begin{tikzpicture}
%            \begin{axis}[
%                axis lines = center,
%                xlabel = \(\omega\),
%                ylabel = {\(U_\text{eff}\)},
%                ymax=4,
%                yticklabels={,,},
%                xticklabels={,,},
%                ticks=none,
%            ]
%            %Below the red parabola is defined
%            \addplot [
%                domain=-2.5:2.5, 
%                samples=100, 
%                color=red,
%            ]
%            {2 - x^2};
%    %         \addlegendentry{$U_\text{eff}$}
%            %Here the blue parabola is defined
%            \end{axis}
%          \end{tikzpicture}
%          \begin{tikzpicture}
%            \begin{axis}[
%                    xlabel = \(\rho\),
%                    ylabel = {\(\omega\)},
%                    zlabel = {\(U_\text{eff}\)},
%                    yticklabels={,,},
%                    xticklabels={,,},
%                    zticklabels={,,},
%                    ticks=none,
%              ]
%                \addplot3 [
%                    surf,
%                    faceted color=blue,
%                    samples=15,
%                    domain=0:9,y domain=-5:5,
%                  ] {x^2*(3 - y^2)};
%            \end{axis}
%          \end{tikzpicture} 
%        \end{center}
        
        \begin{enumerate}
          \item
            $\quad$\\
            \begin{center}
              \begin{tikzpicture}
                \begin{axis}[
                    axis lines = center,
                    xlabel = \(\rho\),
                    ylabel = {\(U_\text{eff}\)},
                    ymax=4,
                    yticklabels={,,},
                    xticklabels={,,},
                    ticks=none,
                ]
                %Below the red parabola is defined
                \addplot [
                    domain=0:2.5, 
                    samples=100, 
                    color=red,
                ]
                {x^2};
                \end{axis}
              \end{tikzpicture}
            \end{center}
              Here, the potential term is a positive parabola.
              The easiest way to visualize this is imagine that the wire is stationary in the gravitaional field.
              For some initial conditions there will be a $\rho_{\text{max}}$ that $\rho$ cannot increase beyond--for the energy to remain constant $\dot \rho$ must decrease when $\rho$ increases, and eventually $\dot \rho$ must become zero.
              When the bead has zero velocity in the $\rho$ direction it will fall down the wire and $\dot \rho$ will increase while $\rho$ decreases.
              With $\rho$ decreasing, it will eventually become zero, and all the particle's energy will be purely kinetic.
              The bead will move past the origin and swing back to ``$-\rho_{\text{max}}$" and the cycle will continue.
              The bead essentially oscillates on the wire--it can be seen that the bead's generalized energy looks more or less like that of a simple harmonice oscillator with a slight perturbation connecting $\rho$ and $\dot \rho$.

          \item
            $\quad$\\
            \begin{center}
              \begin{tikzpicture}
                \begin{axis}[
                    axis lines = center,
                    xlabel = \(\rho\),
                    ylabel = {\(U_\text{eff}\)},
                    ymax=4,
                    yticklabels={,,},
                    xticklabels={,,},
                    ticks=none,
                ]
                %Below the red parabola is defined
                \addplot [
                    domain=0:2.5, 
                    samples=100, 
                    color=red,
                ]
                {-x^2};
                \end{axis}
              \end{tikzpicture}
            \end{center}

            In this case, as the bead falls down the potential, $\rho$ increases, and the bead heads off to infinity.
            It might be possible(probably but cannot be sure without an explicit check of the system's integrability) to start with $\rho \dot \rho$ negative and small enough that the bead will move down and change direction before crossing $\rho = 0$ and then heading off to $\rho \to \pm \infty$.
            Essentially trying to climb the hill of the potential, failling and rolling back down the hill.
            If the initial velocity is great enough, the bead might swing past $\rho = 0$ and head off to the infinity on the side of the wire opposite to where it started.

          \item
            $\quad$\\
            \begin{center}
              \begin{tikzpicture}
                \begin{axis}[
                    axis lines = center,
                    xlabel = \(\rho\),
                    ylabel = {\(U_\text{eff}\)},
                    ymax=4,
                    yticklabels={,,},
                    xticklabels={,,},
                    ticks=none,
                ]
                %Below the red parabola is defined
                \addplot [
                    domain=0:2.5, 
                    samples=100, 
                    color=red,
                ]
                {0};
                \end{axis}
              \end{tikzpicture}
            \end{center}
            Here, the potential term is zero.
            More specifically, the potential and generalized energy are completely indepedent of the gravitaional strength parameter $g$.
            This implies that we can imagine the bead on the wire in free space.
            One option for the bead's trajectory is to stay stationary at a given point $\rho$ for eternity.
            Another option is for the bead to move in a monotonous direction along the wire and move towards infinity increasingly slowly.
            This point is a bifurcation in the system.
        \end{enumerate}
      \item
        The energy is not conserved here as the motor is giving energy input into the system.
        Thus we do not expect a conseved energy-like quantity to have the form of the total energy.
        Mathematically they are not equivalent because the kinetic term is not a simple quadratic in the velocities.
    \end{enumerate}
  \item
    \begin{enumerate}
      \item
        The time it takes to traverse the wire is:
        \begin{equation*}
          T[x(y)] = \int_{y_A}^{y_B} (\dot y)^{-1} dy
        \end{equation*}
        $\dot y$ can be determined from the conservation of energy.
        Along the bead's path, the conservation of the bead's energy constrains $\dot y$, $\dot x$ and $y$ so that:
        \begin{equation*}
          0 = \frac{1}{2} m (\dot x^2 + \dot y^2) - m g y = \frac{1}{2} m (x'^2 + 1) \dot y^2 - m g y
        \end{equation*}
        Rerranging this:
        \begin{align*}
                      & \dot y^2 = \frac{2gy}{1 + x'^2}\\
          \Rightarrow & \dot y = \frac{\sqrt{2gy}}{\sqrt{1 + x'^2}}\\
        \end{align*}
        Substituting this into the time functional makes the $y$ dependence explicit:
        \begin{equation*}
          T[y(x)] = \int_{y_A}^{y_B} \frac{\sqrt{1 + x'^2}}{\sqrt{2gy}} dy
        \end{equation*}
      \item
        In the time actional, there is an equivalent of a Lagragian:
        \begin{equation*}
          L(x, x', y) = \frac{\sqrt{1 + x'^2}}{\sqrt{2gy}}
        \end{equation*}
        $x$ is cyclical here, which implies that:
        \begin{equation*}
          \frac{\partial L}{\partial x'} = \frac{x'}{\sqrt{2gy(1 + x'^2)}}
        \end{equation*}
        is conserved with changing $y$ or:
        \begin{equation*}
          0 = \frac{d}{dy}\frac{\partial L}{\partial x'} \Rightarrow \frac{x'}{\sqrt{2gy(1 + x'^2)}} = Q
        \end{equation*}
        Since the derivative is a total derivative in $y$ and $x(y)$.
        The following algebra ensues(some of it might be superfluous since there are squaring and square-rooting steps):
        \begin{align*}
                      & Q = - \frac{x'}{\sqrt{2gy(1 + x'^2)}}\\
          \Rightarrow & x'^2  = 2Q^2gy(1 + x'^2)\\
          \Rightarrow & x'^2  = \frac{2Q^2gy}{1 - 2Q^2gy}\\
          \Rightarrow & \frac{dx}{dy}  = \frac{\sqrt{y}}{\sqrt{1/2Q^2g - y}}
        \end{align*}
        This equation is separable so it can be integrated to get:
        \begin{align*}
          x = \int_{x_A}^{x}d\mathtt{x} + x_A = \int_{y_A}^{y}\frac{\sqrt{\mathtt{y}}}{\sqrt{1/2Q^2g - \mathtt{y}}} d \mathtt{y} + x_A
        \end{align*}
        This integral can be solved with the substitution $y = \frac{1}{2Q^2g}\sin^2\frac{\theta}{2}$:\footnote{I got stuck here and cheated by looking at \href{https://www.ucl.ac.uk/~ucahmto/latex_html/chapter2_latex2html/node7.html}{this link}}
        \begin{align*}
             \int_{y_A}^{y}\frac{\sqrt{\mathtt{y}}}{\sqrt{1/2Q^2g - \mathtt{y}}} d \mathtt{y} + x_A
          &= \int_{\theta_A}^{\theta}\frac{\sin(\theta/2)}{\cos(\theta/2)} \frac{1}{2Q^2g} \sin \frac{\theta}{2} \cos \frac{\theta}{2} d {\theta} + x_A\\
          &= \frac{1}{2Q^2g} \int_{\theta_A}^{\theta}\sin^2(\theta/2) d {\theta} + x_A\\
          &= \frac{1}{2Q^2g}\frac{1}{2}\left(\theta - \sin \theta \right) - \frac{1}{Q^2g}\frac{1}{2}\left(\theta_A - \sin \theta_A \right) + x_A\\
          &= \frac{a}{2}\left(\theta - \sin \theta \right)
        \end{align*}
        Where $a = \frac{1}{2Q^2 g}$, and the dependence on the conserved quantity is clear, and $y$ is:
        \begin{equation*}
          y = \frac{1}{2Q^2g}\sin^2\frac{\theta}{2} = \frac{a}{2}(1 - \cos \theta)
        \end{equation*}
        Determining $a$ in terms of $x_a, x_b, y_a, y_b$:
        \begin{align*}
          x_B - x_A = \frac{a}{2} \left(\theta_B - \theta_A +\sin \theta_A -\sin\theta_B\right)
        \end{align*}
        And:
        \begin{equation*}
          \theta_{*} = 2\sin^{-1}\left(\sqrt{\frac{ y_{*}}{a}}\right)
        \end{equation*}
        So that:
        \begin{align*}
          x_B - x_A &= \frac{a}{2} \left[2\sin^{-1}\sqrt{\frac{ y_{B}}{a}} - 2\sin^{-1}\sqrt{\frac{ y_{A}}{a}} + \sin\left(2\sin^{-1}\left(\sqrt{\frac{ y_{A}}{a}}\right)\right) + \sin\left(2\sin^{-1}\left(\sqrt{\frac{ y_{B}}{a}}\right)\right)\right] \\
                    &= a \left[\sin^{-1}\sqrt{\frac{ y_{B}}{a}} - \sin^{-1}\sqrt{\frac{ y_{A}}{a}} + \sqrt{\frac{ y_{A}}{a}}\cos\left(\sin^{-1}\left(\sqrt{\frac{ y_{A}}{a}}\right)\right)\right. \\
                    &\qquad \qquad \left. - \sqrt{\frac{ y_{B}}{a}}\cos\left(\sin^{-1}\left(\sqrt{\frac{ y_{B}}{a}}\right)\right)\right] \\
                    &= a \left[\sin^{-1}\sqrt{\frac{ y_{B}}{a}} - \sin^{-1}\sqrt{\frac{ y_{A}}{a}} + \sqrt{\frac{ y_{A}}{a}}\sqrt{1 - \frac{ y_{A}}{a}} - \sqrt{\frac{ y_{B}}{a}}\sqrt{1 - \frac{ y_{B}}{a}}\right] \\
        \end{align*}
        And this is an implicit relation for $a$ in terms of the coordinates.
        
    \end{enumerate}
  \item
    \begin{enumerate}
      \item
        Noted.
      \item
        Consider a variation of the form:
        \begin{equation*}
          q(t) \leadsto q(t) + \epsilon \tilde{\delta}^f_s q(t) = q(t) + \epsilon f(t) \tilde{\delta}_s q(t)
        \end{equation*}
        with $f(t_0) = f(t_1) = 0$.
        Now the velocity varies as:
        \begin{equation*}
          \dot q(t) \leadsto \dot q(t) + \epsilon \tilde{\delta}^f_s \dot q(t) 
        \end{equation*}
        With:
        \begin{equation*}
          \tilde{\delta}^f_s \dot q(t) = \dot f \tilde{\delta}_s q + f(t) \tilde{\delta}_s \dot q
        \end{equation*}
        The variation in the Lagrangian now is:
        \begin{align*}
          \tilde{\delta}_s^f L &= \frac{\partial L}{\partial q^i} \tilde{\delta}_s^fq^i + \frac{\partial L}{\partial \dot q^i} \tilde{\delta}_s^f\dot q^i\\
                               &= \frac{\partial L}{\partial q^i} f(t)\tilde{\delta}_sq^i + \frac{\partial L}{\partial \dot q^i} \dot f \tilde{\delta}_s q^i + \frac{\partial L}{\partial \dot q^i} f(t) {\delta}_s \dot q^i\\
                               &= f(t)\left(\frac{\partial L}{\partial q^i} \tilde{\delta}_sq^i + \frac{\partial L}{\partial \dot q^i} {\delta}_s \dot q^i\right) + \frac{\partial L}{\partial \dot q^i} \dot f \tilde{\delta}_s q^i \\
                               &= f(t)\left(\tilde{\delta}_sL \right) + \frac{\partial L}{\partial \dot q^i} \dot f \tilde{\delta}_s q^i \\
                               &= f(t)\frac{d}{dt} R_s + \frac{\partial L}{\partial \dot q^i} \dot f \tilde{\delta}_s q^i \\
                               &= \frac{d}{dt}\left(f(t) R_s\right) - R_s \dot f+ \frac{\partial L}{\partial \dot q^i} \dot f \tilde{\delta}_s q^i \\
                               &= \frac{d}{dt}\left(f(t) R_s\right) + \left(\frac{\partial L}{\partial \dot q^i}\tilde{\delta}_s q^i - R_s\right)\dot f\\
                               &= \frac{d}{dt}\left(f(t) R_s\right) + Q_s \dot f\\
        \end{align*}
        Now the variation in the action is:
        \begin{align*}
          \tilde{\delta}_s^fS &= \int_{t_0}^{t_1} \frac{d}{dt}\left(f(t) R_s\right) + Q_s \dot f dt\\
                             &= \left.f(t) R_s\right|_{t_0}^{t_1} + \int_{t_0}^{t_1}Q_s \dot f dt\\
                             &= \int_{t_0}^{t_1} dt Q_s \dot f
        \end{align*}
        Where the first integral vanishes owing to $f$'s boundary conditions.
      \item
        On shell, the action's variation goes to zero:
        \begin{align*}
          0 &= \int_{t_0}^{t_1} dtQ_s \dot f\\
            &= \int_{f_0}^{f_1} df Q_s \\
            &= \left.fQ_s \right|_{t_0}^{t_1} + \int_{t_0}^{t_1} f \frac{dQ_s}{dt} dt \\
            &= \int_{t_0}^{t_1} f \frac{dQ_s}{dt} dt
        \end{align*}
        Since $f$ is an arbitrary function, integral being identically zero implies:
        \begin{align*}
          \frac{dQ_s}{dt} = 0
        \end{align*}
      \item
        The time translation symmetry is $\tilde{\delta}_s^f q = f \dot q$:
        \begin{align*}
          \tilde{\delta}_s^f S = 
          \int_{t_0}^{t_1} dt \tilde{\delta}_s^f L %&= \int_{t_0}^{t_1} dt- \frac{\partial V}{\partial q} \tilde{\delta}_s^fq^i + m \dot q \tilde{\delta}_s^f\dot q^i\\
                               &= \int_{t_0}^{t_1} dt \left[ - \frac{\partial V}{\partial q} f(t) \dot q^i + m \dot q \dot f \dot q + m \dot q f(t) \ddot  q^i\right]\\
                               &= \int_{t_0}^{t_1} dt \left[ - \frac{d V}{d t} f(t) + m \dot q ^2\dot f  + m \dot q f(t) \ddot  q^i\right]\\
                               &= \int_{t_0}^{t_1} dt \left[V \dot f  + \frac{1}{2}m \dot q^2 \dot f  + \frac{d}{dt} \left(\frac{1}{2}m\dot q^2 f(t) \right)- \frac{d}{dt} \left(Vf\right)\right]\\
                               &= \int_{t_0}^{t_1} dt \left(V   + \frac{1}{2}m \dot q^2\right) \dot f
        \end{align*}
        The total time derivative terms go to zero upon integration because $f(t_1) = f(t_0) = 0$, and what is left is the conserved energy times $\dot f$.
        Note that this only works because the potential has no explicit time dependence.
        This derivation is a nice way to see that time dependent potentials do not conserve energy.

    \end{enumerate}
\end{enumerate}

\end{document}
