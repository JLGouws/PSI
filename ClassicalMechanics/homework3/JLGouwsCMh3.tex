\documentclass[12pt,a4]{article}
\usepackage{physics, amsmath,amsfonts,amsthm,amssymb, mathtools,steinmetz, gensymb, siunitx}	% LOADS USEFUL MATH STUFF
\usepackage{xcolor,graphicx}
\usepackage[left=45pt, top=20pt, right=45pt, bottom=45pt ,a4paper]{geometry} 				% ADJUSTS PAGE
\usepackage{setspace}
\usepackage{caption}
\usepackage{tikz}
\usepackage{pgf,tikz,pgfplots,wrapfig}
\usepackage{mathrsfs}
\usepackage{fancyhdr}
\usepackage{float}
\usepackage{array}
\usepackage{booktabs,multirow}
\usepackage{bm}
\usepackage{tensor}

\usetikzlibrary{decorations.text, calc}
\pgfplotsset{compat=1.7}

\usetikzlibrary{decorations.pathreplacing,decorations.markings}
\usepgfplotslibrary{fillbetween}

\newcommand{\vect}[1]{\boldsymbol{#1}}

\usepackage{hyperref}
%\usepackage[style= ACM-Reference-Format, maxbibnames=6, minnames=1,maxnames = 1]{biblatex}
%\addbibresource{references.bib}


\hypersetup{pdfborder={0 0 0},colorlinks=true,linkcolor=black,urlcolor=cyan,}
%\hypersetup{
%
%    colorlinks=true,
%
%    linkcolor=blue,
%
%    filecolor=magenta,      
%
%    urlcolor=cyan,
%
%    pdftitle={An Example},
%
%    pdfpagemode=FullScreen,
%
%    }
%}

\title{
\textsc{CM Homework 3}
}
\author{\textsc{J L Gouws}
}
\date{\today
\\[1cm]}



\usepackage{graphicx}
\usepackage{array}




\begin{document}
\thispagestyle{empty}

\maketitle

\begin{enumerate}
  \item
    \begin{enumerate}
      \item
        \begin{enumerate}
          \item
            The Maxwell's equations for the vector potential are:
            \begin{align*}
              \Box A_\nu -\nabla_\nu\nabla^\mu A_mu &= - 4 \pi j_\nu\\
              \nabla_\mu A^\mu &= \Gamma
            \end{align*}
          \item 
            For the Lorentz gauge, $A_\mu \leadsto A_\mu + \nabla_\mu \xi$ such that Maxwell's equations simplify to:
            \begin{align*}
              \Box A_\mu &= -4 \pi j_\mu\\
              \nabla_\mu A^\mu &= 0
            \end{align*}
        \end{enumerate}

      \item
        For the second equation:
        \begin{align*}
          0 = \nabla_\nu A^\nu &= \eta^{\mu\nu} a_\mu\frac{\partial}{\partial x^{\nu}} \exp(ik_\nu x^\nu)\\
                           &= \eta^{\mu\nu} a_\mu \exp(ik_\nu x^\nu)i k_\nu\\
                           &= (\mathbf{a} \cdot \mathbf{k} - \omega \phi) \exp(ik_\nu x^\nu)\\
                           &\Rightarrow \mathbf{a} \cdot \mathbf{k} = \omega \phi
        \end{align*}
        For the first equation:
        \begin{align*}
          0 = \Box A_\mu  &= \nabla^\nu \nabla_\nu A_\mu\\
                          &= \eta^{\rho\nu}\nabla_\rho \nabla_\nu A_\mu\\
                          &= \eta^{\rho\nu}\nabla_\rho a_\mu \exp(ik_\mu x^\mu) i k_\nu \\
                          &= - \eta^{\rho\nu} a_\mu \exp(ik_\mu x^\mu) k_\nu k_\rho \\
%                          &= \left(-\frac{\partial^2}{\partial t^2} + \frac{\partial }{\partial x^i \partial x_i}\right) a_\mu \exp(ik_\mu x^\mu)\\
                          &= a_\mu (\omega^2 - |\mathbf{k}|^2)\exp(ik_\mu x^\mu)
        \end{align*}
        Now one posibility for the coefficients to solve the equation is for each component of $a$ to be zero, this is trivial since there is not potential at all so not of interest.
        Thus the interesting condition is:
        \begin{align*}
          \omega^2 = |\mathbf{k}|^2
        \end{align*}
      \item
        The electromagnetic stress tensor is required from the potential:
        \begin{align*}
          \tensor{F}{_\mu_\nu} &= \nabla_\mu A_\nu - \nabla_\nu A_\mu\\
                               &= ik_\mu a_\nu \exp(ik_\rho x^\rho) - ik_\nu a_\mu\exp(ik_\rho x^\rho)\\
                               &= i(k_\mu a_\nu - k_\nu a_\mu) \exp(ik_\rho x^\rho)
        \end{align*}
        Now for the electric field:
        \begin{align*}
          E^i = \tensor{F}{_i_0} &= i(k_i a_0 - k_0 a_i) \exp[ik_\rho x^\rho]\\
                                 &= - i(k^i \phi + \omega a^i) \exp[i(-\omega t + \mathbf{k}\cdot \mathbf{x})]\\
        \end{align*}
        Now dotting this with $\mathbf{n} = \mathbf{k} / k$ and noting no appearance of the metric since this is the spacelike hypersurface:
        \begin{align*}
          \mathbf{E} \cdot \mathbf{n} =  E^i k_i / k &=  - i(k^i k_i\omega + \omega a^i k_i) \exp[ik_\rho x^\rho] / k\\
                                                     &=  - i(|\mathbf{k}|^2 \phi + \omega \mathbf{a} \cdot \mathbf{k}) \exp[ik_\rho x^\rho] / k\\
                                                     &=  - i(\omega^2 \phi + \omega^2 \phi) \exp[ik_\rho x^\rho] / k\\
                                                     &= 0
        \end{align*}
        For the magnetic flux density:
        \begin{align*}
                      & \tensor{\epsilon}{^i^j_k} B^k = \tensor{F}{_i_j} \\
          \Rightarrow & \tensor{\epsilon}{^i^j^k}\tensor{\epsilon}{^i^j_l} B^l = \tensor{\epsilon}{^i^j^k} \tensor{F}{_i_j} \\
          \Rightarrow & 2\delta^k_l B^l = \tensor{\epsilon}{^i^j^k} \tensor{F}{_i_j} \\
          \Rightarrow & B^k = \frac{1}{2}\tensor{\epsilon}{^i^j^k} \tensor{F}{_i_j} 
        \end{align*}
        Now, calculating the components of the flux density is easier:
        \begin{align*}
          B^k &= \frac{1}{2}\tensor{\epsilon}{^i^j^k} \tensor{F}{_i_j} \\
              &= \frac{1}{2}\tensor{\epsilon}{^i^j^k} \left[i(k_i a_j - k_j a_i) \exp(ik_\rho x^\rho)\right] \\
%              &= \frac{i}{2}\exp(ik_\rho x^\rho)\left[\tensor{\epsilon}{^i^j^k} k_i a_j + \tensor{\epsilon}{^j^i^k}k_j a_i) \right] \\
%              &= \frac{i}{2}\exp(ik_\rho x^\rho)\left[\tensor{\epsilon}{^i^j^k} k_i a_j + \tensor{\epsilon}{^i^j^k}k_i a_j) \right] \\
              &= i\tensor{\epsilon}{^i^j^k} k_i a_j \exp(ik_\rho x^\rho) 
        \end{align*}
        Now:
        \begin{align*}
          \mathbf{B} \cdot \mathbf{n} =  B^i k_i / k &= \frac{1}{2}\tensor{\epsilon}{^j^k^i} \left[i(k_j a_k - k_k a_j) \exp(ik_\rho x^\rho)\right] k_i / k\\
                                                     &= \frac{i}{2}\exp(ik_\rho x^\rho) \left[\tensor{\epsilon}{^j^k^i}k_j a_k k_i - \tensor{\epsilon}{^j^k^i} k_k a_j k_i \right]/k \\
                                                     &= \frac{i}{2}\exp(ik_\rho x^\rho) \left[\tensor{\epsilon}{^j^k^i}k_j a_k k_i - \tensor{\epsilon}{^k^i^j} k_i a_k k_j \right]/k \\
                                                     &= \frac{i}{2}\exp(ik_\rho x^\rho) \left[\tensor{\epsilon}{^j^k^i}k_j a_k k_i - \tensor{\epsilon}{^j^k^i} k_i a_k k_j \right]/k \\
                                                     &= 0 
        \end{align*}
        Now for the cross product:
        \begin{align*}
          \mathbf{B} \times \mathbf{n} &= \tensor{\epsilon}{^i^j_k} B^k k_j / k \\
                                       &= \tensor{F}{_i_j} k_j / k \\
                                       &= i(k_i a_j - k_j a_i) \exp(ik_\rho x^\rho) k_j / k \\
                                       &= i(k_i \mathbf{a}\cdot \mathbf{k} - |\mathbf{k}|^2 a_i) \exp(ik_\rho x^\rho) / k \\
                                       &= i(k_i \omega \phi - \omega^2 a_i) \exp(ik_\rho x^\rho) / \omega \\
                                       &= i(k_i \phi - \omega a_i) \exp(ik_\rho x^\rho)\\
                                       &= E_i
        \end{align*}
      \item
        Now $A^\mu$ becomes $(0, \mathbf{A})$: so that:
        \begin{align*}
          E^i   &= - i(k^i \phi + \omega a^i) \exp[i(-\omega t + \mathbf{k}\cdot \mathbf{x})]\\
                &= - i\omega a^i \exp[i(-\omega t + \mathbf{k}\cdot \mathbf{x})]\\
                &=  a^i \frac{\partial}{\partial t} \exp[i(-\omega t + \mathbf{k}\cdot \mathbf{x})]\\
        \end{align*}
        Which in vector form is
        \begin{align*}
          \mathbf{E} = \dot{\mathbf{ A}}
        \end{align*}
        and:
        \begin{align*}
          \mathbf{B} = \mathbf{n} \times \mathbf{E} = \mathbf{n} \times \dot{\mathbf{ A}}
        \end{align*}
      \item
        Note that:
        \begin{align*}
          \frac{\partial}{\partial t} \varepsilon &= \mathbf{E} \cdot \frac{\partial \mathbf{E}}{\partial t} + \mathbf{B} \cdot \frac{\partial \mathbf{B}}{\partial t}\\
        \end{align*}          
        And using Maxwell's equations:
        \begin{align*}
          \nabla \cdot \mathbf{S} &=
          \nabla \cdot \left(\mathbf{E} \times \mathbf{B} \right) \\
                                                  &= (\nabla \times \mathbf{E}) \cdot \mathbf{B} - \mathbf{E} \cdot (\nabla \times \mathbf{B})\\
                                                  &= - \mathbf{B} \cdot \partial_t \mathbf{B} - \mathbf{E} \cdot \partial_t \mathbf{E} \\
                                                  &= - \partial_t \varepsilon
        \end{align*}          
        In this sense, the Poynting vector is the current associated with the energy $\epsilon$ (charge).
        The Poynting vector describes the instantaneous flow of energy out of a surface.
      \item
        This is just the rate of change of the total energy out of some bounding surface:
        \begin{align*}
          P = \frac{d E}{dt} &= - \frac{d}{dt} \int_V d^3x \epsilon\\
                             &= - \int_V d^3x \partial_t \epsilon\\
                             &= \int_V d^3x \nabla \cdot \mathbf{S}\\
                             &= \int_{\partial V} d \mathbf{a} \cdot \mathbf{S} d
        \end{align*}
        For a sphere:
        \begin{align*}
          d \mathbf{a} = \vec{s} R^2 \sin \theta d \theta  d \psi
        \end{align*}
        Where $\vec{s}$ is a unit vector normal to the sphere, an not:
        \begin{align*}
          P &= R^2\int_{\partial V} \vec{s} \cdot \mathbf{S}  \sin \theta d \theta  d \psi
        \end{align*}
      \item
        The Poynting vector here is:
        \begin{align*}
          \mathbf{S} &=
          \dot{\mathbf{A}} \times \left(\mathbf{n} \times \dot{\mathbf{A}}\right) \\
                                                  &= |\dot {\mathbf{A}}|^2\mathbf{n} - \left(\dot{\mathbf{A}} \cdot \mathbf{n}\right) \dot{\mathbf{A}}\\
                                                  &= |{\mathbf{E}}|^2\mathbf{n} - \left(\mathbf{E} \cdot \mathbf{n}\right) \dot{\mathbf{A}}\\
                                                  &= |{\mathbf{E}}|^2\mathbf{n} 
        \end{align*}          
        but note that since $\mathbf{n} \cdot \dot{\mathbf{A}} = 0$ and $\mathbf{n}| = 1$:
        \begin{align*}
          |\mathbf{B}|^2 = |\dot{\mathbf{A}} \times \mathbf{n}|^2 = |\dot{\mathbf{A}}|^2 = |\mathbf{E}|^2
        \end{align*}
        Thus,
        \begin{align*}
          \mathbf{S} 
                  &= \left(\frac{1}{2}|{\mathbf{E}}|^2 + \frac{1}{2}|{\mathbf{B}}|^2 \right)\mathbf{n} \\
                  &= \varepsilon\mathbf{n} 
        \end{align*}          
    \end{enumerate}
  \item
    \begin{enumerate}
      \item
        Looking at the denomenator of the integrand:
        \begin{align*}
          \frac{1}{|\mathbf{r} - \mathbf{r}'|} &= \frac{1}{|\mathbf{r}|^2 + |\mathbf{r}'|^2 - 2|\mathbf{r}||\mathbf{r}'|\cos \theta}\\
                                               &= \frac{1}{|\mathbf{r}|^2} - \frac{2 |\mathbf{r}'| - 2|\mathbf{r}|}{|\mathbf{r}|^4} + \mathcal{O}(\mathbf{r}'^2)
        \end{align*}
        
        Since $|\mathbf{r}| \gg |\mathbf{r}'|$ only the first term is significant, and
        \begin{align*}
          \frac{1}{|\mathbf{r} - \mathbf{r}'|} &\approx \frac{1}{|\mathbf{r}|^2}
        \end{align*}
        And the potential becomes:
        \begin{equation*}
          \mathbf{A}(t, \mathbf{r}) \approx \frac{1}{|r|} \int d^3 r' \mathbf{j}(t_R, \mathbf{r}')
        \end{equation*}
      \item
        Taylor expanding:
        \begin{align*}
          |\mathbf{r} - \mathbf{r}'| &= \sqrt{|\mathbf{r}|^2 + |\mathbf{r}'|^2 - 2 \mathbf{r} \cdot \mathbf{r}'}\\
                                     &= \sqrt{|\mathbf{r}|^2 + |\mathbf{r}'|^2 - 2 |\mathbf{r}| |\mathbf{r}'| \cos \theta}\\
                                     &= \sqrt{|\mathbf{r}|^2} + \frac{|\mathbf{r}'|^2}{\sqrt{|\mathbf{r}|^2}} - \frac{|\mathbf{r}| |\mathbf{r}'| \cos \theta}{\sqrt{|\mathbf{r}|^2}} + \mathcal{O}\left(\frac{1}{|\mathbf{r}|^2}\right)\\
                                     &\approx |\mathbf{r}|  - \mathbf{n}_\mathbf{r} \cdot \mathbf{r}'
        \end{align*}
        Now, the retarded time becomes:
        \begin{align*}
          t_R &\approx t - |\mathbf{r}|  + \mathbf{n}_\mathbf{r} \cdot \mathbf{r}'\\
              &=       t'  + \mathbf{n}_\mathbf{r} \cdot \mathbf{r}'
        \end{align*}
        Now the current can be approximated:
        \begin{align*}
          \mathbf{j}(t_R) &\approx \mathbf{j}(t' + \mathbf{n}_\mathbf{r} \cdot \mathbf{r}')\\
                          &\approx \mathbf{j}(t')  + \frac{d \mathbf{j}}{dt'}(t') \mathbf{n}_\mathbf{r} \cdot \mathbf{r}'
        \end{align*}
        Upon substituting these expressions into the expression for the vector potential:
        \begin{align*}
          \mathbf{A}(t,\mathbf{r}) &\approx \frac{1}{|\mathbf{r}|} \int d^3r' \left(\mathbf{j}(t')  + \frac{d \mathbf{j}}{dt'}(t') \mathbf{n}_\mathbf{r} \cdot \mathbf{r}'\right)\\
                                   &=       \frac{1}{|\mathbf{r}|} \int d^3r' \mathbf{j}(t')  + \frac{1}{|\mathbf{r}|} \int d^3r'\frac{d \mathbf{j}}{dt'}(t') (\mathbf{n}_\mathbf{r} \cdot \mathbf{r}')\\
                                   &=       \frac{1}{|\mathbf{r}|} \int d^3r' \mathbf{j}(t')  + \frac{1}{|\mathbf{r}|} \frac{\partial }{\partial t'}\int d^3r'(\mathbf{n}_\mathbf{r} \cdot \mathbf{r}') \mathbf{j}(t') 
        \end{align*}
      \item
        Only taking the first part of the vector potential:
        \begin{align*}
          \mathbf{A}(t,\mathbf{r}) &= \frac{1}{|\mathbf{r}|} \int d^3r' \mathbf{j}(t') \\
                                   &= \frac{1}{|\mathbf{r}|} \int d^3r' \mathbf{j} \cdot \nabla'(\mathbf{r}') \\
                                   &= -\frac{1}{|\mathbf{r}|} \int d^3r' \nabla' \cdot \mathbf{j} \mathbf{r}' + \int \mathbf{r'} \mathbf{j} \cdot d \mathbf{a}\\
                                   &= \frac{1}{|\mathbf{r}|} \int d^3r' \dot \rho \mathbf{r}'  \\
                                   &= \frac{1}{|\mathbf{r}|} \frac{\partial}{\partial t}\int d^3r' \rho \mathbf{r}'  \\
                                   &= \frac{\dot{\mathbf{d}}}{|\mathbf{r}|}
        \end{align*}
        Where we used the fact that there are no currents just outside the region where the charges are contained.
        With
        \begin{equation*}
          \dot{\mathbf{d}} = \int d^3r' \rho \mathbf{r}'
        \end{equation*}

        For some point charges the charge density is:
        \begin{equation*}
          \rho = \sum_\alpha e_\alpha \delta^{(3)}(\mathbf{r} - \mathbf{r}_\alpha)
        \end{equation*}
        And the dipole moment becomes:
        \begin{align*}
          \dot{\mathbf{d}}(t') &= \sum_\alpha e_\alpha \int d^3r'\delta^{(3)}(\mathbf{r}' - \mathbf{r}'_\alpha)  \mathbf{r}'\\
                               &= \sum_\alpha e_\alpha \mathbf{r}'_\alpha
        \end{align*}
      \item
        The Poynting vector for the dipole radiation is:
        \begin{align*}
          \mathbf{S} &= \mathbf{E} \times \mathbf{B}\\
                     &= \dot{\mathbf{A}} \times \left(\mathbf{n} \times \dot{\mathbf{A}}\right)\\
                     &= \left(\dot{\mathbf{A}}\right)^2\mathbf{n} - \left(\mathbf{n} \cdot \dot{\mathbf{A}}\right) \dot{\mathbf{A}}
        \end{align*}
        Where $\mathbf{n}$ is the direction of the radiation, which will approximately be: $\mathbf{r}/|\mathbf{r}|$ in the large $r$ limit.
        Now to find the power dissipated by dipole accross a surface, the most convenient surface is a sphere of radius $r$, and the coordinates are chosen such that $\dot{\mathbf{A}}$ is along the $z$-axis.
        The coordinate system is chosen so that the angle btween $\vec{s} = \mathbf{r}/\mathbf{r}$ and $\dot{\mathbf{A}}$ is simply $\theta$.
        Then:
        \begin{align*}
          P &= r^2\int_{S^2} \vec{s} \cdot \mathbf{S} \sin\theta d\theta d \psi\\
            &= r^2\int_{S^2} \mathbf{n} \cdot \left[\left(\dot{\mathbf{A}}\right)^2\mathbf{n} - \left(\mathbf{n} \cdot \dot{\mathbf{A}}\right) \dot{\mathbf{A}} \right]\sin\theta d\theta d \psi\\
            &= r^2\int_{S^2} \left[\left(\dot{\mathbf{A}}\right)^2 - \left(\mathbf{n} \cdot \dot{\mathbf{A}}\right)^2  \right]\sin\theta d\theta d \psi\\
            &= r^2\left|\dot{\mathbf{A}}\right|^2\int_{S^2} \sin\theta d\theta d \psi - r^2\int_{S^2}\left|\dot{\mathbf{A}}\right|^2\cos^2\theta \sin\theta d\theta d \psi\\
            &= r^2\left|\dot{\mathbf{A}}\right|^2 4\pi - 2\pi r^2\left|\dot{\mathbf{A}}\right|^2\left.\frac{\cos^3\theta}{3}\right|_0^\pi \\
            &= r^2\left|\dot{\mathbf{A}}\right|^2 4\pi - 2\pi r^2\left|\dot{\mathbf{A}}\right|^2\frac{2}{3} \\
            &= 4\pi r^2\left(\left|\dot{\mathbf{A}}\right|^2  - \left|\dot{\mathbf{A}}\right|^2\frac{1}{3}\right)
        \end{align*}
        Now along the surface of constant $\left|\mathbf{r}\right|$:
        \begin{equation*}
          \dot{\mathbf{A}} = \frac{d}{dt'} \frac{\dot{\mathbf{d}}(t')}{|\mathbf{r}|} = \frac{\ddot{\mathbf{d}}(t')}{|\mathbf{r}|}
        \end{equation*}
        Using this in the expression for $P$:
        \begin{align*}
          P = 4\pi \frac{2}{3} |\ddot{\mathbf{d}}(t')|^2
        \end{align*}
      \item
        \begin{align*}
          \ddot{\mathbf{d}}(t') = \sum_{\alpha = 1}^{N} e_\alpha \ddot{\mathbf{r}}'_\alpha
        \end{align*}
        But, Newton's second law gives $\mathbf{F} = m\ddot{\mathbf{r}}$, and the force on the charge labelled by $\beta$ charge is (in a convenient choice of units):
        \begin{align*}
          & \mathbf{F}_\alpha= \sum_{\beta = 1, \beta \neq \alpha}^N \frac{e_\alpha e_\beta}{|\mathbf{r}_\alpha - \mathbf{r}_\beta|^2} \hat{\mathbf{r}}_{\alpha\beta}\\
          \Rightarrow & \ddot{\mathbf{r}}_{\alpha} = \sum_{\beta = 1, \beta \neq \alpha }^N \frac{e_\alpha e_\beta}{m_\alpha|\mathbf{r}_\alpha - \mathbf{r}_\beta|^2} \hat{\mathbf{r}}_{\alpha\beta} = \sum_{\beta = 1, \beta \neq \alpha }^N \frac{k e_\beta }{|\mathbf{r}_\alpha - \mathbf{r}_\beta|^2} \hat{\mathbf{r}}_{\alpha\beta}
        \end{align*}
        Where $k$ is the ratio of acharge to mass and $\hat{\mathbf{r}}_{\alpha\beta} = \frac{\mathbf{r}_\alpha - \mathbf{r}_\beta}{|\mathbf{r}_\alpha - \mathbf{r}_\beta|}$.
      \item
        \begin{enumerate}
          \item
            The integral expression for the power essentialy defines the flux of the field $\mathbf{S}$, but the flux of a field produced by a contained source does not depend on the geometry of the surface, just as in Gauss' law.
            Thus, the power not depending on the radius of the surface is natural.
          \item
            This formula for the power depends on the second derivative of the dipole moment.
            For a few dicrete charges, the charge of each does not depend on time, so the time dependence comes from the position of the charge $\mathbf{r}'_\alpha$ and the second derivative of this gives the charge's acceleration.
            Thus, if the charge has no acceleration, it does not radiate any power.
          \item
            Now for the dipole term:
            \begin{align*}
              \ddot{\mathbf{d}}(t') &= \sum_{\alpha = 1}^{N} e_\alpha \sum_{\beta = 1, \beta \neq \alpha }^N \frac{k e_\beta }{|\mathbf{r}_\alpha - \mathbf{r}_\beta|^2} \hat{\mathbf{r}}_{\alpha\beta}\\
                                    &= \sum_{\alpha = 1}^{N} \sum_{\beta = 1, \beta \neq \alpha }^N \frac{k e_\alpha e_\beta }{|\mathbf{r}_\alpha - \mathbf{r}_\beta|^2} \hat{\mathbf{r}}_{\alpha\beta}\\
    %                                &= \frac{1}{2}\sum_{\alpha = 1}^{N} \sum_{\beta = 1, \beta \neq \alpha }^N \frac{k e_\alpha e_\beta }{|\mathbf{r}_\alpha - \mathbf{r}_\beta|^2} \hat{\mathbf{r}}_{\alpha\beta} + \frac{1}{2}\frac{1}{2}\sum_{\alpha = 1}^{N} \sum_{\beta = 1, \beta \neq \alpha }^N \frac{k e_\alpha e_\beta }{|\mathbf{r}_\alpha - \mathbf{r}_\beta|^2} \hat{\mathbf{r}}_{\alpha\beta}\\
    %                                &=\frac{1}{2} \sum_{\alpha = 1}^{N} \sum_{\beta = 1, \beta \neq \alpha }^N \frac{k e_\alpha e_\beta }{|\mathbf{r}_\alpha - \mathbf{r}_\beta|^2} \hat{\mathbf{r}}_{\alpha\beta} + \frac{1}{2} \sum_{\beta= 1}^{N} \sum_{\alpha = 1, \beta \neq \alpha }^N \frac{k e_\beta e_\alpha }{|\mathbf{r}_\beta - \mathbf{r}_\alpha|^2} \hat{\mathbf{r}}_{\beta\alpha}\\ 
    %                                &= \sum_{\alpha = 1}^{N} \sum_{\beta = 1}^{\alpha - 1} \frac{k e_\alpha e_\beta }{|\mathbf{r}_\alpha - \mathbf{r}_\beta|^2} \hat{\mathbf{r}}_{\alpha\beta} + \sum_{\alpha = 1}^{N} \sum_{\beta = \alpha + 1}^N \frac{k e_\alpha e_\beta }{|\mathbf{r}_\alpha - \mathbf{r}_\beta|^2} \hat{\mathbf{r}}_{\alpha\beta}\\
    %                                &= \sum_{\beta = 1}^{N -1} \sum_{\alpha = 1}^{\beta + 1} \frac{k e_\alpha e_\beta }{|\mathbf{r}_\alpha - \mathbf{r}_\beta|^2} \hat{\mathbf{r}}_{\alpha\beta} + \sum_{\alpha = 1}^{N} \sum_{\beta = \alpha + 1}^N \frac{k e_\alpha e_\beta }{|\mathbf{r}_\alpha - \mathbf{r}_\beta|^2} \hat{\mathbf{r}}_{\alpha\beta}\\
            \end{align*}
            Notice that the sum here is the sum of all off diagonal entries in an $N \times N$ matrix.
            This sum can be done over the rows or the columns first:
            \begin{equation*}
              \sum_{\alpha = 1}^{N} \sum_{\beta = 1, \beta \neq \alpha }^N \frac{k e_\alpha e_\beta }{|\mathbf{r}_\alpha - \mathbf{r}_\beta|^2} \hat{\mathbf{r}}_{\alpha\beta} = \sum_{\beta = 1}^{N} \sum_{\alpha = 1, \beta \neq \alpha }^N \frac{k e_\alpha e_\beta }{|\mathbf{r}_\alpha - \mathbf{r}_\beta|^2} \hat{\mathbf{r}}_{\alpha\beta}
            \end{equation*}
            Thus after splitting the sum, and in the second line relabelling the indices in the second sum:
            \begin{align*}
              \ddot{\mathbf{d}}(t') &= \frac{1}{2}\sum_{\alpha = 1}^{N} \sum_{\beta = 1, \beta \neq \alpha }^N \frac{k e_\alpha e_\beta }{|\mathbf{r}_\alpha - \mathbf{r}_\beta|^2} \hat{\mathbf{r}}_{\alpha\beta} + \frac{1}{2}\sum_{\beta = 1}^{N} \sum_{\alpha = 1, \beta \neq \alpha }^N \frac{k e_\alpha e_\beta }{|\mathbf{r}_\alpha - \mathbf{r}_\beta|^2} \hat{\mathbf{r}}_{\alpha\beta}\\
                                    &= \frac{1}{2}\sum_{\alpha = 1}^{N} \sum_{\beta = 1, \beta \neq \alpha }^N \frac{k e_\alpha e_\beta }{|\mathbf{r}_\alpha - \mathbf{r}_\beta|^2} \hat{\mathbf{r}}_{\alpha\beta} + \frac{1}{2}\sum_{\alpha = 1}^{N} \sum_{\beta = 1, \beta \neq \alpha }^N \frac{k e_\beta e_\alpha}{|\mathbf{r}_\beta - \mathbf{r}_\alpha|^2} \hat{\mathbf{r}}_{\beta\alpha}\\
                                    &= \frac{1}{2}\sum_{\alpha = 1}^{N} \sum_{\beta = 1, \beta \neq \alpha }^N \frac{k e_\alpha e_\beta }{|\mathbf{r}_\alpha - \mathbf{r}_\beta|^2} \hat{\mathbf{r}}_{\alpha\beta} - \frac{1}{2}\sum_{\alpha = 1}^{N} \sum_{\beta = 1, \beta \neq \alpha }^N \frac{k e_\beta e_\alpha}{|\mathbf{r}_\alpha - \mathbf{r}_\beta|^2} \hat{\mathbf{r}}_{\alpha\beta}\\
                                    &= 0
            \end{align*}
          \item
            A system where all the charges have the same charge to mass ratio is a bit artificial, because all the charges will have the same sign of there charge.
            This is not typical of a natural system, because a bunch of charges of the same sign will tend to disperse without an external force holding them together.            
            Also not many particles have the same charge to mass ratio unless they are the same type of particle and typically different types of particles group together.
        \end{enumerate}
    \end{enumerate}
  \item
    \begin{enumerate}
      \item
      For the second integral in the expression for $\mathbf{A}$ in component form:
      \begin{align}
        \int d^3r' j^i {r'}^j n_j &= \int j^i  {r'}^j n_j + \frac{1}{2}j^j n_j{r'}^i - \frac{1}{2}j^j n_j{r'}^i d^3r' \nonumber \\
                                   &= \int j^i  {r'}^j n_j + \frac{1}{2}j^j n_j{r'}^i - \frac{1}{2}j^j n_j{r'}^i d^3r' \nonumber\\
                                   &=\frac{1}{2} \int j^i  {r'}^j n_j + j^j n_j{r'}^id^3r' + \frac{1}{2}\int j^i  {r'}^j n_j - j^j n_j{r'}^i d^3r' \label{eq:vecPot}
      \end{align}
      For the first integral in in Eq.~\ref{eq:vecPot}:
      \begin{align*}
%        \frac{1}{2} \int j^i  {r'}^j n_j + j^j n_j{r'}^id^3r' &= \frac{1}{2} \int j^i  {r'}^j n_j d^3r' + \frac{1}{2} \int j^j n_j{r'}^id^3r'\\
%                                                              &= \frac{1}{2} \int j^k  {r'}^j n_j \partial_k {r'}^i d^3r' + \frac{1}{2} \int j^j n_j{r'}^id^3r'\\
%                                                              &= \frac{1}{2} \int {r'}^i \partial_k(j^k  {r'}^j n_j)   d^3r' + \frac{1}{2} \int j^j n_j{r'}^id^3r'\\
%                                                              &= -\frac{1}{2} \int {r'}^i \nabla \cdot \mathbf{j}  {r'}^j n_j + {r'}^i j^k n_j \partial'_k {r'}^jd^3r' + \frac{1}{2} \int j^j n_j{r'}^id^3r'\\
%                                                              &= \frac{1}{2} \int \dot \rho {r'}^i {r'}^j n_j + {r'}^i j^j n_j d^3r' + \frac{1}{2} \int j^j n_j{r'}^id^3r'\\
%%                                                              &= \frac{1}{2} \int \dot \rho {r'}^i {r'}^j n_j d^3r' + \frac{1}{2} \int j^j n_j({r'}^i + 1)d^3r'\\
%                                                              &= \frac{1}{2} \int \dot \rho {r'}^i {r'}^j n_j + {r'}^i j^j n_j d^3r' + \frac{1}{2} \int j^j n_j{r'}^id^3r'\\
%                                                              &= \frac{1}{2} \int \dot \rho {r'}^i {r'}^j n_j d^3r' +  \int j^j n_j{r'}^id^3r'\\
%                                                              &= \frac{1}{2} \int \dot \rho {r'}^i {r'}^j n_j d^3r' +  n_j\int j^j {r'}^i d^3r'\\
%                                                              &= \frac{1}{2} \int \dot \rho {r'}^i {r'}^j n_j d^3r' +  n_j\int j^k \partial_k{r'}^j {r'}^l\partial_l{r'}^i d^3r'\\
%                                                              &= \frac{1}{2} \int \dot \rho {r'}^i {r'}^j n_j d^3r' +  n_j\int j^j {r'}^k\partial'_k {r'}^i d^3r'\\
%                                                              &= \frac{1}{2} \int \dot \rho {r'}^i {r'}^j n_j d^3r' +  n_j\int j^k {r'}^i\partial_k d^3r'\\
%                                                              &= \frac{1}{2} \int \dot \rho {r'}^i {r'}^j n_j d^3r' +  n_j\int j^j {r'}^k \partial_k {r'}^i d^3r'\\
        \frac{1}{2} \int_{V'} j^i  {r'}^j n_j + j^j n_j{r'}^id^3r' &= \frac{1}{2} \int_{V'} j^i  {r'}^j n_j d^3r' + \frac{1}{2} \int_{V'} j^j n_j{r'}^id^3r'\\
                                                              &= \frac{1}{2} n_j\int_{V'} j^k  {r'}^j \partial_k {r'}^i d^3r' + \frac{1}{2} n_j\int j^k {r'}^i\partial_k {r'}^j d^3r'\\
                                                              &= \frac{1}{2} n_j\int_{V'} j^k  {r'}^j \partial_k {r'}^i  +  j^k {r'}^i\partial_k {r'}^j d^3r'\\
                                                              &= \frac{1}{2} n_j\int_{V'} j^k  ({r'}^j\partial'_k {r'}^i  +  {r'}^i\partial'_k {r'}^j) d^3r'\\
                                                              &= \frac{1}{2} n_j\int_{V'} j^k \partial'_k ({r'}^j {r'}^i) d^3r'\\
                                                              &= \frac{1}{2} n_j\int_{V'} \partial'_k (j^k {r'}^j {r'}^i) d^3r' - n_j\int_{V'} (\partial'_k j^k) {r'}^j {r'}^i d^3r'\\
                                                              &= \frac{1}{2} n_j\int_{\partial V'} \partial'_k n_k j^k {r'}^j {r'}^i d^3r' + n_j\int_{V'} \dot \rho {r'}^j {r'}^i d^3r'  \\
                                                              &= \frac{1}{2} \int_{V'} \dot \rho n_j{r'}^j {r'}^i d^3r'\\
                                                              &= \frac{1}{2} \frac{\partial}{\partial t} \int_{V'} \rho n_j{r'}^j {r'}^i d^3r'\\
                                                              &= \frac{1}{6} \frac{\partial}{\partial t} \left[\int_{V'} 3 \rho {r'}^j {r'}^i d^3r'\right]n_j\\
                                                              &:= \frac{1}{6} \frac{\partial}{\partial t} Q \mathbf{n} 
%        \frac{1}{2} \int j^i  {r'}^j n_j + j^j n_j{r'}^id^3r' &= \frac{1}{2} \int j^i  {r'}^j n_j d^3r' + \frac{1}{2} \int j^j n_j{r'}^id^3r'\\
%                                                              &= \frac{1}{2} \int j^k  {r'}^j n_j \partial_k {r'}^i d^3r' + \frac{1}{2} \int j^j n_j{r'}^id^3r'\\
%                                                              &= -\frac{1}{2} \int (\partial_k j^k)  {r'}^j n_j {r'}^i d^3r' + \frac{1}{2} \int j^j n_j{r'}^id^3r'\\
%                                                              &= \frac{1}{2} \int \dot \rho  {r'}^j n_j {r'}^i d^3r' + \frac{1}{2} \int j^j n_j{r'}^id^3r'\\
      \end{align*}
      With
      \begin{equation*}
        Q^{ij} = \int_{V'} 3 \rho {r'}^j {r'}^i d^3r'
      \end{equation*}
      Which in the discrete form becomes:
      \begin{equation*}
        Q^{ij} = \sum_\alpha 3 e_\alpha {r'}^i_\alpha {r'}^j_\alpha 
      \end{equation*}
      Missing the term $\delta^{ij}r^2$ which makes the tensor traceless, maybe this happens with a specific coordinate choice or by exploiting gauge freedom.

      After writing the second integral in vector form:
      \begin{align*}
        \frac{1}{2}\int \mathbf{j}  (\mathbf{n}\cdot \mathbf{r'}) - (\mathbf{j}\cdot \mathbf{n})\mathbf{r'} d^3r'
      \end{align*}
      looking at it for too much time, one sees the triple product structure:
      \begin{align*}
        \int d^3r' j^i {r'}^j n_j &= \frac{1}{2}\int (\mathbf{n} \times (\mathbf{j} \times \mathbf{r}'))^i d^3r'\\
                                  &= \frac{1}{2}\left[\int \mathbf{r}' \times \mathbf{j}d^3r' \right]\times \mathbf{n}
      \end{align*}
      and define:
      \begin{equation*}
         \mathbf{m} = \frac{1}{2}\int \mathbf{r}' \times \mathbf{j}d^3r'
      \end{equation*}
      In the discrete case for a single charge:
      \begin{equation*}
        \mathbf{j}_{\text{q}} = q \mathbf{v} \leadsto \mathbf{j} = \sum_\alpha e_\alpha \delta^{3}(\mathbf{x} - \mathbf{x}_\alpha)\mathbf{v}_\alpha
      \end{equation*}
      So that:
      \begin{equation*}
         \mathbf{m} = \frac{1}{2}\sum_\alpha e_\alpha\mathbf{r}'_\alpha \times \mathbf{v}_\alpha
      \end{equation*}

      \item
        The Poynting vector is again:
        \begin{align*}
          \mathbf{S} &= \mathbf{E} \times \mathbf{B}\\
                     &= \dot{\mathbf{A}} \times \left(\mathbf{n} \times \dot{\mathbf{A}}\right)\\
                     &= \left(\dot{\mathbf{A}}\right)^2\mathbf{n} - \left(\mathbf{n} \cdot \dot{\mathbf{A}}\right) \dot{\mathbf{A}}
        \end{align*}
        Writing $\dot{\mathbf{A}}$ in index notation:
        \begin{equation*}
          \dot{A}^i = \frac{1}{r} \left(\ddot{d}^i + \frac{\dddot{Q}^{ij}n_j}{6} + \tensor{\epsilon}{^i_j_k}\dot m^j n^k\right)
        \end{equation*}
        And $\mathbf{n} \times \dot{\mathbf{A}} $:
        \begin{align*}
          \langle n_i n_j\rangle = \frac{1}{3}\delta_{ij}
        \end{align*}
        \begin{align*}
          \langle n_i n_jn_kn_l\rangle = \frac{1}{15}(\delta_{ij}\delta_{kl} + \delta_{ik}\delta_{jl} + \delta_{il}\delta_{jk})
        \end{align*}
        \begin{equation*}
          \tensor{\epsilon}{^i_j_k}n^j \dot {A}^k = \frac{1}{r} \tensor{\epsilon}{^i_j_k}n^j  \left(\ddot{d}^j + \frac{\dddot{Q}^{jl}n_l}{6} + \tensor{\epsilon}{^j_l_m}\dot m^l n^m\right)
        \end{equation*}
        Now:
        And $\dot{\mathbf{A}} \times (\mathbf{n} \times \dot{\mathbf{A}})$:
        And $\dot{\mathbf{A}}^2$:
        \begin{align*}
          &\frac{1}{r^2}\left(\ddot{d}^i + \frac{\dddot{Q}^{ij}n_j}{6} + \tensor{\epsilon}{^i_j_k}\ddot m^j n^k\right) \left(\ddot{d}_i + \frac{\tensor{\dddot{Q}}{_i^k}n_k}{6} + \tensor{\epsilon}{_i_l_m}\ddot m^l n^m\right)\\
          = &\frac{1}{r^2}\left(\ddot{d} \cdot \ddot{d} + \frac{\dddot{Q}^{ij}n_j}{6} \frac{\tensor{\dddot{Q}}{_i^k}n_k}{6} + \tensor{\epsilon}{^i_j_k}\dot m^j n^k \tensor{\epsilon}{_i_l_m}\dot m^l n^m\right) \\
            &\quad + \frac{1}{r^2}\left(\frac{\ddot{d}^i \dddot{Q}^{ij}n_j}{3} + \frac{ \tensor{\epsilon}{_i_l_m}\dddot{Q}^{ij}n_j \ddot m^l n^m}{6} + \frac{ \tensor{\epsilon}{^i_j_k}\tensor{\dddot{Q}}{_i^l}n_l \ddot m^j n^k}{6} \right)\\
          = &\frac{1}{r^2}\left(\ddot{d} \cdot \ddot{d} + \frac{\dddot{Q}^{ij}n_j}{6} \frac{\tensor{\dddot{Q}}{_i^k}n_k}{6} + \tensor{\epsilon}{^i_j_k}\dot m^j n^k \tensor{\epsilon}{_i_l_m}\dot m^l n^m\right) \\
            &\quad + \frac{1}{r^2}\left(\frac{\ddot{d}^i \dddot{Q}^{ij}n_j}{3} + \frac{ \tensor{\epsilon}{^i_l_m}\dddot{Q}^{ij}n_j \ddot m^l n^m}{3}  \right)
        \end{align*}
        And $\mathbf{n} \cdot \mathbf{n} = 1$ so this value must be integrated over a sphere.
        For which the quanitity
        \begin{align*}
          \langle n_i n_j\rangle %= \frac{1}{3}\delta_{ij}
        \end{align*}
        is required.
        Look at the function:
        \begin{equation*}
          z(\mathbf{r}) = \int_{S^2} d \Omega e^{n^i r_i} = \int_{S^2} d \Omega e^{r \cos \theta}= 2 \pi \frac{e^{r} - e^{-r}}{r} = 4 \pi\left(1 +\frac{r^ir_i}{6}+\frac{(r^ir_i)^2}{120}\right) + \mathcal{O}(r^6)
        \end{equation*}
        And note:
        \begin{equation*}
          \left.\frac{\partial}{\partial r_i}z(\mathbf{r})\right|_{r = 0} = \int_{S^2} n_i d \Omega \quad \text{and} \quad \left.\frac{\partial}{\partial r_i}\frac{\partial}{\partial r_j}z(\mathbf{r})\right|_{r = 0} = \int_{S^2} n_i n_j d \Omega
        \end{equation*}
        Now:
        \begin{equation*}
          \frac{\partial}{\partial r_i}z(\mathbf{r}) = 4 \pi\left(\frac{r^i}{3}+\frac{2(r^jr_j)2r^i}{120}\right) + \mathcal{O}(r^5)
        \end{equation*}
        And for $\mathbf{r} = 0$ this vanishes, but:
        \begin{equation*}
          \frac{\partial}{\partial r_j}\frac{\partial}{\partial r_i}z(\mathbf{r}) = 4 \pi\left(\frac{\delta^{ij}}{3}+\frac{(r^jr_j)\delta^{ij}}{30}+\frac{r^jr^i}{15}\right) + \mathcal{O}(r^4)
        \end{equation*}
        And evaluated at $\mathbf{r} = 0$ gives:
        \begin{equation*}
          \int_{S^2} d \Omega = 4 \pi\left(\frac{\delta^{ij}}{3}\right)
        \end{equation*}
        \begin{align*}
          \langle\dot A^2\rangle
          = &\frac{4\pi}{r^2}\left(\ddot{\mathbf{d}} \cdot \ddot{\mathbf{d}} + \frac{\dddot{Q}^{ij}\tensor{\dddot{Q}}{_i^k} \delta^{jk}}{108} + \frac{\tensor{\epsilon}{^i_j_k}\tensor{\epsilon}{_i_l_m}\dot m^j \dot m^l \delta^{mk}}{9}\right) \\
            &\quad + \frac{4\pi}{r^2}\left(\frac{ \tensor{\epsilon}{^i_l_m}\dddot{Q}^{ij}\delta^{jm} \dot m^l}{9} \right)\\
          = &\frac{4\pi}{r^2}\left(\ddot{\mathbf{d}} \cdot \ddot{\mathbf{d}} + \frac{\dddot{\mathbf{Q}}:\dddot{\mathbf{Q}}}{108} + \frac{\tensor{\epsilon}{^i_j_k}\tensor{\epsilon}{_i_l_k}\dot m^j \dot m^l}{3}\right) \\
            &\quad + \frac{4\pi}{r^2}\left(\frac{ \tensor{\epsilon}{^i_l_m}\dddot{Q}^{im} \ddot m^l}{9} \right)\\
        \end{align*}
        For the last term, $\tensor{\epsilon}{^i_l_m}$ is antisymmetric in $i$ and $m$, but $Q$ is symmetric in those indices, so the term is zero.
        \begin{align*}
          \langle\dot A^2\rangle
          = &\frac{4\pi}{r^2}\left(\ddot{\mathbf{d}} \cdot \ddot{\mathbf{d}} + \frac{\dddot{\mathbf{Q}}:\dddot{\mathbf{Q}}}{108} + \frac{2 \delta^{jl}\ddot m^j \ddot m^l}{3}\right) \\
          = &\frac{4\pi}{r^2}\left(\ddot{\mathbf{d}} \cdot \ddot{\mathbf{d}} + \frac{\dddot{\mathbf{Q}}:\dddot{\mathbf{Q}}}{108} + \frac{2}{3}\ddot{\mathbf{ m}}\cdot \ddot{\mathbf{ m}}\right) \\
          = &\frac{4\pi}{r^2}\left(\ddot{\mathbf{d}} \cdot \ddot{\mathbf{d}} + \frac{\dddot{\mathbf{Q}}:\dddot{\mathbf{Q}}}{108} + \frac{2}{3}\ddot{\mathbf{ m}}\cdot \ddot{\mathbf{ m}}\right) 
        \end{align*}
        And $\dot{\mathbf{A}} \cdot \mathbf{n}$:
        \begin{align*}
          \dot{A}^in_i = \frac{1}{r^2}\left(\ddot{d}^in_i + \frac{\tensor{\dddot{Q}}{^i^j}n_in_j}{6}\right)
        \end{align*}
        Then
        \begin{align*}
          (\dot{A}^in_i)^2 &= \frac{1}{r^2}\left(\ddot{d}^in_i + \frac{\tensor{\dddot{Q}}{^i^j}n_in_j}{6}\right)\left(\ddot{d}^ln_l + \frac{\tensor{\dddot{Q}}{^k^l}n_kn_l}{6}\right)\\
                           &= \frac{1}{r^2}\left(\ddot{d}^in_i \ddot{d}^ln_l + 2\ddot{d}^in_i\frac{\tensor{\dddot{Q}}{^k^l}n_kn_l}{6} + \frac{\tensor{\dddot{Q}}{^i^j}n_in_j\tensor{\dddot{Q}}{^k^l}n_kn_l}{36}\right)
        \end{align*}
        To integrate this over the sphere, the following are needed:
        \begin{equation*}
          \left.\frac{\partial}{\partial r_k}\frac{\partial}{\partial r_j}\frac{\partial}{\partial r_i}z(\mathbf{r})\right|_{r = 0} = \int_{S^2} n_i n_j n_k d \Omega\quad \text{and} \quad \left.\frac{\partial}{\partial r_l}\frac{\partial}{\partial r_k}\frac{\partial}{\partial r_j}\frac{\partial}{\partial r_i}z(\mathbf{r})\right|_{r = 0} = \int_{S^2} n_i n_j n_k n_l d \Omega
        \end{equation*}
        Now:
        \begin{equation*}
          \frac{\partial}{\partial r_k}\frac{\partial}{\partial r_j}\frac{\partial}{\partial r_i}z(\mathbf{r}) = 4 \pi\left(\frac{r^k\delta^{ij}}{30}+\frac{\delta^{jk}r^i + r^j\delta^{ik}}{30}\right) + \mathcal{O}(r^3)
        \end{equation*}
        Which is zero for $\mathbf{r} = 0$, but differentiating again gives:
        Integrating this over the sphere gives:
        \begin{equation*}
          \frac{\partial}{\partial r_l}\frac{\partial}{\partial r_k}\frac{\partial}{\partial r_j}\frac{\partial}{\partial r_i}z(\mathbf{r}) = 4 \pi\left(\frac{\delta^{lk}\delta^{ij}}{15}+\frac{\delta^{jk}\delta^{il} + \delta^{jl}\delta^{ik}}{15}\right) + \mathcal{O}(r^2)
        \end{equation*}
        Which gives for $\mathbf{r} = 0$:
        \begin{equation*}
          \int_{S^2} d\Omega n_i n_jn_kn_l = \frac{4\pi}{15}(\delta_{ij}\delta_{kl} + \delta_{ik}\delta_{jl} + \delta_{il}\delta_{jk})
        \end{equation*}
        \begin{align*}
          \int(\dot{A}^in_i)^2 d\Omega &= \frac{1}{r^2}\left(\ddot{d}^in_i + \frac{\tensor{\dddot{Q}}{^i^j}n_in_j}{6}\right)\left(\ddot{d}^ln_l + \frac{\tensor{\dddot{Q}}{^k^l}n_kn_l}{6}\right)\\
                                       &= \frac{1}{r^2}\left(\frac{1}{3}\ddot{d}^i \ddot{d}^l\delta_{il} + \frac{\tensor{\dddot{Q}}{^i^j}\tensor{\dddot{Q}}{^k^l}}{36}\frac{1}{15}(\delta_{ij}\delta_{kl} + \delta_{ik}\delta_{jl} + \delta_{il}\delta_{jk})\right)
        \end{align*}
        Notice that since the quadrupole moment is traceless so that:
        \begin{align*}
          \tensor{\dddot{Q}}{^i^j}\tensor{\dddot{Q}}{^k^l} \delta_{ij} \delta_{kl} = 0
        \end{align*}
        Which gives:
        \begin{align*}
          \int(\dot{A}^in_i)^2 d\Omega &= \frac{1}{r^2}\left(\frac{1}{3}\ddot{d}^i \ddot{d}^l\delta_{il} + \frac{1}{15}\frac{\tensor{\dddot{Q}}{^i^j}\tensor{\dddot{Q}}{^i^j} + \tensor{\dddot{Q}}{^i^j}\tensor{\dddot{Q}}{^j^i}}{36}\right)\\
                                       &= \frac{1}{r^2}\left(\frac{1}{3}\ddot{d}^i \ddot{d}^l\delta_{il} + \frac{2}{15}\frac{\tensor{\dddot{Q}}{^i^j}\tensor{\dddot{Q}}{^i^j}}{36}\right)\\
                                       &= \frac{1}{r^2}\left(\frac{1}{3}\ddot{\mathbf{d}}^i \cdot \ddot{\mathbf{d}}^l + \frac{1}{15}\frac{\dddot{\mathbf{Q}}:\dddot{\mathbf{Q}}}{18}\right)
        \end{align*}
        And subtracting this from the other term and multiplying by $r^2$ to get the power results in:
        \begin{equation*}
          P = 4\pi \left(\frac{2}{3}\ddot{\mathbf{d}} \cdot \ddot{\mathbf{d}} + \frac{\dddot{\mathbf{Q}}:\dddot{\mathbf{Q}}}{180} + \frac{2}{3}\ddot{\mathbf{ m}}\cdot \ddot{\mathbf{ m}}\right) 
        \end{equation*}

      \item
        \begin{align*}
          \dot{\mathbf{m}}^i &= \frac{1}{2} \sum_\alpha e_\alpha (\mathbf{r}_\alpha \times \dot{\mathbf{v}}_\alpha)^i + \frac{1}{2} \sum_\alpha e_\alpha (\dot{\mathbf{r}_\alpha} \times \mathbf{v}_\alpha)^i\\
                             &= \frac{1}{2} \sum_\alpha e_\alpha (\mathbf{r}_\alpha \times \mathbf{a}_\alpha)^i + \frac{1}{2} \sum_\alpha e_\alpha (\mathbf{v}_\alpha \times \mathbf{v}_\alpha)^i\\
                             &= \frac{1}{2} \sum_\alpha e_\alpha (\mathbf{r}_\alpha \times \mathbf{a}_\alpha)^i 
        \end{align*}

        And for a closed system of charges
            \begin{equation*}
              \sum_{\alpha = 1}^{N} \sum_{\beta = 1, \beta \neq \alpha }^N \frac{k e_\alpha e_\beta }{|\mathbf{r}_\alpha - \mathbf{r}_\beta|^2} \hat{\mathbf{r}}_{\alpha\beta} = \sum_{\beta = 1}^{N} \sum_{\alpha = 1, \beta \neq \alpha }^N \frac{k e_\alpha e_\beta }{|\mathbf{r}_\alpha - \mathbf{r}_\beta|^2} \hat{\mathbf{r}}_{\alpha\beta}
            \end{equation*}
        \begin{align*}
          \dot{\mathbf{m}}^i &= \frac{1}{2} \sum_\alpha \sum_{\beta = 1, \beta \neq \alpha }^N \frac{k e_\alpha e_\beta}{|\mathbf{r}_\alpha - \mathbf{r}_\beta|^2} (\mathbf{r}_\alpha \times \hat{\mathbf{r}}_{\alpha\beta})^i \\
                             &= \frac{1}{2} \sum_\alpha \sum_{\beta = 1, \beta \neq \alpha }^N \frac{k e_\alpha e_\beta}{|\mathbf{r}_\alpha - \mathbf{r}_\beta|^3} (\mathbf{r}_\alpha \times (\mathbf{r}_{\alpha} - \mathbf{r}_{\beta}))^i \\
                             &= -\frac{1}{2} \sum_\alpha \sum_{\beta = 1, \beta \neq \alpha }^N \frac{k e_\alpha e_\beta}{|\mathbf{r}_\alpha - \mathbf{r}_\beta|^3} (\mathbf{r}_\alpha \times \mathbf{r}_{\beta})^i 
        \end{align*}
        Now the cross product is antisymmetric in $\alpha$ and $\beta$, but the factor infront of it is symmetric.
        Thus, summing over the indices gives zero as can be shown in the same way as was done for the electric dipole.
        The quadrupole moment probably survives with all it's time derivatives.

    \end{enumerate}
  \item
    I worked alone, but got help from Jackie. Thanks for the hint about the spherical integrals.
\end{enumerate}

\end{document}
