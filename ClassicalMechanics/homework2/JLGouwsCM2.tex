\documentclass[12pt,a4]{article}
\usepackage{physics, amsmath,amsfonts,amsthm,amssymb, mathtools,steinmetz, gensymb, siunitx}	% LOADS USEFUL MATH STUFF
\usepackage{xcolor,graphicx}
\usepackage[left=45pt, top=20pt, right=45pt, bottom=45pt ,a4paper]{geometry} 				% ADJUSTS PAGE
\usepackage{setspace}
\usepackage{caption}
\usepackage{tikz}
\usepackage{pgf,tikz,pgfplots,wrapfig}
\usepackage{mathrsfs}
\usepackage{fancyhdr}
\usepackage{float}
\usepackage{array}
\usepackage{booktabs,multirow}
\usepackage{bm}
\usepackage{tensor}

\usetikzlibrary{decorations.text, calc}
\pgfplotsset{compat=1.7}

\usetikzlibrary{decorations.pathreplacing,decorations.markings}
\usepgfplotslibrary{fillbetween}

\newcommand{\vect}[1]{\boldsymbol{#1}}

\usepackage{hyperref}

%\usepackage[style= ACM-Reference-Format, maxbibnames=6, minnames=1,maxnames = 1]{biblatex}
%\addbibresource{references.bib}


\hypersetup{pdfborder={0 0 0},colorlinks=true,linkcolor=black,urlcolor=cyan,}
\allowdisplaybreaks
%\hypersetup{
%
%    colorlinks=true,
%
%    linkcolor=blue,
%
%    filecolor=magenta,      
%
%    urlcolor=cyan,
%
%    pdftitle={An Example},
%
%    pdfpagemode=FullScreen,
%
%    }
%}

\title{
\textsc{CM Homework 2}
}
\author{\textsc{J L Gouws}
}
\date{\today
\\[1cm]}



\usepackage{graphicx}
\usepackage{array}




\begin{document}
\thispagestyle{empty}

\maketitle

\begin{enumerate}
  \item
    \begin{enumerate}
      \item 
        When $u$ goes to zero, $v'$ should go to $v$ so the coefficient of $v$ is $+$.
        If $v$ is zero then in the moving $K'$ frame the object will move backwards with $-u$.
        In the limit as $u$ goes to $-c$ and $v$ goes to $c$, a plus sign in the denominator would result in $v'$ going to infinity which is not physical.
        But a minus sing would result in $(c + c) / 2$ in the limit, as expected.
      \item
        The infinitesmal displacement of the object in the $K'$ frame is $d{\bf x'} = {\bf v'}dt' = {\bf v'}_{\parallel} dt' + {\bf v'}_{\perp}dt' = d {\bf x}'_\parallel + d {\bf x}'_\perp$. Where:
        \begin{equation*}
          {\bf v'}_{\parallel} dt' = d {\bf x}'_\parallel \qquad {\bf v'}_{\perp}dt' = d {\bf x}'_\perp
        \end{equation*}
        Which is justified because we are working in a flat space.
        The infinitesmal displacement of the object in the $K$ frame is ${\bf v}dt = {\bf v}_{\parallel} dt + {\bf v}_{\perp}dt$.
        \begin{equation*}
          {\bf v}_{\parallel} dt = d {\bf x}_\parallel \qquad {\bf v}_{\perp}dt = d {\bf x}_\perp
        \end{equation*}
        Now under the Lorentz transformation, only the parallel directions are affected:
        \begin{align}
          d {\bf x}'_\parallel &= \gamma (u) (d {\bf x}_\parallel - {\bf u} dt) \label{eq:x1}\\
          d {\bf x}'_\perp &= d {\bf x}_\perp \label{eq:x2}\\
          d t' &= \gamma (u) (d t - \frac{{\bf u}}{c^2} \cdot d{\bf x}_\parallel)   \label{eq:x3}
        \end{align}
        Note that:
        \begin{align}
          d t' &= \gamma(u) (dt - \frac{u^2}{c^2} dt - \gamma(u)^{-1}\frac{{\bf u}}{c^2} \cdot d {\bf x}'_\parallel) \nonumber\\
               &= \gamma(u) (\gamma(u)^{-2} dt - \gamma(u)^{-1}\frac{{\bf u}}{c^2} \cdot d {\bf x}'_\parallel)\nonumber\\
          \Rightarrow & dt = \gamma(u) (dt' + \frac{{\bf u}}{c^2} \cdot d {\bf x}'_\parallel) \label{eq:x4}
        \end{align}
        Using this, Eq.~\ref{eq:x1} can also be inverted:\footnote{This is working out the inverse Lorentz transform, which seemed to defeat the point of doing the part with the four vector. There is no clear way to answer the question without inverting the transform, however.}
        \begin{align}
          d {\bf x}_\parallel = \gamma (u) (d {\bf x}'_\parallel + {\bf u} dt) \label{ex:vpara}
        \end{align}
        For $v_\parallel$ divide Eq.~\ref{ex:vpara} by Eq.~\ref{eq:x3}:
        \begin{align*}
          {\bf v}_\parallel = \frac{d {\bf x}_\parallel}{d t}  &= \frac{\gamma (u) (d {\bf x}'_\parallel - {\bf u} dt')}{\gamma (u) (d t' + \frac{\bf u}{c^2}\cdot d{\bf x}'_\parallel)}\\
                                                                  &= \frac{{\bf v}'_\parallel + {\bf u}' }{1 + \frac{\bf u}{c^2} \cdot {\bf v}'}\\
        \end{align*}
        For $v_\perp$ divide Eq.~\ref{eq:x2} by Eq.~\ref{eq:x4}:
        \begin{align*}
          {\bf v}_\perp = \frac{d {\bf x}_\perp}{d t} &= \frac{d {\bf x}'_\perp}{\gamma(u) (dt' + \frac{{\bf u}}{c^2} \cdot d {\bf x}'_\parallel)}\\
                                                      &= \frac{{\bf v}'_\perp}{\gamma (u) (1 + \frac{{\bf u}}{c^2}\cdot {\bf v}'_\parallel)}\\
                                                      &= \frac{{\bf v}'_\perp}{\gamma (u) (1 + \frac{{\bf u}\cdot {\bf v}'}{c^2})}
        \end{align*}
        As required.
      \item 
        Let $\mathbf{c}$ be the velocity vector for the light in the $K$ frame. 
        Then the speed squared in the $K$ frame is:
        \begin{align*}
          \mathbf{c}_\parallel^2 + \mathbf{c}_\perp^2 &= \frac{{c'}_\parallel^2 + 2 u {c'}_\parallel + u^2}{(1 + \mathbf{u} \cdot \mathbf{c'}/c^2)^2} + \frac{(1 - u^2 / c^2) {c'}_\perp^2}{(1 + \mathbf{u} \cdot \mathbf{c'}/c^2)^2}\\
                                                      &= \frac{{c'}_\parallel^2 + 2 u {c'}_\parallel' + u^2 + {c'}_\perp'^2 - u^2 {c'}_\perp^2/ c^2}{(1 + \mathbf{u} \cdot \mathbf{c'}/c^2)^2}\\
                                                       &= \frac{c^2 + 2 u {c'}_\parallel + u^2(c^2 - {c'}_\perp^2)/ c^2}{(1 + \mathbf{u} \cdot \mathbf{c}/c^2)^2}\\
                                                       &= \frac{c^2 + 2 u {c'}_\parallel + (u{c'}_\parallel)^2/ c^2}{(1 + \mathbf{u} \cdot \mathbf{c}/c^2)^2}\\
                                                       &= \frac{c^2 + 2\mathbf{u} \cdot \mathbf{c}  + (\mathbf{u} \cdot \mathbf{c})^2/ c^2}{(1 + \mathbf{u} \cdot \mathbf{c}/c^2)^2}\\
                                                       &= \frac{c^2 (1 + \mathbf{u} \cdot \mathbf{c}/c^2)^2}{(1 + \mathbf{u} \cdot \mathbf{c}/c^2)^2}\\
                                                       &= c^2
        \end{align*}
        So that the speed of the light beam is still c.
        Then:
        \begin{align*}
          {\bf u} \cdot {\bf c}_\parallel  = u c \cos \theta & = \frac{{\bf u} \cdot {\bf c'}_\parallel + u^2}{1 + \frac{{\bf u} \cdot {\bf c'}}{c^2}}\\
                                                             & = \frac{{\bf u} \cdot ({\bf c'}_\parallel + {\bf c'}_\perp) + 1}{1 + \frac{{\bf u} \cdot {\bf c'}}{c^2}}\\
                                                  & = \frac{uc \cos \theta ' + u^2}{1 + \frac{u \cos \theta '}{c}}\\
                                      \Rightarrow &\cos\theta = \frac{\frac{u}{c} + \cos \theta '}{1 + \frac{u}{c} \cos\theta ' }
        \end{align*}
      \item
        This is just a long and tedious algebra exercise combining the parts of $\mathbf{v}$ perpendicular and parallel to $\mathbf{u}$.
        \begin{align*}
          {\bf v} &= {\bf v}_\parallel + {\bf v}_\perp\\
                  &= \frac{1}{1 + \frac{{\bf u} \cdot {\bf v}'}{c^2}} \left( {\bf u} + {\bf v'}_\parallel + \gamma^{-1}(  u) {\bf v}'_\perp\right)\\
                  &= \frac{1}{1 + \frac{{\bf u} \cdot {\bf v}'}{c^2}} \left( {\bf u} + {\bf v'} - {\bf v'}_\perp + \gamma^{-1}(  u) {\bf v}'_\perp\right)\\
                  &= \frac{1}{1 + \frac{{\bf u} \cdot {\bf v}'}{c^2}} \left[ {\bf u} + {\bf v'} + \left(\gamma^{-1}(  u) - 1\right) {\bf v}'_\perp\right]\\
                  &= \frac{1}{1 + \frac{{\bf u} \cdot {\bf v}'}{c^2}} \left[ {\bf u} + {\bf v'} + \left(\gamma^{-1}(  u) - 1\right) \left({\bf v}' - \frac{{\bf u}\cdot{\bf v}'}{u^2} {\bf u}\right)\right]\\
                  &= \frac{1}{1 + \frac{{\bf u} \cdot {\bf v}'}{c^2}} \left[ {\bf u} + {\bf v'} - \frac{1}{u^2}\left(\gamma^{-1}(  u) - 1\right) \left(({\bf u}\cdot{\bf v}'){\bf u} - u^2{\bf v}'\right)\right]\\
                  &= \frac{1}{1 + \frac{{\bf u} \cdot {\bf v}'}{c^2}} \left[ {\bf u} + {\bf v'} - \frac{1}{u^2}\left(\gamma^{-1}(  u) - 1\right) {\bf u} \times \left({\bf u} \times {\bf v}'\right)\right]\\
                  &= \frac{1}{1 + \frac{{\bf u} \cdot {\bf v}'}{c^2}} \left[ {\bf u} + {\bf v'} - \frac{c^2}{u^2}\left(\gamma^{-1}(  u) - 1\right) \frac{\bf u}{c} \times \left(\frac{\bf u}{c} \times {\bf v}'\right)\right]\\
                  &= \frac{1}{1 + \frac{{\bf u} \cdot {\bf v}'}{c^2}} \left[ {\bf u} + {\bf v'} - \frac{c^2}{u^2}\left(\frac{\gamma (1 - \gamma)}{\gamma^{2}( u)} \right) \frac{\bf u}{c} \times \left(\frac{\bf u}{c} \times {\bf v}'\right)\right]\\
                  &= \frac{1}{1 + \frac{{\bf u} \cdot {\bf v}'}{c^2}} \left[ {\bf u} + {\bf v'} - \frac{c^2}{u^2}\left(\frac{\gamma (1 - \gamma)}{\frac{1}{1 - \frac{u^2}{c^2}}} \right) \frac{\bf u}{c} \times \left(\frac{\bf u}{c} \times {\bf v}'\right)\right]\\
                  &= \frac{1}{1 + \frac{{\bf u} \cdot {\bf v}'}{c^2}} \left[ {\bf u} + {\bf v'} + \left(\frac{\gamma (1 - \gamma)}{\frac{-\frac{u^2}{c^2}}{1 - \frac{u^2}{c^2}}} \right) \frac{\bf u}{c} \times \left(\frac{\bf u}{c} \times {\bf v}'\right)\right]\\
                  &= \frac{1}{1 + \frac{{\bf u} \cdot {\bf v}'}{c^2}} \left[ {\bf u} + {\bf v'} + \left(\frac{\gamma (1 - \gamma)}{ 1 - \frac{1}{1 - \frac{u^2}{c^2}}} \right) \frac{\bf u}{c} \times \left(\frac{\bf u}{c} \times {\bf v}'\right)\right]\\
                  &= \frac{1}{1 + \frac{{\bf u} \cdot {\bf v}'}{c^2}} \left[ {\bf u} + {\bf v'} + \left(\frac{\gamma (1 - \gamma)}{ 1 - \gamma^2} \right) \frac{\bf u}{c} \times \left(\frac{\bf u}{c} \times {\bf v}'\right)\right]\\
                  &= \frac{1}{1 + \frac{{\bf u} \cdot {\bf v}'}{c^2}} \left[ {\bf u} + {\bf v'} + \left(\frac{\gamma(u) }{ 1 + \gamma(u)} \right) \frac{\bf u}{c} \times \left(\frac{\bf u}{c} \times {\bf v}'\right)\right]\\
        \end{align*}
        The non-linearity in $\mathbf{u}$ is pretty clear because there is a double cross product, but this will be worked out to first order in $\frac{1}{c^2}$ later.
        The fraction with the Lorentz factor to lowest non-trivial order is:
        \begin{align*}
          \gamma (1 + \gamma)^{-1} &= \left(1 +\frac{1}{2}\frac{u^2}{c^2} + \mathcal{O}(1/c^3)\right)\left(\frac{1}{2} - \frac{1}{2}(-\frac{1}{2}(-u^2/c^2))+ \mathcal{O}(1/c^3)\right)\\
                                &\approx \left(1 +\frac{1}{2}\frac{u^2}{c^2}\right)\left(\frac{1}{2} - \frac{1}{4}u^2/c^2\right)\\
                                &= \frac{1}{2} -\frac{1}{8}\frac{u^4}{c^4}
        \end{align*}
        And this term is multliplying a term in $\frac{1}{c^2}$, so only ther first term is required.
        Also for the  front most factor:
        \begin{align*}
          \frac{1}{1 + \frac{\mathbf{u}\cdot\mathbf{v'}}{c^2}} = 1 - \frac{\mathbf{u} \cdot \mathbf{v'}}{c^2} + \mathcal{O}(1/c^3)
        \end{align*}
        the second term is of order $1/c^2$ so it will only mutltiply the sum of the velocities and not the cross product term.
        Combining these results in:
        \begin{equation*}
          \mathbf{u}\oplus \mathbf{v'} \approx \mathbf{u} + \mathbf{v}' + \frac{1}{2c^2}\mathbf{u}\times(\mathbf{u} \times \mathbf{v}') - \mathbf{u}\frac{\mathbf{u}\cdot\mathbf{v}'}{c^2} - \mathbf{v}'\frac{\mathbf{u}\cdot\mathbf{v}'}{c^2}
        \end{equation*}
        Now for linearity:
        \begin{align*}
          (\lambda \mathbf{u})\oplus (\lambda \mathbf{v'}) &\approx \lambda\mathbf{u} + \lambda \mathbf{v}' + \frac{\lambda^3}{2c^2}\mathbf{u}\times(\mathbf{u} \times \mathbf{v}') - \lambda^3\mathbf{u}\frac{\mathbf{u}\cdot\mathbf{v}'}{c^2} - \lambda^3\mathbf{v}'\frac{\mathbf{u}\cdot\mathbf{v}'}{c^2}\\
                                                           &=      \lambda\left(\mathbf{u} + \mathbf{v}' + \frac{\lambda^2}{2c^2}\mathbf{u}\times(\mathbf{u} \times \mathbf{v}') - \lambda^2\mathbf{u}\frac{\mathbf{u}\cdot\mathbf{v}'}{c^2} - \lambda^2\mathbf{v}'\frac{\mathbf{u}\cdot\mathbf{v}'}{c^2}\right)\\
                                                           &\neq \lambda(\mathbf{u}\oplus \mathbf{v'})
        \end{align*}
        For commutavity:
      \begin{align*}
          \mathbf{v'}\oplus \mathbf{u} \approx \mathbf{v}' + \mathbf{u} + \frac{1}{2c^2}\mathbf{v'}\times(\mathbf{v'} \times \mathbf{u}) - \mathbf{v}'\frac{\mathbf{v'}\cdot\mathbf{u}}{c^2} - \mathbf{u}\frac{\mathbf{v'}\cdot\mathbf{u}}{c^2}
      \end{align*}
      But:
      \begin{equation*}
        \mathbf{v'}\times(\mathbf{v'} \times \mathbf{u}) \propto  v'^2 u \qquad \text{and} \qquad (\mathbf{u}\times \mathbf{u} \times \mathbf{v'}) \propto u^2 v
      \end{equation*}
      The resulting vectors have different lengths in general, and cannot be equal.
      This result implies that this addition operator is not commutative.
      Checking associativity at the first nontrivial order in $\frac{1}{c}$:
        \begin{align*}
          (\mathbf{u}\oplus \mathbf{v'})\oplus \mathbf{w'} &\approx \mathbf{u} + \mathbf{v}' + \frac{1}{2c^2}\mathbf{u}\times(\mathbf{u} \times \mathbf{v}') - \mathbf{u}\frac{\mathbf{u}\cdot\mathbf{v}'}{c^2} - \mathbf{v}'\frac{\mathbf{u}\cdot\mathbf{v}'}{c^2}\\
                                                           & \quad + \mathbf{w'} +\frac{1}{2c^2} (\mathbf{u} + \mathbf{v}') \times \left[(\mathbf{u} + \mathbf{v}') \times \mathbf{w'}\right]\\
                                                           &= \mathbf{u} + \mathbf{v}' + \frac{1}{2c^2}\mathbf{u}\times(\mathbf{u} \times \mathbf{v}') - \mathbf{u}\frac{\mathbf{u}\cdot\mathbf{v}'}{c^2} - \mathbf{v}'\frac{\mathbf{u}\cdot\mathbf{v}'}{c^2}\\
                                                           & \quad + \mathbf{w'} +\frac{1}{2c^2} (\mathbf{u} + \mathbf{v}') \times \left[\mathbf{u} \times \mathbf{w'} + \mathbf{v}'\times \mathbf{w'} \right]\\
                                                           &= \mathbf{u} + \mathbf{v}' + \frac{1}{2c^2}\mathbf{u}\times(\mathbf{u} \times \mathbf{v}') - \mathbf{u}\frac{\mathbf{u}\cdot\mathbf{v}'}{c^2} - \mathbf{v}'\frac{\mathbf{u}\cdot\mathbf{v}'}{c^2}\\
                                                           & \quad + \mathbf{w'} +\frac{1}{2c^2} (\mathbf{u} \times \mathbf{u} \times \mathbf{w'} + \mathbf{v}'\times \mathbf{v}'\times \mathbf{w'}) 
        \end{align*}
        And:
        \begin{align*}
          \mathbf{u}\oplus (\mathbf{v'}\oplus \mathbf{w'}) &\approx \mathbf{u} + \mathbf{v}' + \mathbf{w}' + \frac{1}{2c^2}\mathbf{v'}\times(\mathbf{v'} \times \mathbf{w'}) - \mathbf{v}'\frac{\mathbf{v}'\cdot\mathbf{w}'}{c^2} - \mathbf{w}'\frac{\mathbf{v}'\cdot\mathbf{w}'}{c^2}\\
                                                           & \quad + \frac{1}{2c^2} \mathbf{u} \times \left[\mathbf{u}  \times (\mathbf{v'} + \mathbf{w'})\right]\\
                                                           &= \mathbf{u} + \mathbf{v}' + \mathbf{w}' + \frac{1}{2c^2}\mathbf{v'}\times(\mathbf{v'} \times \mathbf{w'}) - \mathbf{v}'\frac{\mathbf{v}'\cdot\mathbf{w}'}{c^2} - \mathbf{w}'\frac{\mathbf{v}'\cdot\mathbf{w}'}{c^2}\\
                                                           & \quad + \frac{1}{2c^2} \left[\mathbf{u} \times \mathbf{u}\times \mathbf{v'} +  \mathbf{u}\times \mathbf{u} \times \mathbf{w}'\right]
        \end{align*}
        The question of associativity comes down to:
        \begin{equation*}
          \mathbf{u}\frac{\mathbf{u}\cdot\mathbf{v}'}{c^2} + \mathbf{v}'\frac{\mathbf{u}\cdot\mathbf{v}'}{c^2} \stackrel{?}{=} \mathbf{v}'\frac{\mathbf{v}'\cdot\mathbf{w}'}{c^2} + \mathbf{w}'\frac{\mathbf{v}'\cdot\mathbf{w}'}{c^2}
        \end{equation*}
        One side depends on $\mathbf{w}'$ and the other one does not so these are not equal in general.
      \item
        The 4 velocity is a contravariant 4-vector so it transforms as:
        \begin{align*}
          v'^\mu &= \tensor{\Lambda}{^{\mu}_\nu} v^\nu\\
          v^\mu &= \tensor{\left(\Lambda^{-1}\right)}{^{\mu}_\nu} v^\nu
        \end{align*}
        Now, this transform can be performed for an arbitary frame with no convenient axis, but that will result in impractically copius algebra. $K$'s axes here will simply be the ones such that:
        \begin{equation*}
          u = \left(
                cu^0, u, 0, 0
              \right)
        \end{equation*}
        Where $u^0$ is the ratio rate of time passing in frame $K$ with respect to propper time.
        Under these conditions the Lorentz transform with this boost in the ${\bf u}$ direction is:
        \begin{equation*}
          \Lambda = \left(\begin{matrix}
                      \gamma          &         -\gamma u / cu^0 & 0 & 0\\ 
                      -\gamma u / cu^0 & \gamma                 & 0 & 0\\ 
                      0               & 0                      & 1 & 0\\ 
                      0               & 0                      & 0 & 1
                    \end{matrix}\right)
        \end{equation*}
        And the inverse of this matrix is a matrix such that $\Lambda^{-1}\Lambda = 1$. To this end see that:
        Now let $u = u / u^0$ which is just the magnitude of ${\bf u}$ in the $K$ frame, technically $u_0 = 1$ since the $K$ frame is the rest frame.
        \begin{equation*}
          \gamma^2 - \frac{u^2}{c^2} \gamma ^2 = \gamma^2(1 - \frac{u^2}{c^2}) = \gamma^2 (\gamma^{-2}) = 1
        \end{equation*}
        To this end, the matrix
        \begin{equation*}
          \Lambda^{-1} = \left(\begin{matrix}
                      \gamma         &        \gamma u / c & 0 & 0\\ 
                      \gamma u / c & \gamma                & 0 & 0\\ 
                      0              & 0                     & 1 & 0\\ 
                      0              & 0                     & 0 & 1
                    \end{matrix}\right)
        \end{equation*}
        Is the inverse which is the Lorentz transform for $u = (cu^0, - u, 0, 0)$.
        Now:
        \begin{align*}
          v &= \Lambda^{-1} v'\\
            & = \left(\begin{matrix}
                      \gamma         &        \gamma u / c & 0 & 0\\ 
                      \gamma u / c & \gamma                & 0 & 0\\ 
                      0              & 0                     & 1 & 0\\ 
                      0              & 0                     & 0 & 1
                    \end{matrix}\right)
                  \left(\begin{matrix}
                    c v'^0    \\
                    v'^1    \\
                    v'^2 \\
                    v'^3
                  \end{matrix}\right)\\
            &=              \left(\begin{matrix}
                                c\gamma (v'^0   + u v'^1/ c^2) \\ 
                                \gamma (u v'^0 + v'^1)   \\ 
                                v'^2                \\ 
                                v'^3
                            \end{matrix}\right)\\
            &=
                  \left(\begin{matrix}
                    c v^0  \\
                    v^1    \\
                    v^2    \\
                    v^3
                  \end{matrix}\right)
        \end{align*}
        Now to find the velocity measured in frame $K$ the spacial components (derivatives with respect to proper time) must be divided by the ratio of local time to proper time (the vectors first component except for the factor of $c$).
        Note that $v^1$ and $v'^1$ are the components parallel to ${\bf u}$ by the choice of reference frame, and the other components are parallel.
        \begin{align*}
                \left(\begin{matrix}
                    c   \\
                    {\bf v}   
                  \end{matrix}\right)
            =       
                  \left(\begin{matrix}
                    c   \\
                    v^1/v^0    \\
                    v^2/v^0   \\
                    v^3/v^0
                  \end{matrix}\right)
            &=              \left(\begin{matrix}
                                c \\ 
                                \gamma (u v'^0 + v'^1)/\left[ \gamma (v'^0   + u v'^1/ c^2)\right]   \\ 
                                v'^2 / \gamma (v'^0   + u v'^1/ c^2)/\left[ \gamma (v'^0   + u v'^1/ c^2)\right]              \\ 
                                v'^3 / \gamma (v'^0   + u v'^1/ c^2)/\left[ \gamma (v'^0   + u v'^1/ c^2)\right]
                            \end{matrix}\right)\\
            &=              \left(\begin{matrix}
                                c \\ 
                                (u  + v'^1/v'^0)/\left[ (1  + u \frac{v'^1}{v'0}/ c^2)\right]   \\ 
                                v'^2 /v'^0/\left[ \gamma (1   + u \frac{v'^1}{v'0}/ c^2)\right]              \\ 
                                v'^3 /v'^0/\left[ \gamma (1   + u \frac{v'^1}{v'0}/ c^2)\right]
                            \end{matrix}\right)\\
        \end{align*}
        Notice that:
        \begin{equation*}
          u \frac{v'^1}{v'^0} = {\bf u} \cdot {\bf v}'
        \end{equation*}
        and $v'^j/v'^0$ are just the components of the local three velocity ${\bf v}'$ of moving body.
      \item
        The composition of multiple velocities relates to boosting a boost and the question then comes down to:
        The question then becomes is the boost of a boost another boost?
        \begin{equation*}
          B(\mathbf{v}) B(\mathbf{w}) \stackrel{?}{=} B(\mathbf{v} \oplus \mathbf{w})
        \end{equation*}
        The general answer is no, unless the boosts are along collinear directions.
        This boils down to the Lorentz lie group being a closed and connected group that has both pure boosts and rotations.
        Thus the velocity composition operator does not define a group operation because subgroup of pure Lorentz boosts is not closed.
    \end{enumerate}
  \item
    \begin{enumerate}
      \item
        Note that ${\bf v} \times {\bf v} = 0$, which gives:
        \begin{align*}
          \delta {\bf v}' &= (-{\bf v}) \oplus ({\bf v + \delta v})\\
                          &= \frac{1}{1 - \frac{ v^2 - {\bf v}\cdot \delta {\bf v}}{c^2}} \left( \delta {\bf v} + \frac{\gamma(v)}{1 + \gamma(v)} \frac{- {\bf v}}{c} \times \left(\frac{- {\bf v}}{c} \times \delta {\bf v}\right)\right)\\
                          &= \frac{1}{1 - \frac{ v^2 - {\bf v}\cdot \delta {\bf v}}{c^2}}\left[ \delta {\bf v}_\parallel + \delta {\bf v}_\perp + \frac{\gamma(v)}{1 + \gamma(v)}\frac{1}{c^2} \left({\bf v}({\bf v} \cdot \delta {\bf v}) - \delta {\bf v}({\bf v} \cdot {\bf v})\right)\right]\\
                          &= \frac{1}{1 - \frac{ v^2 - {\bf v}\cdot \delta {\bf v}}{c^2}}\left[ \delta {\bf v}_\parallel + \delta {\bf v}_\perp - \frac{\gamma(v)}{1 + \gamma(v)}\frac{v^2}{c^2} \left(\delta {\bf v}({\bf v} \cdot {\bf v}) - {\bf v}({\bf v} \cdot \delta {\bf v}) )\right)\right]\\
                          &= \frac{1}{1 - \frac{ v^2 - {\bf v}\cdot \delta {\bf v}}{c^2}}\left[ \delta {\bf v}_\parallel + \delta {\bf v}_\perp - \frac{\gamma(v)}{1 + \gamma(v)}\frac{v^2}{c^2} \delta {\bf v}_\perp\right]\\
                          &= \frac{1}{1 - \frac{ v^2 - {\bf v}\cdot \delta {\bf v}}{c^2}}\left[ \delta {\bf v}_\parallel + \delta {\bf v}_\perp - (1 - \gamma(v)^{-1}) \delta {\bf v}_\perp\right]\\
                          &= \frac{1}{1 - \frac{ v^2 - {\bf v}\cdot \delta {\bf v}}{c^2}}\left[ \delta {\bf v}_\parallel +  \gamma(v)^{-1} \delta {\bf v}_\perp\right]\\
        \end{align*}
        The front factor in the last expression can be expanded about $\gamma(v)^2$:
        \begin{align*}
          \frac{1}{1 - \frac{ v^2 - {\bf v}\cdot \delta {\bf v}}{c^2}} &= \frac{1}{1 - \frac{ v^2 }{c^2} - \frac{{\bf v}\cdot \delta {\bf v}}{c^2}}\\
                                                                       &\approx \frac{1}{1 - \frac{ v^2 }{c^2} } - \frac{1}{\left(1 - \frac{ v^2 }{c^2}\right)^2 }\left(- \frac{{\bf v}\cdot \delta {\bf v}}{c^2}\right)\\
                                                                       &= \gamma(v)^2 + \gamma(v)^4 \frac{{\bf v}\cdot \delta {\bf v}}{c^2}
        \end{align*}
        Thus:
        \begin{align*}
          \delta {\bf v}' &= \gamma(v)^2 \delta {\bf v}_\parallel +  \gamma(v) \delta {\bf v}_\perp
        \end{align*}
        After dropping terms of higher order in $\delta$.
      \item
        The problem is best solved(?) if basis vectors are chosen such that ${\bf v} = (v, 0, 0)$ where $v$ is the length of ${\bf v}$.
        In this case ${\bf \delta v} = ( \delta v_\parallel, \delta v_2, \delta v_2)$.
        But now the reference frame can be rotated about ${\bf v}$ so that ${\bf \delta v} = ( \delta v_\parallel, \delta v_\perp, 0)$.
        Now, the problem at hand essentially has two space dimensions.%\footnote{Mathematica was not available at the time of writing this, and there was a necesity to avoid extra computation.}
%        For $B({\bf v + \delta v})$:
%        \begin{align*}
%          \psi''_1 &= \tanh^{-1}(v + \delta v) \\
%                  &= \frac{1}{2}\left(\ln(1 + v + \delta v_\parallel) - \ln(1 - v - \delta v_\parallel)\right)\\
%                  &= \frac{1}{2}\left(\ln(1 + v) - \ln(1 - v)\right) + \frac{\delta v_\parallel}{2}\left(\frac{1}{1 + v} + \frac{1}{1 - v}\right) + \mathcal{O}(\delta v_\parallel^2)\\
%                  &= \psi_1 + \frac{\delta v_\parallel}{2}\left(\frac{2}{1 - v^2}\right) + \mathcal{O}(\delta v_\parallel^2)\\
%                  &= \psi_1 + \gamma^2 \delta v_\parallel + \mathcal{O}(\delta v_\parallel^2)\\
%        \end{align*}
%        Similarly:
%        \begin{align*}
%          \psi''_2 &= \tanh^{-1}(\delta v_\perp) \\
%                  &= \frac{1}{2}\left(\ln(1 + \delta v_\perp) - \ln(1 - \delta v_\perp)\right)\\
%                  &= \frac{1}{2}\left(\ln(1) - \ln(1)\right) + \frac{\delta v_\perp}{2}\left(1 + 1\right) + \mathcal{O}(\delta v_\perp^2)\\
%                  &= \delta v_\perp + \mathcal{O}(\delta v_\perp^2)
%        \end{align*}
%        And $\psi''_3 = 0$.
%        For $B(-{\bf v})$:
%        \begin{align*}
%          \tanh^{-1}(-v )
%                  &= \frac{1}{2}\left(\ln(1 - v ) - \ln(1 + v )\right)\\
%                  &= - \frac{1}{2}\left(\ln(1 + v ) - \ln(1 - v )\right)\\
%                  &= - \psi_1
%        \end{align*}
%        Only keeping the terms of order $\delta v_*$ and using the BCH formula:
%        \begin{equation*}
%          \exp (A) \exp(B) = \exp(A + B + \frac{1}{2}[A, B] + \frac{1}{12}[A, [A, B]] + ...)
%        \end{equation*}
%        Now isolating the exponents and inserting the boosts:
%        \begin{align*}
%          &\psi_1 K_1 + \gamma^2 \delta v_\parallel K_1 + \delta v_\perp K_2 - \psi_1 K_1 + \frac{1}{2} \left[\gamma^2 \delta v_\parallel K_1 - \delta v_\perp K_2, \psi_1 K_1\right]\\
%          =&  \gamma^2\delta v_\parallel K_1 +  \delta v_\perp K_2  - \frac{1}{2} \left[ \delta v_\perp K_2, \psi_1 K_1\right]\\
%          =&  \gamma^2\delta v_\parallel K_1 +  \delta v_\perp K_2  + \frac{1}{2} \delta v_\perp \psi_1 J_3\\
%        \end{align*}
%        \begin{align*}
%          &\exp\left\{\psi_1 K_1 + \gamma^2 \delta v_\parallel K_1 + \delta v_\perp K_2 \right\}\\
%          &\exp\left\{\delta v_\perp K_2 + \psi_1 K_1 + \gamma^2 \delta v_\parallel K_1 \right\}\\
%          &\exp\left\{\delta v_\perp K_2 + \frac{1}{2}\delta v_\perp\psi_1 J_3\right\} \exp\left\{\psi_1 K_1 + \gamma^2 \delta v_\parallel K_1 \right\}\\
%        \end{align*}
        \begin{align*}
          B({\bf v} + \delta {\bf v}) =
          \left(
        \begin{matrix}
          \tilde \gamma    &-\tilde\gamma (v + \delta v_\parallel)/c                   &-\tilde\gamma \delta v_\perp/c                   &0  \\                  
          -\tilde\gamma (v + \delta v_\parallel)/c       &1+(\tilde\gamma-1)\frac{(v + \delta v_\parallel)^2}  {\tilde v^2}&  (\tilde\gamma-1)\frac{(v + \delta v_\parallel)\delta v_\perp}{\tilde v^2}&  0   \\
        -\tilde\gamma v_\perp/c&  (\tilde\gamma-1)\frac{(v + \delta v_\parallel)\delta v_\perp}{\tilde v^2}&1+(\tilde\gamma-1)\frac{\delta v_\perp^2}{\tilde v^2}&  0                              \\
          0                                             &  0                             & 0 & 1                               
        \end{matrix}
          \right)
        \end{align*}
        Where:
        \begin{align*}
          \tilde\gamma &= \gamma(|{\bf v} + \delta {\bf v}|)\\
                       &= \frac{1}{\sqrt{1 - \frac{v^2 - 2 v \delta v_\parallel}{c^2} }}\\
                       &= \gamma(v) + \frac{v \delta v_\parallel}{c^2} \frac{1}{\sqrt{1 - v^2 }^3} + \mathcal{O}(\delta v_\parallel^2)\\
                       &= \gamma(v) + v \delta v_\parallel \gamma^3/c^2 + \mathcal{O}(\delta v_\parallel^2)
        \end{align*}
        And:
        \begin{equation*}
          \tilde v^2 = |{\bf v} + \delta {\bf v}|^2 = (v + \delta v_\parallel)^2 + \delta v_\perp^2 = (v + \delta v_\parallel)^2
        \end{equation*}
        \begin{align*}
          B({\bf v} + \delta {\bf v}) &=
          \left(
        \begin{matrix}
          \tilde \gamma    &-\tilde\gamma (v + \delta v_\parallel)/c                   &-\tilde\gamma \delta v_\perp/c                   &0  \\                  
          -\tilde\gamma (v + \delta v_\parallel)/c       &   \tilde\gamma                                                  &  (\tilde\gamma-1)\frac{\delta v_\perp}{(v + \delta v_\parallel)}&  0   \\
          -\tilde\gamma \delta v_\perp/c&  (\tilde\gamma-1)\frac{\delta v_\perp}{(v + \delta v_\parallel)}               &  1                               & 0\\
          0                                             &  0                             & 0 & 1                               
        \end{matrix}
          \right)\\
          &=
          \left(
        \begin{matrix}
          \gamma + v \delta v_\parallel \gamma^3/c^2    &- (\gamma v + v^2 \delta v_\parallel \gamma^3/c^2 + \gamma\delta v_\parallel)/c                   &-\gamma \delta v_\perp/c                   &0  \\                  
          - (\gamma v + v^2 \delta v_\parallel \gamma^3/c^2 + \gamma\delta v_\parallel)/c &   \gamma + v \delta v_\parallel \gamma^3/c^2                      &  (\gamma-1)\frac{\delta v_\perp}{v}&  0   \\
          -\gamma \delta v_\perp/c&  (\gamma-1)\frac{\delta v_\perp}{v}               &  1                               & 0\\
          0                                             &  0                             & 0 & 1                               
        \end{matrix}
          \right)\\
          &=
          \left(
        \begin{matrix}
          \gamma + v \delta v_\parallel \gamma^3/c^2    &- (\gamma v + \delta v_\parallel \gamma^3)/c                   &-\gamma \delta v_\perp/c                   &0  \\                  
          - (\gamma v + \delta v_\parallel \gamma^3 )/c &   \gamma + v \delta v_\parallel \gamma^3/c^2                      &  (\gamma-1)\frac{\delta v_\perp}{v}&  0   \\
          -\gamma \delta v_\perp/c&  (\gamma-1)\frac{\delta v_\perp}{v}               &  1                               & 0\\
          0                                             &  0                             & 0 & 1                               
        \end{matrix}
          \right)
        \end{align*}
        since
        \begin{align*}
          \gamma v + v^2 \delta v_\parallel \gamma^3/c^2 + \gamma\delta v_\parallel 
          &= \gamma v + \delta v_\parallel \gamma (v^2 \delta v_\parallel \gamma^2/c^2 + 1)\\
          &= \gamma v + \delta v_\parallel \gamma \left(\frac{v^2 /c^2}{1 - v^2/c^2} + 1\right)\\
          &= \gamma v + \delta v_\parallel \gamma^3  
        \end{align*}
        For $B({\bf v + \delta v})$:
        \begin{align*}
          B(-\mathbf{v}) =
          \left(
          \begin{matrix}
          \gamma       & \gamma v/c                     &0                               &0                                \\
          \gamma v  /c &   \gamma                       &0                               &0                                \\
          0            &0                               &1                               &0                                \\
          0            &0                               &0                               &1
          \end{matrix}
          \right)
        \end{align*}

        \begin{align*}
          &B({\bf v} + \delta \mathbf{v})B(- \mathbf{v})\\
          &=
          \left(
          \begin{matrix*}
            \gamma^2(1-v^2/c^2)        & v^2 \delta v_\parallel \gamma^4 / c^3 - \gamma^4\delta v_\parallel /c     &\gamma(\gamma - 1) \delta  v_\perp /c-\gamma^2 \delta  v_\perp/c                      &0                                \\
            v^2 \delta v_\parallel \gamma^4 / c^3 - \gamma^4\delta v_\parallel /c & \gamma^2(1-v^2/c^2)            &(\gamma-1)\frac{\delta v_\perp}{v}                               &0                                \\
            \gamma(\gamma - 1) \delta  v_\perp /c-\gamma^2 \delta  v_\perp /c        &\gamma (\gamma-1)\frac{\delta v_\perp}{v} - \gamma^2 v \delta v_\perp/c^2                              &1                               &0                                \\
            0                          &0                               &0                               &1
          \end{matrix*}
          \right)\\
          &=
          \left(
          \begin{matrix*}
            1                          & \frac{1}{c}\gamma^4\delta v_\parallel (v^2   / c^2 - 1)     &- \gamma \delta  v_\perp /c                      &0                                \\
            \frac{1}{c}\gamma^4\delta v_\parallel (v^2   / c^2 - 1) &  1                             &(\gamma-1)\frac{\delta v_\perp}{v}                               &0                                \\
            - \gamma \delta  v_\perp /c       &v\delta v_\perp(-\gamma + \gamma^2 -\gamma^2v^2/c^2)                               &1                               &0                                \\
            0                          &0                               &0                               &1
          \end{matrix*}
          \right)\\
          &=
          \left(
          \begin{matrix*}
            1                          & -\gamma^2\delta v_\parallel/c    &- \gamma \delta  v_\perp /c                      &0                                \\
            -\gamma^2\delta v_\parallel/c &  1                             &\frac{\gamma-1}{v^2}v\delta v_\perp                               &0                                \\
            - \gamma \delta  v_\perp /c       &- \frac{\gamma-1}{v^2}v\delta v_\perp                               &1                               &0                                \\
            0                          &0                               &0                               &1
          \end{matrix*}
          \right)\\
          &=
          I 
          -\gamma^2\delta v_\parallel/c
          \left(
          \begin{matrix}
          0            & 1                              &0                               &0                                \\
          1            & 0                              &0                               &0                                \\
          0            &0                               &0                               &0                                \\
          0            &0                               &0                               &0
          \end{matrix}
          \right)
          - \gamma \delta  v_\perp /c
          \left(
          \begin{matrix}
          0            & 0                              &1                               &0                                \\
          0            & 0                              &0                               &0                                \\
          1            &0                               &0                               &0                                \\
          0            &0                               &0                               &0
          \end{matrix}
          \right)
          -\frac{\gamma-1}{v^2} v \delta v_\perp
          \left(
          \begin{matrix}
          0            & 0                              &0                               &0                                \\
          0            & 0                              &-1                              &0                                \\
          0            & 1                              &0                               &0                                \\
          0            &0                               &0                               &0
          \end{matrix}
          \right)\\
          &=
          I 
          -(\gamma^2\delta v_\parallel/c) K_1
          -(\gamma \delta  v_\perp /c) K_2
          -\frac{\gamma-1}{v^2} v \delta v_\perp K_3\\
          &=
          I 
          -\delta \boldsymbol{\psi}'\cdot \mathbf{K}
          -\delta \boldsymbol{\theta}\cdot \mathbf{J}
        \end{align*}
        Since $\boldsymbol{\theta}$ only has a component in the third index:
        \begin{align*}
          \delta \boldsymbol{\theta} = \frac{\gamma-1}{v^2} \mathbf{v} \times \mathbf{\delta v}_\perp = \frac{\gamma^2-1}{(1 +\gamma)v^2 /c^2}\frac{1}{c^2} \mathbf{v} \times \mathbf{\delta v} = \frac{\frac{v^2/c^2}{1- v^2 / c^2}}{(1 +\gamma )v^2 /c^2}\frac{1}{c^2} \mathbf{v} \times \mathbf{\delta v} = \frac{1}{c^2}\frac{\gamma^2}{1 +\gamma } \mathbf{v} \times \mathbf{\delta v}
        \end{align*}
      \item
        Boosts do not form a subgroup of the Lorentz group.
        The last question showed that the combination of two boosts is a boost times a rotation(not a boost).
        Therefore, boosts do not satisfy the property of groups that requires the group to be closed under the combination operation.
      \item
        An acelerating observer is locally boosted along a trajectory with respect to the local change in velocity.
        This is a Lorentz transfrom between two frames that the accelerating frame has at two different times.
        An external laboratory frame cannot find out the local boost of the accelerating frame directly, because it is a non-inertial frame.
        The two frames, however, are instantaneously inertial frames and so the laboratory can find transforms, boosts between the two frames and the laborotory.
        When the laboratory combines the two transforms, the resulting transform is a boost and a rotation.
        This occurs between every instantaneous inertial frame of the accelerating object, so the accelerating object has an extra rotation in the laborotory frame resulting in a deviation in its orbit.
        The deviation could create a precession.
        If the accelerating object experiences a balanced force, so the object does not spiral towards or away from the acceleration's source, the laborotory can never see the object crash into the source or go to infinity so it must see the object's orbit precess.
      \item
        Considering that normal orbit changes by $\delta \boldsymbol{\theta}$ for every small time interval $\delta t$, $\boldsymbol{\omega}_L$ would be the limit of $\frac{\delta \boldsymbol{\theta}}{\delta t}$ as $\delta t$ goes to zero.
        \begin{align*}
          \boldsymbol{\omega}_T &= \lim_{\delta t \to 0} \frac{\delta \boldsymbol{\theta}}{\delta t}\\
                                &= \lim_{\delta t \to 0} \frac{\gamma - 1}{v^2} \mathbf{v} \times \frac{\mathbf{\delta v}}{\delta t}\\
                                &= \frac{\gamma - 1}{v^2} \mathbf{v} \times \mathbf{a}\\
                                &= \frac{1}{m}\frac{\gamma - 1}{v^2} \mathbf{v} \times \mathbf{F}\\
                                &= \frac{1}{m}\frac{\gamma - 1}{v^2} \mathbf{v} \times \frac{d V}{dr} (-\mathbf{\hat r})\\
                                &= \frac{1}{mr}\frac{\gamma - 1}{v^2}  \frac{d V}{dr} \mathbf{r}\times \mathbf{v} \\
                                &= \frac{1}{m^2r}\frac{\gamma - 1}{v^2}  \frac{d V}{dr} \mathbf{r}\times \mathbf{p} \\
                                &= \frac{1}{m^2r}\frac{\gamma - 1}{v^2}  \frac{d V}{dr} \mathbf{L} 
        \end{align*}
        Now the Lorentz factor can be expanded:
        \begin{equation*}
          \gamma \approx 1 + \frac{1}{2} \frac{v^2}{c^2} + \mathcal{O}\left(\frac{v^4}{c^4}\right)
        \end{equation*}
        This result give a first order approximation for the Thomas precession frequency.
        \begin{align*}
          \boldsymbol{\omega}_T &= \frac{1}{m^2c^2}\frac{1}{2}\frac{1}{r}  \frac{d V}{dr} \mathbf{L} 
        \end{align*}
        Which is off by a minus sign, but the dependences look good.
    \end{enumerate}
\end{enumerate}
\appendix
\section{The product of a shifted boost and it's inverse}

I started working on this and I thought it was a bit instructive, but I could not get a clear answer.

        For $B({\bf v + \delta v})$:
        \begin{align*}
          \psi''_1 &= \tanh^{-1}(v + \delta v) \\
                  &= \frac{1}{2}\left(\ln(1 + v + \delta v_\parallel) - \ln(1 - v - \delta v_\parallel)\right)\\
                  &= \frac{1}{2}\left(\ln(1 + v) - \ln(1 - v)\right) + \frac{\delta v_\parallel}{2}\left(\frac{1}{1 + v} + \frac{1}{1 - v}\right) + \mathcal{O}(\delta v_\parallel^2)\\
                  &= \psi_1 + \frac{\delta v_\parallel}{2}\left(\frac{2}{1 - v^2}\right) + \mathcal{O}(\delta v_\parallel^2)\\
                  &= \psi_1 + \gamma^2 \delta v_\parallel + \mathcal{O}(\delta v_\parallel^2)\\
        \end{align*}
        Similarly:
        \begin{align*}
          \psi''_2 &= \tanh^{-1}(\delta v_\perp) \\
                  &= \frac{1}{2}\left(\ln(1 + \delta v_\perp) - \ln(1 - \delta v_\perp)\right)\\
                  &= \frac{1}{2}\left(\ln(1) - \ln(1)\right) + \frac{\delta v_\perp}{2}\left(1 + 1\right) + \mathcal{O}(\delta v_\perp^2)\\
                  &= \delta v_\perp + \mathcal{O}(\delta v_\perp^2)
        \end{align*}
        And $\psi''_3 = 0$.
        For $B(-{\bf v})$:
        \begin{align*}
          \tanh^{-1}(-v )
                  &= \frac{1}{2}\left(\ln(1 - v ) - \ln(1 + v )\right)\\
                  &= - \frac{1}{2}\left(\ln(1 + v ) - \ln(1 - v )\right)\\
                  &= - \psi_1
        \end{align*}
        Only keeping the terms of order $\delta v_*$ and using the BCH formula:
        \begin{equation*}
          \exp (A) \exp(B) = \exp(A + B + \frac{1}{2}[A, B] + \frac{1}{12}\left([A, [A, B]] + [B, [B, A]]\right) + ...)
        \end{equation*}
        Now isolating the exponents and inserting the basis vectors, but only looking at the first commutator:
        \begin{align*}
          &\psi_1 K_1 + \gamma^2 \delta v_\parallel K_1 + \delta v_\perp K_2 - \psi_1 K_1 + \frac{1}{2} \left[\gamma^2 \delta v_\parallel K_1 - \delta v_\perp K_2, \psi_1 K_1\right]\\
          =&  \gamma^2\delta v_\parallel K_1 +  \delta v_\perp K_2  - \frac{1}{2} \left[ \delta v_\perp K_2, \psi_1 K_1\right]\\
          =&  \gamma^2\delta v_\parallel K_1 +  \delta v_\perp K_2  + \frac{1}{2} \delta v_\perp \psi_1 J_3
        \end{align*}
        It is now clear that there is a rotation around the $3$ axis that results from the boost basis vectors not commuting.
        There are many terms like this in order $\delta v_\perp$ alternating on the basis $K_2$ and $J_3$.
        The factor in front of $K_2$ probably comes to $\cosh \psi_1$, but the $J_3$ cofactor might look a bit wierd.
        Working it out would probably be done using \href{https://en.wikipedia.org/wiki/Baker\%E2\%80\%93Campbell\%E2\%80\%93Hausdorff_formula}{Dynkin's formula} and combinatorial analyis since certain combination of $J_3$, $K_1$ and $K_2$ will give specific terms and the sum can split.
        That being said, working it out explicity is probably quite heavy going.

\end{document}
